% sections/introducao/sobre_as_ferramentas.tex
% !TeX root = ../../main.tex

\section{Sobre as ferramentas}

Conforme explicado acima, o Google \gls{workspace} oferece aos usuários um conjunto de 
ferramentas online e gratuitas que possibilitam a criação, edição e compartilhamento de 
diferentes tipos de \gls{arquivo}s, tanto individualmente quanto de forma colaborativa. 
Nas próximas seções, abordaremos as principais funcionalidades desse conjunto, contemplando 
a explicação e o passo a passo para o uso de cada uma delas, que estão brevemente 
apresentadas abaixo:

\begin{itemize}
    \item \textbf{Google Documentos} (Docs): Editor de textos online correspondente ao Microsoft 
    Word, que permite criar, formatar e compartilhar documentos;
 
    \item \textbf{Google Planilhas} (Sheets): Ferramenta com a mesma funcionalidade do Microsoft 
    Excel, utilizada para organização de dados e aplicação de fórmulas, criação de 
    gráficos e análises;

    \item \textbf{Google Apresentações} (Slides): Equivalente ao PowerPoint, permitindo a criação 
    de apresentações de \gls{slide} dinâmicas, com imagens, vídeos e animações;

    \item \textbf{Google Formulários} (Forms): Recurso para a criação de questionários, pesquisas e 
    enquetes, possibilitando também o acompanhamento dos resultados em tempo real e a 
    exportação deles para o Google Planilhas;

    \item \textbf{Google Drive}: Serviço de armazenamento em \gls{nuvem} da Google, onde todos os 
    \gls{arquivo}s gerados pelas outras ferramentas são armazenados de forma segura e podem 
    ser compartilhados;

    \item \textbf{Google Agenda}: O Google Agenda é um calendário digital para organização de 
    eventos, reuniões e demais lembretes pessoais e/ou compartilhados, com opção de 
    integração a outras ferramentas;

    \item \textbf{Gmail}: Serviço de e-mail da Google amplamente utilizado e que conta com recursos 
    de integração ao Drive, Agenda e demais aplicativos do Google \gls{workspace};

    \item \textbf{Motor de Busca}: Principal ferramenta de busca/pesquisa na internet, utilizada 
    para localizar informações de forma rápida e organizada por meio dos índices. 
\end{itemize}