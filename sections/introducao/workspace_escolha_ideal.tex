% sections/introducao/workspace_escolha_ideal.tex
% !TeX root = ../../main.tex

\section{Por que o Google Workspace é a escolha ideal?}

Atualmente, contamos com diversos pacotes de ferramentas conhecidas no mercado, como o 
pacote Microsoft Office, LibreOffice, Google \gls{workspace} e outras. Esses pacotes, em 
geral, possuem as ferramentas mais comuns e essenciais para o mercado corporativo e 
educacional, como editores de texto, editores de planilhas eletrônicas e apresentadores 
de \gls{slide}. A escolha de uso de determinado pacote depende de vários aspectos, tais 
como valor, compatibilidade, complexidade e ambiente de execução. 

Atualmente, o pacote da Microsoft, também conhecido como Microsoft 365, é considerado o 
padrão de mercado, adotado por diversas empresas, principalmente por estar presente desde 
1990. A Microsoft conta com ferramentas como: o Word para edição de texto, Excel para 
planilhas eletrônicas e PowerPoint para apresentações de \gls{slide}. Essas ferramentas 
possuem uma gama de recursos embutidos, tornando-se complexas e robustas. Além disso, o 
pacote está disponível para uso apenas com uma assinatura e é disponibilizado no formato 
desktop, ou seja, para instalar no computador. 

O pacote LibreOffice também ganhou destaque no mercado principalmente por possuir as 
mesmas ferramentas da Microsoft, porém de forma gratuíta. As aplicações que ganham 
destaque são: o Writer para edição de textos, o Calc para planilhas eletrônicas e o 
Impress para apresentações de slides. Este pacote também é disponibilizado no formato 
\gls{desktop}, mas possui funções e recursos menos avançados quando comparado com os 
demais pacotes, além de dispor de uma interface pouco intuitiva e que exige mais atenção 
do profissional que o utilizará.

Por fim, destacamos o Google \gls{workspace} que, de igual forma, possui as mesmas 
funcionalidades: o Google Docs para edição de textos, o Google Sheets para planilhas 
eletrônicas e o Google Slides para apresentações. O grande diferencial do pacote da 
Google está em sua gratuidade e qualidade das ferramentas, contando com opções de 
compartilhamento e acesso via computador ou celular. Além disso, o pacote da Google é 
\gls{online} sem ter a necessidade de instalar nada no computador para utilizar. 

Em vista disso, as ferramentas da Google foram escolhidas para a criação deste material, 
com aplicações gratuitas, fáceis, intuitivas e que pudessem ser úteis para o maior número 
de pessoas nas suas determinadas funções. O intuito do material é esclarecer o uso e tirar 
dúvidas sobre essas ferramentas, fazendo com que mais pessoas tenham acesso a um pacote 
office de qualidade e gratuíto. E também, busca-se incentivar o uso e a prática de cada 
recurso, gerando um benefício para cada profissional em sua área de atuação, desde a 
organização do dia-a-dia até a elaboração de planilhas e compartilhamento remoto. 
