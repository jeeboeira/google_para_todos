% sections/introducao/introducao.tex
% !TeX root = ../main.tex

\section{Apresentação da apostila}

Olá, seja bem-vindo(a) à apostila “Google Para Todos”! Neste material, 
você irá encontrar um passo-a-passo completo para aprender a utilizar as 
principais ferramentas online e gratuitas do Google Workspace. Além de 
operar essas funcionalidades, você terá a habilidade de aplicá-las em 
cenários do cotidiano que, cada vez mais, estão imersos na tecnologia, 
como trabalho, estudos e até projetos pessoais. Este é um guia feito para 
facilitar seu aprendizado e ajudar você a utilizar, de forma confiante, 
a tecnologia no dia a dia.

Para contextualizar o tema desta apostila, apresentamos alguns dados de 
pesquisas sobre o cenário tecnológico no Brasil: 

\begin{itemize}
    \item Nos últimos anos, segundo o Movimento Brasil Competitivo (MBC; FGV, 2022), 
    as ocupações profissionais relacionadas às atividades digitais apresentaram 
    um crescimento de \SI{4.9}{\percent} em relação às demais ocupações, o que torna evidente 
    a importância do conhecimento tecnológico e digital para o mercado;
    
    \item A Pesquisa Nacional por Amostra de Domicílios Contínua – PNAD Contínua (2022),
    também apresentou a relevância das ferramentas digitais ao indicar que 7,4 
    milhões de brasileiros estavam trabalhando por meios remotos (home office) 
    no ano de 2022.
    
    \item Segundo a Pesquisa de Inovação Semestral 2022 do IBGE (2023), a computação em 
    nuvem é a tecnologia avançada mais adotada entre as empresas brasileiras, 
    sendo que 87\% delas possuem 500 ou mais empregados, 76,8\% de 250 a 499 
    empregados e 68\% de 100 a 249 empregados. Isso revela que, tecnologias em 
    nuvem, como o Google Workspace, são as mais utilizadas no mercado de trabalho, 
    visto que são soluções acessíveis, colaborativas e produtivas. 
\end{itemize}
    