% sections/extras/definicao_glossariotex
% !TeX root = ../../main.tex

% ======== Criar glossários por categoria ========
\newglossary[gg]{geral}{glg}{gls}{Geral}
\newglossary[dd]{documentos}{dlg}{dls}{Documentos}
\newglossary[pp]{planilhas}{plg}{pls}{Planilhas}
\newglossary[aa]{apresentacoes}{alg}{als}{Apresentações}
\newglossary[ag]{agendagmail}{agg}{ags}{Agenda e Gmail}
\newglossary[dr]{drivebusca}{drg}{drs}{Drive e Motor de Busca}
\newglossary[ff]{formularios}{flg}{fls}{Formulários}

% =========================================================
% ====================== GERAL =============================
% =========================================================
\newglossaryentry{aba}{type=geral,name=Aba,description={Parte de uma janela que permite abrir várias páginas ou seções ao mesmo tempo, como em um navegador}}
\newglossaryentry{algoritmo}{type=geral,name={Algoritmo},description={Sequência de raciocínios, passos ou instruções definidas para alcançar um objetivo, sendo necessário que as etapas sejam finitas e executadas sistematicamente}}
\newglossaryentry{aplicativos}{type=geral,name={Aplicativos (Apps)},description={Programas que você instala e usa no celular, tablet ou computador para realizar tarefas}}
\newglossaryentry{arquivo}{type=geral,name={Arquivo},description={Documento ou item salvo no computador, como textos, imagens ou vídeos}}
\newglossaryentry{backup}{type=geral,name={Backup},description={Cópia de segurança dos seus arquivos, feita para evitar a perda de dados em caso de problemas no dispositivo}}
\newglossaryentry{barrapesquisa}{type=geral,name={Barra de pesquisa},description={Campo onde você digita palavras para procurar informações ou arquivos}}
\newglossaryentry{caractere}{type=geral,name={Caractere},description={Cada símbolo utilizado na escrita digital, como letras, números, espaços ou sinais de pontuação}}
\newglossaryentry{card}{type=geral,name={Card},description={Elemento visual usado em interfaces para apresentar informações de forma organizada e destacada, geralmente com imagem, título e descrição}}
\newglossaryentry{cursor}{type=geral,name={Cursor},description={Seta ou linha piscando na tela que mostra onde você está digitando ou clicando}}
\newglossaryentry{default}{type=geral,name={Default},description={O modo padrão de algo}}
\newglossaryentry{desktop}{type=geral,name={Desktop (Área de trabalho)},description={Tela principal do computador onde ficam ícones de arquivos, pastas e programas Também pode significar o tipo de programa que roda de forma integrada ao computador, diferente de um site (web) ou aplicativo móvel}}
\newglossaryentry{diretorio}{type=geral,name={Diretório (ou Pasta)},description={Local usado para guardar e organizar arquivos no computador}}
\newglossaryentry{download}{type=geral,name={Download},description={Ação de copiar um arquivo da internet para o seu computador, celular ou outro dispositivo}}
\newglossaryentry{extensoes}{type=geral,name={Extensões},description={Programas extras que adicionam novas funções a um navegador ou aplicativo}}
\newglossaryentry{feedback}{type=geral,name={Feedback},description={Informação de retorno fornecida sobre uma ação, desempenho ou resposta, com o objetivo de reforçar acertos ou indicar áreas para melhoria}}
\newglossaryentry{googlemaps}{type=geral,name={Google Maps},description={Serviço de mapas online do Google que permite visualizar locais, traçar rotas e explorar regiões do mundo por meio de imagens de satélite e mapas interativos}}
\newglossaryentry{html}{type=geral,name={HTML},description={Sigla para \textit{HyperText Markup Language}, utilizada para criar e estruturar páginas web}}
\newglossaryentry{link}{type=geral,name={Link},description={Endereço clicável que leva você a uma página da internet ou a um arquivo}}
\newglossaryentry{login}{type=geral,name={Login},description={Processo de entrar em uma conta usando seu nome de usuário e senha para acessar serviços}}
\newglossaryentry{mobile}{type=geral,name={Mobile},description={Dispositivo móvel}}
\newglossaryentry{nuvem}{type=geral,name={Nuvem},description={Espaço na internet onde você pode guardar arquivos e acessá-los de qualquer lugar com internet}}
\newglossaryentry{offline}{type=geral,name={Offline},description={Quando um dispositivo não está conectado à internet}}
\newglossaryentry{online}{type=geral,name={Online},description={Quando um dispositivo está conectado à internet}}
\newglossaryentry{pdf}{type=geral,name={PDF},description={Formato de arquivo que mantém a aparência original de textos, imagens e gráficos, permitindo que o documento seja visualizado e impresso da mesma forma em qualquer dispositivo}}
\newglossaryentry{popup}{type=geral,name={Pop-up},description={Janela ou elemento que aparece automaticamente na tela, geralmente para exibir mensagens, avisos ou anúncios}}
\newglossaryentry{qrcode}{type=geral,name={QRCode},description={Código bidimensional que pode ser escaneado por câmeras ou aplicativos, direcionando o usuário a um site, texto ou outro conteúdo digital}}
\newglossaryentry{software}{type=geral,name={Software},description={Programa que você usa no computador ou celular para realizar tarefas, como escrever textos ou acessar a internet}}
\newglossaryentry{spam}{type=geral,name={Spam},description={Mensagens indesejadas, geralmente enviadas em massa, que aparecem no e-mail ou em outros meios digitais}}
\newglossaryentry{upload}{type=geral,name={Upload},description={Ação de enviar um arquivo do seu dispositivo para a internet ou para um sistema online}}
\newglossaryentry{url}{type=geral,name={URL},description={Endereço que indica o local específico de um recurso localizado na internet}}
\newglossaryentry{web}{type=geral,name={Web},description={Conjunto de páginas e serviços acessados pela internet}}
\newglossaryentry{workspace}{type=geral,name={Workspace (Área de trabalho)},description={Espaço virtual onde você organiza e acessa seus arquivos, projetos ou ferramentas dentro de um software}}
\newglossaryentry{zip}{type=geral,name={zip},description={Formato de compactação de arquivos que reduz o tamanho dos dados e permite agrupar vários arquivos em um único pacote, facilitando o armazenamento e o envio}}

% =========================================================
% ====================== DOCUMENTOS ========================
% =========================================================
\newglossaryentry{areatransferencia}{type=documentos,name={Área de transferência},description={Espaço temporário do sistema usado para armazenar dados copiados ou recortados, permitindo colá-los em outro local}}
\newglossaryentry{usb}{type=documentos,name={USB (Universal Serial Bus)},description={Padrão de conexão e comunicação que permite ligar periféricos ao computador, transmitindo também energia elétrica}}
\newglossaryentry{libreofficewriter}{type=documentos,name={LibreOffice Writer},description={Processador de textos de código aberto e gratuito, utilizado para criar e editar documentos compatíveis com o formato odt}}
\newglossaryentry{microsoftword}{type=documentos,name={Microsoft Word},description={Processador de textos da Microsoft com recursos avançados de formatação, compatível com o formato docx}}
\newglossaryentry{ortografia}{type=documentos,name={Ortografia},description={Conjunto de regras que define o modo correto de escrever as palavras de uma língua}}
\newglossaryentry{gramatica}{type=documentos,name={Gramática},description={Estudo da estrutura das palavras e das regras que governam a combinação dessas palavras em frases e textos}}

% =========================================================
% ====================== PLANILHAS =========================
% =========================================================
\newglossaryentry{celula}{type=planilhas,name={Célula},description={Espaço individual em uma planilha onde você pode digitar números, textos ou fórmulas}}
\newglossaryentry{dashboard}{type=planilhas,name={Dashboard (Painel de controle)},description={Tela que mostra informações organizadas de forma visual para ajudar na análise e tomada de decisões}}
\newglossaryentry{ecommerce}{type=planilhas,name={E-commerce (Comércio eletrônico)},description={Loja ou negócio que vende produtos ou serviços pela internet}}
\newglossaryentry{marketplace}{type=planilhas,name={Marketplace},description={Plataformas online onde vários vendedores oferecem produtos ou serviços, como se fosse um shopping virtual}}
\newglossaryentry{script}{type=planilhas,name={Script},description={Conjunto de comandos ou instruções usados para automatizar tarefas em programas ou páginas da internet}}

% =========================================================
% ====================== APRESENTAÇÕES =====================
% =========================================================
\newglossaryentry{braille}{type=apresentacoes,name={Braille},description={Sistema de escrita e leitura tátil para pessoas cegas ou com baixa visão, composto por combinações de pontos em relevo}}
\newglossaryentry{layout}{type=apresentacoes,name={Layout},description={Organização visual dos elementos em uma página, tela ou documento}}
\newglossaryentry{notificacao}{type=apresentacoes,name={Notificação},description={Mensagem ou alerta gerado por um aplicativo para informar o usuário sobre eventos ou ações que requerem atenção}}
\newglossaryentry{slide}{type=apresentacoes,name={Slide},description={Página individual de uma apresentação digital usada para exibir informações visuais e textuais}}

% =========================================================
% ====================== AGENDA E GMAIL ====================
% =========================================================
\newglossaryentry{ics}{type=agendagmail,name={ics},description={Extensão de arquivo utilizada para armazenar informações de calendário, como eventos e compromissos}}
\newglossaryentry{caldav}{type=agendagmail,name={CalDAV},description={Protocolo de internet que permite o acesso e a sincronização de calendários entre dispositivos e aplicativos}}
\newglossaryentry{chat}{type=agendagmail,name={Chat},description={Aplicativo do Google para enviar mensagens instantâneas e colaborar em tempo real}}
\newglossaryentry{email}{type=agendagmail,name={E-mail},description={Serviço de envio e recebimento de mensagens digitais pela internet, podendo incluir textos e anexos}}
\newglossaryentry{meet}{type=agendagmail,name={Meet},description={Plataforma do Google para videoconferências e reuniões online}}
\newglossaryentry{phishing}{type=agendagmail,name={Phishing},description={Tipo de golpe digital que tenta enganar o usuário para obter informações pessoais por meio de mensagens falsas}}
\newglossaryentry{tooltip}{type=agendagmail,name={Tooltip},description={Pequena caixa de texto que aparece quando o usuário passa o cursor sobre um elemento, exibindo uma breve explicação ou dica}}

% =========================================================
% ====================== DRIVE E BUSCA =====================
% =========================================================
\newglossaryentry{canva}{type=drivebusca,name={Canva},description={Plataforma de design gráfico para criação de conteúdo visual como artes, pôsteres e apresentações}}
\newglossaryentry{docusign}{type=drivebusca,name={DocuSign},description={Plataforma online de assinatura eletrônica e gerenciamento de contratos}}
\newglossaryentry{gatilho}{type=drivebusca,name={Gatilho},description={Evento ou condição que inicia automaticamente uma sequência de ações em um processo de automação}}
\newglossaryentry{googleads}{type=drivebusca,name={Google Ads},description={Plataforma de publicidade online do Google para criar anúncios pagos exibidos em resultados de pesquisa e sites parceiros}}
\newglossaryentry{googlebusiness}{type=drivebusca,name={Google Business},description={Ferramenta do Google que ajuda empresas a gerenciar sua presença online}}
%\newglossaryentry{marketplace}{type=drivebusca,name={Google Workspace Marketplace},description={Loja online de aplicativos e extensões que adicionam funcionalidades aos serviços do Google Workspace}}
\newglossaryentry{homepage}{type=drivebusca,name={Homepage (Página inicial)},description={Primeira página que aparece ao abrir um site ou navegador}}
\newglossaryentry{ifttt}{type=drivebusca,name={IFTTT},description={Plataforma que permite criar cadeias de automação com base na lógica "Se isso acontecer, então faça aquilo"}}
\newglossaryentry{keep}{type=drivebusca,name={Keep (Google Keep)},description={Aplicativo do Google para criar notas, listas e lembretes}}
\newglossaryentry{malware}{type=drivebusca,name={Malware (Software Malicioso)},description={Termo para softwares criados com intenção de causar danos ou roubar dados}}
\newglossaryentry{miro}{type=drivebusca,name={Miro},description={Plataforma online usada por equipes para criação de fluxogramas, mapas mentais e colaboração visual}}
\newglossaryentry{modeloer}{type=drivebusca,name={Modelo ER (Entidade-Relacionamento)},description={Diagrama visual usado para planejar e estruturar um banco de dados}}
\newglossaryentry{machinelearning}{type=drivebusca,name={Modelos de Machine Learning},description={Algoritmos treinados com grandes volumes de dados para identificar padrões e realizar previsões ou classificações}}
\newglossaryentry{notebook}{type=drivebusca,name={Notebook},description={Arquivo interativo que combina anotações de texto, código executável, gráficos e resultados}}
\newglossaryentry{pandadoc}{type=drivebusca,name={PandaDoc},description={Plataforma para automação do fluxo de documentos de vendas e assinatura eletrônica}}
\newglossaryentry{zapier}{type=drivebusca,name={Zapier},description={Plataforma que integra diferentes aplicativos e serviços web para automatizar fluxos de trabalho}}

% =========================================================
% ====================== FORMULÁRIOS =======================
% =========================================================
\newglossaryentry{escalalinear}{type=formularios,name={Escala Linear},description={Tipo de pergunta em formulários que utiliza uma escala numérica para quantificar opinião ou satisfação}}
\newglossaryentry{iframe}{type=formularios,name={Iframe},description={Elemento HTML usado para incorporar outro conteúdo dentro de uma página web}}
\newglossaryentry{template}{type=formularios,name={Template},description={Estrutura pré-definida de um formulário com layout e conteúdo base para simplificar a criação de documentos}}

\makeglossaries
