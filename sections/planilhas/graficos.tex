% !TeX root = ../../main.tex
% sections/planilhas/graficos.tex

\section{Transformando dados em visualizações}
Após a organização e a formatação dos dados em uma planilha, o próximo passo para extrair valor real dessas informações é a visualização. Os gráficos traduzem a complexidade dos dados em um formato visual que o cérebro pode processar de forma muito mais rápida e intuitiva.

No entanto, a eficácia de um gráfico depende da escolha do tipo correto para os dados e para a visualização que se deseja obter. Cada tipo de gráfico tem uma finalidade específica. A tabela abaixo serve como um guia rápido para ajudar na seleção do gráfico mais apropriado para sua necessidade.

\begin{table}[!ht]
	\centering
	\resizebox{\textwidth}{!}{%
		\begin{tabular}{@{}p{2.5cm}p{4cm}p{5cm}p{4.5cm}@{}}
			\toprule
			\textbf{Tipo de Gráfico} & \textbf{Finalidade Principal} & \textbf{Exemplo de Pergunta que Responde} & \textbf{Estrutura de Dados Ideal} \\
			\midrule
			Coluna & Comparar valores entre categorias distintas. & “Qual produto vendeu mais em janeiro?” & Uma coluna para categorias (texto) e colunas subsequentes para valores (números). \\
			Linha & Mostrar a evolução de dados ao longo do tempo. & “Como nossas vendas mensais mudaram ao longo do último ano?” & Uma coluna para o eixo do tempo (datas, meses) e colunas subsequentes para os valores a serem acompanhados. \\
			Pizza & Exibir a proporção de cada categoria em um todo. & “Qual a porcentagem de nosso orçamento é gasta com marketing?” & Uma coluna para categorias (texto) e uma única coluna para os valores correspondentes (números). \\
			\bottomrule
		\end{tabular}
	}
	\caption{Relação entre tipos de gráficos, suas finalidades e estrutura de dados recomendada.}
	\label{tab:tipos_graficos}
\end{table}

\subsection{Criando seu primeiro gráfico}

A criação de um gráfico no Google Sheets é um processo direto, onde a qualidade e a clareza do seu gráfico final são um reflexo direto da organização da sua tabela de origem.

\subsubsection{Passo 1: Seleção de dados}
Antes de inserir um gráfico, é fundamental garantir que seus dados estejam estruturados de forma lógica e clara. A prática recomendada é organizar os dados em colunas com um cabeçalho claro e descritivo na primeira linha de cada coluna. Por exemplo, em uma tabela de vendas, a primeira coluna pode ser "Mês", a segunda "Produto A" e a terceira "Produto B".

Após certificar que os dados estão devidamente estruturados, é preciso selecionar o intervalo de células que você deseja visualizar. Clique na primeira célula do seu conjunto de dados e arraste o mouse para incluir todas as linhas e colunas relevantes, incluindo os cabeçalhos.

\subsubsection{Passo 2: Inserindo o gráfico}
Com o intervalo de dados selecionado, navegue até a Barra de Menus e clique em \textbf{Inserir > Gráfico}.

\begin{figure}[h]
	\centering
	\includegraphics[width=.6\textwidth]{images/planilhas/imagem_26.png}
	\caption{Inserção do gráfico}
	\label{fig:planilhas:graficos1}
\end{figure}

O Google Planilhas analisará os dados selecionados e inserirá automaticamente um tipo de gráfico que ele considera mais apropriado. Um painel lateral chamado Editor de Gráficos também aparecerá à direita da tela. É através deste editor que todo o controle sobre o gráfico é exercido.

\subsection{Conhecendo o Editor de Gráficos}

O Editor de Gráficos é dividido em duas abas principais, a primeira serve para definir a estrutura e os dados, a segunda para definir a aparência.

\subsubsection{Aba Configuração}

Nesta aba se encontram os fundamentos do gráfico, definindo o tipo de gráfico, o intervalo de dados e como as colunas e linhas são usadas para os Eixos e as Séries. É o local para garantir que os dados estão sendo representados corretamente.

\subsubsection{Aba Personalizar}

Nesta aba se encontram todos os aspectos estéticos do gráfico, como cores, fontes, títulos, legendas, linhas de grade e etc. É aqui que um gráfico funcional se transforma em uma visualização de fácil leitura.

\section{Principais tipos de gráficos}

Apesar de o Google Planilhas apresentar diversos tipos de gráficos, três deles formam a base para a grande maioria dos gráficos: Coluna, Linha e Pizza. É importante notar que, embora visualmente distintos, a estrutura de dados deles segue um padrão comum: colunas que contêm categorias ou rótulos (texto, datas) e colunas que contêm valores (números).


\subsection{Gráfico de Coluna}
\begin{itemize}
	\item \textbf{O que é?} Uma categoria de gráficos com colunas dispostas na vertical, ideal para comparar valores entre diferentes categorias discretas.
	\item \textbf{Para que serve?} Perfeito para responder a perguntas como “Qual vendedor teve o melhor desempenho?” ou “Qual foi a receita em cada trimestre?”. 
	\item \textbf{Estrutura de dados ideal:} Os dados devem ser organizados com as categorias a serem comparadas em uma coluna (que se tornará o eixo horizontal, ou eixo X) e os valores numéricos correspondentes em colunas adjacentes (que formarão as barras no eixo vertical, ou eixo Y).
\end{itemize}

\begin{figure}[h]
	\centering
	\includegraphics[width=.6\textwidth]{images/planilhas/imagem_22.png}
	\caption{Exemplo de gráfico de coluna}
	\label{fig:planilhas:graficos2}
\end{figure}


\subsection{Gráfico de Linha}
\begin{itemize}
	\item \textbf{O que é?} Uma categoria de gráficos que conecta pontos de dados com uma linha, ideal para identificar tendências, padrões, flutuações e aceleração ou desaceleração em uma métrica. 
	\item \textbf{Para que serve?} Perfeito para responder perguntas como “Nossa base de usuários está crescendo?” ou “Como a temperatura variou ao longo do dia?”.
	\item \textbf{Estrutura de dados ideal:} Uma primeira coluna com os dados do eixo do tempo (dias, meses, anos) e as colunas seguintes com os valores numéricos que você deseja acompanhar ao longo desse tempo.
\end{itemize}

\begin{figure}[h]
	\centering
	\includegraphics[width=.6\textwidth]{images/planilhas/imagem_23.png}
	\caption{Exemplo de gráfico de linha}
	\label{fig:planilhas:graficos3}
\end{figure}


\subsection{Gráfico de Pizza}
\begin{itemize}
	\item \textbf{O que é?} Uma categoria de gráficos onde um círculo é subdividido em partes individuais que se relacionam com um todo. Cada parte representa uma categoria, e o tamanho da fatia é proporcional à sua porcentagem do total. 
	\item \textbf{Para que serve?} Ideal para responder a perguntas como “Qual porcentagem do nosso tráfego vem de cada rede social?” ou “Como nosso orçamento está dividido entre os departamentos?”.
	\item \textbf{Estrutura de dados ideal:} Uma coluna para os nomes das categorias e uma única coluna adjacente com seus valores numéricos correspondentes. 
\end{itemize}

\begin{figure}[h]
	\centering
	\includegraphics[width=.6\textwidth]{images/planilhas/imagem_24.png}
	\caption{Exemplo de gráfico de pizza}
	\label{fig:planilhas:graficos4}
\end{figure}

\subsection{Personalização Avançada}
Após a criação de um gráfico básico, o Google Sheets oferece uma vasta gama de opções de personalização para refinar a aparência e a clareza da visualização.  
O \textbf{Editor de Gráficos}, acessível no painel lateral, é a ferramenta central para todas essas modificações, dividindo as opções em duas abas principais: \textbf{Configurações} e \textbf{Personalizar}.  
Enquanto a aba de \textbf{Configuração} lida com a estrutura e os dados, a aba de \textbf{Personalizar} é onde a mágica estética realmente acontece, transformando um gráfico funcional em uma ferramenta de comunicação visual impactante.

\subsubsection{Estilo do Gráfico}
O estilo geral define a primeira impressão do gráfico. Nesta seção do Editor de Gráficos, é possível ajustar elementos como o tipo de gráfico, cores de fundo, bordas e fonte padrão.  

\begin{itemize}
	\item \textbf{Tipo de Gráfico:} Embora o Google Sheets sugira um tipo inicial, é possível alterá-lo para explorar outras representações visuais.
	\item \textbf{Cor de Fundo:} Define a cor de plano de fundo da área do gráfico.
	\item \textbf{Borda do Gráfico:} Adiciona uma borda ao redor, destacando-o do restante do conteúdo.
	\item \textbf{Fonte:} Afeta a legibilidade de todos os textos do gráfico, incluindo títulos, rótulos e legendas.
\end{itemize}

\begin{figure}[h]
	\centering
	\includegraphics[width=.6\textwidth]{images/planilhas/imagem_25.png}
	\caption{Personalização de gráficos}
	\label{fig:planilhas:graficos5}
\end{figure}

\subsubsection{Títulos, Eixos e Legendas}
Esses elementos são cruciais para contextualizar e interpretar o gráfico.  
\begin{itemize}
	\item \textbf{Título do Gráfico:} Deve ser curto, claro e indicar o principal insight.
	\item \textbf{Títulos dos Eixos:} Identificam o que está sendo representado, como “Meses”, “Vendas (R\$)” ou “Número de Clientes”.
	\item \textbf{Legendas:} Identificam séries de dados por meio de cores ou padrões e podem ter posição, fonte e tamanho personalizados.
\end{itemize}

\subsubsection{Formatando as Séries de Dados}
As séries de dados são o elemento central do gráfico.  
\begin{itemize}
	\item \textbf{Cor da Série:} Altere as cores das barras, linhas ou fatias de pizza.
	\item \textbf{Estilo da Linha ou Marcadores:} Ajuste espessura, tipo de traço e marcadores (círculos, quadrados etc.).
	\item \textbf{Rótulos de Dados:} Exibem valores diretamente no gráfico, devendo ser usados com moderação.
	\item \textbf{Linha de Tendência:} Evidencia padrões ou projeções (linear, exponencial, polinomial etc.).
\end{itemize}

\begin{figure}[h]
	\centering
	\includegraphics[width=.6\textwidth]{images/planilhas/imagem_30.png}
	\caption{Personalização da série de dados}
	\label{fig:planilhas:graficos6}
\end{figure}

\subsubsection{Linhas de Grade e Marcas}
As linhas de grade e marcas de escala facilitam a leitura e interpretação dos valores.  
\begin{itemize}
	\item \textbf{Linhas de Grade:} Ajudam na estimativa de valores. Podem ser principais ou secundárias, com cor e estilo personalizáveis.
	\item \textbf{Marcas de Escala (ticks):} Traços posicionados nos eixos para indicar valores correspondentes. Ajustáveis em posição, comprimento e frequência.
\end{itemize}
