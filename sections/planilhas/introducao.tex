% !TeX root = ../../main.tex
% sections/planilhas/introducao.tex

\section{Introdução}
Uma das ferramentas mais poderosas e conhecidas dentro do \textbf{Workspace}, é o \textbf{Google Planilhas} (conhecido também como \textbf{Google Sheets}). Seja para organizar as finanças pessoais, gerenciar um projeto da faculdade ou analisar dados de um negócio, ele se torna um grande aliado. Nesta seção, será introduzida a ferramenta, juntamente com suas funcionalidades.

\subsection{O que é?}
De maneira simples, uma planilha é um documento composto por linhas e colunas de células. Nessas células é possível inserir textos, números e fórmulas que permitam fazer cálculos, análises e etc.

O Google Planilhas é uma alternativa online e gratuita ao Microsoft Excel, ferramenta que possui ampla base de usuários de longa data. Ele funciona diretamente no navegador, não sendo necessário instalar (o que pode ser uma vantagem em relação ao Excel) e além disso, permite integração com outras ferramentas do Workspace, como o Google Drive e o Google Forms.

\subsection{Para que serve?}

O Google Planilhas é útil para:
\begin{itemize}
	\item Organizar informações pessoais em tabelas de forma clara e estruturada.
	\item Realizar cálculos automáticos com fórmulas e funções matemáticas.
	\item Criar gráficos e dashboards para análise visual de dados.
	\item Compartilhar documentos com edição simultânea por várias pessoas.
	\item Planejamentos pessoais e administrativos, cronogramas de projetos, planos de estudo.
\end{itemize}

\subsection{Exemplos práticos de uso}
Podemos pensar no Google Sheets como um caderno digital inteligente, capaz de realizar tarefas e organizar informações de forma muito eficiente. Abaixo estão alguns exemplos e cenários em que essa poderosa ferramenta pode ser sua aliada:
\begin{itemize}
	\item Orçamento pessoal, anotando as suas despesas do mês, como aluguel e alimentação, além de automatizar cálculos e previsões para você saber para onde seu dinheiro está indo.
	
	\item Planejamento de viagem, criando um roteiro dia a dia com os passeios, endereços e horários, além de controlar os gastos previstos com transporte e hospedagem.
	
	\item Lista de convidados para um evento, elaborando uma lista com o nome de todas as pessoas que você quer convidar e anotando ao lado quem já confirmou presença.
	
	\item Controle de estoque simples, listando todos os seus produtos para saber exatamente quantos itens ainda tem disponíveis para venda.
	
	\item Registro de treinos na academia, anotando os exercícios, pesos e repetições de cada dia, ajudando a visualizar seu progresso ao longo do tempo.
\end{itemize}

\subsection{Primeiros passos}
Assumindo que você já possui uma conta \textbf{Google} (caso não possua, é elucidado anteriormente nesta apostila), há duas principais maneiras de criar uma planilha.

Se já tens conhecimento da ferramenta \textbf{Google Drive} (também elucidada aqui), é interessante que crie seus documentos a partir de lá, pois assim podem ser organizados ao seu diretório pessoal em nuvem, junto aos seus outros documentos. Nada o impede de importá-lo ao seu diretório posteriormente, caso opte por criar o arquivo diretamente no site da ferramenta.

\begin{itemize}
	\item Google Drive (\href{https://drive.google.com/}{drive.google.com}): Acesse (ou crie) o diretório que deseja hospedar o arquivo e no canto superior esquerdo vá no botão “Novo” > “Planilhas Google”.
	
	\item Site da ferramenta (\href{https://sheets.google.com/}{sheets.google.com}): No início da página, clique no botão “+”, rotulado como “Nova planilha em branco”.
	
	\item Bônus: Na barra de endereços do seu navegador em uma nova guia, apenas escreva “\href{https://sheet.new/}{sheet.new}”. Esse link te leva para uma planilha nova rapidamente!
\end{itemize}

Antes de trabalhar com a planilha e os dados em si, vamos primeiramente renomear o arquivo (essencial para que possamos ter um espaço organizado): vá ao canto superior esquerdo e onde está o nome do arquivo (que por padrão é “Planilha sem título”) e simplesmente dê um clique único com o botão esquerdo do mouse e assim, pode reescrevê-lo com o que desejar.

\begin{figure}[htbp]
	\centering
	\includegraphics[width=.9\textwidth]{images/planilhas/imagem_1.png}
	\caption{Interface do Google Planilhas}
	\label{fig:planilhas:introducao1}
\end{figure}


\begin{dica}
	Evite usar espaços no nome do arquivo, para garantir uma melhor compatibilidade com a eventual exportação do arquivo.
\end{dica}

