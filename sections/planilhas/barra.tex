% !TeX root = ../../main.tex
% sections/planilhas/barra.tex

\section{Entendendo a barra superior}

Para dominar o Google Planilhas o primeiro passo é compreender a sua barra superior. Localizada na parte superior da interface da tela, é o principal centro de controle para todas as ações que você realizará em suas planilhas. É através dela que são organizadas as ferramentas da plataforma de maneira lógica e eficiente.

A barra superior do Google Planilhas é composta por três componentes distintos, cada um com uma finalidade específica para otimizar seu fluxo de trabalho. Compreender a função de cada um é fundamental para navegar de forma clara na ferramenta.   

\begin{figure}[h]
	\centering
	\includegraphics[width=.9\textwidth]{images/planilhas/imagem_2.png}
	\caption{Barra superior do Google Planilhas}
	\label{fig:planilhas:barra1}
\end{figure}

\begin{itemize}
	\item Barra de Menus: A fileira de texto no topo (Arquivo, Editar, Ver, etc.). Nela contém um índice completo de todas as funcionalidades disponíveis na plataforma.   
	
	\item Barra de Ferramentas de Acesso Rápido: A fileira de ícones logo abaixo da Barra de Menus. Ela oferece atalhos visuais para as ações mais comuns e frequentes, como formatação de texto e de números.   
	
	\item Barra de Fórmulas: A área que começa com o símbolo “fx”, sendo o principal local para a edição de fórmulas.   
\end{itemize}

\subsection{Barra de Menus}

A Barra de Menus (Arquivo, Editar, Ver, Inserir, Formatar, Dados, Ferramentas, Extensões, Ajuda) funciona como um mapa completo de tudo que o Google Planilhas pode fazer. Cada menu agrupa comandos relacionados por categoria, tornando a localização de qualquer ferramenta uma tarefa lógica e intuitiva. Mesmo que uma função tenha um atalho visual na barra de ferramentas, sua versão completa e todas as suas opções sempre poderão ser encontradas aqui.

Apesar disso, ações que demandam recorrer constantemente às opções dessa parte da interface, como aplicar negrito ou alterar a cor de uma célula, podem se tornar lentas e trabalhosas. É nesse ponto que a Barra de Ferramentas de Acesso Rápido entra em cena, pois ela é otimizada para eficiência, transformando ações repetitivas em um simples clique.

\subsection{Barra de Ferramentas de Acesso Rápido}

Servindo como uma forma de atalho para ações da Barra de Menus, a Barra de Ferramentas de Acesso Rápido tem a função de agilizar o trabalho. Ao invés de navegar pelos menus, é possível executar formatações e executar comandos com apenas um clique. Aqui se encontram ferramentas para desfazer e refazer ações, imprimir, aplicar formatações de uma célula para outra, controlar o zoom, formatar números como moeda ou porcentagem, alterar o estilo da fonte (negrito, itálico), definir cores de preenchimento e de texto, adicionar bordas e alinhar o conteúdo das células.

\subsection{Barra de Fórmulas}

Identificada pelo ícone “fx”, é uma das partes mais importantes da interface de uma planilha. Ela possui duas funções:
\begin{enumerate}
	\item Exibir o conteúdo real: Uma célula pode exibir um valor, como por exemplo “90”, mas seu conteúdo real pode ser uma função que calcula esse valor, por exemplo “=SOMA(D12:D18)”. A barra de fórmulas sempre mostra o conteúdo real por trás da célula.
	\item Permitir edições precisas: Através dela é possível editar e criar fórmulas longas e complexas com maior facilidade, oferecendo maior clareza e espaço do que a edição direta na célula. 
\end{enumerate}

\subsection{Navegando pela Barra de Menus}

	A seguir um detalhamento de cada menu que responderá a duas questões principais: \textbf{o que são} e \textbf{para que servem} suas principais funcionalidades, explicando com exemplos práticos para facilitar a compreensão e aplicação no dia a dia.

\subsection{Menu Arquivo}
	Este menu concentra todas as ações que afetam o arquivo da planilha como um todo, desde sua criação até o compartilhamento e exportação.

\begin{figure}[h]
	\centering
	\includegraphics[width=.3\textwidth]{images/planilhas/imagem_3.png}
	\caption{Menu arquivo}
	\label{fig:planilhas:barra2}
\end{figure}

\begin{itemize}
	\item \textbf{Novo / Abrir:}
	\begin{itemize}
		\item \textbf{O que são?} O botão “Novo” é para criar uma nova planilha do zero ou a partir de um modelo. O “Abrir” para abrir um arquivo já existente no seu Google Drive.
		\item \textbf{Para que servem?} Iniciar um novo projeto ou continuar um anterior.
		\item \textbf{Como usar?} Clique em \textbf{Arquivo > Novo > Planilha} para um arquivo em branco. Para abrir clique em \textbf{Arquivo > Abrir} e navegue pelos seus arquivos no Google Drive.
	\end{itemize}
	
	\item \textbf{Fazer uma cópia:}
	\begin{itemize}
		\item \textbf{O que é?} Cria um clone exato e independente da planilha atual, mas com um novo nome. 
		\item \textbf{Para que serve?} Recriar planilhas a partir de uma estrutura pronta ao invés de criar tudo do zero novamente. 
		\item \textbf{Exemplo prático?} Você tem uma planilha de "Orçamento Mensal" com toda a estrutura de despesas e receitas. No início de cada mês você vai em \textbf{Arquivo > Fazer uma cópia}, renomeia para "Orçamento - {Novo mês}" e preenche com os novos dados, mantendo o modelo original intacto para os meses seguintes.
	\end{itemize}
	
	\item \textbf{Compartilhar:}
	\begin{itemize}
		\item \textbf{O que é?} Cria um link do arquivo para que outras pessoas tenham acesso a sua planilha de forma on-line. (Para mais informações consultar o capítulo Compartilhamento).
		\item \textbf{Para que serve?} Trabalhar em equipe no mesmo arquivo simultaneamente e em tempo real, controlando quem pode apenas ver, quem pode comentar e quem pode editar o conteúdo.
		\item \textbf{Exemplo prático?} Você cria uma planilha para organizar um churrasco com amigos. Usando o botão "Compartilhar", você envia um link de "Editor" para eles, permitindo que todos adicionem os itens que vão levar e vejam as atualizações dos outros instantaneamente.
	\end{itemize}
	
	\item \textbf{Baixar:}
	\begin{itemize}
		\item \textbf{O que é?} Exporta e salva uma cópia da sua planilha no seu computador em diferentes formatos.
		\item \textbf{Para que serve?} Para compartilhar seu trabalho com pessoas que usam outros softwares - como Microsoft Excel - ou para criar uma versão estática do seu documento, como um PDF.
		\item \textbf{Exemplo prático?} Você precisa enviar um relatório de vendas para um cliente que só utiliza Excel. Você pode ir em \textbf{Arquivo > Baixar > Microsoft Excel (.xlsx)}. Um arquivo.xlsx será baixado e o cliente poderá abri-lo sem problemas.
	\end{itemize}
	
	\item \textbf{Histórico de versões:}
	\begin{itemize}
		\item \textbf{O que é?} Registra todas as alterações feitas na planilha, mostrando quem as fez e quando.
		\item \textbf{Para que serve?} Permite visualizar e restaurar versões anteriores do seu trabalho, sendo uma ferramenta poderosa para reverter erros ou recuperar informações apagadas.
		\item \textbf{Exemplo prático?} Imagine que você deletou acidentalmente uma aba inteira com dados importantes. Em vez de se desesperar, você pode acessar \textbf{Arquivo > Histórico de versões > Ver histórico de versões}. Uma barra lateral mostrará todas as versões salvas. Você pode encontrar a versão de 5 minutos atrás, antes do erro, e clicar em "Restaurar esta versão" para recuperar todo o seu trabalho perdido.
	\end{itemize}
	
\end{itemize}

\subsection{Menu Editar:}
Este menu contém as ferramentas fundamentais para manipular o conteúdo das células.
	
	\begin{itemize}
		\begin{figure}[h]
			\centering
			\includegraphics[width=.3\textwidth]{images/planilhas/imagem_4.png}
			\caption{Menu editar}
			\label{fig:planilhas:barra3}
		\end{figure}
		
		\item \textbf{Desfazer / Refazer:}
		\begin{itemize}
			\item \textbf{O que são?} Comandos básicos para reverter a última ação (Desfazer) ou reaplicar uma ação que foi desfeita (Refazer).
			\item \textbf{Para que serve?} Para corrigir erros de forma instantânea.
		\end{itemize}
		\item \textbf{Recortar / Copiar / Colar:}
		\begin{itemize}
			\item \textbf{O que é?} Ações para mover (Recortar) ou duplicar (Copiar) dados de um local para outro (Colar).
			\item \textbf{Para que serve?} Para reorganizar a estrutura da sua planilha ou replicar informações e fórmulas rapidamente.
		\end{itemize}
		\item \textbf{Colar especial:}
		\begin{itemize}
			\item \textbf{O que é?} Uma versão avançada do comando “Colar” que permite escolher exatamente o que você quer colar de uma célula copiada.
			\item \textbf{Para que serve?} Evitar problemas comuns, como copiar uma fórmula e ela quebrar em um novo local ou copiar um valor e trazer junto uma formatação indesejada.
			\item \textbf{Exemplos práticos:}
			\begin{itemize}
				\item \textbf{Colar apenas os valores:} Você tem uma célula com a fórmula “=SOMA(B2:B10)” que resulta em “R\$1.500,00”. Se você copiar e colar normalmente, a fórmula será ajustada para o novo local. Mas se usar \textbf{Editar > Colar especial > Apenas os valores}, você colará o texto estático “1.500” sem a fórmula ou a formatação de moeda.
				\item \textbf{Colar apenas a formatação:} Você criou um cabeçalho com fundo azul, texto branco e em negrito. Para aplicar este mesmo estilo a outro cabeçalho, copie o original, selecione o novo cabeçalho e vá em \textbf{Editar > Colar especial > Apenas a formatação}. O estilo será aplicado sem alterar o texto.
			\end{itemize}
		\end{itemize}
	\end{itemize}
	
	\subsection{Menu Ver:}
	Controla a aparência da área de trabalho, permitindo personalizar o que é exibido na tela sem alterar os dados.
	\begin{itemize}
		\begin{figure}[h]
			\centering
			\includegraphics[width=.3\textwidth]{images/planilhas/imagem_5.png}
			\caption{Menu editar}
			\label{fig:planilhas:barra4}
		\end{figure}
		\item \textbf{Mostrar:}
		\begin{itemize}
			\item \textbf{O que é?} Submenu que permite ativar ou desativar elementos da interface.
			\item \textbf{Para que serve?} Para limpar a tela e focar no que lhe é importante.
			\item \textbf{Exemplo prático:} Antes de apresentar um dashboard, vá em \textbf{Ver > Mostrar} e desmarque as Linhas de grade. Isso remove as linhas cinzas que separam as células, dando à planilha uma aparência mais limpa e profissional.
		\end{itemize}
		\item \textbf{Congelar:}
		\begin{itemize}
			\item \textbf{O que é?} Ferramenta para travar linhas e colunas no campo de visão enquanto você navega.
			\item \textbf{Para que serve?} Garante que cabeçalhos ou identificadores importantes estejam sempre visíveis.
			\item \textbf{Exemplo prático:} Em uma lista com 200 produtos, clique na linha de cabeçalho da tabela e vá em \textbf{Ver > Congelar > 1 linha}. Assim, os cabeçalhos permanecem visíveis ao rolar a planilha.
		\end{itemize}
		\item \textbf{Zoom:}
		\begin{itemize}
			\item \textbf{O que é?} Ajusta o nível de ampliação da planilha.
			\item \textbf{Para que serve?} Melhora a legibilidade (aumentando o zoom) ou oferece uma visão geral (diminuindo o zoom).
			\item \textbf{Como usar:} Vá em \textbf{Ver > Zoom} e selecione a porcentagem desejada.
		\end{itemize}
	\end{itemize}
	
	\subsection{Menu Inserir:}
	Permite adicionar elementos visuais, interativos e informativos à planilha.
	\begin{itemize}
		\begin{figure}[htbp]
			\centering
			\includegraphics[width=.3\textwidth]{images/planilhas/imagem_6.png}
			\caption{Menu inserir}
			\label{fig:planilhas:barra5}
		\end{figure}
		\item \textbf{Gráfico:}
		\begin{itemize}
			\item \textbf{O que é?} Forma visual de representar dados em colunas, linhas, pizza, etc.
			\item \textbf{Para que serve?} Facilita a análise e a apresentação de informações complexas.
			\item \textbf{Exemplo prático:} Selecione os dados e clique em \textbf{Inserir > Gráfico}. Escolha o tipo desejado para comparar visualmente o desempenho dos produtos.
		\end{itemize}
		\item \textbf{Imagem:}
		\begin{itemize}
			\item \textbf{O que é?} Insere figuras ou ilustrações dentro das células ou na planilha.
			\item \textbf{Para que serve?} Melhora a visualização e contextualização dos dados.
			\item \textbf{Exemplo prático:} Em uma planilha de vendedores, insira a foto de cada pessoa em \textbf{Inserir > Imagem > Na célula}.
		\end{itemize}
		\item \textbf{Link:}
		\begin{itemize}
			\item \textbf{O que é?} Cria um endereço clicável que leva a outra página, documento ou parte da planilha.
			\item \textbf{Para que serve?} Facilita o acesso rápido a informações relacionadas.
			\item \textbf{Exemplo prático:} Em uma planilha de e-commerce, adicione links diretos para cada transação.
		\end{itemize}
		\item \textbf{Caixa de seleção:}
		\begin{itemize}
			\item \textbf{O que é?} Recurso interativo para marcar ou desmarcar opções em uma célula.
			\item \textbf{Para que serve?} Útil para criar listas de verificação e indicar status de tarefas.
			\item \textbf{Exemplo prático:} Crie uma coluna “Status” e insira caixas de seleção em \textbf{Inserir > Caixa de seleção} para marcar tarefas concluídas.
		\end{itemize}
	\end{itemize}
	
	\subsection{Menu Formatar:}
	Permite alinhar, ajustar espaçamentos e aplicar estilos visuais.
	\begin{itemize}
		\begin{figure}[htbp]
			\centering
			\includegraphics[width=.3\textwidth]{images/planilhas/imagem_7.png}
			\caption{Menu formatar}
			\label{fig:planilhas:barra6}
		\end{figure}
		\item \textbf{Tema:}
		\begin{itemize}
			\item \textbf{O que é?} Conjunto de estilos que define aparência da planilha.
			\item \textbf{Para que serve?} Uniformiza cores e fontes.
			\item \textbf{Exemplo prático:} Aplique um tema corporativo com as cores do logotipo da empresa.
		\end{itemize}
		\item \textbf{Número:}
		\begin{itemize}
			\item \textbf{O que é?} Define como valores numéricos são exibidos (moeda, porcentagem, data etc.).
			\item \textbf{Para que serve?} Facilita a compreensão dos dados.
			\item \textbf{Exemplo prático:} Selecione uma coluna e vá em \textbf{Formatar > Número > Moeda} para exibir “R\$” nos valores.
		\end{itemize}
		\item \textbf{Negrito / Itálico / Alinhamento:}
		\begin{itemize}
			\item \textbf{O que é?} Formatações básicas que alteram a aparência e posição do texto.
			\item \textbf{Para que serve?} Destaca informações importantes e melhora a leitura.
			\item \textbf{Exemplo prático:} Aplique negrito aos resultados finais e alinhe colunas numéricas.
		\end{itemize}
		\item \textbf{Formatação condicional:}
		\begin{itemize}
			\item \textbf{O que é?} Aplica estilos automáticos com base em regras definidas.
			\item \textbf{Para que serve?} Destaca dados relevantes e padrões.
			\item \textbf{Exemplo prático:} Use \textbf{Formatar > Formatação condicional} para destacar valores negativos em vermelho.
		\end{itemize}
		\item \textbf{Cores alternadas:}
		\begin{itemize}
			\item \textbf{O que é?} Alterna cores entre linhas.
			\item \textbf{Para que serve?} Facilita a leitura de grandes tabelas.
			\item \textbf{Exemplo prático:} Aplique \textbf{Formatar > Cores alternadas} para diferenciar visualmente cada linha.
		\end{itemize}
	\end{itemize}
	
	\subsection{Menu Dados:}
	Oferece ferramentas para organizar e analisar informações.
	\begin{itemize}
		\begin{figure}[htbp]
			\centering
			\includegraphics[width=.3\textwidth]{images/planilhas/imagem_8.png}
			\caption{Menu dados}
			\label{fig:planilhas:barra7}
		\end{figure}
		\item \textbf{Classificar:}
		\begin{itemize}
			\item \textbf{O que é?} Organiza dados em ordem crescente ou decrescente.
			\item \textbf{Para que serve?} Facilita a localização de informações.
			\item \textbf{Exemplo prático:} Use \textbf{Dados > Classificar de A a Z} para ordenar nomes de funcionários.
		\end{itemize}
		\item \textbf{Criar um filtro:}
		\begin{itemize}
			\item \textbf{O que é?} Exibe apenas dados que atendem a critérios específicos.
			\item \textbf{Para que serve?} Facilita análises sem alterar a planilha.
			\item \textbf{Exemplo prático:} Aplique \textbf{Dados > Criar filtro} para visualizar apenas vendas de um produto.
		\end{itemize}
		\item \textbf{Validação de dados:}
		\begin{itemize}
			\item \textbf{O que é?} Define regras para os valores aceitos em uma célula.
			\item \textbf{Para que serve?} Garante precisão e padronização nos dados.
			\item \textbf{Exemplo prático:} Configure para aceitar apenas números inteiros entre 1 e 10.
		\end{itemize}
	\end{itemize}
	
	\subsection{Menu Ferramentas:}
	Oferece funcionalidades adicionais que expandem as capacidades da planilha.
	\begin{itemize}
		\begin{figure}[htbp]
			\centering
			\includegraphics[width=.3\textwidth]{images/planilhas/imagem_9.png}
			\caption{Menu ferramentas}
			\label{fig:planilhas:barra8}
		\end{figure}
		
		\item \textbf{Criar um formulário:}
		\begin{itemize}
			\item \textbf{O que é?} Cria formulários vinculados à planilha.
			\item \textbf{Para que serve?} Coleta informações automaticamente.
			\item \textbf{Exemplo prático:} Crie um formulário escolar com nome, idade e turma, com respostas registradas na planilha.
		\end{itemize}
		\item \textbf{Verificação ortográfica:}
		\begin{itemize}
			\item \textbf{O que é?} Verifica a ortografia do texto nas células.
			\item \textbf{Para que serve?} Corrige erros de escrita automaticamente.
			\item \textbf{Exemplo prático:} Use para corrigir nomes de cidades digitados incorretamente.
		\end{itemize}
		\item \textbf{Regras de notificação:}
		\begin{itemize}
			\item \textbf{O que é?} Configura alertas para mudanças em planilhas compartilhadas.
			\item \textbf{Para que serve?} Mantém usuários informados sobre alterações.
			\item \textbf{Exemplo prático:} Em \textbf{Ferramentas > Regras de notificação}, configure alertas imediatos ou diários para modificações.
		\end{itemize}
	\end{itemize}
	
	\subsection{Menu Extensões:}
	Permite expandir as funcionalidades do Google Planilhas.
	\begin{itemize}
		\begin{figure}[htbp]
			\centering
			\includegraphics[width=.3\textwidth]{images/planilhas/imagem_11.png}
			\caption{Menu extensões}
			\label{fig:planilhas:barra10}
		\end{figure}
		\item \textbf{Complementos:}
		\begin{itemize}
			\item \textbf{O que é?} Ferramentas extras que adicionam recursos ao Google Planilhas.
			\item \textbf{Para que serve?} Aumenta as possibilidades da ferramenta.
			\item \textbf{Exemplo prático:} Instale complementos para importar dados do Google Analytics ou gerar gráficos personalizados.
		\end{itemize}
		\item \textbf{Apps Script / Macros:}
		\begin{itemize}
			\item \textbf{O que é?} Ferramentas para automatizar tarefas repetitivas ou criar scripts personalizados.
			\item \textbf{Para que serve?} Economizam tempo e reduzem erros manuais.
			\item \textbf{Exemplo prático:} Crie uma macro para formatar tabelas automaticamente ou scripts que enviem relatórios por e-mail.
		\end{itemize}
	\end{itemize}
	
	\subsection{Menu Ajuda:}
	Recurso essencial para encontrar informações, resolver dúvidas e aprender a usar a ferramenta.
	\begin{itemize}
		\begin{figure}[htbp]
			\centering
			\includegraphics[width=.3\textwidth]{images/planilhas/imagem_12.png}
			\caption{Menu ajuda}
			\label{fig:planilhas:barra12}
		\end{figure}
		\item \textbf{Ajuda do Planilhas:}
		\begin{itemize}
			\item \textbf{O que é?} Área de suporte e tutoriais sobre o Google Planilhas.
			\item \textbf{Para que serve?} Ensina funções e soluções para diversos problemas.
			\item \textbf{Exemplo prático:} Use para relembrar funções ou aprender a criar gráficos passo a passo.
		\end{itemize}
		\item \textbf{Pesquisar os menus:}
		\begin{itemize}
			\item \textbf{O que é?} Barra de pesquisa que encontra rapidamente comandos dentro dos menus.
			\item \textbf{Para que serve?} Economiza tempo ao localizar opções sem navegar manualmente.
			\item \textbf{Exemplo prático:} Digite “formatação condicional” na barra e acesse o comando diretamente.
		\end{itemize}
	\end{itemize}
