% !TeX root = ../../main.tex
% sections/planilhas/dicas.tex

\section{Dicas e boas práticas}
A criação de uma planilha eficaz vai além da simples inserção de dados e fórmulas. Ela envolve um conjunto de boas práticas e princípios de design que garantem que o documento seja claro, confiável, fácil de usar e escalável.

\subsection{Organização e estrutura de dados}
A base de uma planilha eficiente é uma boa organização e uma estruturação de dados objetiva. Uma base sólida garante que as ferramentas mais avançadas da plataforma funcionem de maneira previsível e eficiente.

\subsection{Nomenclatura de arquivos clara e consistente}
A clareza começa com os nomes. Nomes genéricos como "Planilha sem título" ou "Página1" geram confusão e dificultam a navegação.

\begin{itemize}
	\item \textbf{Arquivos:} Adote um padrão de nomenclatura consistente para os arquivos que inclua informações relevantes como o conteúdo, a área responsável e a data (ex: Relatorio-Vendas\_Marketing\_2025-10). É recomendado também evitar o uso de espaços nos nomes dos arquivos para garantir uma melhor compatibilidade durante a exportação para outros formatos.   
	\item \textbf{Páginas:} Renomeie sempre as páginas com nomes curtos e descritivos que indiquem o seu conteúdo, como "Dados Brutos", "Dashboard", "Resumo Mensal" ou "Instruções".
\end{itemize}

\subsection{Cabeçalhos únicos e descritivos}
Cada coluna deve ter um cabeçalho único e descritivo, localizado na primeira linha do conjunto de dados. Estes cabeçalhos são identificadores funcionais dos conteúdos da respectiva coluna. Ferramentas essenciais, como a Ordenação, os Filtros e as Tabelas Dinâmicas dependem diretamente deles. A ausência de cabeçalhos ou a existência de cabeçalhos duplicados é a causa mais comum de erros ao tentar analisar dados.

\subsection{Evitar células mescladas}
A funcionalidade de mesclar células é útil para criar títulos que se estendem por várias colunas acima de uma tabela de dados. No entanto, utilizar células mescladas dentro de um intervalo de dados estruturado é uma prática que deve ser evitada pois ela quebra a estrutura de grade da planilha, onde cada dado ocupa uma única célula numa intersecção de linha e coluna.  

A principal consequência é o impedimento da seleção correta de colunas inteiras, causando falhas em operações de ordenação e filtro, tornando impossível a criação de uma Tabela Dinâmica a partir desses dados. Como alternativa para centralizar um título sobre uma única coluna, utilize a opção "Centralizar" no alinhamento horizontal.

\subsection{Legibilidade e impacto visual}
Uma planilha bem formatada não é apenas esteticamente agradável; ela é mais fácil e rápida de ler e interpretar. 

\subsection{Formatação com propósito}
O objetivo da formatação é clareza, não decoração. O excesso de cores, fontes e bordas pode poluir a visualização e dificultar a compreensão em vez de a facilitar. Algumas dicas de formatação:
Use negrito para destacar cabeçalhos e totais.   
Use cores de fundo sutis para agrupar informações relacionadas ou destacar linhas e colunas importantes.
Use bordas para separar claramente diferentes blocos de informação.  
Use cores de texto com significado, como verde para valores positivos/receitas e vermelho para valores negativos/despesas.   

\begin{figure}[h]
	\centering
	\includegraphics[width=.9\textwidth]{images/planilhas/imagem_37.png}
	\caption{Tabela formatada}
	\label{fig:planilhas:dica1}
\end{figure}

\subsection{Padronização eficiente com Pintar Formatação}
Para garantir a consistência visual em toda a planilha utilize a ferramenta "Pintar Formatação", representada pelo ícone de um rolo de pintura na barra de ferramentas. Esta ferramenta permite copiar todo o estilo de uma célula (fonte, cor de fundo, bordas, formato de número, etc.) e aplicá-lo rapidamente a outra célula ou intervalo. É a forma mais eficiente de garantir que todos os cabeçalhos, totais ou blocos de dados tenham uma aparência padronizada.

\begin{figure}[h]
	\centering
	\includegraphics[width=.9\textwidth]{images/planilhas/imagem_38.png}
	\caption{Barra de formatação, com a ferramenta "Pintar Formatação" destacada}
	\label{fig:planilhas:dica2}
\end{figure}

\subsection{Cores alternadas}
Aplicar uma formatação que alterna a cor de fundo entre linhas claras e escuras melhora drasticamente a legibilidade de tabelas largas e densas. Isto ajuda o olho a seguir uma linha específica da esquerda para a direita sem se desviar para as linhas adjacentes, reduzindo o cansaço visual e a probabilidade de erros de leitura. É possível automatizar esse processo através de \textbf{Formatar > Cores alternadas}.

\begin{figure}[h]
	\centering
	\includegraphics[width=.9\textwidth]{images/planilhas/imagem_39.png}
	\caption{Painel "Cores alternadas"}
	\label{fig:planilhas:dica3}
\end{figure}
