% !TeX root = ../../main.tex
% sections/planilhas/tabelas.tex

\section{Tabelas Dinâmicas}
Após a inserção e organização dos dados em uma planilha, uma das formas mais eficientes de analisar informações e identificar padrões é utilizar as tabelas dinâmicas. Esse recurso permite organizar, resumir e filtrar grandes volumes de dados de maneira flexível, facilitando a visualização de resultados sem a necessidade de fórmulas complexas.

As tabelas dinâmicas são especialmente úteis em contextos de relatórios financeiros, controle de despesas, acompanhamento de desempenho e quaisquer cenários em que seja necessário cruzar informações de forma dinâmica.

\subsection{Criando uma Tabela Dinâmica}
O processo de criação de uma tabela dinâmica no Google Planilhas é simples e intuitivo. A ferramenta automaticamente reconhece o intervalo de dados e oferece opções para definir como as informações serão resumidas e exibidas.

\subsubsection{Passo 1: Seleção dos dados}
Antes de criar a tabela dinâmica, é importante garantir que a planilha esteja bem estruturada, com cabeçalhos claros (devidamente preenchidos) e dados organizados em colunas. No exemplo abaixo, há uma tabela com as colunas “Descrição”, “Valor” e “Situação”, representando diferentes despesas.

\subsubsection{Passo 2: Inserindo a tabela dinâmica}
Com o intervalo selecionado, acesse o menu superior e clique em Inserir > Tabela dinâmica.

\begin{figure}[h]
	\centering
	\includegraphics[width=.5\textwidth]{images/planilhas/imagem_27.png}
	\caption{Menu de criação de tabela dinâmica}
	\label{fig:planilhas:tabelas1}
\end{figure}

Uma janela será exibida pedindo para confirmar o intervalo de dados e escolher onde a tabela será criada: em uma nova página ou em uma página existente. A opção mais comum é criar em uma nova página, mantendo o relatório separado da planilha original.

\begin{figure}[h]
	\centering
	\includegraphics[width=.5\textwidth]{images/planilhas/imagem_28.png}
	\caption{Exemplo de criação de tabela dinâmica}
	\label{fig:planilhas:tabelas2}
\end{figure}

Ao clicar em Criar, uma nova aba será aberta com a estrutura da tabela dinâmica e o Editor de tabela dinâmica visível à direita da tela.

\subsection{Conhecendo o Editor de Tabela Dinâmica}
O editor é o painel de controle dessa ferramenta. É nele que o usuário define quais dados serão exibidos, como serão organizados e quais cálculos serão aplicados. Ele possui quatro principais áreas de configuração:
\begin{itemize}
	\item \textbf{Linhas:} determina os itens que aparecerão listados verticalmente na tabela (por exemplo, “Descrição”).
	\item \textbf{Colunas:} organizam as categorias na horizontal (por exemplo, “Mês” ou “Situação”).
	\item \textbf{Valores:} representam os dados numéricos que serão calculados. Podem ser somas, médias, contagens, entre outros.
	\item \textbf{Filtros:} permitem exibir apenas informações específicas conforme critérios definidos.
\end{itemize}
No exemplo ilustrado abaixo, o campo “Descrição” foi adicionado às Linhas, e o campo “Valor” foi adicionado em Valores, resultando em um resumo simples e direto das despesas listadas.

\begin{figure}[h]
	\centering
	\includegraphics[width=.5\textwidth]{images/planilhas/imagem_29.png}
	\caption{Exemplo de gráfico feito com tabela dinâmica}
	\label{fig:planilhas:tabelas3}
\end{figure}

\subsection{Personalizando a Exibição dos Dados}

As tabelas dinâmicas do Google Planilhas oferecem uma série de opções de personalização para tornar os dados mais claros e visualmente organizados:

\begin{itemize}
	\item \textbf{Ordenação:} é possível ordenar os valores em ordem crescente ou decrescente, tanto para linhas quanto para colunas.
	\item \textbf{Totais:} pode-se exibir o total geral ou subtotais de grupos de dados.
	Agrupamento: categorias semelhantes podem ser agrupadas para uma análise mais segmentada.
	\item \textbf{Filtros adicionais:} permitem restringir a visualização a determinados períodos, categorias ou faixas de valores.
\end{itemize}

Essas funcionalidades tornam as tabelas dinâmicas extremamente versáteis, permitindo explorar diferentes perspectivas sobre os mesmos dados sem alterar a planilha original.

\begin{dica}
	\begin{itemize}
		\item Mantenha os dados de origem sempre atualizados, pois as tabelas dinâmicas se ajustam automaticamente conforme as alterações.
		\item Evite células mescladas no intervalo de origem, pois isso pode dificultar a leitura dos dados pela ferramenta.
		\item Use nomes claros nos cabeçalhos, garantindo que cada coluna tenha uma identificação única.
		\item Combine com gráficos dinâmicos para transformar os resultados em visualizações mais intuitivas.
	\end{itemize}
\end{dica}
