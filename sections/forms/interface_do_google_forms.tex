% sections/forms/interface_do_google_forms.tex
% !TeX root = ../../../main.tex

\section{Interface do Google Forms}
\subsection{Visão Geral da Tela Inicial:}

A tela inicial do \textbf{Google Forms} é simples e intuitiva. Na barra
superior, o usuário pode definir o nome do formulário, acessar as
configurações, personalizar o layout e utilizar o botão de envio.

\begin{figure}[H]
	\centering
	\includegraphics[width=.9\textwidth]{images/forms/interface_do_google_forms/Imagem 3.png}
\end{figure}

\textbf{Barra Superior} (canto superior direito):

\begin{figure}[H]
	\centering
	\includegraphics[width=.7\textwidth]{images/forms/interface_do_google_forms/Imagem 16.png}
\end{figure}

A barra superior concentra as opções de edição, visualização,
personalização e publicação do formulário. Abaixo, uma análise de suas
funcionalidades:

\begin{itemize}
\item
  \textbf{Personalizar tema:}
\end{itemize}

\begin{quote}
É utilizado para alterar o tema, cores, imagens e a fonte
\end{quote}

\begin{figure}[H]
	\centering
	\includegraphics[width=.5\textwidth]{images/forms/interface_do_google_forms/Imagem 21.png}
\end{figure}

\begin{itemize}
\item
  \textbf{Visualizar:}
\end{itemize}


\begin{quote}
Permite verificar como o formulário será exibido para as pessoas que
irão responder.
\end{quote}

\begin{figure}[H]
	\centering
	\includegraphics[width=.9\textwidth]{images/forms/interface_do_google_forms/Imagem 22.png}
\end{figure}

\begin{itemize}
\item
  \textbf{Desfazer e refazer:}
\end{itemize}

\begin{figure}[H]
	\centering
	\includegraphics[width=.7\textwidth]{images/forms/interface_do_google_forms/Imagem 6.png}
\end{figure}

\begin{quote}
Essas opções controlam alterações feitas no documento, permitindo
desfazer as últimas modificações ou refazer o que foi desfeito.
\end{quote}

\begin{itemize}
\item
  \textbf{Copiar Link:}
\end{itemize}

\begin{figure}[H]
	\centering
	\includegraphics[width=.7\textwidth]{images/forms/interface_do_google_forms/Imagem 8.png}
\end{figure}

\begin{quote}
Permite copiar o \textbf{link} do formulário para compartilhamento. Após,
basta encaminhar o \textbf{link} de acesso para os respondentes.
\end{quote}

\begin{itemize}
\item
  \textbf{Compartilhar:}
\end{itemize}

\begin{figure}[H]
	\centering
	\includegraphics[width=.7\textwidth]{images/forms/interface_do_google_forms/Imagem 9.png}
\end{figure}

\begin{quote}
É possível compartilhar o link para determinados e-mails definindo-os
como participantes ou editores do formulário.
\end{quote}

\begin{itemize}
\item
  \textbf{Publicar:}
\end{itemize}

\begin{figure}[H]
	\centering
	\includegraphics[width=.7\textwidth]{images/forms/interface_do_google_forms/Imagem 9+1.png}
\end{figure}

\begin{quote}
Serve para publicar o formulário, disponibilizando-o para que os
participantes possam responder.
\end{quote}

\begin{itemize}
\item
  \textbf{Botão ``Mais''} (ícone de três pontinhos na
  vertical)\textbf{:}
\end{itemize}


\begin{figure}[H]
	\centering
	\includegraphics[width=.39\textwidth]{images/forms/interface_do_google_forms/Imagem 23.png}
\end{figure}

Habilita algumas opções adicionais como: fazer uma cópia do formulário,
mover para lixeira, preencher automaticamente, incorporar HTML,
Imprimir, \textbf{Apps Script} - que permite escrever um código na
linguagem HTML para incorporar ao formulário -, instalar complementos no
formulário, cancelar a publicação e configurar atalhos do teclado.


\subsection{Aba Respostas:}

\begin{figure}[H]
	\centering
	\includegraphics[width=.8\textwidth]{images/forms/interface_do_google_forms/Imagem 10.png}
\end{figure}

Nesta aba é possível verificar as respostas do formulário e exportá-las
para o \textbf{Google} \textbf{Planilhas}, se desejado. Contempla, também,
mais algumas opções: receber notificações por e-mail, selecionar destino
para as respostas, desvincular formulário, baixar as respostas em
formato .\textbf{csv} (para planilhas), imprimir ou excluir as respostas.

\begin{figure}[H]
	\centering
	\includegraphics[width=.7\textwidth]{images/forms/interface_do_google_forms/Imagem 11.png}
\end{figure}

\subsection{Aba Configurações:}

Aqui pode ser configurado o formulário, por meio da criação de testes
(menu ``Criar Teste''), gerenciar como as respostas são coletadas (menu
``Respostas''), enviar uma cópia das respostas aos participantes,
permitir a edição das respostas e limitar para apenas 1 resposta por
e-mail, por exemplo.

\begin{figure}[H]
	\centering
	\includegraphics[width=.8\textwidth]{images/forms/interface_do_google_forms/Imagem 13.png}
\end{figure}

Essa parte trata da apresentação do formulário ao usuário, podendo ser
habilitada a exibição da barra de progresso e o embaralhamento de
perguntas. Temos, ainda, algumas configurações após envio da resposta,
como: mostrar mensagem de confirmação após envio, mostrar link para
enviar outra resposta, exibir resumo dos resultados e uma restrição
especial para desativar o salvamento automático dos participantes, caso
eles saiam da página antes de finalizar as respostas não são salvas.

\begin{figure}[H]
	\centering
	\includegraphics[width=.7\textwidth]{images/forms/interface_do_google_forms/Imagem 14.png}
\end{figure}

Por fim, temos alguns padrões, como coletar e-mails por padrão e tornar
as perguntas obrigatórias.

\subsection{Barra Lateral}

Na barra lateral encontramos seis opções como:

\begin{figure}[H]
	\centering
	\includegraphics[width=.7\textwidth]{images/forms/interface_do_google_forms/Imagem 15.png}
\end{figure}

\begin{itemize}
\item
  Adicionar pergunta
\item
  Importar pergunta
\item
  Adicionar título e descrição
\item
  Adicionar imagens
\item
  Adicionar vídeos
\item
  Adicionar seção
\end{itemize}
