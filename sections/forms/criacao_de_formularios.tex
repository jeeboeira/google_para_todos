% sections/forms/criacao_de_formularios.tex
% !TeX root = ../../../main.tex

\section{Criação de Formulários}
\subsection{Criar um Novo Formulário}

Na plataforma Google Formulários, para iniciar a criação de um novo
formulário, acesse a página principal selecionando o ícone roxo no canto
superior esquerdo da tela (caixa de diálogo ``Formulários''). Em
seguida, será possível identificar algumas opções para iniciar um novo
formulário. A opção mais convencional é a criação de um formulário ``em
branco'', onde utiliza-se o \textbf{template} original do Google
Formulários sem estrutura, conteúdo ou estilizações pré-definidas. Além
disso, a galeria também oferece modelos prontos como: Avaliação de
Curso, Teste em Branco, Atividade Final, Avaliação, Título da Planilha,
etc.

\begin{figure}[H]
	\centering
	\includegraphics[width=.9\textwidth]{images/forms/criacao_de_formularios/Imagem 1.png}
\end{figure}

Esses modelos podem ser utilizados como base, dependendo da necessidade.
Cada um deles possui um conjunto de elementos essenciais que simplifica
a criação de documentos similares.

Um \textbf{template} (exemplo) de formulário com estilo ``Convite para
Festa'':

\begin{figure}[H]
	\centering
	\includegraphics[width=.9\textwidth]{images/forms/criacao_de_formularios/Imagem 2.png}
\end{figure}

\subsection{Título e Descrição do Formulário}

Após selecionar a opção de formulário, as primeiras etapas são definir o
título e a descrição do formulário. Esses campos aparecem no topo da
página de edição.

\begin{figure}[H]
	\centering
	\includegraphics[width=.9\textwidth]{images/forms/criacao_de_formularios/Imagem 3.png}
\end{figure}

O título (exemplo da imagem: \textbf{Teste em branco}) identifica o
formulário, enquanto a descrição pode ser usada para fornecer instruções
ou informações adicionais para os respondentes, como o objetivo da
atividade ou o prazo para envio da resposta.

O título e a descrição são elementos essenciais na criação de um
formulário, pois são os primeiros itens visualizados pelos respondentes
e possuem papel fundamental na contextualização do conteúdo. O título
deve ser claro e específico, permitindo que a pessoa compreenda de
imediato o assunto abordado. Um bom título transmite seriedade e
facilita a organização dos formulários, especialmente quando há vários
em uso.

Já a descrição tem a função de complementar o título, oferecendo
orientações mais detalhadas sobre o que se espera do respondente. Nela,
é possível incluir informações como o objetivo do formulário, instruções
sobre como preencher, prazos para envio, quem deve responder, se as
respostas são anônimas, entre outros dados relevantes. Uma descrição bem
redigida torna o preenchimento mais fluido e evita o surgimento de
dúvidas, demonstrando atenciosidade na comunicação com o público. Assim,
investir tempo na definição adequada do título e da descrição contribui
diretamente para garantir eficácia e clareza ao formulário como um todo.

\subsection{Adição de Perguntas}

Após definir o título e a descrição do formulário, o próximo passo é a
adição das perguntas. Essa etapa é fundamental para coletar as
informações desejadas de forma clara e objetiva. O Google Formulários
oferece uma variedade de formatos para a criação das perguntas,
permitindo que o conteúdo seja adaptado conforme o tipo de dado que se
pretende obter. Entre as opções disponíveis, estão: múltipla escolha,
resposta curta, parágrafo, caixas de seleção, lista suspensa, upload de
arquivo, escala linear, classificação, grade de múltipla escolha, grade
de caixa de seleção, data e horário.

\begin{figure}[H]
	\centering
	\includegraphics[width=.9\textwidth]{images/forms/criacao_de_formularios/Imagem 4.png}
\end{figure}

O criador do formulário pode selecionar o tipo de pergunta mais
adequado, de acordo com a necessidade da atividade. Além disso, é
possível adicionar, duplicar ou excluir perguntas existentes, e definir
se cada uma delas deverá ter preenchimento obrigatório ou não. Essa
flexibilidade permite a construção de formulários personalizados e
eficientes, facilitando tanto a aplicação quanto a análise posterior das
respostas.
