% sections/forms/coleta_e_analise_de_respostas.tex
% !TeX root = ../../../main.tex

\section{Coleta e Análise de Respostas}
\subsection{Acesso às respostas}

Para acessar as respostas que foram enviadas ao seu formulário, basta
acessar a aba de ``Respostas'', estando com acesso de administrador
(``Proprietário''), através do link correspondente

\begin{figure}[H]
	\centering
	\includegraphics[width=.9\textwidth]{images/forms/Coleta e Análise de Respostas - 15-09/Imagem 1.png}
\end{figure}


\subsection{Resumo, por pergunta e individual}

Dentro da aba de respostas, disponibilizam-se diversas métricas e
resumos das respostas dadas, tanto ao nível da pergunta, em um contexto
geral de respostas, quanto ao nível individual de respostas do usuário

\begin{figure}[H]
	\centering
	\includegraphics[width=.9\textwidth]{images/forms/Coleta e Análise de Respostas - 15-09/Imagem 2.png}
\end{figure}

\begin{figure}[H]
	\centering
	\includegraphics[width=.9\textwidth]{images/forms/Coleta e Análise de Respostas - 15-09/Imagem 3.png}
\end{figure}

Diferente do resumo geral, que mostra gráficos e estatísticas de todas
as respostas juntas, esta função permite que você analise cada pergunta
individualmente. Ela isola uma única questão e lhe permite visualizar
como cada participante respondeu a ela, um por um. Essa perspectiva
contribui para a identificação de desvios e padrões, garantindo uma
análise mais aprofundada das respostas.

Em um cenário onde a precisão é fundamental, ou onde você precisa
entender a lógica por trás de cada escolha, o "Resumo Individual por
Pergunta" se torna uma ferramenta indispensável para uma análise crítica
e eficiente.

\subsection{Exportação para Google Planilhas}

Essa funcionalidade é o coração da análise avançada. Com apenas um
clique, todas as respostas do seu formulário são exportadas para uma
nova planilha ou para uma já existente. Cada linha da planilha
representa uma resposta individual, e cada coluna corresponde a uma
pergunta do formulário. Isso transforma os dados em um formato apto para
filtragem, ordenação, cálculo e manipulação, permitindo uma análise mais
profunda do que aquilo que o \textbf{Forms} pode oferecer sozinho. É ideal
para quando você precisa cruzar dados, fazer cálculos complexos ou
preparar informações para outras ferramentas, especialmente o
\textbf{Google Planilhas}, que utiliza arquivos no formato \textbf{.csv}.

\begin{figure}[H]
	\centering
	\includegraphics[width=.9\textwidth]{images/forms/Coleta e Análise de Respostas - 15-09/Imagem 4.png}
\end{figure}

É possível exportar os dados para uma nova planilha ou importar de
alguma já existente:

\begin{figure}[H]
	\centering
	\includegraphics[width=.9\textwidth]{images/forms/Coleta e Análise de Respostas - 15-09/Imagem 5.png}
\end{figure}

\begin{figure}[H]
	\centering
	\includegraphics[width=.9\textwidth]{images/forms/Coleta e Análise de Respostas - 15-09/Imagem 6.png}
\end{figure}

\subsection{Gráficos automáticos}

Assim que as respostas começam a chegar, o Google Forms cria um resumo
visual com gráficos automáticos para a maioria dos tipos de perguntas.
Para perguntas de múltipla escolha, por exemplo, ele gera gráficos de
barras ou de pizza, mostrando a distribuição das respostas. Para
perguntas de escala, ele exibe a média e a distribuição das pontuações.
Esses gráficos são atualizados em tempo real, fornecendo uma visão geral
imediata do desempenho do formulário, sem que você precise fazer
qualquer trabalho manual. Esses recursos são perfeitos para uma análise
rápida ou para apresentar um resumo visual dos resultados sem que haja a
necessidade de manipular planilhas.

\begin{figure}[H]
	\centering
	\includegraphics[width=.9\textwidth]{images/forms/Coleta e Análise de Respostas - 15-09/Imagem 7.png}
\end{figure}
