% sections/forms/logica_e_condicional.tex
% !TeX root = ../../../main.tex

\section{Lógica e Condicional}
\subsection{Uso de seções}

O conceito de seções no \textbf{Google Forms} se assemelha aos capítulos
de um livro. Ele permite que você divida seu formulário em partes
menores e mais organizadas. Em vez de ter uma longa lista de perguntas,
você pode agrupar perguntas relacionadas em seções separadas. Isso torna
o formulário mais fácil de preencher para o usuário, melhorando a
separação das informações e sua interpretação.

\begin{figure}[H]
	\centering
	\includegraphics[width=.14\textwidth]{images/forms/logica_e_condicional/Imagem 1.png}
	\caption{Barra lateral de ferramentas, destacando o ícone de "Adicionar seção" (duas barras paralelas), que permite dividir o formulário em várias páginas.}
\end{figure}

É possível criar diversas sessões no formulário, bem como atribuir
títulos e descrições para cada uma delas, garantindo mais clareza.


\begin{figure}[H]
	\centering
	\includegraphics[width=.8\textwidth]{images/forms/logica_e_condicional/Imagem 2.png}
	\caption{Visualização do formulário após a aplicação de seções, mostrando a estrutura com "Seção 2 de 3" e "Seção 3 de 3". Também exibe a opção de configurar o fluxo após o preenchimento de uma seção.}
\end{figure}

\subsection{Direcionamento baseado em respostas}

O direcionamento baseado em respostas é uma funcionalidade que permite
criar um fluxo personalizado para o seu formulário. Em vez de o usuário
seguir uma sequência linear de perguntas, ele será enviado para uma
seção específica com base na resposta que ele escolher.

Esse recurso é extremamente útil para tornar o formulário mais dinâmico
e relevante, pois o usuário só verá as perguntas que se aplicam a ele ou
ao contexto escolhido.

Como funciona:

1 - Você cria uma pergunta (geralmente de múltipla escolha ou lista
suspensa).

2 - Para cada opção de resposta, você define para qual seção o
formulário deve ``ir para a seção com base na resposta''.

\begin{figure}[H]
	\centering
	\includegraphics[width=.8\textwidth]{images/forms/logica_e_condicional/Imagem 3.png}
	\caption{Menu de opções adicionais de uma pergunta de Múltipla Escolha, onde a funcionalidade "Ir para a seção com base na resposta" (lógica condicional) é ativada.}
\end{figure}

\begin{figure}[H]
	\centering
	\includegraphics[width=.8\textwidth]{images/forms/logica_e_condicional/Imagem 4.png}
	\caption{Pergunta de Múltipla Escolha com a lógica condicional ativada, mostrando o menu suspenso ao lado de cada opção de resposta, que define para qual seção o usuário será direcionado com base na escolha.}
\end{figure}

\begin{figure}[H]
	\centering
	\includegraphics[width=.8\textwidth]{images/forms/logica_e_condicional/Imagem 5.png}
	\caption{Demonstração do uso da lógica condicional, onde a opção de resposta está configurada para "Ir para a seção 2 (Seção 2)", direcionando o usuário para a página correta do formulário.}
\end{figure}

\subsection{Perguntas obrigatórias}

Uma pergunta obrigatória é aquela que o usuário deve responder para
poder enviar o formulário. O \textbf{Google Forms} impede o envio do
formulário se alguma pergunta obrigatória não for preenchida. Essa
funcionalidade garante que você colete todas as informações essenciais.
Perguntas obrigatórias sempre estarão indicadas por um pequeno asterisco
vermelho (*) ao lado da questão.

Podem ser úteis para obtenção de dados cruciais para a sua análise ou
registro, como nome, e-mail e informações de contato.

\begin{figure}[H]
	\centering
	\includegraphics[width=.8\textwidth]{images/forms/logica_e_condicional/Imagem 6.png}
	\caption{Edição de uma pergunta de Múltipla Escolha, destacando a chave seletora "Obrigatória", que força o participante a responder à questão antes de poder enviar o formulário ou avançar.}
\end{figure}

\begin{figure}[H]
	\centering
	\includegraphics[width=.4\textwidth]{images/forms/logica_e_condicional/Imagem 7.png}
	\caption{Visualização de uma pergunta no modo de edição que foi marcada como obrigatória, indicada pelo asterisco vermelho (*) ao lado do título da pergunta.}
\end{figure}
