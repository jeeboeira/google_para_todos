% sections/forms/compartilhamento_do_formulario.tex
% !TeX root = ../../../main.tex

\section{Compartilhamento do Formulário}
\subsection{Como enviar (link, e-mail, QR Code):}

Você pode enviar seu formulário de várias maneiras, mas deve sempre
basear-se na forma com maior alcance ao seu público-alvo:

\begin{itemize}
\item
  \textbf{Link direto}:
\end{itemize}

Dessa forma, você copia diretamente o link gerado pelo serviço de
Formulário, podendo ser enviado via Whatsapp, Telegram, corpo do texto
de um e-mail ou qualquer outro canal de comunicação por mensagens

\begin{figure}[H]
	\centering
	\includegraphics[width=.8\textwidth]{images/forms/compartilhamento_do_formulario/Imagem 12.png}
\end{figure}

\begin{itemize}
\item
  \textbf{E-mail}:
\end{itemize}

É possível enviar diretamente o formulário por e-mail, personalizando a
mensagem de convite e selecionando uma lista específica de
destinatários.

\begin{figure}[H]
	\centering
	\includegraphics[width=.7\textwidth]{images/forms/compartilhamento_do_formulario/Imagem 13.png}
\end{figure}

\begin{itemize}
\item
  \textbf{Permissões de visualização e edição:}
\end{itemize}

Podem ser alteradas através do botão compartilhar (ícone de ``perfil e
mais''), na aba superior do \textbf{Google Forms}. É possível definir quem
pode responder, habilitando acesso ao público geral (qualquer pessoa com
o link) ou restringindo a usuários específicos, como apenas para pessoas
da sua organização. Nesse caso, você deve selecionar a opção
``Restrito''. Quanto às permissões de ``visualização de editor'',
normalmente apenas o criador pode alterar o formulário (``Restrito'' por
padrão), mas a edição colaborativa é permitida, e você pode atribuí-la,
com cuidado, ao selecionar a opção ``Qualquer pessoa com o link''.

\begin{figure}[H]
	\centering
	\includegraphics[width=.6\textwidth]{images/forms/compartilhamento_do_formulario/Imagem 14.png}
\end{figure}

\subsection{Incorporação em sites:}

Trata-se de um recurso um pouco mais avançado. No ícone ``mais'' (três
pontinhos na vertical), selecione a opção ``Incorporar HTML''. Copie o
código HTML que pretendes incorporar e cole no seu site para que o
formulário apareça diretamente na página. Ele funciona como um
\textbf{\textbf{iframe}}, permitindo que os usuários respondam ao
formulário sem ter que sair do site.

\begin{figure}[H]
	\centering
	\includegraphics[width=.5\textwidth]{images/forms/compartilhamento_do_formulario/Imagem 15.png}
\end{figure}

\begin{figure}[H]
	\centering
	\includegraphics[width=.7\textwidth]{images/forms/compartilhamento_do_formulario/Imagem 16.png}
\end{figure}

%
