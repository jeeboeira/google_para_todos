% sections/forms/formularios.tex
% !TeX root = ../../../main.tex

\section{Introdução ao Google Forms}

O \textbf{Google Forms} é um aplicativo da web, gratuito, desenvolvido
pela Google, que faz parte do pacote do \textbf{Google Docs Editors} e, de
acordo com o site TechTudo, é uma ferramenta que vai além da criação de
formulários tradicionais. Com ele, é possível montar pesquisas e
enquetes utilizando diferentes tipos de perguntas, como múltipla
escolha, escalas de avaliação e respostas abertas. Pode ser usado para
diversos fins, como organizar inscrições em eventos, coletar e-mails de
contato, aplicar questionários e até mesmo criar \textbf{quizzes}. Outro
ponto positivo é que ele possibilita o trabalho em equipe
(colaborativo), através da edição e criação de formulários
simultaneamente, por meio de conexão à internet, por múltiplos usuários.

Alguns de seus usos mais comuns estão voltados à realização de pesquisas
de satisfação e coleta de \textbf{feedbacks}, de modo a avaliar a
experiência de alguns consumidores com serviços, produtos e
atendimentos. Também é amplamente utilizado para cadastro em eventos,
registro de participantes, aplicações de provas e questionários
à distância, bem como em situações no ambiente de trabalho, como para
solicitar férias, aferir as pesquisas de clima e registrar chamados.

\section{Acesso à Ferramenta}

O acesso ao \textbf{Google Forms} pode ser realizado de duas formas
principais: através do \textbf{Google Drive} ou diretamente pelo
\textbf{link} da web: ``forms.google.com''. É importante salientar que, para
acessar a ferramenta, é necessário possuir uma conta Google. Abaixo, o
passo a passo de acesso para cada um desses métodos:

\begin{itemize}
\item
  Pelo Google Drive:

  \begin{itemize}
  \item
    Na barra de pesquisa do navegador, digite ``drive.google.com'';
  \item
    No site aberto, clique em
    \textbf{``Novo''}→\textbf{``Mais''}→\textbf{``Formulários Google''};
  \item
    Pronto, uma nova página será aberta e você já pode começar a criar.
  \end{itemize}
\end{itemize}

\begin{itemize}
\item
  Diretamente pelo site do Forms: 

  \begin{itemize}
  \item
    Na barra de pesquisa do navegador, digite ``forms.google.com'';
  \item
    Selecione a opção ``\textbf{Em branco}'' ou escolha um modelo
    pronto;
  \item
    Uma nova página será criada e você já pode começar a editar.
  \end{itemize}
\end{itemize}

\section{Interface do Google Forms}

\subsection{Visão Geral da Tela Inicial:}

A tela inicial do \textbf{Google Forms} é simples e intuitiva. Na barra
superior, o usuário pode definir o nome do formulário, acessar as
configurações, personalizar o layout e utilizar o botão de envio.

(IMAGEM 3)

\textbf{Barra Superior} (canto superior direito):

(IMAGEM 16)

A barra superior concentra as opções de edição, visualização,
personalização e publicação do formulário. Abaixo, uma análise de suas
funcionalidades:

\begin{itemize}
\item
  \textbf{Personalizar tema:}
\end{itemize}

(IMAGEM 4)

\begin{quote}
É utilizado para alterar o tema, cores, imagens e a fonte
\end{quote}

(IMAGEM 21).

\begin{itemize}
\item
  \textbf{Visualizar:}
\end{itemize}

(IMAGEM 5)

\begin{quote}
Permite verificar como o formulário será exibido para as pessoas que
irão responder.
\end{quote}

(IMAGEM 22)

\begin{itemize}
\item
  \textbf{Desfazer e refazer:}
\end{itemize}

(IMAGEM 6 E 7)

\begin{quote}
Essas opções controlam alterações feitas no documento, permitindo
desfazer as últimas modificações ou refazer o que foi desfeito.
\end{quote}

\begin{itemize}
\item
  \textbf{Copiar Link:}
\end{itemize}

(IMAGEM 8)

\begin{quote}
Permite copiar o \textbf{link} do formulário para compartilhamento. Após,
basta encaminhar o \textbf{link} de acesso para os respondentes.
\end{quote}

\begin{itemize}
\item
  \textbf{Compartilhar:}
\end{itemize}

(IMAGEM 9)

\begin{quote}
É possível compartilhar o link para determinados e-mails definindo-os
como participantes ou editores do formulário.
\end{quote}

\begin{itemize}
\item
  \textbf{Publicar:}
\end{itemize}

(IMAGEM 9)

\begin{quote}
Serve para publicar o formulário, disponibilizando-o para que os
participantes possam responder.
\end{quote}

\begin{itemize}
\item
  \textbf{Botão ``Mais''} (ícone de três pontinhos na
  vertical)\textbf{:}
\end{itemize}

\begin{quote}
(IMAGEM 23)

Habilita algumas opções adicionais como: fazer uma cópia do formulário,
mover para lixeira, preencher automaticamente, incorporar HTML,
Imprimir, \textbf{Apps Script} - que permite escrever um código na
linguagem HTML para incorporar ao formulário -, instalar complementos no
formulário, cancelar a publicação e configurar atalhos do teclado.
\end{quote}

\subsection{Aba Respostas:}

(IMAGEM 10)

Nesta aba é possível verificar as respostas do formulário e exportá-las
para o \textbf{Google} \textbf{Planilhas}, se desejado. Contempla, também,
mais algumas opções: receber notificações por e-mail, selecionar destino
para as respostas, desvincular formulário, baixar as respostas em
formato .\textbf{csv} (para planilhas), imprimir ou excluir as respostas.

(IMAGEM 11)

\subsection{Aba Configurações:}

(IMAGEM 12)

Aqui pode ser configurado o formulário, por meio da criação de testes
(menu ``Criar Teste''), gerenciar como as respostas são coletadas (menu
``Respostas''), enviar uma cópia das respostas aos participantes,
permitir a edição das respostas e limitar para apenas 1 resposta por
e-mail, por exemplo.

(IMAGEM 13)

Essa parte trata da apresentação do formulário ao usuário, podendo ser
habilitada a exibição da barra de progresso e o embaralhamento de
perguntas. Temos, ainda, algumas configurações após envio da resposta,
como: mostrar mensagem de confirmação após envio, mostrar link para
enviar outra resposta, exibir resumo dos resultados e uma restrição
especial para desativar o salvamento automático dos participantes, caso
eles saiam da página antes de finalizar as respostas não são salvas.

(IMAGEM 14)

Por fim, temos alguns padrões, como coletar e-mails por padrão e tornar
as perguntas obrigatórias.

\subsection{Barra Lateral}

Na barra lateral encontramos seis opções como:

(IMAGEM 15)

\begin{itemize}
\item
  Adicionar pergunta
\item
  Importar pergunta
\item
  Adicionar título e descrição
\item
  Adicionar imagens
\item
  Adicionar vídeos
\item
  Adicionar seção
\end{itemize}

\section{Criação de Formulários}

\subsection{Criar um Novo Formulário}

Na plataforma Google Formulários, para iniciar a criação de um novo
formulário, acesse a página principal selecionando o ícone roxo no canto
superior esquerdo da tela (caixa de diálogo ``Formulários''). Em
seguida, será possível identificar algumas opções para iniciar um novo
formulário. A opção mais convencional é a criação de um formulário ``em
branco'', onde utiliza-se o \textbf{template} original do Google
Formulários sem estrutura, conteúdo ou estilizações pré-definidas. Além
disso, a galeria também oferece modelos prontos como: Avaliação de
Curso, Teste em Branco, Atividade Final, Avaliação, Título da Planilha,
etc.

(IMAGEM 1)

Esses modelos podem ser utilizados como base, dependendo da necessidade.
Cada um deles possui um conjunto de elementos essenciais que simplifica
a criação de documentos similares.

Um \textbf{template} (exemplo) de formulário com estilo ``Convite para
Festa'':

(IMAGEM 2)

\subsection{Título e Descrição do Formulário}

Após selecionar a opção de formulário, as primeiras etapas são definir o
título e a descrição do formulário. Esses campos aparecem no topo da
página de edição.

(IMAGEM 3)

O título (exemplo da imagem: \textbf{Teste em branco}) identifica o
formulário, enquanto a descrição pode ser usada para fornecer instruções
ou informações adicionais para os respondentes, como o objetivo da
atividade ou o prazo para envio da resposta.

O título e a descrição são elementos essenciais na criação de um
formulário, pois são os primeiros itens visualizados pelos respondentes
e possuem papel fundamental na contextualização do conteúdo. O título
deve ser claro e específico, permitindo que a pessoa compreenda de
imediato o assunto abordado. Um bom título transmite seriedade e
facilita a organização dos formulários, especialmente quando há vários
em uso.

Já a descrição tem a função de complementar o título, oferecendo
orientações mais detalhadas sobre o que se espera do respondente. Nela,
é possível incluir informações como o objetivo do formulário, instruções
sobre como preencher, prazos para envio, quem deve responder, se as
respostas são anônimas, entre outros dados relevantes. Uma descrição bem
redigida torna o preenchimento mais fluido e evita o surgimento de
dúvidas, demonstrando atenciosidade na comunicação com o público. Assim,
investir tempo na definição adequada do título e da descrição contribui
diretamente para garantir eficácia e clareza ao formulário como um todo.

\subsection{Adição de Perguntas}

Após definir o título e a descrição do formulário, o próximo passo é a
adição das perguntas. Essa etapa é fundamental para coletar as
informações desejadas de forma clara e objetiva. O Google Formulários
oferece uma variedade de formatos para a criação das perguntas,
permitindo que o conteúdo seja adaptado conforme o tipo de dado que se
pretende obter. Entre as opções disponíveis, estão: múltipla escolha,
resposta curta, parágrafo, caixas de seleção, lista suspensa, upload de
arquivo, escala linear, classificação, grade de múltipla escolha, grade
de caixa de seleção, data e horário.

(IMAGEM 4)

O criador do formulário pode selecionar o tipo de pergunta mais
adequado, de acordo com a necessidade da atividade. Além disso, é
possível adicionar, duplicar ou excluir perguntas existentes, e definir
se cada uma delas deverá ter preenchimento obrigatório ou não. Essa
flexibilidade permite a construção de formulários personalizados e
eficientes, facilitando tanto a aplicação quanto a análise posterior das
respostas.

\section{Tipos de Perguntas Disponíveis}

\subsection{Resposta curta}

Essa categoria é utilizada para respostas rápidas, geralmente textos com
poucas palavras ou caracteres, como: nome, e-mail, número de matrícula,
e similares.

(IMAGEM 24)

\subsection{Parágrafo}

Ao contrário da categoria anterior, costuma ser utilizado quando a
resposta precisa ser mais longa em extensão, como no caso da descrição
de um serviço ou objeto, opinião ou \textbf{feedback} mais detalhado.

(IMAGEM 25)

\subsection{Múltipla escolha}

Fornece uma lista de opções de escolha. Somente uma pode ser
selecionada.

(IMAGEM 26)

\subsection{Caixas de seleção}

Semelhante à categoria anterior, mas permite a seleção de mais de uma
opção entre as alternativas.

(IMAGEM 27)

\subsection{Menu suspenso}

Muito parecido com a ``Múltipla Escolha'', porém, é apresentado em
formato de um menu suspenso, contendo as opções disponíveis para seleção
do respondente.

(IMAGEM 28)

(IMAGEM 29)

(IMAGEM 30)

\subsection{Upload de arquivos}

Permite que o participante envie um arquivo, sendo ele um documento,
planilha, PDF, vídeo, apresentação, desenho, imagem ou áudio. Pode-se
limitar a quantidade de arquivos e o tamanho máximo do arquivo (limite
máximo de 1GB para a totalidade de arquivos anexos). É necessário que o
usuário esteja logado em sua conta Google para utilizar esse recurso.

(IMAGEM 31)


\subsection{Escala linear}


Com essa função, o participante pode, por exemplo, avaliar algum serviço
ou atendimento ao atribuir uma nota presente na escala de 1 a 5, 1 a 10
ou 0 a 10, conforme determinado pelo criador do formulário.

(IMAGEM 32)

(IMAGEM 33)


\subsection{Classificação}


Semelhante ao último tipo de avaliação, porém com uma opção para
estilização dos ícones selecionáveis (``coração'', ``estrela'' ou
``polegar / positivo'').

(IMAGEM 34)

(IMAGEM 35)


\subsection{Grade de múltipla escolha}


Possibilita a criação de uma ``tabela'', onde, a cada linha criada,
pode-se atribuir um conjunto de opções para seleção de múltipla escolha
ao usuário (resposta única por linha).

(IMAGEM 36)

(IMAGEM 37)


\subsection{Grade de caixas de seleção}


Semelhante à categoria anterior, porém, permite que mais de uma opção
possa ser selecionada por linha.

(IMAGEM 38)

\subsection{Data e Horário}

Essas opções permitem que o usuário insira uma data, que contempla dia,
mês e ano (formato \textbf{dd}/\textbf{mm}/\textbf{aaaa}), e/ou um horário,
contendo horas e minutos (formato \textbf{hh}:\textbf{mm}).

(IMAGEM 39)

(IMAGEM 40)

(IMAGEM 41)

(IMAGEM 42)

\section{Personalização do Formulário}

\subsection{Mudança de tema (cores, imagens de fundo):}

Dentro dos formulários, é possível alterar temas e personalizar,
escolhendo entre cores predefinidas ou atribuindo uma paleta de sua
preferência. Também é possível adicionar imagens de fundo para tornar o
visual mais atrativo e contextualizar o seu questionário. Também é
possível alterar fontes, tamanhos de letras e planos de fundo.

(IMAGEM 1)

\subsection{Layouts visuais e Imagens e Vídeos nas perguntas}

O Google Forms oferece layouts simples e intuitivos, mas é possível
organizar perguntas de forma personalizada utilizando seções e páginas,
criando uma navegação fluida e visualmente agradável, bem como vídeos e
outros tópicos interativos.

(IMAGEM 2)

\section{Configurações do Formulário}

\subsection{Coleta de e-mails:}

A configuração da coleta de e-mails na ferramenta pode ser configurada
no menu expansível ``Respostas'', dentro da guia ``Configurações'', que
está presente na aba superior do \textbf{Google Formulários}.

A coleta de e-mails possui 3 formatos: ``verificado'', onde o
participante deverá estar logado exclusivamente em sua conta Google para
submissão da resposta; ``entrada do participante'', na qual o
respondente pode inserir seu e-mail manualmente; e ``não coletar'',
desabilitando essa funcionalidade.

(IMAGEM 7)

\subsection{Permitir ou não edições após envio}

Na mesma aba ``Respostas'', o campo de seleção ``permitir edição das
respostas'' define se o respondente poderá ou não alterar suas respostas
depois de enviar o formulário. Quando esta opção está ativa, o usuário
recebe um link para revisar e corrigir suas respostas - uma alternativa
bastante utilizada em pesquisas ou cadastros que demandam atualizações
futuras. Quando está desativada, as respostas tornam-se definitivas e
inalteráveis, garantindo maior confiabilidade. Esta opção é
especialmente útil no caso de avaliações, inscrições e outros processos
que requerem mais formalidade.

(IMAGEM 8)

\subsection{Mostrar barra de progresso}

Encontra-se na aba ``Apresentação'' (abaixo de ``Respostas''). Ativar
essa opção faz com que, no rodapé do formulário, seja exibida uma linha
indicativa que informa quanto já foi respondido em relação ao total de
seções do formulário. É bastante útil para orientar o respondente, dando
noção de avanço e tempo necessário para concluir o preenchimento.

(IMAGEM 9)

\subsection{Embaralhar ordem das perguntas}

Na mesma guia ``Apresentação'', esta opção altera automaticamente a
sequência em que as questões aparecem para cada respondente. Esse
recurso é amplamente utilizado em avaliações, pois ajuda a evitar que
participantes copiem respostas uns dos outros, com base no ordenamento
de questões.

\begin{dica} 
  Esse recurso pode impactar diretamente na coleta de
  opiniões, já que a ordem das perguntas pode influenciar a forma com que
  elas são respondidas.
\end{dica} 

(IMAGEM 10)

\subsection{Transformar em teste (modo quiz)}

Ativar a opção ``Criar teste'' (acima da guia ``Respostas'') permite que
o formulário seja usado como uma avaliação. Desse modo, o criador pode
atribuir pontos às perguntas, definir respostas corretas e incluir
\textbf{feedback} automático para acertos e erros. ele é muito útil em
ambientes educacionais e treinamentos, pois possibilita a correção de
respostas imediata e o cálculo de notas automaticamente. Além disso,
pode-se escolher se o resultado deve ser exibido logo após o envio ou
posteriormente, mediante autorização do criador.

(IMAGEM 11)

\section{Lógica e Condicional}

\subsection{Uso de seções}

O conceito de seções no \textbf{Google Forms} se assemelha aos capítulos
de um livro. Ele permite que você divida seu formulário em partes
menores e mais organizadas. Em vez de ter uma longa lista de perguntas,
você pode agrupar perguntas relacionadas em seções separadas. Isso torna
o formulário mais fácil de preencher para o usuário, melhorando a
separação das informações e sua interpretação.

(IMAGEM 1)

É possível criar diversas sessões no formulário, bem como atribuir
títulos e descrições para cada uma delas, garantindo mais clareza.

(IMAGEM 2)

\subsection{Direcionamento baseado em respostas}

O direcionamento baseado em respostas é uma funcionalidade que permite
criar um fluxo personalizado para o seu formulário. Em vez de o usuário
seguir uma sequência linear de perguntas, ele será enviado para uma
seção específica com base na resposta que ele escolher.

Esse recurso é extremamente útil para tornar o formulário mais dinâmico
e relevante, pois o usuário só verá as perguntas que se aplicam a ele ou
ao contexto escolhido.

Como funciona:

1 - Você cria uma pergunta (geralmente de múltipla escolha ou lista
suspensa).

2 - Para cada opção de resposta, você define para qual seção o
formulário deve "ir para a seção com base na resposta".

(IMAGEM 3)

(IMAGEM 4)

\subsection{Perguntas obrigatórias}

Uma pergunta obrigatória é aquela que o usuário deve responder para
poder enviar o formulário. O \textbf{Google Forms} impede o envio do
formulário se alguma pergunta obrigatória não for preenchida. Essa
funcionalidade garante que você colete todas as informações essenciais.
Perguntas obrigatórias sempre estarão indicadas por um pequeno asterisco
vermelho (*) ao lado da questão.

Podem ser úteis para obtenção de dados cruciais para a sua análise ou
registro, como nome, e-mail e informações de contato.

(IMAGEM 5)

(IMAGEM 6)

\section{Compartilhamento do Formulário}

\subsection{Como enviar (link, e-mail, QR Code):}

Você pode enviar seu formulário de várias maneiras, mas deve sempre
basear-se na forma com maior alcance ao seu público-alvo:

\begin{itemize}
\item
  \textbf{Link direto}:
\end{itemize}

Dessa forma, você copia diretamente o link gerado pelo serviço de
Formulário, podendo ser enviado via Whatsapp, Telegram, corpo do texto
de um e-mail ou qualquer outro canal de comunicação por mensagens

(IMAGEM 12)

\begin{itemize}
\item
  \textbf{E-mail}:
\end{itemize}

É possível enviar diretamente o formulário por e-mail, personalizando a
mensagem de convite e selecionando uma lista específica de
destinatários.

(IMAGEM 13)

\begin{itemize}
\item
  \textbf{Permissões de visualização e edição:}
\end{itemize}

Podem ser alteradas através do botão compartilhar (ícone de ``perfil e
mais''), na aba superior do \textbf{Google Forms}. É possível definir quem
pode responder, habilitando acesso ao público geral (qualquer pessoa com
o link) ou restringindo a usuários específicos, como apenas para pessoas
da sua organização. Nesse caso, você deve selecionar a opção
``Restrito''. Quanto às permissões de ``visualização de editor'',
normalmente apenas o criador pode alterar o formulário (``Restrito'' por
padrão), mas a edição colaborativa é permitida, e você pode atribuí-la,
com cuidado, ao selecionar a opção ``Qualquer pessoa com o link''.

(IMAGEM 13)

\subsection{Incorporação em sites:}

Trata-se de um recurso um pouco mais avançado. No ícone ``mais'' (três
pontinhos na vertical), selecione a opção ``Incorporar HTML''. Copie o
código HTML que pretendes incorporar e cole no seu site para que o
formulário apareça diretamente na página. Ele funciona como um
\textbf{\textbf{iframe}}, permitindo que os usuários respondam ao
formulário sem ter que sair do site.

(IMAGEM 14)

(IMAGEM 15)

\section{Coleta e Análise de Respostas}

\subsection{Acesso às respostas}

Para acessar as respostas que foram enviadas ao seu formulário, basta
acessar a aba de ``Respostas'', estando com acesso de administrador
(``Proprietário''), através do link correspondente

(IMAGEM 1)

\subsection{Resumo, por pergunta e individual}

Dentro da aba de respostas, disponibilizam-se diversas métricas e
resumos das respostas dadas, tanto ao nível da pergunta, em um contexto
geral de respostas, quanto ao nível individual de respostas do usuário

(IMAGEM 2)

(IMAGEM 3)

Diferente do resumo geral, que mostra gráficos e estatísticas de todas
as respostas juntas, esta função permite que você analise cada pergunta
individualmente. Ela isola uma única questão e lhe permite visualizar
como cada participante respondeu a ela, um por um. Essa perspectiva
contribui para a identificação de desvios e padrões, garantindo uma
análise mais aprofundada das respostas.

Em um cenário onde a precisão é fundamental, ou onde você precisa
entender a lógica por trás de cada escolha, o "Resumo Individual por
Pergunta" se torna uma ferramenta indispensável para uma análise crítica
e eficiente.

\subsection{Exportação para Google Planilhas}

Essa funcionalidade é o coração da análise avançada. Com apenas um
clique, todas as respostas do seu formulário são exportadas para uma
nova planilha ou para uma já existente. Cada linha da planilha
representa uma resposta individual, e cada coluna corresponde a uma
pergunta do formulário. Isso transforma os dados em um formato apto para
filtragem, ordenação, cálculo e manipulação, permitindo uma análise mais
profunda do que aquilo que o \textbf{Forms} pode oferecer sozinho. É ideal
para quando você precisa cruzar dados, fazer cálculos complexos ou
preparar informações para outras ferramentas, especialmente o
\textbf{Google Planilhas}, que utiliza arquivos no formato \textbf{.csv}.

(IMAGEM 4)

É possível exportar os dados para uma nova planilha ou importar de
alguma já existente:

(IMAGEM 5)

(IMAGEM 6)

\subsection{Gráficos automáticos}

Assim que as respostas começam a chegar, o Google Forms cria um resumo
visual com gráficos automáticos para a maioria dos tipos de perguntas.
Para perguntas de múltipla escolha, por exemplo, ele gera gráficos de
barras ou de pizza, mostrando a distribuição das respostas. Para
perguntas de escala, ele exibe a média e a distribuição das pontuações.
Esses gráficos são atualizados em tempo real, fornecendo uma visão geral
imediata do desempenho do formulário, sem que você precise fazer
qualquer trabalho manual. Esses recursos são perfeitos para uma análise
rápida ou para apresentar um resumo visual dos resultados sem que haja a
necessidade de manipular planilhas.

(IMAGEM 7)

\section{Dicas e Boas Práticas}

\subsection{Organização e clareza das perguntas:}

Mantenha as questões objetivas, com enunciados simples e bem
estruturados. Evite termos técnicos ou muito específicos para que
qualquer pessoa compreenda facilmente o que está sendo solicitado.

\subsection{Testar o formulário antes do envio:}

Faça uma simulação respondendo o formulário como se fosse um
participante. Isso ajuda a identificar possíveis erros, questões mal
formuladas ou problemas de lógica que podem prejudicar a coleta de
respostas. O recurso ``Visualizar'', na aba superior, também é útil para
ter um vislumbre do resultado final durante o desenvolvimento do
formulário.

\subsection{Limitar número de respostas, quando necessário:}

Em alguns casos, é importante controlar a quantidade de respostas, com o
intuito de evitar duplicidades, restringir o acesso a um número
específico de participantes e garantir mais organização na análise dos
dados posteriormente.

\subsection{Garantir acessibilidade:}

Verifique se o formulário é compatível com diferentes dispositivos
(computador, tablet e celular) e se pode ser acessado por pessoas com
necessidades especiais, assim pode-se ampliar o alcance e assegurar a
participação de todos, com clareza e acessibilidade.
