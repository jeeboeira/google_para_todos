% sections/forms/tipos_de_perguntas_disponiveis.tex
% !TeX root = ../../../main.tex

\section{Tipos de Perguntas Disponíveis}
\subsection{Resposta curta}

Essa categoria é utilizada para respostas rápidas, geralmente textos com
poucas palavras ou caracteres, como: nome, e-mail, número de matrícula,
e similares.

\begin{figure}[H]
	\centering
	\includegraphics[width=.7\textwidth]{images/forms/tipos_de_perguntas_disponiveis/Imagem 24.png}
	\caption{Pergunta configurada como "Resposta curta", destinada à coleta de texto conciso (ex: nome, idade).}
\end{figure}

\subsection{Parágrafo}

Ao contrário da categoria anterior, costuma ser utilizado quando a
resposta precisa ser mais longa em extensão, como no caso da descrição
de um serviço ou objeto, opinião ou \textbf{feedback} mais detalhado.

\begin{figure}[H]
	\centering
	\includegraphics[width=.7\textwidth]{images/forms/tipos_de_perguntas_disponiveis/Imagem 25.png}
	\caption{Pergunta configurada como "Parágrafo" (Resposta Longa), ideal para respostas dissertativas ou abertas.}
\end{figure}

\subsection{Múltipla escolha}

Fornece uma lista de opções de escolha. Somente uma pode ser
selecionada.

\begin{figure}[H]
	\centering
	\includegraphics[width=.7\textwidth]{images/forms/tipos_de_perguntas_disponiveis/Imagem 26.png}
	\caption{Pergunta configurada como "Múltipla escolha", onde o participante pode selecionar apenas uma opção entre as listadas.}
\end{figure}

\subsection{Caixas de seleção}

Semelhante à categoria anterior, mas permite a seleção de mais de uma
opção entre as alternativas.

\begin{figure}[H]
	\centering
	\includegraphics[width=.7\textwidth]{images/forms/tipos_de_perguntas_disponiveis/Imagem 27.png}
	\caption{Pergunta configurada como "Caixas de seleção", que permite ao participante escolher várias opções das listadas (múltipla resposta).}
\end{figure}

\subsection{Menu suspenso}

Muito parecido com a ``Múltipla Escolha'', porém, é apresentado em
formato de um menu suspenso, contendo as opções disponíveis para seleção
do respondente.

\begin{figure}[H]
	\centering
	\includegraphics[width=.7\textwidth]{images/forms/tipos_de_perguntas_disponiveis/Imagem 28.png}
	\caption{Pergunta configurada como "Lista suspensa" (Dropdown), que economiza espaço na página, mostrando as opções somente após o clique.}
\end{figure}

\begin{figure}[H]
	\centering
	\includegraphics[width=.7\textwidth]{images/forms/tipos_de_perguntas_disponiveis/Imagem 30.png}
	\caption{Visualização da "Lista suspensa" no formulário do participante, exibindo as opções configuradas ("Opção 1," "Opção 2," etc.).}
\end{figure}

\subsection{Upload de arquivos}

Permite que o participante envie um arquivo, sendo ele um documento,
planilha, PDF, vídeo, apresentação, desenho, imagem ou áudio. Pode-se
limitar a quantidade de arquivos e o tamanho máximo do arquivo (limite
máximo de 1GB para a totalidade de arquivos anexos). É necessário que o
usuário esteja logado em sua conta Google para utilizar esse recurso.

\begin{figure}[H]
	\centering
	\includegraphics[width=.9\textwidth]{images/forms/tipos_de_perguntas_disponiveis/Imagem 31.png}
	\caption{Pergunta configurada como "Upload de arquivo", que permite ao participante enviar documentos. A imagem detalha as opções de restrição, como permitir apenas tipos de arquivo específicos (Documento, PDF, Imagem, etc.), definir o número máximo e o tamanho máximo do arquivo.}
\end{figure}


\subsection{Escala linear}


Com essa função, o participante pode, por exemplo, avaliar algum serviço
ou atendimento ao atribuir uma nota presente na escala de 1 a 5, 1 a 10
ou 0 a 10, conforme determinado pelo criador do formulário.

\begin{figure}[H]
	\centering
	\includegraphics[width=.7\textwidth]{images/forms/tipos_de_perguntas_disponiveis/Imagem 32.png}
	\caption{Pergunta configurada como "Escala linear", um tipo de questão Likert que define um intervalo numérico (de 1 a 10, neste exemplo) e permite rotular as extremidades ("Avaliação Mínima" e "Avaliação Máxima").}
\end{figure}

\begin{figure}[H]
	\centering
	\includegraphics[width=.7\textwidth]{images/forms/tipos_de_perguntas_disponiveis/Imagem 33.png}
	\caption{Visualização da "Escala linear" no formulário do participante, mostrando a série de botões de rádio para a seleção do valor.}
\end{figure}


\subsection{Classificação}

Semelhante ao último tipo de avaliação, porém com uma opção para
estilização dos ícones selecionáveis (``coração'', ``estrela'' ou
``polegar / positivo'').

\begin{figure}[H]
	\centering
	\includegraphics[width=.7\textwidth]{images/forms/tipos_de_perguntas_disponiveis/Imagem 34.png}
	\caption{Configuração de uma pergunta tipo "Classificação" (Rating), que utiliza um número predefinido de estrelas (neste caso, 5) para que o participante atribua uma nota.}
\end{figure}

\subsection{Grade de múltipla escolha}

Possibilita a criação de uma ``tabela'', onde, a cada linha criada,
pode-se atribuir um conjunto de opções para seleção de múltipla escolha
ao usuário (resposta única por linha).

\begin{figure}[H]
	\centering
	\includegraphics[width=.7\textwidth]{images/forms/tipos_de_perguntas_disponiveis/Imagem 36.png}
	\caption{Configuração de uma pergunta tipo "Grade de múltipla escolha" (Matriz de Múltipla Escolha). Este tipo organiza itens em Linhas e Colunas, onde o participante deve selecionar apenas uma opção por linha (a chave "Exigir uma resposta em cada linha" está desativada no momento da captura).}
\end{figure}

\begin{figure}[H]
	\centering
	\includegraphics[width=.7\textwidth]{images/forms/tipos_de_perguntas_disponiveis/Imagem 37.png}
	\caption{Visualização da "Grade de múltipla escolha" no formulário do participante, demonstrando que apenas um rádio botão pode ser selecionado por linha.}
\end{figure}

\subsection{Grade de caixas de seleção}


Semelhante à categoria anterior, porém, permite que mais de uma opção
possa ser selecionada por linha.

\begin{figure}[H]
	\centering
	\includegraphics[width=.7\textwidth]{images/forms/tipos_de_perguntas_disponiveis/Imagem 38.png}
	\caption{Visualização da "Grade da caixa de seleção" (Matriz de Caixas de Seleção). Diferente da anterior, esta permite que o participante selecione várias opções por linha (múltiplas caixas de seleção).}
\end{figure}

\subsection{Data e Horário}

Essas opções permitem que o usuário insira uma data, que contempla dia,
mês e ano (formato \textbf{dd}/\textbf{mm}/\textbf{aaaa}), e/ou um horário,
contendo horas e minutos (formato \textbf{hh}:\textbf{mm}).

\begin{figure}[H]
	\centering
	\includegraphics[width=.7\textwidth]{images/forms/tipos_de_perguntas_disponiveis/Imagem 39.png}
	\caption{Configuração de uma pergunta tipo "Data", que exige que o participante insira uma data.}
\end{figure}

\begin{figure}[H]
	\centering
	\includegraphics[width=.7\textwidth]{images/forms/tipos_de_perguntas_disponiveis/Imagem 40.png}
	\caption{Visualização da pergunta de "Data" no formulário do participante, mostrando o widget de calendário para facilitar a seleção da data.}
\end{figure}

\begin{figure}[H]
	\centering
	\includegraphics[width=.7\textwidth]{images/forms/tipos_de_perguntas_disponiveis/Imagem 41.png}
	\caption{Configuração de uma pergunta tipo "Horário", que exige que o participante insira um horário.}
\end{figure}

\begin{figure}[H]
	\centering
	\includegraphics[width=.7\textwidth]{images/forms/tipos_de_perguntas_disponiveis/Imagem 42.png}
	\caption{Visualização da pergunta de "Horário" no formulário do participante, exibindo os campos de entrada de horas e minutos.}
\end{figure}
   

