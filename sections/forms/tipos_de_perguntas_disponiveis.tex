% sections/forms/tipos_de_perguntas_disponiveis.tex
% !TeX root = ../../../main.tex

\section{Tipos de Perguntas Disponíveis}
\subsection{Resposta curta}

Essa categoria é utilizada para respostas rápidas, geralmente textos com
poucas palavras ou caracteres, como: nome, e-mail, número de matrícula,
e similares.

\begin{figure}[H]
	\centering
	\includegraphics[width=.7\textwidth]{images/forms/tipos_de_perguntas_disponiveis/Imagem 24.png}
\end{figure}

\subsection{Parágrafo}

Ao contrário da categoria anterior, costuma ser utilizado quando a
resposta precisa ser mais longa em extensão, como no caso da descrição
de um serviço ou objeto, opinião ou \textbf{feedback} mais detalhado.

\begin{figure}[H]
	\centering
	\includegraphics[width=.7\textwidth]{images/forms/tipos_de_perguntas_disponiveis/Imagem 25.png}
\end{figure}

\subsection{Múltipla escolha}

Fornece uma lista de opções de escolha. Somente uma pode ser
selecionada.

\begin{figure}[H]
	\centering
	\includegraphics[width=.7\textwidth]{images/forms/tipos_de_perguntas_disponiveis/Imagem 26.png}
\end{figure}

\subsection{Caixas de seleção}

Semelhante à categoria anterior, mas permite a seleção de mais de uma
opção entre as alternativas.

\begin{figure}[H]
	\centering
	\includegraphics[width=.7\textwidth]{images/forms/tipos_de_perguntas_disponiveis/Imagem 27.png}
\end{figure}

\subsection{Menu suspenso}

Muito parecido com a ``Múltipla Escolha'', porém, é apresentado em
formato de um menu suspenso, contendo as opções disponíveis para seleção
do respondente.

\begin{figure}[H]
	\centering
	\includegraphics[width=.7\textwidth]{images/forms/tipos_de_perguntas_disponiveis/Imagem 28.png}
\end{figure}

\begin{figure}[H]
	\centering
	\includegraphics[width=.7\textwidth]{images/forms/tipos_de_perguntas_disponiveis/Imagem 30.png}
\end{figure}

\subsection{Upload de arquivos}

Permite que o participante envie um arquivo, sendo ele um documento,
planilha, PDF, vídeo, apresentação, desenho, imagem ou áudio. Pode-se
limitar a quantidade de arquivos e o tamanho máximo do arquivo (limite
máximo de 1GB para a totalidade de arquivos anexos). É necessário que o
usuário esteja logado em sua conta Google para utilizar esse recurso.

\begin{figure}[H]
	\centering
	\includegraphics[width=.7\textwidth]{images/forms/tipos_de_perguntas_disponiveis/Imagem 31.png}
\end{figure}


\subsection{Escala linear}


Com essa função, o participante pode, por exemplo, avaliar algum serviço
ou atendimento ao atribuir uma nota presente na escala de 1 a 5, 1 a 10
ou 0 a 10, conforme determinado pelo criador do formulário.

\begin{figure}[H]
	\centering
	\includegraphics[width=.7\textwidth]{images/forms/tipos_de_perguntas_disponiveis/Imagem 32.png}
\end{figure}

\begin{figure}[H]
	\centering
	\includegraphics[width=.7\textwidth]{images/forms/tipos_de_perguntas_disponiveis/Imagem 33.png}
\end{figure}


\subsection{Classificação}

Semelhante ao último tipo de avaliação, porém com uma opção para
estilização dos ícones selecionáveis (``coração'', ``estrela'' ou
``polegar / positivo'').

\begin{figure}[H]
	\centering
	\includegraphics[width=.7\textwidth]{images/forms/tipos_de_perguntas_disponiveis/Imagem 34.png}
\end{figure}

\subsection{Grade de múltipla escolha}

Possibilita a criação de uma ``tabela'', onde, a cada linha criada,
pode-se atribuir um conjunto de opções para seleção de múltipla escolha
ao usuário (resposta única por linha).

\begin{figure}[H]
	\centering
	\includegraphics[width=.7\textwidth]{images/forms/tipos_de_perguntas_disponiveis/Imagem 36.png}
\end{figure}

\begin{figure}[H]
	\centering
	\includegraphics[width=.7\textwidth]{images/forms/tipos_de_perguntas_disponiveis/Imagem 37.png}
\end{figure}

\subsection{Grade de caixas de seleção}


Semelhante à categoria anterior, porém, permite que mais de uma opção
possa ser selecionada por linha.

\begin{figure}[H]
	\centering
	\includegraphics[width=.7\textwidth]{images/forms/tipos_de_perguntas_disponiveis/Imagem 38.png}
\end{figure}

\subsection{Data e Horário}

Essas opções permitem que o usuário insira uma data, que contempla dia,
mês e ano (formato \textbf{dd}/\textbf{mm}/\textbf{aaaa}), e/ou um horário,
contendo horas e minutos (formato \textbf{hh}:\textbf{mm}).

\begin{figure}[H]
	\centering
	\includegraphics[width=.7\textwidth]{images/forms/tipos_de_perguntas_disponiveis/Imagem 39.png}
\end{figure}

\begin{figure}[H]
	\centering
	\includegraphics[width=.7\textwidth]{images/forms/tipos_de_perguntas_disponiveis/Imagem 40.png}
\end{figure}

\begin{figure}[H]
	\centering
	\includegraphics[width=.7\textwidth]{images/forms/tipos_de_perguntas_disponiveis/Imagem 41.png}
\end{figure}

\begin{figure}[H]
	\centering
	\includegraphics[width=.7\textwidth]{images/forms/tipos_de_perguntas_disponiveis/Imagem 42.png}
\end{figure}
