% sections/forms/dicas_e_boas_praticas.tex
% !TeX root = ../../../main.tex

\section{Dicas e Boas Práticas}
\subsection{Organização e clareza das perguntas:}

Mantenha as questões objetivas, com enunciados simples e bem
estruturados. Evite termos técnicos ou muito específicos para que
qualquer pessoa compreenda facilmente o que está sendo solicitado.

\subsection{Testar o formulário antes do envio:}

Faça uma simulação respondendo o formulário como se fosse um
participante. Isso ajuda a identificar possíveis erros, questões mal
formuladas ou problemas de lógica que podem prejudicar a coleta de
respostas. O recurso ``Visualizar'', na aba superior, também é útil para
ter um vislumbre do resultado final durante o desenvolvimento do
formulário.

\subsection{Limitar número de respostas, quando necessário:}

Em alguns casos, é importante controlar a quantidade de respostas, com o
intuito de evitar duplicidades, restringir o acesso a um número
específico de participantes e garantir mais organização na análise dos
dados posteriormente.

\subsection{Garantir acessibilidade:}

Verifique se o formulário é compatível com diferentes dispositivos
(computador, tablet e celular) e se pode ser acessado por pessoas com
necessidades especiais, assim pode-se ampliar o alcance e assegurar a
participação de todos, com clareza e acessibilidade.
