% sections/forms/configuracoes_do_formulario.tex
% !TeX root = ../../../main.tex

\section{Configurações do Formulário}
\subsection{Coleta de e-mails:}

A configuração da coleta de e-mails na ferramenta pode ser configurada
no menu expansível ``Respostas'', dentro da guia ``Configurações'', que
está presente na aba superior do \textbf{Google Formulários}.

A coleta de e-mails possui 3 formatos: ``verificado'', onde o
participante deverá estar logado exclusivamente em sua conta Google para
submissão da resposta; ``entrada do participante'', na qual o
respondente pode inserir seu e-mail manualmente; e ``não coletar'',
desabilitando essa funcionalidade.

\begin{figure}[H]
	\centering
	\includegraphics[width=.8\textwidth]{images/forms/configuracoes_do_formulario/Imagem 7.png}
\end{figure}

\subsection{Permitir ou não edições após envio}

Na mesma aba ``Respostas'', o campo de seleção ``permitir edição das
respostas'' define se o respondente poderá ou não alterar suas respostas
depois de enviar o formulário. Quando esta opção está ativa, o usuário
recebe um link para revisar e corrigir suas respostas - uma alternativa
bastante utilizada em pesquisas ou cadastros que demandam atualizações
futuras. Quando está desativada, as respostas tornam-se definitivas e
inalteráveis, garantindo maior confiabilidade. Esta opção é
especialmente útil no caso de avaliações, inscrições e outros processos
que requerem mais formalidade.


\begin{figure}[H]
	\centering
	\includegraphics[width=.8\textwidth]{images/forms/configuracoes_do_formulario/Imagem 8.png}
\end{figure}

\subsection{Mostrar barra de progresso}

Encontra-se na aba ``Apresentação'' (abaixo de ``Respostas''). Ativar
essa opção faz com que, no rodapé do formulário, seja exibida uma linha
indicativa que informa quanto já foi respondido em relação ao total de
seções do formulário. É bastante útil para orientar o respondente, dando
noção de avanço e tempo necessário para concluir o preenchimento.


\begin{figure}[H]
	\centering
	\includegraphics[width=.8\textwidth]{images/forms/configuracoes_do_formulario/Imagem 9.png}
\end{figure}

\subsection{Embaralhar ordem das perguntas}

Na mesma guia ``Apresentação'', esta opção altera automaticamente a
sequência em que as questões aparecem para cada respondente. Esse
recurso é amplamente utilizado em avaliações, pois ajuda a evitar que
participantes copiem respostas uns dos outros, com base no ordenamento
de questões.

\begin{dica} 
  Esse recurso pode impactar diretamente na coleta de
  opiniões, já que a ordem das perguntas pode influenciar a forma com que
  elas são respondidas.
\end{dica} 


\begin{figure}[H]
	\centering
	\includegraphics[width=.8\textwidth]{images/forms/configuracoes_do_formulario/Imagem 10.png}
\end{figure}

\subsection{Transformar em teste (modo quiz)}

Ativar a opção ``Criar teste'' (acima da guia ``Respostas'') permite que
o formulário seja usado como uma avaliação. Desse modo, o criador pode
atribuir pontos às perguntas, definir respostas corretas e incluir
\textbf{feedback} automático para acertos e erros. ele é muito útil em
ambientes educacionais e treinamentos, pois possibilita a correção de
respostas imediata e o cálculo de notas automaticamente. Além disso,
pode-se escolher se o resultado deve ser exibido logo após o envio ou
posteriormente, mediante autorização do criador.


\begin{figure}[H]
	\centering
	\includegraphics[width=.8\textwidth]{images/forms/configuracoes_do_formulario/Imagem 11.png}
\end{figure}
