% sections/forms/personalizacao_do_formulario.tex
% !TeX root = ../../../main.tex

\section{Personalização do Formulário}
\subsection{Mudança de tema (cores, imagens de fundo):}

Dentro dos formulários, é possível alterar temas e personalizar,
escolhendo entre cores predefinidas ou atribuindo uma paleta de sua
preferência. Também é possível adicionar imagens de fundo para tornar o
visual mais atrativo e contextualizar o seu questionário. Também é
possível alterar fontes, tamanhos de letras e planos de fundo.

\begin{figure}[H]
	\centering
	\includegraphics[width=.6\textwidth]{images/forms/personalizacao_do_formulario/Imagem 1.png}
	\caption{Painel lateral de "Tema", que permite a personalização completa da aparência do formulário. Ele inclui opções para definir o Estilo de texto (Fonte e Tamanho) para Cabeçalho, Pergunta e Texto, escolher uma imagem para o cabeçalho e selecionar a Cor e o Plano de fundo do formulário.}
\end{figure}

\subsection{Layouts visuais e Imagens e Vídeos nas perguntas}

O Google Forms oferece layouts simples e intuitivos, mas é possível
organizar perguntas de forma personalizada utilizando seções e páginas,
criando uma navegação fluida e visualmente agradável, bem como vídeos e
outros tópicos interativos.

\begin{figure}[H]
	\centering
	\includegraphics[width=.7\textwidth]{images/forms/personalizacao_do_formulario/Imagem 2.png}
	\caption{Visualização da edição de uma pergunta de Múltipla Escolha, destacando (com um contorno vermelho) a Barra Lateral de Ferramentas. Essa barra contém os ícones para adicionar novos elementos ao formulário: Adicionar Pergunta (+), Importar perguntas, Adicionar Título e Descrição (Tt), Adicionar Imagem, Adicionar Vídeo e Adicionar Seção (duas barras paralelas).}
\end{figure}
