% sections/forms/introducao_ao_google_forms.tex
% !TeX root = ../../../main.tex

\section{Introdução ao Google Forms}
O \textbf{Google Forms} é um aplicativo da web, gratuito, desenvolvido
pela Google, que faz parte do pacote do \textbf{Google Docs Editors} e, de
acordo com o site TechTudo, é uma ferramenta que vai além da criação de
formulários tradicionais. Com ele, é possível montar pesquisas e
enquetes utilizando diferentes tipos de perguntas, como múltipla
escolha, escalas de avaliação e respostas abertas. Pode ser usado para
diversos fins, como organizar inscrições em eventos, coletar e-mails de
contato, aplicar questionários e até mesmo criar \textbf{quizzes}. Outro
ponto positivo é que ele possibilita o trabalho em equipe
(colaborativo), através da edição e criação de formulários
simultaneamente, por meio de conexão à internet, por múltiplos usuários.

Alguns de seus usos mais comuns estão voltados à realização de pesquisas
de satisfação e coleta de \textbf{feedbacks}, de modo a avaliar a
experiência de alguns consumidores com serviços, produtos e
atendimentos. Também é amplamente utilizado para cadastro em eventos,
registro de participantes, aplicações de provas e questionários
à distância, bem como em situações no ambiente de trabalho, como para
solicitar férias, aferir as pesquisas de clima e registrar chamados.
