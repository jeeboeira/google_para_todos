% sections/agenda/gerenciamento_entradas_no_calendario.tex
% !TeX root = ../../../main.tex

\section{Gerenciando Entradas no Calendário}
Este guia irá demonstrar o processo completo para compor uma marcação de evento. Para colocar isso em prática, usaremos a seguinte situação: ``Marcar uma reunião com colegas de trabalho da faculdade''. Acompanhe como criar o evento.

\subsection{Como criar um evento}
Na tela inicial, clique no botão ``Criar'' que se encontra na parte esquerda da tela. No menu expansível, selecione a opção ``Evento''.

\begin{figure}[H]
    \centering
    \includegraphics[width=.39\textwidth]{/agenda/gerenciamento_entradas_no_calendario/Imagem1.png}
    \caption{}
\end{figure}

Isso abrirá uma janela sobreposta, com os campos de título, data/hora do evento, etc. Preencha o campo ``Adicionar título'' para definir o título do evento. Em seguida, é necessário definir a data e hora do evento, que no exemplo é uma reunião, clique na região que mostra a data atual, isso irá abrir campos para seleção desses dados. Para selecionar ou digitar os valores desejados, clique nas caixas de seleção de data e  hora de início e fim.

\begin{figure}[H]
    \centering
    \includegraphics[width=.39\textwidth]{/agenda/gerenciamento_entradas_no_calendario/Imagem2.png}
    \caption{Evento com os valores definidos}
\end{figure}

Alternativamente, pode-se definir o evento com duração para o dia inteiro, basta clicar na caixa de seleção ``Dia inteiro''. Ao marcar essa caixa, serão ocultados os seletores de data e horário de início e fim.

Com os dados básicos definidos, já se pode concluir a criação do evento clicando no botão ``Salvar''. A realização dessa ação fechará a janela de cadastro de evento e retornará à tela principal. O evento recém criado poderá ser visto se visualização configurado abranger a data e hora dele.


\subsubsection{Definindo repetição de evento}
Se o evento que está sendo criado ou editado se repete periodicamente, existe a possibilidade de defini-lo para tal. Para marcar a repetição de um evento, clique no menu suspenso com o texto ``Não se repete''. Observe que abrem-se diversas opções para repetição de evento. Como no exemplo apresentado a situação que mais se ajusta é a marcação de reuniões semanais, foi selecionada a opção ``Semanal''.

\begin{figure}[H]
    \centering
    \includegraphics[width=.39\textwidth]{/agenda/gerenciamento_entradas_no_calendario/Imagem3.png}
    \caption{}
\end{figure}

Com essa opção selecionada, se salvarmos o evento da forma que está agora, serão criadas reuniões todas às quintas-feiras no horário definido, mas sem um fim determinado para a repetição.


\paragraph{Definindo repetição de evento com fim determinado}
Para determinar o fim da repetição de um evento, clique no menu suspenso usado para definir a repetição e selecione a opção ``Personalizar''. Isso abrirá a janela ``Recorrência personalizada'', onde poderá se definir melhor a repetição do evento. No exemplo escolhido, as reuniões serão todas as quintas-feiras do mês de Setembro de 2025, portanto selecionamos o dia 25 de Setembro como data final das reuniões.

\begin{figure}[H]
    \centering
    \includegraphics[width=.39\textwidth]{/agenda/gerenciamento_entradas_no_calendario/Imagem4.png}
    \caption{}
\end{figure}

Com a recorrência definida, clique em ``Concluir'' na janela de ``Recorrência personalizada''.


\subsubsection{Adicionar convidados}
Para adicionar convidados, basta clicar na opção ``Adicionar convidados'' e digitar o nome ou e-mail de quem você deseja convidar. Enquanto estiver digitando, o sistema irá te fornecer sugestões, se encontrar quem quiser convidar na lista exibida, é só clicar no nome do convidado que ele será adicionado ao evento.

\begin{figure}[H]
    \centering
    \includegraphics[width=.39\textwidth]{/agenda/gerenciamento_entradas_no_calendario/Imagem5.png}
    \caption{}
\end{figure}

Note que quando adicionar o primeiro convidado ao evento, o Google Agenda já irá pré-definir uma reunião pelo Google Meet. Mais a frente comentaremos sobre configuração de videoconferência.


\paragraph{Permissões de convidados}
Logo abaixo da lista de convidados, temos a opção ``Permissões de convidados'', clicando nela aparecerão caixas de seleção com as quais podemos definir o que os convidados podem fazer com o evento. No exemplo, foi removida a permissão ``Convidar outras pessoas'' e mantida a opção ``Ver lista de convidados''.


\subsubsection{Configurando videoconferência do Google Meet}
Logo abaixo das permissões de convidados temos a seção de configurações de reunião pelo Google Meet. No lado direito desta seção temos 4 (quatro) botões, em ordem da esquerda para a direita: 

\begin{itemize}
    \item Copiar informações da videoconferência: Copia as informações da videoconferência para que você as compartilhe por Whatsapp, por exemplo.
    \item Opções da videochamada: Abre uma janela para configurar as opções da videoconferência. As opções são autoexplicativas, portanto não entraremos em detalhes de cada uma delas.
    \item Mais detalhes da videoconferência: Exibe informações extras sobre a videoconferência.
    \item Remover conferência: Remove as configurações da videoconferência, como o exemplo é uma reunião presencial, essa opção foi usada.
\end{itemize}


\subsubsection{Adicionar local de evento}
Para adicionar o local do evento, clique na opção ``Adicionar local'' e digite o nome ou endereço do local do evento. Veja que, tendo um local selecionado, será exibido um botão ``Visualizar no Maps'', permitindo que você veja o local selecionado através da aplicação Google Maps.

\begin{figure}[H]
    \centering
    \includegraphics[width=.39\textwidth]{/agenda/gerenciamento_entradas_no_calendario/Imagem6.png}
    \caption{}
\end{figure}


\subsubsection{Adicionar descrição}
Para adicionar uma descrição, clique na opção ``Adicionar descrição ou um anexo do Google Drive'', isso abre uma caixa de texto com opções de formatação. Digite nessa caixa a descrição que deseja dar ao evento.


\paragraph{Adicionar ata da reunião}
Se o evento necessitar de uma ata, temos a opção de criar uma a partir do próprio evento, por meio da opção ``Criar ata da reunião''. Caso você tenha clicado por engano na opção, você pode remover a ata clicando no ícone ``X'' que se encontra ao lado do nome do arquivo da ata.


\paragraph{Adicionar anexo do Google Drive}
Para adicionar um anexo do Google Drive ao evento, basta clicar na opção ``Adicionar um anexo do Google Drive''. Isso abrirá uma janela para que você selecione o arquivo desejado. Para mais informações sobre como operar o Google Drive, leia o capítulo específico que trata dessa ferramenta.

Com os itens selecionados, clique no botão inserir.

Caso tenha anexado algum arquivo por engano e queira removê-lo dos anexos, clique no ícone ``X'' ao lado do nome do anexo que pretende remover.

\begin{figure}[H]
    \centering
    \includegraphics[width=.39\textwidth]{/agenda/gerenciamento_entradas_no_calendario/Imagem7.png}
    \caption{}
\end{figure}


\subsubsection{Configurações extras}
Para acessar configurações de disponibilidade, visibilidade e notificações, clique na área da janela logo abaixo da seção onde se configura a descrição do evento. Geralmente essa seção possui o seu nome, conforme a imagem a seguir.

\begin{figure}[H]
    \centering
    \includegraphics[width=.39\textwidth]{/agenda/gerenciamento_entradas_no_calendario/Imagem8.png}
    \caption{}
\end{figure}

Essa seção possui as seguintes opções, apresentadas na próxima sequência de subitens.


\paragraph{Ícone de pasta --- Disponibilidade}
Tem a finalidade de definir sua disponibilidade durante o evento. No caso, pode-se dizer se estará disponível ou ocupado.


\paragraph{Ícone de cadeado --- Visibilidade do evento}
Serve para definir o nível de visibilidade do evento. Ao selecionar diferentes opções deste menu, o ícone de ajuda (ponto de interrogação dentro de um círculo) abrirá um balão de dica (tooltip) sobre a opção selecionada.


\paragraph{Ícone de sino --- Notificações}
Podem-se configurar notificações para se lembrar do evento. Por padrão, o Google Agenda emite uma notificação com 10 minutos de antecedência ao horário marcado. Para mudar o horário da primeira notificação, clique no menu suspenso que contém o texto ``10 minutos antes''.

Há a possibilidade de personalizar a notificação selecionando a opção ``Personalizar\ldots'', a qual abre uma janela com mais configurações para a notificação, como a forma de entrega dela e o tempo de antecedência ao evento. 

Podemos configurar mais notificações clicando no botão ``Adicionar notificação''. 

Para remover uma notificação, arraste o mouse para cima da linha da notificação que deseja excluir até que apareça um ícone ``X''. Clique neste ícone para remover a notificação.


\subsubsection{Botão ``Mais opções''}
Para acessar mais opções de criação de agenda, clique no botão ``Mais opções''.

\begin{figure}[H]
    \centering
    \includegraphics[width=.39\textwidth]{/agenda/gerenciamento_entradas_no_calendario/Imagem9.png}
    \caption{}
\end{figure}

Ao clicar nesse botão, a janela sobreposta irá ocupar a página inteira, dando uma visão geral do evento. Essa visualização também permite que se alterem quaisquer campos e opções mencionados anteriormente.

\begin{figure}[H]
    \centering
    \includegraphics[width=.39\textwidth]{/agenda/gerenciamento_entradas_no_calendario/Imagem10.png}
    \caption{}
\end{figure}


\subsubsection{Aba ``Encontrar um horário''}
Essa aba permite que tenhamos uma visão de onde o evento se encaixa no seu cronograma. Também pode-se ver eventos dos cronogramas dos convidados caso estes não estejam privados.


\subsubsection{Concluir criação de evento}
A qualquer momento após ter definido o título do evento, clique no botão ``Salvar'' para criar o evento.

\begin{figure}[H]
    \centering
    \includegraphics[width=.39\textwidth]{/agenda/gerenciamento_entradas_no_calendario/Imagem11.png}
    \caption{}
\end{figure}

\begin{figure}[H]
    \centering
    \includegraphics[width=.39\textwidth]{/agenda/gerenciamento_entradas_no_calendario/Imagem12.png}
    \caption{}
\end{figure}

Se adicionar convidados ao evento, uma janela de diálogo vai pedir se pretendes enviar e-mails de convite. Selecione a opção desejada.

\begin{figure}[H]
    \centering
    \includegraphics[width=.39\textwidth]{/agenda/gerenciamento_entradas_no_calendario/Imagem13.png}
    \caption{}
\end{figure}

Após a criação do evento, o mesmo será exibido na agenda.


\subsection{Como criar uma tarefa}
Este guia irá demonstrar o processo completo para compor uma marcação de tarefa. Para colocar isso em prática, usaremos a seguinte situação: ``Marcar um lembrete de ir ao mercado''. Acompanhe como criar a tarefa:

Na tela inicial, clique no botão ``Criar'' que se encontra na parte esquerda da tela. No menu expansível, selecione a opção ``Tarefa''.

\begin{figure}[H]
    \centering
    \includegraphics[width=.39\textwidth]{/agenda/gerenciamento_entradas_no_calendario/Imagem14.png}
    \caption{}
\end{figure}

Isso abrirá uma janela sobreposta muito semelhante à de criação de agenda, embora com menos opções.

Preencha o campo ``Adicionar título'' para definir o título da tarefa. Em seguida, é necessário definir a data e hora da tarefa, que no exemplo é um lembrete. Clique na região que mostra a data atual, isso irá abrir campos para seleção desses dados. Para selecionar ou digitar os valores desejados, clique nas caixas de seleção de data e de hora de início e fim.

\begin{figure}[H]
    \centering
    \includegraphics[width=.39\textwidth]{/agenda/gerenciamento_entradas_no_calendario/Imagem15.png}
    \caption{}
\end{figure}

Alternativamente, pode-se definir a tarefa para o dia inteiro, basta clicar na caixa de seleção ``Dia inteiro''. Marcando essa caixa, serão ocultados os seletores de data e horário. Com esses dados básicos definidos, já se pode concluir a criação dessa tarefa clicando no botão ``Salvar''.


\subsubsection{Definindo repetição de tarefa}
Como nos eventos, também há a possibilidade de configurar recorrência para tarefas. As opções disponíveis são as mesmas encontradas na criação de eventos.


\subsubsection{Adicionar descrição}
Para adicionar uma descrição à tarefa, basta clicar na caixa de texto onde consta ``Adicionar uma descrição'' e digitar a descrição desejada.


\subsubsection{Selecionar a lista de tarefas}
Logo abaixo da descrição da tarefa tem um menu suspenso com o texto ``Minhas tarefas''. Essa é a lista de tarefas padrão do Google Agenda. Se você já tiver outra lista de tarefas cadastrada, pode selecioná-la usando este menu.


\subsubsection{Concluir criação de tarefa}
Com os campos preenchidos, clique no botão ``Salvar'' para criar a tarefa. Ao clicar nesse botão, a tarefa aparecerá em sua agenda.

\begin{figure}[H]
    \centering
    \includegraphics[width=.39\textwidth]{/agenda/gerenciamento_entradas_no_calendario/Imagem16.png}
    \caption{}
\end{figure}


\subsection{Como criar um agendamento de horário.}
O presente guia tem por objetivo demonstrar o processo de configuração de agendamentos de horário com reserva. O exemplo utilizado para ilustrar essa função será o da configuração de agendamentos de um médico. Abaixo, seguem os passos necessários para se fazer tal configuração. 

Na tela inicial, clique no botão criar e no menu suspenso clique em agendamento de horários. 

\begin{figure}[H]
    \centering
    \includegraphics[width=.39\textwidth]{/agenda/gerenciamento_entradas_no_calendario/Imagem17.png}
    \caption{}
\end{figure}

Fazendo isso, será aberto um painel à esquerda da tela para a configuração do cronograma de agendamentos. Note que não há todas as opções visíveis nesta etapa, como indicado pelo botão ``Avançar'' no final desse painel. Será discutido brevemente sobre cada uma das seções de ambas as partes deste painel.


\subsubsection{Título do Agendamento e Duração dos Horários}
Primeiro, preencha o campo ``Adicionar título'' para poder identificar a grade de agendamento. Em seguida, tem-se a opção de duração dos horários. Clicando sobre o menu suspenso, abrem-se mais opções de duração, com a possibilidade de personalizar a duração, caso necessário.


\subsubsection{Disponibilidade Geral e Janela de Programação}
Em seguida, temos a seção ``Disponibilidade geral''. Nela é definido em quais dias da semana haverá disponibilidade de agendamento, bem como a recorrência da grade que está sendo configurada. No exemplo, o médico não poderá atender a partir das 11:30 nas terças e quintas-feiras, sendo necessário configurar o ajuste como na imagem a seguir.

\begin{figure}[H]
    \centering
    \includegraphics[width=.39\textwidth]{/agenda/gerenciamento_entradas_no_calendario/Imagem18.png}
    \caption{}
\end{figure}

Logo abaixo, temos a seção ``Janela de programação''. É nela onde se definem início e fim de disponibilidade de reserva, bem como tempos máximo e mínimo de antecedência para marcação de reserva. 


\subsubsection{Disponibilidade Ajustada}
Após configurar a janela de agendamento, temos a configuração de ``Disponibilidade ajustada''. Nela podem ser adicionados ajustes na grade de horários em dias específicos.


\subsubsection{Configurações de Eventos Agendados}
Essa seção conta com opções para configurar intervalo entre os horários, número máximo de agendamentos por dia e permissões para que os convidados possam convidar outras pessoas.


\subsubsection{Verificação de Disponibilidade}
Ao configurar uma grade de horários de agendamento, na sessão ``Agendas'', pode-se deixar horários de agendamento indisponíveis caso estes coincidam com marcações de eventos nas agendas pertencentes a conta Google que se está realizando a configuração. Essa opção vem marcada por padrão, mas pode ser desmarcada, se assim o desejar.


\subsubsection{Definindo Coorganizadores}
Ao organizar uma grade de agendamentos, ao final do painel de configuração, há uma seção denominada ``Coorganizadores''. É por meio dela que se pode gerir quais outros usuários também podem editar a grade.


\subsubsection{Foto e nome da página de agendamento}
Esta seção se torna visível após clicar sobre o botão ``Avançar'' na etapa inicial de criação de grade de agendamentos.

Nela pode-se visualizar como será exibida a sua identidade na página de agendamento.


\subsubsection{Local e videoconferência}
Nesta seção é definido o meio pelo qual o atendimento será feito: se será remoto ou presencial. Para definir essa opção, clique no menu suspenso ``Selecionar como e onde fazer uma reunião''.


\subsubsection{Descrição}
Há a possibilidade de adicionar uma descrição para a grade de agendamentos. Com ela, pode-se descrever melhor o serviço/atendimento prestado.


\subsubsection{Personalizar formulário de reserva}
Logo após a seção de descrição da grade, você pode personalizar o formulário de reserva. Por padrão, o Google Agenda pré-configura os campos ``Nome'', ``Sobrenome'' e ``Endereço on-line'', os quais são obrigatórios e não podem ser alterados. Para adicionar itens, basta clicar no botão ``Adicionar um item''. Há também a possibilidade de habilitar obrigatoriedade de verificação de e-mail.


\subsubsection{Lembretes e confirmações}
Nesta seção é onde se configuram os e-mails de lembrete. Por padrão, o Google Agenda pré-configura um e-mail para um dia antes do agendamento. Podem ser adicionados mais lembretes conforme a necessidade.


\subsubsection{Salvando a grade de agendamento}
Para concluir a criação da grade, clique no botão ``Salvar'' que se encontra ao final do painel depois de clicar no botão ``Avançar''. Ao clicar nesse botão, o painel de configuração será fechado e, em seguida, a sua visualização da agenda será atualizada, exibindo os dias de agendamento com uma faixa de destaque. À esquerda da tela será exibido uma caixa com um resumo da grade de agendamentos recém-criada, conforme imagem a seguir.

\begin{figure}[H]
    \centering
    \includegraphics[width=.39\textwidth]{/agenda/gerenciamento_entradas_no_calendario/Imagem19.png}
    \caption{}
\end{figure}


\subsection{Como configurar um evento de ausência}
A seguir será demonstrado como criar uma evento de ausência no Google Agenda. O exemplo que será usado é um evento de ausência por causa de férias. Confira a seguir os passos para criar esse tipo de registro:

Na tela inicial do Google Agenda, clique no botão ``Criar'' e no menu suspenso, clique na opção ``Ausente''. 

\begin{figure}[H]
    \centering
    \includegraphics[width=.39\textwidth]{/agenda/gerenciamento_entradas_no_calendario/Imagem20.png}
    \caption{Acessado a criação de evento de ausência}
\end{figure}

Essa ação abrirá uma janela para configuração do evento de ausência.


\subsubsection{Título, período da ausência, recorrência e recusa de reuniões}
Como na configuração de eventos, aqui também há configurações de título, duração e recorrência nesse tipo de entrada da agenda. 

Note que o período de ausência pode abranger tanto um único dia, em horário específico, quanto vários dias inteiros. Por padrão, o Google Agenda inicia o evento com a data do dia seguinte, abrangendo o dia inteiro. Como exemplo, será marcada uma semana para representar o período de férias do dono da agenda.

\begin{figure}[H]
    \centering
    \includegraphics[width=.39\textwidth]{/agenda/gerenciamento_entradas_no_calendario/Imagem21.png}
    \caption{}
\end{figure}

Veja que logo abaixo da configuração de recorrência há opção de recusar reuniões de forma automática, selecione as opções de acordo com o desejado.


\subsubsection{Mensagem do evento de ausência}
Logo abaixo das configurações de recusa de reuniões, há a configuração da mensagem do evento. Use esse campo para deixar um breve recado do motivo da ausência.


\subsubsection{Visibilidade da ausência}
Seguindo a definição da mensagem, há a configuração do nível de visibilidade do evento de ausência, marcado com um ícone de cadeado. Ao lado direito do menu suspenso desta opção, encontra-se um ícone de ajuda (ponto de exclamação dentro de um círculo). 

Ao passar o mouse em cima dele, é exibido um ``tooltip'', que descreve o efeito de cada opção do menu.


\subsubsection{Salvando evento de ausência}
Clique no botão ``Salvar'' para concluir a criação do evento de ausência de acordo com as configurações selecionadas. Realizar essa ação exibirá uma caixa de diálogo pedindo se confirma recusa automática conforme configurado, como exemplificado na imagem abaixo:

\begin{figure}[H]
    \centering
    \includegraphics[width=.39\textwidth]{/agenda/gerenciamento_entradas_no_calendario/Imagem22.png}
    \caption{}
\end{figure}

Clicar no botão ``Salvar e recusar'' fecha a janela de configuração e exibe na agenda as marcações de ausência, caso o período configurado coincida com o da visualização da tela antes de iniciar o processo de criação do evento.

\begin{figure}[H]
    \centering
    \includegraphics[width=.39\textwidth]{/agenda/gerenciamento_entradas_no_calendario/Imagem23.png}
    \caption{Aviso de férias criado}
\end{figure}


\subsection{Como configurar um evento ``Hora de se concentrar''}
Esse tipo de evento serve para avisar às outras pessoas que você estará indisponível e não aceitará interrupções durante o período marcado. Para criar um evento de ``Hora de se concentrar'', na tela inicial do Google Agenda, clique no botão ``Criar'' e, em seguida, em ``Hora de se concentrar''. Essa ação abrirá uma janela como a da imagem a seguir.

\begin{figure}[H]
    \centering
    \includegraphics[width=.39\textwidth]{/agenda/gerenciamento_entradas_no_calendario/Imagem24.png}
    \caption{}
\end{figure}


\subsubsection{Título, data, duração e recorrência}
As mesmas configurações de título, data, duração e recorrência encontradas na marcação de eventos genéricos também aparecem na criação deste tipo de evento. Para mais informações, leia as seções que tratam destas configurações no guia ``Como criar um evento''.


\subsubsection{Silenciar notificações de chat}
Logo abaixo das opções de data, duração e recorrência, há a opção ``Não perturbe'' a qual permite bloquear notificações de chat do Google Workspace. Essa opção é útil caso não queira ser interrompido por alguma mensagem do chat do Google.


\subsubsection{Recusa automática de reuniões}
Esta opção já foi explorada anteriormente no guia ``Como configurar um evento de ausência''. Leia o item correspondente para mais explicações sobre esta configuração.


\subsubsection{Adicionar local}
Há, também, a possibilidade de adicionar um local para este tipo de evento. Para mais informações, leia o item que trata desta configuração no guia ``Como criar um evento''.


\subsubsection{Adicionar descrição ou um anexo do Google Drive}
Esta opção é idêntica à encontrada no guia ``Como criar um evento'', confira o item correspondente para mais informações.


\subsubsection{Configurações extras}
Para acessar configurações de visibilidade e notificações, clique na seção da janela logo abaixo da seção onde se configura a descrição do evento. A disponibilidade estará fixa como ``Ocupado''. Para mais detalhes sobre estas configurações, recorra ao item encontrado no guia ``Como criar um evento''.


\subsubsection{Salvando o evento de ``Hora de se concentrar''}
Com as configurações feitas, clique no botão ``Salvar''. 

Observação: Caso tenha marcado para recusar reuniões automaticamente, aparecerá uma caixa de diálogo pedindo se realmente deseja fazer a recusa automática. Clique em ``Salvar e recusar'' para confirmar e concluir a criação do evento.

Com a criação do evento concluída, o mesmo será exibido na agenda caso sua data e hora coincidam com o período selecionado em tela, como observado na imagem a seguir.

\begin{figure}[H]
    \centering
    \includegraphics[width=.39\textwidth]{/agenda/gerenciamento_entradas_no_calendario/Imagem25.png}
    \caption{}
\end{figure}


\subsection{Como configurar um local de trabalho}
Para configurar um local de trabalho, a partir da tela inicial do Google Agenda, clique no botão ``Criar'' e, em seguida, na opção ``Local de trabalho''. Essa ação exibirá uma janela como a mostrada na imagem a seguir.

\begin{figure}[H]
    \centering
    \includegraphics[width=.39\textwidth]{/agenda/gerenciamento_entradas_no_calendario/Imagem26.png}
    \caption{}
\end{figure}


\subsubsection{Configurar Período, Horário e Recorrência}
Pode-se configurar estas opções da mesma forma que outros tipos de eventos. Leia o exposto no guia ``Como criar um evento'' para mais detalhes.


\subsubsection{Escolher local}
Por padrão, o Google Agenda já disponibiliza dois locais: ``Casa'' e ``Escritório''. Para mais opções, clique no botão ``Outros locais'' e selecione a opção desejada no menu suspenso.


\subsubsection{Salvar local de trabalho}
Com as opções selecionadas, clique no botão ``Salvar'' para guardar as alterações. A realização dessa ação fechará a janela de configuração e exibirá o local de trabalho na agenda conforme exemplificado na imagem a seguir.

\begin{figure}[H]
    \centering
    \includegraphics[width=.39\textwidth]{/agenda/gerenciamento_entradas_no_calendario/Imagem27.png}
    \caption{}
\end{figure}


\subsection{Excluir eventos da agenda}
Nas seções a seguir, haverão instruções envolvendo exclusão temporária, recuperação e exclusão definitiva de entradas da agenda.


\subsubsection{Exclusão de registros}
Para excluir qualquer registro da agenda, clique no registro desejado. Essa ação fará com que seja exibida uma janela com um resumo sobre o evento selecionado, conforme a imagem abaixo.

\begin{figure}[H]
    \centering
    \includegraphics[width=.39\textwidth]{/agenda/gerenciamento_entradas_no_calendario/Imagem28.png}
    \caption{}
\end{figure}

Dentro dessa janela, clique no ícone de lixeira. 

\begin{dica}
Se o evento for recorrente, será exibida uma janela de diálogo. Selecione a opção desejada e clique em ``OK''.
\end{dica}


\subsubsection{Como acessar a lixeira}
Para acessar a lixeira, a partir da tela inicial do Google Agenda, abra o menu de Configurações (ícone de engrenagem na parte superior da página) e clique na opção ``Lixeira''.

\begin{figure}[H]
    \centering
    \includegraphics[width=.39\textwidth]{/agenda/gerenciamento_entradas_no_calendario/Imagem29.png}
    \caption{Caminho para acessar a lixeira}
\end{figure}


\subsubsection{Como recuperar eventos da lixeira}
Para recuperar eventos da lixeira, primeiro acesse-a e a partir deste ponto há algumas formas de recuperação:


\paragraph{Recuperar um único evento}
Na linha do evento que se deseja recuperar, clique no ícone de seta torcida para a esquerda, conforme a imagem a seguir.

\begin{figure}[H]
    \centering
    \includegraphics[width=.39\textwidth]{/agenda/gerenciamento_entradas_no_calendario/Imagem30.png}
    \caption{Observe o ícone em destaque: clicar neste ícone na linha do registro desejado faz com que ele seja restaurado}
\end{figure}


\paragraph{Recuperar múltiplos eventos}
Para recuperar mais de um evento, selecione os eventos desejados usando as caixas de seleção que se encontram na parte esquerda da lista. Se preferir restaurar todos, há uma caixa de seleção na linha de cabeçalho da lista que permite selecionar todos os itens. Selecionar qualquer um dos itens exibe logo acima da lista um contador de itens selecionado, acompanhado de botões para restaurar e excluir. Clique no botão de restaurar para recuperar os itens selecionados.

\begin{figure}[H]
    \centering
    \includegraphics[width=.39\textwidth]{/agenda/gerenciamento_entradas_no_calendario/Imagem31.png}
    \caption{Exemplo de restauração de múltiplos itens. Clicar no ícone em destaque restaura os itens selecionados}
\end{figure}


\subsubsection{Como excluir eventos definitivamente}
Para excluir definitivamente eventos da lixeira, primeiro acesse-a. A partir deste ponto, há algumas formas de exclusão:


\paragraph{Excluir um único evento}
O processo é similar ao de recuperação de um único evento, basta clicar no ícone de lixeira na linha do evento que se deseja excluir.


\paragraph{Excluir múltiplos eventos}
O processo é o mesmo da restauração de múltiplos eventos, com a diferença de que o ícone a ser clicado é o da lixeira.


\paragraph{Esvaziar lixeira}
Clique no botão ``Esvaziar lixeira''. Essa ação exibirá uma janela de diálogo pedindo se realmente deseja fazer isso. Clique em ``Esvaziar'' para concluir o processo ou em ``Cancelar'' se deseja manter os itens.

\begin{figure}[H]
    \centering
    \includegraphics[width=.39\textwidth]{/agenda/gerenciamento_entradas_no_calendario/Imagem32.png}
    \caption{Caixa de diálogo exibida ao clicar no botão ``Esvaziar lixeira''}
\end{figure}