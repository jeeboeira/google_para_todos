% sections/agenda/dicas_e_boas_praticas.tex
% !TeX root = ../../../main.tex

\section{Dicas e boas práticas do Google Agenda}
A utilização de cores diferentes para categorizar eventos pode tornar mais fácil a sua visualização. Pode-se utilizá-las para marcar diferentes tipos de compromissos, como, por exemplo: estudo, pessoal, trabalho, lazer, entre outros.

Utilizar mais de uma agenda para sua rotina pessoal: uma apenas para tarefas do trabalho, outra para uso pessoal, por exemplo, é outra prática que pode te ajudar a se organizar de maneira mais prática.

Adicione descrições relevantes aos eventos, como, por exemplo, um endereço, um link, um documento que será necessário para o compromisso.

Defina lembretes automáticos para os eventos, habilitando notificações por e-mail ou \textit{push} para lembrar-se dos seus compromissos com antecedência.

Para eventos que ocorrem com determinada frequência, evite criar cópias quase idênticas, agen/de-os como “recorrente”.

Realizar a integração do Google Agenda com o Gmail. Assim, quando receber e-mails de confirmação de reuniões, viagens e outros, o Google Agenda irá adicioná-los automaticamente.

Definir os horários nos quais estará disponível. Isso auxiliará outras pessoas que precisam agendar eventos para conferir a disponibilidade de seus convidados.

É possível compartilhar a agenda com outras pessoas para verificar sua disponibilidade. Usufrua deste recurso.

A função “Encontrar um horário” facilita a agenda de um evento com várias pessoas para evitar conflitos com outros encontros.

Ajuste o que as outras pessoas têm acesso para verificar na sua agenda: é possível personalizar as permissões, podendo mostrar apenas quando o usuário está ocupado ou somente os detalhes dos eventos, por exemplo.

É possível realizar pesquisas avançadas para encontrar eventos específicos, mesmo com o calendário cheio. O usuário pode definir critérios de busca por locais, datas, criador de evento, além de poder combinar filtros para uma busca mais precisa.


\subsection{comandos}
Ao ativar a opção de atalhos de teclado nas configurações do Google Agenda, é possível executar ações de maneira rápida e prática. Comandos de atalho simplificam a realização de tarefas como: criar novos eventos, alternar entre as visualizações de dia, semana ou mês, navegar pelos próximos períodos e até o acesso à busca.

\textbf{Principais atalhos:}\newline
\begin{itemize}
    \item \tecla{j} ou \tecla{n} → Alterar a visualização da agenda para o período seguinte;\newline
    \item \tecla{r} → Atualizar a agenda;\newline
    \item \tecla{t} → Ir para o dia atual;\newline
    \item \tecla{+} → Ir para a seção “Adicionar uma agenda”;\newline
    \item \tecla{/} → Posicionar o cursor na caixa de pesquisa;\newline
    \item \tecla{s} → Ir para a página “Configurações”;\newline
    \item \tecla{g} → Ir para uma data específica;\newline
    \item \tecla{1} ou \tecla{d} → Visualização de dia;\newline
    \item \tecla{2} ou \tecla{w} → Visualização de semana;\newline
    \item \tecla{3} ou \tecla{m} → Visualização de mês;\newline
    \item \tecla{4} + \tecla{x} → Visualização personalizada;\newline
    \item \tecla{5} + \tecla{a} → Visualização de compromisso;\newline
    \item \tecla{c} → Criar um novo evento;\newline
    \item \tecla{e} → Ver os detalhes de um evento;\newline
    \item \tecla{Backaspace} ou \tecla{Delete} → Excluir eventos;\newline
    \item \tecla{z} → Desfazer;\newline
    \item \tecla{Ctrl} + \tecla{s} → Salvar o evento a partir da página detalhes;\newline
    \item \tecla{Esc} → Voltar da página de detalhes do evento para a grade da agenda;\newline
\end{itemize}
