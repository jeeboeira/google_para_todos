% sections/agenda/dicas_e_boas_praticas.tex
% !TeX root = ../../../main.tex

\section{Dicas e boas práticas do Google Agenda}

\begin{dica}
\begin{itemize}
    \item Utilize cores diferentes para categorizar eventos, facilitando a visualização. Por exemplo: estudo, pessoal, trabalho e lazer;
    \item Considere usar mais de uma agenda para sua rotina: uma específica para tarefas de trabalho e outra para compromissos pessoais;
    \item Adicione descrições relevantes aos eventos, como endereços, links, documentos importantes ou orientações;
    \item Defina lembretes automáticos para seus eventos, ativando notificações por e-mail ou \textit{push} para lembrar-se com antecedência;
    \item Para compromissos recorrentes, utilize a funcionalidade de eventos recorrentes, evitando duplicar manualmente eventos semelhantes;
    \item Integre o Google Agenda ao Gmail para adicionar automaticamente eventos de confirmações recebidas por e-mail;
    \item Ajuste e defina seus horários disponíveis para facilitar o agendamento de compromissos por outras pessoas;
    \item Compartilhe sua agenda com quem necessita verificar sua disponibilidade, fazendo uso desse recurso colaborativo;
    \item Utilize a função “Encontrar um horário” para marcar eventos com várias pessoas, evitando conflitos de agenda;
    \item Ajuste as permissões de visualização da sua agenda, permitindo mostrar apenas disponibilidade ou detalhes completos, conforme sua preferência;
    \item Utilize a busca avançada para localizar eventos específicos em agendas grandes, filtrando por local, data, criador ou outros critérios.
\end{itemize}
\end{dica}

\subsection{Comandos}
Ao ativar a opção de atalhos de teclado nas configurações do Google Agenda, é possível executar ações de maneira rápida 
e prática. Comandos de atalho simplificam a realização de tarefas como: criar novos eventos, alternar entre as 
visualizações de dia, semana ou mês, navegar pelos próximos períodos e até o acesso à busca. \newline

\textbf{Principais atalhos:}\newline
\begin{itemize}
    \item \tecla{j} ou \tecla{n} → Alterar a visualização da agenda para o período seguinte;
    \item \tecla{r} → Atualizar a agenda;
    \item \tecla{t} → Ir para o dia atual;
    \item \tecla{+} → Ir para a seção “Adicionar uma agenda”;
    \item \tecla{/} → Posicionar o \gls{cursor} na caixa de pesquisa;
    \item \tecla{s} → Ir para a página “Configurações”;
    \item \tecla{g} → Ir para uma data específica;
    \item \tecla{1} ou \tecla{d} → Visualização de dia;
    \item \tecla{2} ou \tecla{w} → Visualização de semana;
    \item \tecla{3} ou \tecla{m} → Visualização de mês;
    \item \tecla{4} + \tecla{x} → Visualização personalizada;
    \item \tecla{5} + \tecla{a} → Visualização de compromisso;
    \item \tecla{c} → Criar um novo evento;
    \item \tecla{e} → Ver os detalhes de um evento;
    \item \tecla{Backaspace} ou \tecla{Delete} → Excluir eventos;
    \item \tecla{z} → Desfazer;
    \item \tecla{Ctrl} + \tecla{s} → Salvar o evento a partir da página detalhes;
    \item \tecla{Esc} → Voltar da página de detalhes do evento para a grade da agenda.
\end{itemize}
