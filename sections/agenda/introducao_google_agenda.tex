% sections/agenda/introducao_google_agenda.tex
% !TeX root = ../../../main.tex

\section{Introdução ao Google Agenda}
O Google Agenda é uma ferramenta de gerenciamento de tempo e agendamento de compromissos, lançada pelo Google em 2006. Desde então, consolidou-se como uma das plataformas mais utilizadas em seu segmento, graças à sua versatilidade, acessibilidade e eficiência na organização de atividades pessoais e profissionais.

Mais do que um simples calendário digital, o Google Agenda funciona como um assistente pessoal inteligente. Um de seus principais diferenciais é a integração com outros serviços do ecossistema Google (\cite{GoogleAgendaBaseConhecimento}). O Gmail, por exemplo, permite a criação automática de eventos a partir de e-mails recebidos, enquanto o Google \Gls{meet} gera links de videoconferência diretamente nos compromissos agendados. Essas funcionalidades ampliam significativamente a utilidade e a praticidade da plataforma.

Além disso, o Google Agenda oferece recursos avançados, como o gerenciamento de múltiplas agendas, lembretes personalizados e opções de compartilhamento (\cite{GoogleAgendaBaseConhecimento}), tornando a administração do tempo mais dinâmica, eficiente e colaborativa.

Com base nesses recursos, este capítulo foi desenvolvido com o objetivo de apresentar as principais funcionalidades do Google Agenda e orientar o usuário em sua utilização no dia a dia, promovendo uma rotina mais organizada e produtiva.