% sections/agenda/configuracoes.tex
% !TeX root = ../../../main.tex

\section{Configurações}
A aba de configurações do Google Agenda é o espaço onde o usuário pode 
personalizar e ajustar o funcionamento de acordo com suas preferências. Nela, é 
possível alterar o idioma, o fuso horário, escolher a forma de exibição dos 
eventos, definir horários de trabalho e disponibilidade, configurar as 
notificações e lembretes de seus eventos e compromissos e também é possível 
gerenciar múltiplas agendas, definir cores para seus diferentes tipos de 
compromisso. Facilitando a vida do usuário, tornando sua experiência mais 
agradável e fácil de usar. Outro ponto importante é a possibilidade de gerenciar 
o compartilhamento de agendas, permitindo que outras pessoas visualizem e 
editem, facilitando bastante na organização de um grupo ou de uma equipe. 


\subsubsection{Como acessar as configurações}
Para acessar as configurações do Google Agenda, clique no ícone de engrenagem 
localizado no canto superior direito da tela.

\begin{figure}[H]
    \centering
    \includegraphics[width=.39\textwidth]{/agenda/configuracoes/Imagem1.png}
    \caption{}
\end{figure}

\begin{itemize}
    \item Lixeira: Todos os eventos que foram previamente excluídos, ficaram lá por até 30 dias e depois serão excluídos permanentemente, ele te dá uma chance de recuperar itens que foram excluídos.
    \item Aparência: Permite personalizar o calendário de acordo com as preferências do usuário, possibilitando a escolha entre diferentes temas e ajustando a densidade das informações exibidas na tela.
    \item Imprimir: Permite gerar e salvar um arquivo PDF ou realizar a impressão de um intervalo de tempo da agenda.
    \item Instalar complemento: Permite integrar ao Google Agenda aplicativos de terceiros para melhorar a experiência do usuário.
\end{itemize}

Para acessar todas as opções de configuração, clique em “Configurações”.

\begin{figure}[H]
    \centering
    \includegraphics[width=.39\textwidth]{/agenda/configuracoes/Imagem2.png}
    \caption{}
\end{figure}


\subsubsection{Configuração geral}
Na Configuração geral, o usuário pode ajustar diversas preferências que personalizam o funcionamento do Google Agenda:

\begin{enumerate}
    \item Idioma e região: Define o idioma, a região e o formato de data e hora da sua agenda
    \item Fuso horário: Define o fuso horário da sua agenda, além de poder adicionar um segundo fuso horário.
    \item Relógio Mundial: Permite exibir no menu principal do Google Agenda vários fusos horários de diversos locais do mundo. 
    
    \begin{figure}[H]
        \centering
        \includegraphics[width=.39\textwidth]{/agenda/configuracoes/Imagem3.png}
        \caption{}
    \end{figure}

    \item Configurações de eventos: Define critérios para criar um novo evento, como duração padrão e as permissões de convidados.
    \item Configurações de notificação: Permite configurar como o usuário será avisado sobre os eventos, podendo alterar como será notificado e o tipo de som da notificação do evento.

    \begin{figure}[H]
        \centering
        \includegraphics[width=.39\textwidth]{/agenda/configuracoes/Imagem4.png}
        \caption{}
    \end{figure}

    \item Opções de visualização: Ajusta a aparência da agenda e a forma como os eventos são mostrados, podendo escolher entre diferentes modos de exibição, como diária, semanal ou mensal.
    \item Recursos inteligentes do Google Workspace: Personaliza a experiência do usuário com recursos de outros produtos do Google, apps como o Gmail, Agenda, Chat, Meet e Drive. Ativando esse recurso você pode permitir que o Gmail crie eventos automaticamente com base em confirmações.
    
    \begin{figure}[H]
        \centering
        \includegraphics[width=.39\textwidth]{/agenda/configuracoes/Imagem5.png}
        \caption{}
    \end{figure}

    \item Horário e local de trabalho: Define sua disponibilidade, informando horários que você trabalha ou estuda para caso alguém tente agendar uma reunião com você.
    \item Atalhos do teclado: Permite utilizar comandos no teclado para facilitar e agilizar a locomoção pela agenda. 
    \item Offline: Permite utilizar da agenda principal mesmo quando estiver sem internet, dando acesso ao usuário a compromissos já sincronizados.

    \begin{figure}[H]
        \centering
        \includegraphics[width=.39\textwidth]{/agenda/configuracoes/Imagem6.png}
        \caption{}
    \end{figure}
\end{enumerate}

\subsection{Adicionar agenda}
Adiciona novos calendários e permite acompanhar outras agendas e eventos públicos, facilitando a organização de seus compromissos e eventos.

\begin{enumerate}
    \item Inscrever-se na agenda: Permite adicionar agendas de outras pessoas, aparecendo a agenda que foram compartilhadas junto com a sua, podendo ver compromissos e eventos de outras pessoas.
        
    \begin{figure}[H]
        \centering
        \includegraphics[width=.39\textwidth]{/agenda/configuracoes/Imagem7.png}
        \caption{}
    \end{figure}

    \item Criar nova agenda: Cria um novo calendário separado da sua agenda principal, útil para manter compromissos separados e organizados por categoria.
    
    \begin{figure}[H]
        \centering
        \includegraphics[width=.39\textwidth]{/agenda/configuracoes/Imagem8.png}
        \caption{}
    \end{figure}

    \item Procurar agendas de interesse: Disponibiliza agendas públicas, assim o usuário pode acompanhar automaticamente eventos como feriados nacionais, eventos esportivos, entre outros.

    \begin{figure}[H]
        \centering
        \includegraphics[width=.39\textwidth]{/agenda/configuracoes/Imagem9.png}
        \caption{}
    \end{figure}

    \item Do URL: Permite adicionar uma agenda externa utilizando um endereço de URL, integrando calendários de outras plataformas.

    \begin{figure}[H]
        \centering
        \includegraphics[width=.39\textwidth]{/agenda/configuracoes/Imagem10.png}
        \caption{}
    \end{figure}
\end{enumerate}


\subsection{Importar e Exportar}
Essas opções permitem transferir eventos entre diferentes contas ou aplicativos de agenda.

\begin{enumerate}
    \item Importar:Permite trazer eventos através de um arquivo no formato .ics para dentro do Google Agenda.
    \item Exportar: Permite baixar os seus eventos do Google agenda em um arquivo .zip que contém um ou mais arquivos no formato .ics.
    
    \begin{figure}[H]
        \centering
        \includegraphics[width=.39\textwidth]{/agenda/configuracoes/Imagem11.png}
        \caption{}
    \end{figure}
\end{enumerate}


\subsubsection{Configurações das minhas agendas}
Logo abaixo do título dessa seção, na barra lateral esquerda, aparecem os nomes das agendas do usuário. Ao clicar em qualquer uma delas, são exibidas as opções de configuração específicas dessa agenda.


\subsubsection{Configurações de agendas criadas}

\begin{enumerate}
    \item Configurações da agenda: Define nome, descrição, fuso horário e outras preferências básicas da agenda selecionada.
    \item Compartilhar com: Permite compartilhar a sua agenda com outras pessoas, além de poder permitir que outros usuários apenas olhem e também possam editar a sua agenda.
    
    \begin{figure}[H]
        \centering
        \includegraphics[width=.39\textwidth]{/agenda/configuracoes/Imagem12.png}
        \caption{}
    \end{figure}

    \item Autorizações de acesso a eventos: Ajusta a quantidade de detalhes que outros usuários poderão ver quando você compartilha a sua agenda.
    \item Notificações de eventos: Define quando e como você será notificado de um evento normal.
    \item Notificações de eventos de dia inteiro: Define quando e como você será notificado de um evento de dia inteiro.
    
    \begin{figure}[H]
        \centering
        \includegraphics[width=.39\textwidth]{/agenda/configuracoes/Imagem13.png}
        \caption{}
    \end{figure}

    \item Outras notificações: Define como você será notificado quando a alguma ação relacionada a sua agenda, como quando alguém adiciona um evento ou quando alguém responde a algum convite.
    \item Integrar agenda: Mostra links e códigos para integrar a sua agenda em outros serviços.
    \item Remover agenda: Permite excluir uma agenda criada ou se desinscrever de uma agenda que compartilharam com você.
    
    \begin{figure}[H]
        \centering
        \includegraphics[width=.39\textwidth]{/agenda/configuracoes/Imagem14.png}
        \caption{}
    \end{figure}
\end{enumerate}


\subsubsection{Configuração da agenda de aniversário}

\begin{enumerate}
    \item Configuração da agenda: Permite ajustar o fuso horário de exibição da agenda.
    \item Configuração de permissão: Define as configurações de permissão da agenda de aniversário.
    \item Notificações de eventos de dia inteiro: Define quando e como você será notificado de um evento de aniversário.
    \item Sincronizar com os contatos:Conecta os aniversários dos seus contatos do Google diretamente na agenda.
    
    \begin{figure}[H]
        \centering
        \includegraphics[width=.39\textwidth]{/agenda/configuracoes/Imagem15.png}
        \caption{}
    \end{figure}
\end{enumerate}