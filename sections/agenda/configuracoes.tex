% sections/agenda/configuracoes.tex
% !TeX root = ../../../main.tex

\section{Configurações}
A aba de configurações do Google Agenda é o espaço onde o usuário pode 
personalizar e ajustar o funcionamento de acordo com suas preferências. Nela, é 
possível alterar o idioma, o fuso horário, escolher a forma de exibição dos 
eventos, definir horários de trabalho e disponibilidade, configurar as 
notificações e lembretes de seus eventos e compromissos e também é possível 
gerenciar múltiplas agendas, definir cores para seus diferentes tipos de 
compromisso. Facilitando a vida do usuário, tornando sua experiência mais 
agradável e fácil de usar. Outro ponto importante é a possibilidade de gerenciar 
o compartilhamento de agendas, permitindo que outras pessoas visualizem e 
editem, facilitando bastante na organização de um grupo ou de uma equipe. 


\subsubsection{Como acessar as configurações}
Para acessar as configurações do Google Agenda, clique no ícone de engrenagem 
localizado no canto superior direito da tela.

\begin{figure}[H]
    \centering
    \includegraphics[width=.39\textwidth]{/agenda/configuracoes/Imagem1.png}
    \caption{}
\end{figure}

\begin{itemize}
    \item Lixeira: Todos os eventos que foram previamente excluídos, ficaram lá por até 30 dias e depois serão excluídos permanentemente, ele te dá uma chance de recuperar itens que foram excluídos.
    \item Aparência: Permite personalizar o calendário de acordo com as preferências do usuário, possibilitando a escolha entre diferentes temas e ajustando a densidade das informações exibidas na tela.
    \item Imprimir: Permite gerar e salvar um arquivo PDF ou realizar a impressão de um intervalo de tempo da agenda.
    \item Instalar complemento: Permite integrar ao Google Agenda aplicativos de terceiros para melhorar a experiência do usuário.
\end{itemize}

Para acessar todas as opções de configuração, clique em “Configurações”.

\begin{figure}[H]
    \centering
    \includegraphics[width=.39\textwidth]{/agenda/configuracoes/Imagem2.png}
    \caption{}
\end{figure}


\subsubsection{Configuração geral}
Na Configuração geral, o usuário pode ajustar diversas preferências que personalizam o funcionamento do Google Agenda:

\begin{enumerate}
    \item Idioma e região: Define o idioma, a região e o formato de data e hora da sua agenda
    \item Fuso horário: Define o fuso horário da sua agenda, além de poder adicionar um segundo fuso horário.
    \item Relógio Mundial: Permite exibir no menu principal do Google Agenda vários fusos horários de diversos locais do mundo. 
\end{enumerate}