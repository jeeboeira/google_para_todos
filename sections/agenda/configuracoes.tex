% sections/agenda/configuracoes.tex
% !TeX root = ../../main.tex

\section{Configurações}
A \gls{aba} de configurações do Google Agenda é o espaço onde o usuário pode 
personalizar e ajustar o funcionamento de acordo com suas preferências. Nela, é 
possível alterar o idioma, o fuso horário, escolher a forma de exibição dos 
eventos, definir horários de trabalho e disponibilidade e configurar as 
notificações e lembretes de seus eventos e compromissos. Também é possível 
gerenciar múltiplas agendas e definir cores para seus diferentes tipos de 
compromisso, o que torna a experiência do usuário mais agradável e auxilia sua organização pessoal. 
Outro ponto importante é a possibilidade de gerenciar 
o compartilhamento de agendas, permitindo que outras pessoas visualizem e 
editem, facilitando a organização de compromissos/tarefas de equipes e demais grupos. 


\subsection{Como acessar as configurações}
Para acessar as configurações do Google Agenda, clique no ícone de engrenagem 
localizado no canto superior direito da tela.

\begin{figure}[H]
    \centering
    \includegraphics[width=.39\textwidth]{images/agenda/configuracoes/Imagem1.png}
    \caption{}
\end{figure}

\begin{itemize}
    \item \textbf{Lixeira}: todos os eventos que foram excluídos ficarão lá por 30 dias e depois serão excluídos permanentemente. Dentro desse período de tempo é possível recuperar os itens que estão na lixeira;
    \item \textbf{Aparência}: permite personalizar o calendário de acordo com as preferências do usuário, possibilitando a escolha entre diferentes temas e ajustando a densidade das informações exibidas na tela;
    \item \textbf{Imprimir}: permite gerar e salvar um \gls{arquivo} \gls{pdf} ou realizar a impressão de um intervalo de tempo da agenda;
    \item \textbf{Instalar complemento}: permite integrar ao Google Agenda \gls{aplicativos} de terceiros para melhorar a experiência do usuário.
\end{itemize}

Para acessar todas as opções de configuração, clique em “Configurações”.

\begin{figure}[H]
    \centering
    \includegraphics[width=.39\textwidth]{images/agenda/configuracoes/Imagem2.png}
    \caption{}
\end{figure}


\subsection{Configuração geral}
Na Configuração geral, o usuário pode ajustar diversas preferências que personalizam o funcionamento do Google Agenda:

\begin{enumerate}
    \item \textbf{Idioma e região}: define o idioma, a região e o formato de data e hora da sua agenda;
    \item \textbf{Fuso horário}: define o fuso horário da sua agenda, além de permitir adicionar um segundo fuso horário;
    \item \textbf{Relógio Mundial}: permite exibir no menu principal do Google Agenda vários fusos horários de diversos locais do mundo; 
    
    \begin{figure}[H]
        \centering
        \includegraphics[width=.39\textwidth]{images/agenda/configuracoes/Imagem3.png}
        \caption{}
    \end{figure}

    \item \textbf{Configurações de eventos}: define critérios para criar um novo evento, como duração padrão e as permissões de convidados;
    \item \textbf{Configurações de notificação}: permite configurar como o usuário será avisado sobre os eventos, podendo alterar como será notificado e o tipo de som da notificação do evento.

    \begin{figure}[H]
        \centering
        \includegraphics[width=.39\textwidth]{/agenda/configuracoes/Imagem4.png}
        \caption{}
    \end{figure}

    \item \textbf{Opções de visualização}: ajusta a aparência da agenda e a forma como os eventos são mostrados, podendo escolher entre diferentes modos de exibição, como diária, semanal ou mensal;
    \item \textbf{Recursos inteligentes do Google \Gls{workspace}}: personaliza a experiência do
     usuário com recursos de outros produtos do Google, apps como o Gmail, Agenda, \Gls{chat}, \Gls{meet} e Drive. 
     Ativando esse recurso você pode permitir que o Gmail crie eventos automaticamente com base em confirmações;
    
    \begin{figure}[H]
        \centering
        \includegraphics[width=.39\textwidth]{/agenda/configuracoes/Imagem5.png}
        \caption{}
    \end{figure}

    \item \textbf{Horário e local de trabalho}: define sua disponibilidade, informando 
    horários que você trabalha ou estuda para caso alguém tente agendar uma reunião com você;
    \item \textbf{Atalhos do teclado}: permite utilizar comandos no teclado para facilitar e agilizar a locomoção pela agenda;
    \item \textbf{\Gls{offline}}: permite utilizar da agenda principal mesmo quando estiver sem internet, dando acesso ao 
    usuário a compromissos já sincronizados.

    \begin{figure}[H]
        \centering
        \includegraphics[width=.39\textwidth]{/agenda/configuracoes/Imagem6.png}
        \caption{}
    \end{figure}
\end{enumerate}

\subsection{Adicionar agenda}
Adiciona novos calendários e permite acompanhar outras agendas e eventos públicos, facilitando a organização de seus compromissos e eventos.

\begin{enumerate}
    \item \textbf{Inscrever-se na agenda}: permite adicionar agendas de outras pessoas à sua e visualizar os compromissos e eventos de forma compartilhada;
        
    \begin{figure}[H]
        \centering
        \includegraphics[width=.39\textwidth]{/agenda/configuracoes/Imagem7.png}
        \caption{}
    \end{figure}

    \item \textbf{Criar nova agenda}: cria um novo calendário separado da sua agenda principal, útil para manter compromissos separados e organizados por categoria;
    
    \begin{figure}[H]
        \centering
        \includegraphics[width=.39\textwidth]{/agenda/configuracoes/Imagem8.png}
        \caption{}
    \end{figure}

    \item \textbf{Procurar agendas de interesse}: disponibiliza agendas públicas, assim o usuário pode acompanhar automaticamente eventos como feriados nacionais, eventos esportivos, entre outros;

    \begin{figure}[H]
        \centering
        \includegraphics[width=.39\textwidth]{/agenda/configuracoes/Imagem9.png}
        \caption{}
    \end{figure}

    \item \textbf{Do \gls{url}}\@: permite adicionar uma agenda externa utilizando um endereço de \gls{url}, integrando calendários de outras plataformas;

    \begin{figure}[H]
        \centering
        \includegraphics[width=.39\textwidth]{/agenda/configuracoes/Imagem10.png}
        \caption{}
    \end{figure}
\end{enumerate}


\subsection{Importar e Exportar}
Essas opções permitem transferir eventos entre diferentes contas ou \gls{aplicativos} de agenda.

\begin{enumerate}
    \item \textbf{Importar}: permite importar eventos através de um \gls{arquivo} no formato .\gls{ics} para dentro do Google Agenda; %chktex 26
    \item \textbf{Exportar}: permite baixar os seus eventos do Google agenda em um \gls{arquivo} .\gls{zip} que contém um ou mais arquivos no formato .\gls{ics}. %chktex 26
    
    \begin{figure}[H]
        \centering
        \includegraphics[width=.39\textwidth]{/agenda/configuracoes/Imagem11.png}
        \caption{}
    \end{figure}
\end{enumerate}


\subsubsection{Configurações das minhas agendas}
Na imagem localizada na próxima seção, na barra lateral esquerda, são exibidos os nomes das agendas do usuário. Ao clicar em qualquer uma delas, são exibidas as opções de configuração específicas para cada uma.


\subsubsection{Configurações de agendas criadas}

\begin{enumerate}
    \item \textbf{Configurações da agenda}: define nome, descrição, fuso horário e outras preferências básicas da agenda selecionada;
    \item \textbf{Compartilhar com}: permite compartilhar a sua agenda com outras pessoas, além de autorizar outros usuários apenas a visualizar ou também a editar a sua agenda;
    
    \begin{figure}[H]
        \centering
        \includegraphics[width=.39\textwidth]{/agenda/configuracoes/Imagem12.png}
        \caption{}
    \end{figure}

    \item \textbf{Autorizações de acesso a eventos}: ajusta a quantidade de detalhes que outros usuários poderão ver quando você compartilha a sua agenda;
    \item \textbf{Notificações de eventos}: define quando e como você será notificado de um evento "normal";
    \item \textbf{Notificações de eventos de dia inteiro}: define quando e como você será notificado de um evento de um dia inteiro de duração;
    
    \begin{figure}[H]
        \centering
        \includegraphics[width=.39\textwidth]{/agenda/configuracoes/Imagem13.png}
        \caption{}
    \end{figure}

    \item \textbf{Outras notificações}: define como você será notificado quando há alguma ação relacionada à sua agenda, como quando alguém adiciona um evento ou quando alguém responde a algum convite;
    \item \textbf{Integrar agenda}: exibe links e códigos para integrar a sua agenda em outros serviços;
    \item \textbf{Remover agenda}: permite excluir uma agenda criada ou se desinscrever de uma agenda compartilhada com você;
    
    \begin{figure}[H]
        \centering
        \includegraphics[width=.39\textwidth]{/agenda/configuracoes/Imagem14.png}
        \caption{}
    \end{figure}
\end{enumerate}


\subsubsection{Configuração da agenda de aniversário}

\begin{enumerate}
    \item \textbf{Configuração da agenda}: permite ajustar o fuso horário de exibição da agenda;
    \item \textbf{Configuração de permissão}: define as configurações de permissão da agenda de aniversário;
    \item \textbf{Notificações de eventos de dia inteiro}: define quando e como você será notificado de um evento de aniversário;
    \item \textbf{Sincronizar com os contatos}: conecta os aniversários dos seus contatos do Google diretamente na agenda.
    
    \begin{figure}[H]
        \centering
        \includegraphics[width=.39\textwidth]{/agenda/configuracoes/Imagem15.png}
        \caption{}
    \end{figure}
\end{enumerate}
