% sections/agenda/gerenciamento_de_multiplas_agendas.tex
% !TeX root = ../../../main.tex

\section{Gerenciamento de múltiplas agendas}
O Google Agenda possui a funcionalidade de gerenciamento de múltiplas agendas. Confira a seguir como criar agendas, se inscrever em agendas de outros usuários, dentre outras funções.

\subsection{Como criar novas agendas}
A partir da tela inicial do Google Agenda, observe o painel localizado na lateral esquerda da página e localize a seção ``Outras agendas''. Clique no botão ``Adicionar outras agendas'' (ícone de sinal de adição à esquerda do título da seção). 

\begin{figure}[H]
    \centering
    \includegraphics[width=.39\textwidth]{/agenda/gerenciamento_de_multiplas_agendas/Imagem1.png}
    \caption{Botão que abre menu para adição de agenda em destaque}
\end{figure}

Ao clicar neste botão, se abre um menu suspenso com diferentes opções para adicionar agendas ao seu Google Agenda. Clique na opção ``Criar nova agenda''. Ao realizar essa ação, a tela mudará para o formulário de criação de agenda, com os seguintes campos:

\begin{itemize}
    \item \textbf{Nome}: nome da nova agenda;
    \item \textbf{Descrição}: descrição em texto simples da agenda;
    \item \textbf{Fuso horário}: defina o fuso horário da nova agenda de acordo com a necessidade;
    \item \textbf{Proprietário}: campo não editável com o nome do proprietário da agenda;
    \item \textbf{Organização}: este campo não editável é somente exibido se a conta Google proprietária da agenda pertencer a alguma organização.
\end{itemize}

Com as informações da agenda preenchidas, clique em ``Criar agenda''.

\begin{figure}[H]
    \centering
    \includegraphics[width=.39\textwidth]{/agenda/gerenciamento_de_multiplas_agendas/Imagem2.png}
    \caption{Exemplo de criação de agenda com detalhes preenchidos e prestes a ser criada}
\end{figure}

Com isso feito, aguarde o sistema do Google Agenda concluir a criação da nova agenda, a qual será exibida no painel do lado esquerdo da página e os campos do formulário de criação de agenda serão limpos, permitindo a criação de outras agendas se assim desejar.


\subsection{Como se inscrever em outras agendas}
A partir da tela inicial do Google Agenda, clique no botão ``Adicionar outras agendas'' e no menu suspenso clique na opção ``Inscrever-se na agenda''. Realizar essa ação levará o usuário para o formulário de inscrição em agendas de contatos e de pessoas que estejam dentro da mesma organização do usuário.


\subsubsection{Como adicionar agendas de interesse}
Na tela inicial do Google Agenda, clique no botão ``Adicionar nova agenda'' e no menu suspenso clique na opção ``Procurar agendas de interesse''. Ao clicar nesta opção, o usuário será direcionado para uma página com várias opções de agendas para se inscrever de acordo com seus interesses.

\begin{figure}[H]
    \centering
    \includegraphics[width=.39\textwidth]{/agenda/gerenciamento_de_multiplas_agendas/Imagem3.png}
    \caption{Tela de inscrição em agendas de interesse}
\end{figure}

Para se inscrever em qualquer uma das agendas disponíveis, basta clicar na caixa de seleção localizada ao lado esquerdo do nome da agenda. 

Há a possibilidade de visualizar agendas das categorias ``Feriados religiosos internacionais'', ``Esportes'' e ``Outros''. Basta posicionar o mouse em cima da opção desejada, isso fará com que seja exibido um ícone de olho na extremidade direita da opção selecionada, clique nesse ícone para abrir uma nova \gls{aba} do navegador com a agenda selecionada.


\subsection{Como adicionar uma agenda por \gls{url}}\label{subsec:agenda_externa}
Na tela inicial do Google Agenda, clique no botão ``Adicionar nova agenda'' e no menu suspenso clique na opção ``Do \gls{url}''. O usuário será direcionado para a página de adição de agenda por \gls{url}.\@

Esta forma de adição serve para se inscrever em agendas \gls{caldav}.\@ Isto é, se o usuário possui um serviço \gls{online} que provê agenda digital, como um provedor de e-mail, pode ser interessante integrar a agenda hospedada nesse provedor externo, assim possibilita centralizar as agendas em uma única ferramenta.

Entre em contato com o seu provedor de serviços \gls{online} para saber se tens serviço de agenda disponível. Caso positivo, localize na ferramenta de seu provedor o local para acessar a \gls{url} da agenda, copie-a e cole-a no campo ``\gls{url} da agenda'', conforme a imagem abaixo.

\begin{figure}[H]
    \centering
    \includegraphics[width=.39\textwidth]{/agenda/gerenciamento_de_multiplas_agendas/Imagem4.png}
    \caption{Adicionando agenda por \gls{url}}
\end{figure}

Ao colar a \gls{url} no campo citado, será habilitado o botão ``Adicionar agenda''. Antes de clicá-lo, note que há uma caixa de seleção localizada logo acima deste botão, marque-a se deseja tornar a agenda prestes a ser incluída disponível para o público. Com as configurações feitas, clique em ``Adicionar agenda'' para concluir o processo de adição. Realizar esta ação adiciona a agenda na seção ``Outras agendas'' no painel esquerdo da tela inicial do Google Agenda.


\subsection{Importação e exportação de agendas}
O Google agenda possui ferramentas para importação e exportação de agendas. Confira a seguir como acessá-las e usá-las.


\subsection{Como importar uma agenda}
A partir da tela inicial do Google Agenda, clique no botão ``Adicionar nova agenda'' e no menu suspenso clique na opção ``Importar''. O usuário será redirecionado para a tela de importação e exportação de agendas.

Antes de iniciar o processo de importação, primeiro exporte uma agenda hospedada em algum serviço externo à plataforma Google. Para mais informações, consulte a Seção~\ref{subsec:agenda_externa} para mais informações sobre agendas fora do Google Agenda. Com a agenda externa exportada, pode-se prosseguir.

Na seção ``Importar'', clique no botão ``Selecionar \gls{arquivo} no seu computador''. Isso abrirá uma janela de seleção de arquivos. Selecione o \gls{arquivo} a ser importado, selecione a agenda de destino e clique no botão ``Importar''. Logo após será exibido uma caixa de diálogo informando quantos eventos foram importados.

\begin{figure}[H]
    \centering
    \includegraphics[width=.39\textwidth]{/agenda/gerenciamento_de_multiplas_agendas/Imagem5.png}
    \caption{Tela de importação/exportação de agendas com um \gls{arquivo} com eventos prestes a serem importados}
\end{figure}

\begin{figure}[H]
    \centering
    \includegraphics[width=.39\textwidth]{/agenda/gerenciamento_de_multiplas_agendas/Imagem6.png}
    \caption{Janela de diálogo informando quantidade de eventos importados}
\end{figure}

Os eventos importados serão exibidos na tela inicial do Google Agenda, veja exemplo na imagem a seguir.

\begin{figure}[H]
    \centering
    \includegraphics[width=.39\textwidth]{/agenda/gerenciamento_de_multiplas_agendas/Imagem7.png}
    \caption{Eventos importados na agenda}
\end{figure}


\subsection{Como exportar uma agenda}
Para acessar a exportação de agendas, pode-se realizar os mesmos passos para acessar a tela de importação. Também se pode chegar na mesma tela acessando as Configurações do Google Agenda.

\begin{figure}[H]
    \centering
    \includegraphics[width=.39\textwidth]{/agenda/gerenciamento_de_multiplas_agendas/Imagem8.png}
    \caption{Acessando as configurações do Google Agenda}
\end{figure}

Na tela de configurações, clique na opção ``Importar e exportar'' localizada no painel localizado na lateral esquerda da página. Na tela de importação/exportação de agendas, clique no botão ``Exportar''. Executar essa ação fará o \gls{download} de um \gls{arquivo} .\gls{zip} contendo um \gls{arquivo} .\gls{ics} com os eventos das agendas selecionadas. Use este \gls{arquivo} para importar os eventos em uma outra ferramenta de gerenciamento de agendamentos. %chktex 26
