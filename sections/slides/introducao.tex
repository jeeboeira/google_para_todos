% sections/slides/introducao.tex% 
% !TeX root = ../../main.tex

\section{Introdução}

O Google Apresentações é uma ferramenta digital gratuita do \textbf{Google}, cuja finalidade é a elaboração, edição e compartilhamento de apresentações em \textbf{slides}. Por se tratar de uma ferramenta \textbf{online}, podemos acessar os conteúdos criados através de outros dispositivos, mesmo estando em diferentes lugares. Isso é essencial para o trabalho em grupo, uma vez que os integrantes podem interagir e realizar alterações simultâneas em um mesmo arquivo remotamente.

Quando comparado ao \textbf{Microsoft PowerPoint}, uma ferramenta tradicionalmente instalada em computadores e equipada com um amplo conjunto de recursos de design, efeitos visuais e opções de personalização, o \textbf{Google Apresentações} se destaca pela praticidade e recursos que permitem colaboração em tempo real. O \textbf{PowerPoint}, por outro lado, se sobressai em termos de sofisticação e coleções de funcionalidades mais avançadas.

\subsection{Criar um projeto}

Ao acessarmos a tela inicial, nos são apresentadas algumas opções para iniciar um novo projeto. Podemos escolher entre modelos pré definidos, como um currículo, álbum de fotos, portfólio, ou até mesmo um projeto completamente em branco. Abaixo desta seção, estarão as apresentações recentemente criadas ou modificadas pelo usuário, se houver.

Para criar uma nova apresentação em branco, basta clicar na caixa com o símbolo de mais ``+'', em destaque na imagem.

\begin{figure}[H]
  \centering
  \includegraphics[width=.39\textwidth]{/slides/introducao/Imagem 1.png}
  \caption{}
\end{figure}

Logo, será aberto um projeto em branco. Uma nova janela se abrirá no navegador e a apresentação estará pronta para ser editada.

\begin{figure}[H]
  \centering
  \includegraphics[width=.39\textwidth]{/slides/introducao/Imagem 2.png}
  \caption{}
\end{figure}

Assim como o \textbf{Planilhas}, \textbf{Documentos} e demais aplicações \textbf{Google}, temos a barra superior principal para nos auxiliar a encontrar as principais funcionalidades através de alguns itens:

\begin{figure}[H]
  \centering
  \includegraphics[width=.39\textwidth]{/slides/introducao/Imagem 3.png}
  \caption{}
\end{figure}

Abaixo, veremos as funcionalidades básicas de cada um destes itens.

\subsection{Abrir um arquivo existente}

Para abrir um arquivo já existente salvo em nuvem (\textbf{Google Drive}, por exemplo) ou no dispositivo local, selecione a guia ``Arquivo'' e clique sobre a opção ``Abrir''.

\begin{figure}[H]
  \centering
  \includegraphics[width=.39\textwidth]{/slides/introducao/Imagem 4.png}
  \caption{}
\end{figure}

Escolha o arquivo que deseja abrir.

\begin{figure}[H]
  \centering
  \includegraphics[width=.39\textwidth]{/slides/introducao/Imagem 5.png}
  \caption{}
\end{figure}

\begin{dica}
  Você pode alterar o local e a forma de busca selecionando os itens de pesquisa na barra superior: ``Meu Drive'', ``Compartilhados comigo'', ``Com estrela'', ``Computadores'' ou ``Upload''
\end{dica}

\subsection{Importar slides}

Você também pode importar arquivos com formatos distintos ou que foram editados em diferentes ferramentas, porém, é necessário ter ciência de que a compatibilidade entre eles pode acarretar em conflitos de dimensão e/ou formatação com alguns elementos, por conta de configurações particulares de cada ferramenta.

Para importar slides de outros arquivos, selecione a opção ``Importar slides'', na aba ``Arquivo'' do menu superior principal, conforme ilustrado na imagem:

\begin{figure}[H]
  \centering
  \includegraphics[width=.39\textwidth]{/slides/introducao/Imagem 6.png}
  \caption{}
\end{figure}

Selecione o ícone do arquivo que deseja abrir (alterne a seleção entre as abas de busca para procurar arquivos em outros locais).

\begin{figure}[H]
  \centering
  \includegraphics[width=.39\textwidth]{/slides/introducao/Imagem 7.png}
  \caption{}
\end{figure}

\begin{figure}[H]
  \centering
  \includegraphics[width=.39\textwidth]{/slides/introducao/Imagem 8.png}
  \caption{}
\end{figure}
