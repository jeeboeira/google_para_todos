% sections/slides/formatar.tex
% !TeX root = ../../main.tex

\textbf{Formatar}

A guia “Formatar”, também no menu superior, possui diversas propriedades para organizar elementos textuais do seu slide, mas também fornece algumas opções úteis para formatação de imagens, gráficos e outros elementos. A seguir, uma breve descrição de suas principais funções:

\section*{Texto}

Inicialmente, selecione o texto que deseja modificar arrastando o mouse com o botão esquerdo pressionado sobre ele. Na aba superior da tela, abaixo do título do documento, selecione a opção “Formatar” $\rightarrow$ “Texto”. Serão abertas algumas opções de formatação para o trecho de texto selecionado:

\begin{itemize}
    \item \textbf{Negrito}: aumenta a espessura dos caracteres selecionados e é normalmente utilizado para destacar palavras ou partes de um texto;
    \item \textbf{Itálico}: popularmente conhecido por “letra deitada”, é principalmente utilizado para sinalizar, no texto, palavras provenientes de outros idiomas, mas pode ter outros usos e finalidades;
    \item \textbf{Sublinhar}: insere um traço abaixo do texto selecionado. Comummente utilizado para dar ênfase a trechos específicos de um texto;
    \item \textbf{Tachar}: insere uma linha que “corta” o texto selecionado. Utilizado para censuras e, na maioria das vezes, indicar que um texto que não tem mais importância no contexto atual, mas não deve ser esquecido (apagado);
    \item \textbf{Sobrescrito}: trata-se de um texto de fonte diminuta escrito no topo da linha. Usualmente utilizado em representações matemáticas exponenciais. Por exemplo: 2x + x\textsuperscript{2} (nesse caso, “2” está sobrescrito);
    
    \textit{Dica: você pode utilizar o atalho [CTRL + .] para sobrescrever um texto}
    
    \item \textbf{Subscrito}: semelhante ao item anterior, porém, o texto diminuto será exibido na extremidade inferior da linha. Também é muito utilizado em representações matemáticas, por exemplo: a\textsubscript{n} (aqui, “n” está subscrito);
    
    \textit{Dica: você pode utilizar o atalho [CTRL + ,] para subscrever um texto}
    
    \item \textbf{Tamanho}: essa opção permite aumentar ou diminuir o tamanho da fonte na porção de texto selecionada. Fonte Menor (10), Fonte Maior (14).
    \item \textbf{Cor}: abre uma paleta de cores que permite selecionar uma cor diferente para o texto selecionado;
    \item \textbf{Cor de destaque}: utilizado para alterar a cor de fundo do texto selecionado;
    \item \textbf{Uso de maiúsculas/minúsculas}: para o texto selecionado, tens a opção de deixar “todas as letras minúsculas”, “TODAS AS LETRAS MAIÚSCULAS” ou “Todas As Primeiras Letras Das Palavras Com Letras Maiúsculas”.
\end{itemize}

\section*{Alinhamento e Recuo}

Para ajustar essas opções, acesse o menu “Formatar” $\rightarrow$ “Alinhar e recuar”:

\begin{itemize}[leftmargin=*]
    \item \textbf{Esquerda}: essa opção é utilizada para formatar todo o parágrafo correspondente ao texto selecionado para a borda esquerda da página;
    
    \begin{flushleft}
    Texto Exemplo, Texto Exemplo, Texto Exemplo, Texto Exemplo, Texto Exemplo, Texto Exemplo, Texto Exemplo, Texto Exemplo, Texto Exemplo, Texto Exemplo
    \end{flushleft}
    
    \item \textbf{Centralizar}: centraliza o conteúdo do parágrafo selecionado na página;
    
    \begin{center}
    Texto Exemplo, Texto Exemplo, Texto Exemplo, Texto Exemplo, Texto Exemplo, Texto Exemplo, Texto Exemplo, Texto Exemplo, Texto Exemplo, Texto Exemplo
    \end{center}
    
    \item \textbf{Direita}: alinha o parágrafo selecionado à borda direita da página;
    
    \begin{flushright}
    Texto Exemplo, Texto Exemplo, Texto Exemplo, Texto Exemplo, Texto Exemplo, Texto Exemplo, Texto Exemplo, Texto Exemplo, Texto Exemplo, Texto Exemplo
    \end{flushright}
    
    \item \textbf{Justificado}: ajusta automaticamente o espaçamento entre palavras das linhas de um parágrafo (com exceção da última) para que ocupem toda a extensão compreendida entre as bordas da página;
    
    Texto Exemplo, Texto Exemplo, Texto Exemplo, Texto Exemplo, Texto Exemplo, Texto Exemplo, Texto Exemplo, Texto Exemplo, Texto Exemplo, Texto Exemplo, Texto Exemplo, Texto Exemplo, Texto Exemplo, Texto Exemplo, Texto Exemplo, Texto Exemplo, Texto Exemplo
    
    \item \textbf{Aumentar recuo}: utilizado para deslocar o texto do parágrafo para a direita, afastando-o da borda esquerda da página;
    \item \textbf{Diminuir recuo}: desloca o texto do parágrafo para a esquerda, aproximando-o da borda esquerda da página;
    \item \textbf{Parte superior}: essa opção tem como função alinhar o texto na extremidade superior da caixa de texto em que está contido;
    \item \textbf{Meio}: alinha o texto na parte central da caixa de texto em que está contido;
    \item \textbf{Parte inferior}: de maneira similar, alinha o conteúdo da caixa de texto em sua extremidade inferior.
\end{itemize}

\section*{Espaçamento entre linhas e parágrafos}

Essas configurações podem ser acessadas, também, através do submenu “Formatar”. Com elas, você pode ajustar o espaço vertical em seu texto para melhorar a legibilidade e a aparência geral do documento. Pode-se alterar o espaçamento entre as linhas com opções pré-definidas (simples; 1,15 ou 1,5) ou personalizadas, e também adicionar espaços-extra antes e/ou depois de cada parágrafo

\section*{Marcadores e numeração}

Acessando os submenus “Formatar” $\rightarrow$ “Marcadores e Numeração/Marcadores” $\rightarrow$ “Menu de lista numerada” ou “Menu de lista com marcadores”, você pode criar uma listagem de itens. Basta selecionar a opção de personalização desejada, escrever um texto e, com a quebra de linha (ao pressionar “Enter”), um novo item será criado.

\textit{Dica: ao escrever os itens de uma lista, se você pressionar a tecla “Tab”, irá criar uma sub lista. Utilize “Shift” + “Tab” para retornar a lista hierarquicamente superior.}

\section*{Formatar Tabela}

Se você possuir uma tabela em seu slide, ao selecioná-la, pode utilizar esta opção do submenu “Formatar” para inserir linhas abaixo ou acima da mesma; inserir colunas à esquerda ou à direita; excluir linhas e colunas; distribuí-las (deixá-las com dimensões equivalentes), mesclar ou desfazer mesclagem de células.

\section*{Imagem}

As opções de formatação para imagens inseridas no slide consistem em:

\begin{itemize}[leftmargin=*]
    \item \textbf{Cortar a imagem}: permite que você redimensione a imagem cortando-a. Após selecioná-la, basta arrastar as extremidades da imagem original, ajustando novas bordas;
    \item \textbf{Substituir a imagem}: permite que você altere a imagem atual por outra proveniente de algum banco de mídia suportado pelo Google Apresentações;
    \item \textbf{Redefinir a imagem}: retoma ao estado original da imagem, desfazendo quaisquer alterações realizadas durante a edição;
\end{itemize}

\section*{Bordas e Linhas}

Para formatar bordas e linhas de elementos contidos em sua apresentação, acesse o sub menu “Formatar” $\rightarrow$ “Bordas e linhas”. As opções disponíveis são:

\begin{itemize}[leftmargin=*]
    \item \textbf{Cor da borda}: utilizada para alterar a cor da borda da imagem ou elemento;
    \item \textbf{Espessura da borda}: mensurada em pixels (px), serve para ajustar a espessura da borda no entorno do elemento;
    \item \textbf{Tipo de borda}: fornece algumas opções alternativas de borda (dupla, tripla, etc.);
    \item \textbf{Linha da borda}: mais relacionada à estilização da borda, ainda assim, muito utilizada. Pode torná-la pontilhada, tachada, contínua, etc.;
    \item \textbf{Decorações de borda}: possibilita adicionar decorações à borda, como ilustrado na imagem a seguir:
\end{itemize}

\textbf{(Imagem 1)}

\textbf{Opções de formatação}

\textbf{(Imagem 2)}

Dentro dessas opções, há a possibilidade de girar o texto, adicionar sombras, reflexos, aumentar a fonte, alterar a posição e mudar o alinhamento do texto.

\textbf{Limpar formatação}: Utilizada para realizar a remoção de toda a formatação direta aplicada a um texto ou a uma tabela, como: negrito, itálico, sublinhado, cores, bordas, sombreados ou estilos de caracteres, restaurando o conteúdo para a formatação padrão ou original.
