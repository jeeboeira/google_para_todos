% sections/slides/slide.tex
% !TeX root = ../../main.tex

\section{Slide}

Você pode adicionar, excluir, reordenar e até adicionar novos temas e estilizações aos slides de uma apresentação para organizá-los da forma desejada. A guia “\gls{slide}”, presente na barra superior do menu principal, concentra todas essas funcionalidades.

\subsection{Novo slide}

Ao clicar em “Novo \gls{slide}”, uma nova página em branco será inserida abaixo da página selecionada atualmente. Alterne a seleção entre as páginas clicando sobre elas na \gls{aba} lateral esquerda do documento.

\subsection{Duplicar slide}

Duplicar o \gls{slide} atual, copiando todos os elementos e estilizações do \gls{slide} atualmente selecionado para uma nova página.

\subsection{Excluir slide}

Remove todo o conteúdo do \gls{slide} selecionado e exclui a página da visualização.

\subsection{Pular slide}

Clicar sobre essa opção oculta o \gls{slide} selecionado durante a apresentação. No entanto, o conteúdo do \gls{slide} não será excluído, podendo, ainda, ser alterado no modo edição.

\subsection{Mover slide}

Na sequência de visualização (\gls{aba} esquerda com numeração “1”, “2”, “3”...), move o \gls{slide} selecionado para a ordem desejada: para cima, para baixo, para o início ou para o fim.

\subsection{Alterar plano de fundo}

Altera o plano de fundo para uma cor sólida (azul, verde, vermelho…), gradiente (combinação de cores) ou para uma imagem, sendo esta obtida a partir de arquivos salvos em \gls{nuvem} no Google Drive ou de arquivos no armazenamento local do computador.

\subsection{Aplicar layout}

Adiciona um novo \gls{slide} com campos de texto ordenados e formatados previamente. Tratam-se de modelos-padrão para preenchimento de título, colunas e cabeçalhos.

\begin{figure}[H]
    \centering
    \includegraphics[width=.39\textwidth]{/slides/slide/Imagem 1.png}
    \caption{}
\end{figure}

\subsection{Transição}

\begin{figure}[H]
    \centering
    \includegraphics[width=.39\textwidth]{/slides/slide/Imagem 2.png}
    \caption{}
\end{figure}

Com esse recurso, você pode configurar a forma como será feita a animação de transição entre slides durante a apresentação: troca simples, efeitos de giro, esmaecer, movimentos geométricos, etc. Além de slides, as transições também podem ser atribuídas a alguns elementos específicos, como textos e imagens.

\begin{dica}
    Utilize animações mais sutis para não criar uma apresentação repleta de movimentos repentinos e exagerados. Isso pode gerar desconforto nos espectadores, inclusive tontura.
\end{dica}

\subsection{Editar tema}

A edição de um tema permite alterar a formatação das caixas de texto, cores, ou elementos do tema de \gls{layout} atualmente escolhido. Será aberta uma janela de visualização com os slides modelos, permitindo que você os personalize da forma que preferir.

\begin{figure}[H]
    \centering
    \includegraphics[width=.75\textwidth]{/slides/slide/Imagem 3.png}
    \caption{}
\end{figure}

\subsection{Alterar tema}

Serve para alterar o tema do \gls{slide} selecionado. Ao lado direito do \gls{layout} atual, será exibida uma lista vertical com uma série de temas prontos disponíveis. Basta clicar sobre um deles para aplicar a estilização em sua apresentação. 

Como exemplo, no \gls{slide} do print abaixo, o tema foi alterado para a opção “Preto simples”. 

\begin{figure}[H]
    \centering
    \includegraphics[width=.34\textwidth]{/slides/slide/Imagem 4.png}
    \caption{}
\end{figure}

\begin{dica}
    Nesta mesma guia, você pode clicar sobre o botão “Importar tema” para adicionar um novo padrão de slides em sua apresentação. Basta procurar e selecionar o \gls{arquivo} de slides baixado em \gls{nuvem} ou no seu computador através da janela de busca.
\end{dica}
