% sections/slides/\gls{arquivo}.tex
% !TeX root = ../../main.tex

\section{Arquivo}

A guia ``\Gls{arquivo}'', na barra de menu principal, encontrada na região superior esquerda da tela, possui algumas propriedades gerais importantes para a definição do seu projeto. Abaixo, segue uma breve descrição dessas funcionalidades.

\subsection{Novo}

\begin{itemize}
  \item \textbf{Apresentação}: abre uma nova guia no navegador, com um novo projeto de apresentação em branco;
  \item \textbf{Da galeria de modelos}: abre uma nova guia no navegador, onde você poderá escolher uma apresentação modelo com elementos pré-definidos.
\end{itemize}

\begin{figure}[H]
    \centering
    \includegraphics[width=.39\textwidth]{/slides/arquivo/Imagem 1.png}
    \caption{Galeria de Modelos}
\end{figure}

\subsection{Abrir}

Essa função abre uma nova janela de busca na tela. Você pode procurar arquivos locais e salvos em \gls{nuvem}, com formato de \textbf{slides} e apresentação compatíveis, para abri-los como um novo projeto, em outra guia do navegador.

\begin{figure}[H]
    \centering
    \includegraphics[width=.39\textwidth]{/slides/arquivo/Imagem 2.png}
    \caption{Abrir Arquivo}
\end{figure}

\subsection{Importar slides}

Se assemelha à funcionalidade ``Abrir'', porém, ao importar slides, você mantém os que já foram criados no projeto e adiciona novos.

\subsection{Fazer uma cópia}

\begin{itemize}
  \item \textbf{Toda a apresentação}: com essa função você copia toda a sua apresentação para um novo \gls{arquivo}, podendo optar entre manter, ou não, as opções de compartilhamento, comentários e anotações originais;
  \item \textbf{Slides selecionados}: faz uma cópia de todos os \textbf{slides} selecionados no documento no presente momento. Assim como na função anterior, pode-se selecionar algumas opções de compartilhamento e anotações.
\end{itemize}

\subsection{E-mail}

\begin{itemize}
  \item \textbf{Enviar este \gls{arquivo} por \gls{email}}: abre uma tela que permite encaminhar a apresentação para a(s) pessoa(s) que deseja, inserindo o endereço de e-mail do(s) destinatário(s) através do campo de inserção de texto, adicionando uma mensagem descritiva e selecionando o formato do documento exportado (``\gls{pdf}'', ``Microsoft PowerPoint'' ou ``Texto simples'');
  \item \textbf{Enviar \gls{email} aos colaboradores}: abrirá uma tela com os nomes das pessoas que já possuem acesso ao documento para que encaminhe um e-mail contendo a apresentação à elas, semelhante à opção anterior. Você também pode optar por encaminhar uma cópia para si mesmo e adicionar uma mensagem informativa.
\end{itemize}

\subsection{Baixar}

Habilita o \textbf{\gls{download}} do \gls{arquivo} no armazenamento local do seu computador (ou celular), em formatos como: Power Point (.pptx), \gls{pdf}, Texto (.txt), Imagem JPEG (.jpg), Imagem PNG (.png) ou Elementos gráficos vetoriais escaláveis (.svg). Os formatos em imagem (últimos três), no entanto, salvam apenas o \gls{slide} atual.

\begin{figure}[H]
    \centering
    \includegraphics[width=.39\textwidth]{/slides/arquivo/Imagem 3.png}
    \caption{Aba com opções de baixar}
\end{figure}

\subsection{Renomear}

Função simples que abre uma caixa editável no nome do \gls{arquivo} para modificação.

\subsection{Adicionar atalho ao Google Drive}

Com essa função, você pode selecionar a pasta ou local onde deseja criar um ``atalho'' de acesso para o projeto atual no seu \textbf{Google Drive}.

\begin{figure}[H]
    \centering
    \includegraphics[width=.39\textwidth]{/slides/arquivo/Imagem 4.png}
    \caption{Adicionar atalho no drive}
\end{figure}

Em seu \textbf{Google Drive}, a visualização ficará desta forma:

\begin{figure}[H]
    \centering
    \includegraphics[width=.39\textwidth]{/slides/arquivo/Imagem 5.png}
    \caption{Atalho adicionado ao drive}
\end{figure}

\subsection{Histórico de versões}

Uma função extremamente útil em edições colaborativas. Com ela, você pode acessar estados específicos do seu projeto salvos cronologicamente. Você pode selecionar entre duas opções:

\begin{itemize}
  \item \textbf{Nomear a versão atual}: pode ser utilizada caso você deseje registrar no histórico a versão atual do seu projeto. Será aberta uma janela para que você possa nomear essa versão, facilitando o controle do histórico;
  \item \textbf{Ver históricos de versões}: essa função abre uma guia na lateral direita da janela. Nela, estão listadas em coluna, cronologicamente, todas as versões salvas do projeto, cada uma contendo o registro de mudanças feitas por cada colaborador no documento. Ao selecionar uma das versões, na parte superior esquerda da tela, surge um botão para ``Restaurar esta versão'', se desejar.
\end{itemize}

\begin{figure}[H]
    \centering
    \includegraphics[width=.39\textwidth]{/slides/arquivo/Imagem 6.png}
    \caption{Histórico de versões}
\end{figure}

\subsection{Detalhes}

Abre, em uma nova janela, uma lista com algumas informações do \gls{arquivo}: data de criação, data da última modificação, proprietário do documento e local em que está salvo.

\subsection{Limitações de segurança}

Caso o \gls{arquivo} tenha alguma limitação de edição, \textbf{\gls{download}} ou compartilhamento, por exemplo, será exibida uma guia na lateral direita contendo essas informações.

\subsection{Configuração da página}

Abre uma janela que permite definir as dimensões da página. Se você selecionar a opção ``Personalizado'', poderá especificar cada uma das dimensões (horizontais e verticais) da página e a unidade de medida desejada: ``Polegadas'', ``Centímetros'', ``Pontos'' ou ``Pixels''.

\begin{figure}[H]
    \centering
    \includegraphics[width=.39\textwidth]{/slides/arquivo/Imagem 7.png}
    \caption{Configuração da página}
\end{figure}

\subsection{Visualizar impressão}

Abre uma janela que exibe a pré-visualização da apresentação para impressão. Serve como uma revisão das configurações aplicadas, visualizando o resultado da impressão antes que uma cópia indesejada seja feita.

\subsection{Imprimir}

A função ``imprimir'' (atalho \tecla{Ctrl + P}) abre uma janela com configurações de impressão para sua apresentação. Selecione o dispositivo desejado ou o formato \gls{pdf} (``Salvar como \gls{pdf}''), especifique as páginas desejadas (\textbf{slides}) e a quantidade de páginas que serão impressas por folha. Inicie a impressão clicando no botão azul ``Imprimir''.
