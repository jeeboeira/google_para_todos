% sections/slides/aplicar_temas.tex
% !TeX root = ../../main.tex

\section{Aplicar Temas}
\subsection{Tema}

No \textbf{Google Apresentações}, a função ``Tema'' define o estilo visual padrão da apresentação, no que se refere à cores, fontes e planos de fundo. Ela é responsável por gerar uma identidade única para todos os slides, de forma consistente.

\begin{figure}[H]
    \centering
    \includegraphics[width=.39\textwidth]{/slides/aplicar_temas/Imagem 1.png}
    \caption{Seletor de temas}
\end{figure}

Ao clicar na função, localizada da barra de menu superior sombreada, será exibida uma guia lateral no lado direito da tela, contendo diversos modelos prontos disponibilizados pela ferramenta. Além disso, é possível importar temas personalizados de outros arquivos clicando em ``Importar tema''.

\begin{dica}
    Caso a função não esteja aparecendo na barra sombreada, clique no ícone de "tres pontos", localizado na sua extremidade direita. Isso deve-se, provavelmente, ao tamanho reduzido da janela do navegador.
\end{dica}

\subsection{Plano de Fundo}

No Google Slides, a função ``Plano de fundo'' permite mudar a cor sólida, escolher gradientes ou inserir uma imagem como fundo para um \gls{slide} específico ou para todos da apresentação. Você pode desfazer as alterações clicando em ``Redefinir'' e pode salvá-las no tema da apresentação, ao clicar em ``Adicionar ao tema''.

\begin{figure}[H]
    \centering
    \includegraphics[width=.39\textwidth]{/slides/aplicar_temas/Imagem 3.png}
    \caption{Customização do tema}
\end{figure}

\subsection{Layout}

No Google Slides, o \gls{layout} organiza os elementos principais do \gls{slide}, como títulos, cabeçalhos e caixas de texto, de maneira simples, padronizando a organização de elementos e poupando esforços. Ao clicar na guia ``\gls{layout}'', também presente na barra de menu superior sombreada, basta selecionar o modelo de organização mais adequado para cada \gls{slide} da sua apresentação.

\begin{figure}[H]
    \centering
    \includegraphics[width=.39\textwidth]{/slides/aplicar_temas/Imagem 4.png}
    \caption{Seletor de layout}
\end{figure}

\subsection{Transição}

A função ``Transição'' adiciona movimentos de animação entre os slides, embelezando-os e tornando a apresentação mais dinâmica. Ao clicar na função, presente no final da barra de menu superior sombreada, será aberta uma guia na lateral direita da tela, com algumas configurações pertinentes para os \textbf{slides} selecionados.

\begin{figure}[H]
    \centering
    \includegraphics[width=.39\textwidth]{/slides/aplicar_temas/Imagem 5.png}
    \caption{Propriedades da transição}
\end{figure}

Em relação ao PowerPoint, possui uma gama de transições mais limitada, voltada à proposta de ser mais prático e objetivo, entregando apenas o necessário para a criação de apresentações claras e simples.
