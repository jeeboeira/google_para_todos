% sections/slides/aplicar_temas.tex
% !TeX root = ../../main.tex

\section{Aplicar Temas}
\subsection{Tema}

No \textbf{Google Apresentações}, a função ``Tema'' define o estilo visual padrão da apresentação, no que se refere à cores, fontes e planos de fundo. Ela é responsável por gerar uma identidade única para todos os slides, de forma consistente.

\textbf{(IMAGEM 1 APONTANDO PARA ABA TEMA)}

Ao clicar na função, será exibida uma guia lateral no lado direito da tela, contendo diversos modelos prontos disponibilizados pela ferramenta. Além disso, é possível importar temas personalizados de outros arquivos clicando em ``Importar tema''.

\subsection{Plano de Fundo}

No Google Slides, a função ``Plano de fundo'' permite mudar a cor sólida, escolher gradientes ou inserir uma imagem como fundo para um slide específico ou para todos da apresentação. Você pode desfazer as alterações clicando em ``Redefinir'' e pode salvá-las no tema da apresentação, ao clicar em ``Adicionar ao tema''.

\textbf{(IMAGEM 3 APONTANDO PARA O GUIA PLANO DE FUNDO E INTERAGINDO COMO GUIA ABRINDO-LO)}

\subsection{Layout}

No Google Slides, o layout organiza os elementos principais do slide, como títulos, cabeçalhos e caixas de texto, de maneira simples, padronizando a organização de elementos e poupando esforços. Ao clicar na guia ``Layout'', basta selecionar o modelo de organização mais adequado para sua apresentação.

\textbf{(IMAGEM 4 APONTANDO PARA O GUIA DE LAYOUT)}

\subsection{Transição}

A função ``Transição'' adiciona movimentos de animação entre os slides, embelezando-os e tornando a apresentação mais dinâmica. Ao clicar na função, será aberta uma guia na lateral direita da tela, com algumas configurações pertinentes para os \textbf{slides} selecionados.

Em relação ao PowerPoint, possui uma gama de transições mais limitada, voltada à proposta de ser mais prático e objetivo, entregando apenas o necessário para a criação de apresentações claras e simples.

\textbf{(IMAGEM 5 APONTANDO PARA O GUIA DE TRANSIÇÃO)}
