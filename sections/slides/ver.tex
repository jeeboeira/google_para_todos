% sections/slides/ver.tex
% !TeX root = ../../main.tex

\section{Ver}

A guia ``Ver'' possui funcionalidades que se concentram, especialmente, na apresentação definitiva do projeto. Abaixo, segue breve descrição destas funções.

\subsection{Modo}

\begin{itemize} 
  \item \textbf{Edição}: habilita o modo de edição para o projeto, onde você pode adicionar, modificar e remover elementos livremente em cada um dos \textbf{slides}. 
  \item \textbf{Comentários}: habilita um modo especial de visualização que não permite edições no projeto, mas apenas a inclusão de comentários nos \textbf{slides} que podem ser visualizados pelos usuários com permissões de edição (atalho \tecla{Ctrl + Alt + M}). 
  \item \textbf{Visualização}: neste modo você pode apenas visualizar o conteúdo dos \textbf{slides} da apresentação, sem permissões para quaisquer adições, alterações ou exclusões. 
\end{itemize}

\subsection{Apresentação de slides}

Ao clicar sobre essa opção, a apresentação entrará em modo ``tela cheia'' para exibição do conteúdo dos \textbf{slides}. Clique em \tecla{Esc} para sair desse modo de visualização.

\subsection{Movimento}

A opção ``Movimento'' abre uma guia na lateral direita da tela que permite visualizar definições de transição entre \textbf{slides} ou animações em objetos selecionados.

\subsection{Criador de tema}

Ao clicar nesta opção, será aberta uma nova janela no centro da tela para que você possa criar e personalizar um tema para os \textbf{slides} de sua apresentação. É possível alterar o tamanho das caixas de texto, modificar cores etc.

\subsection{Comentários}

\begin{itemize}
  \item \textbf{Ocultar comentários}: oculta a visualização dos comentários criados no projeto. 
  \item \textbf{Abrir comentários}: volta a exibir os comentários que foram ocultados em seus respectivos \textbf{slides} da apresentação. 
  \item \textbf{Mostrar todos os comentários}: abre uma guia lateral que lista todos os comentários feitos no arquivo da sua apresentação. 
\end{itemize}

\subsection{Visualização em grade}

A opção “Visualização em grade” amplia a barra lateral esquerda que exibe a ordem de visualização dos slides em formato de coluna e os ordena em forma de grade: com linhas e colunas.

\subsection{Exibir régua}

Mostra duas barras com medidas em cm (centímetros) na extremidade superior e lateral esquerda do slide. Útil para redimensionamentos de elementos que exigem precisões milimétricas.

\subsection{Guias}

As guias são linhas verticais e horizontais adicionais que servem para organizar e alinhar objetos dentro dos slides de forma precisa, garantindo uniformidade e consistência visual. É possível adicioná-las, removê-las ou até arrastá-las para as posições desejadas.

\textbf{Dica:} ao arrastar as guias, note o número exibido na tela: ele corresponde à distância, em cm, que a guia está posicionada em relação à extremidade superior do documento (horizontal) ou à extremidade esquerda (vertical). Use essa informação para ajustá-las adequadamente.

\subsection{Alinhar a}

Organiza e padroniza a diagramação dos elementos visuais na apresentação. Você pode escolher entre duas opções de referência para a organização dos itens: ``Grade'' ou ``Guias''.

\subsection{Ponteiros ao vivo}

A guia ``Ponteiros ao vivo'' possui duas opções que você pode habilitar ou desabilitar: ``Mostrar ponteiros dos colaboradores'' e ``Mostrar meu ponteiro''. Isso refere-se à visualização do cursor do \textbf{mouse} no modo colaborativo de edição, em tempo real.

\textbf{Dica:} no modo ``Apresentação de slides'', em tela cheia, aperte ``L'' para transformar o cursor do \textbf{mouse} em uma ferramenta visual semelhante a um laser, útil para destacar figuras, gráficos ou textos importantes.

\subsection{Mostrar tira de filme}

A tira de filme corresponde à barra lateral esquerda onde é possível visualizar e navegar pelos \textbf{slides} do projeto sequencialmente. A função apenas habilita ou desabilita a exibição dessa barra.

\begin{figure}[H]
    \centering
    \includegraphics[width=.39\textwidth]{/slides/ver/Imagem 1.png}
    \caption{}
\end{figure}

\subsection{Menu de zoom}

Essa função fornece algumas opções que permitem ao usuário ampliar (atalho \tecla{Ctrl + ``=''}) ou reduzir (\tecla{Ctrl + ``-''}) a visualização do \textbf{slide} no documento.

\subsection{Tela inteira}

Ao clicar sobre essa opção, os controles de edição e criação de apresentações (abas superiores de guias e ferramentas) são ocultados.
