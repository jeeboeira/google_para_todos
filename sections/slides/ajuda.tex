% sections/slides/ajuda.tex
% !TeX root = ../../main.tex

\section{Ajuda}

Outra opção disponível no menu da barra superior do \textbf{Google Apresentações}, e que pode ser bastante útil para resolver questionamentos e orientar o usuário durante a utilização da ferramenta, é a guia ``Ajuda''. A seguir, uma breve descrição de suas funcionalidades:

\begin{itemize}
  \item \textbf{Pesquisar nos menus}: essa opção habilita uma \gls{barrapesquisa} que permite encontrar diretamente algumas funções específicas, tais como: ``Colar'', ``Inserir imagem'', ``Inserir vídeo'', ``Configuração da página'', entre outros;
\end{itemize}

\begin{figure}[H]
    \centering
    \includegraphics[width=.39\textwidth]{images/slides/ajuda/Imagem 1.png}
    \caption{Pesquisa nos menus}
\end{figure}


\begin{itemize}
  \item \textbf{Ajuda do app Apresentações}: abre um menu \textbf{\gls{popup}} que contém alguns tópicos de ajuda específicos. Ao clicar no item desejado, será exibida uma explicação detalhada sobre a funcionalidade;
\end{itemize}

\begin{figure}[H]
    \centering
    \includegraphics[width=.39\textwidth]{/slides/ajuda/Imagem 2.png}
    \caption{Central de ajuda}
\end{figure}

\begin{itemize}
  \item \textbf{Treinamento}: abre uma nova guia, direcionando-o para o ``Centro de aprendizagem do Google \Gls{workspace}''. A página apresenta uma série de tutoriais e dicas de boas práticas para a utilização do \textbf{Google Apresentações};
\end{itemize}

\begin{figure}[H]
    \centering
    \includegraphics[width=.39\textwidth]{/slides/ajuda/Imagem 3.png}
    \caption{Página de treinamento}
\end{figure}

\begin{itemize}
  \item \textbf{Atualizações}: abre um \gls{popup} exibindo as atualizações mais recentes da ferramenta. Também informa que é possível acessar essas informações através do blog \textbf{Google \Gls{workspace} Updates};
  \item \textbf{Ajude a melhorar o app Apresentações}: abre uma janela com um campo para inserção de texto no lado direito do seu monitor, possibilitando que envie um \textbf{\gls{feedback}} sobre a ferramenta ao \textbf{Google};
  \item \textbf{Atalhos do teclado}: exibe uma lista com todos os atalhos de teclado que podem ser utilizados no ambiente da ferramenta, como o da função ``Copiar'', o famoso \tecla{CTRL + C}.
\end{itemize}

