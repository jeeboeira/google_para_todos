% sections/slides/colaboracao_e_compartilhamento.tex
% !TeX root = ../../main.tex

\section{Colaboração e Compartilhamento}
\subsection{Compartilhar}

Ao acessar o menu ``Arquivo'', clique sobre a guia ``Compartilhar''. Após dar um nome ao seu projeto ou apresentação, serão exibidas duas funções específicas de compartilhamento: uma que direciona para usuários e outra para a web.

\textbf{IMAGEM 1 SETA APONTANDO PARA OPÇÕES DE COMPARTILHAR}

``Compartilhar com outras pessoas'' disponibiliza o acesso ao conteúdo da apresentação para usuários ou grupos selecionados. Na nova janela, selecione a opção ``Restrito'' caso queira que somente aos participantes/e-mails adicionados tenham acesso ao conteúdo da apresentação. Adicione e-mails e participantes pela barra de inserção de texto, na parte superior da nova janela. Se você quiser compartilhar um link que garante acesso à apresentação a cada um que o possuir, selecione a opção ``Qualquer pessoa com o link''. É extremamente importante atentar-se às permissões de acesso (``Leitor'', ``Comentador'' e ``Editor''). Garanta que a cada um dos participantes seja dada a função adequada.

\textbf{IMAGEM 2 SETA APONTANDO EXPLORADOR DE PARTICIPANTES DO PROJETO, SETA APONTANDO PARA TIPO DE ACESSO E SETA APONTANDO PARA LINK.}

\textbf{IMAGEM 3 SETA APONTANDO PARA OPÇÕES DE ACESSO DA ABA DE TIPO DE ACESSO}

A opção ``Publicar na Web'' abre uma janela para geração de um \textbf{link} ou código para incorporar a apresentação. Depois de publicada, a apresentação pode ser visualizada como slides de apresentação e você tem a escolha de configurar a passagem automática dos slides em intervalos de tempo específicos. O principal ponto é que, sempre que você fizer uma alteração na apresentação original, a versão publicada na web será atualizada automaticamente.

\textbf{IMAGEM 4 SETA APONTANDO PARA OPÇÕES DE TEMPORIZADOR DE AVANÇO AUTOMÁTICO E SETA APONTANDO PARA ÍCONES DE COMPARTILHAMENTO}

\textbf{IMAGEM 5 APONTANDO PARA SELEÇÃO DE TAMANHOS DOS SLIDES EM ABAS DE OPÇÕES}
