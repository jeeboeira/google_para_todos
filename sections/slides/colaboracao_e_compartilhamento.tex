% sections/slides/colaboracao_e_compartilhamento.tex
% !TeX root = ../../main.tex

\section{Colaboração e Compartilhamento}
\subsection{Compartilhar}

Ao acessar o menu ``\Gls{arquivo}'', clique em uma das opções presentes na guia ``Compartilhar''. Após dar um nome ao seu projeto ou apresentação, você pode esolher entre duas funções específicas de compartilhamento: uma que direciona para usuários e outra para a \gls{web}.

\begin{figure}[H]
    \centering
    \includegraphics[width=.39\textwidth]{/slides/colaboracao_e_compartilhamento/Imagem 1.png}
    \caption{Opções de Compartilhar}
\end{figure}

A opção ``Compartilhar com outras pessoas'' disponibiliza o acesso ao conteúdo da apresentação para usuários ou grupos selecionados. Na nova janela, selecione a opção ``Restrito'' caso queira que somente aos participantes/e-mails adicionados tenham acesso ao conteúdo da apresentação. Adicione e-mails e participantes pela barra de inserção de texto, na parte superior da nova janela. Se você quiser compartilhar um \gls{link} que garante acesso à apresentação a cada um que o possuir, selecione a opção ``Qualquer pessoa com o \gls{link}''. É extremamente importante atentar-se às permissões de acesso (``Leitor'', ``Comentador'' e ``Editor''). Garanta que a cada um dos participantes seja dada a função adequada.

\begin{figure}[H]
    \centering
    \includegraphics[width=.39\textwidth]{/slides/colaboracao_e_compartilhamento/Imagem 2.png}
    \caption{Explorador de Participantes do Projeto}
\end{figure}

\begin{figure}[H]
    \centering
    \includegraphics[width=.39\textwidth]{/slides/colaboracao_e_compartilhamento/Imagem 3.png}
    \caption{Opções de Acesso}
\end{figure}

A opção ``Publicar na \gls{web}'' abre uma janela para geração de um \textbf{\gls{link}} ou código para incorporar a apresentação. Depois de publicada, a apresentação pode ser visualizada como slides de apresentação e você tem a escolha de configurar a passagem automática dos slides em intervalos de tempo específicos. Sempre que você fizer uma alteração na apresentação original, a versão publicada na \gls{web} será atualizada automaticamente.

\begin{figure}[H]
    \centering
    \includegraphics[width=.39\textwidth]{/slides/colaboracao_e_compartilhamento/Imagem 4.png}
    \caption{Opções de Temporizador de Avanço Automático}
\end{figure}

\begin{figure}[H]
    \centering
    \includegraphics[width=.39\textwidth]{/slides/colaboracao_e_compartilhamento/Imagem 5.png}
    \caption{Seleção de Tamanho dos Slides}
\end{figure}
