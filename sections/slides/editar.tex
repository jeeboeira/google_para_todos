% sections/slides/editar.tex
% !TeX root = ../../main.tex

\section{Editar}

Na barra de menu principal, acessando o menu ``Editar'', temos as seguintes opções:

\begin{figure}[H]
    \centering
    \includegraphics[width=.39\textwidth]{/slides/editar/Imagem 1.png}
    \caption{}
\end{figure}

\subsection{Desfazer}

A função ``desfazer'' (atalho \tecla{Ctrl + X}) restaura o estado imediatamente anterior do seu documento, ou seja, desfaz alterações acidentais ou indesejadas. Ao utilizá-la, as alterações serão desfeitas, sequencialmente, a cada clique.

\subsection{Refazer}

A função de ``refazer'' (atalho \tecla{Ctrl + Y}) é parecida com a de desfazer, porém, realiza a ação contrária, anulando uma ação de ``desfazer''. Ou seja, retoma um estado de documento anterior a uma ação de ``desfazer'', a cada clique.

\subsection{Recortar}

A função recortar é utilizada quando você deseja copiar uma imagem, forma ou texto para outro local ou \gls{slide}, sem que ela permaneça no local de onde foi copiada após a ação.

\subsection{Copiar}

A função ``copiar'', assim como a ``recortar'', serve para copiar o objeto selecionado, que pode ser um texto, imagem, setas, etc. No entanto, ela não remove o objeto originalmente copiado.

\subsection{Colar}

Após copiar algum elemento do \gls{slide}, utiliza-se a função ``colar'' para que o conteúdo copiado anteriormente seja colado na área ou campo selecionado.

\subsection{Colar sem formatação}

A função ``colar sem formatação'' serve para que você possa colar o conteúdo copiado sem que o tamanho de letras, cores e fontes originais sejam mantidos na apresentação. Dessa forma você não precisará formatar todo o texto ao final da edição para que ele fique com as mesmas propriedades.

\subsection{Selecionar tudo}

A função ``selecionar tudo'' é utilizada quando há a necessidade de selecionar todo o conteúdo presente na página do \gls{slide} ou na caixa de texto, de uma só vez.

\subsection{Excluir}

A função ``excluir'' é utilizada para apagar objetos ou slides selecionados na sua apresentação.

\subsection{Duplicar}

A função ``duplicar'' é utilizada quando tem-se a necessidade de replicar igualmente (com os mesmos elementos e conteúdos) um \gls{slide} da apresentação.

\subsection{Localizar e substituir}

A função ``localizar e substituir'' é utilizada quando, em um determinado \gls{slide}, há uma palavra com diversas repetições, possibilitando a correção de todas as ocorrências de uma só vez. Basta preencher o campo ``Localizar'' com o texto procurado e informar a nova palavra em ``Substituir por''. Utilize as opções ``Anterior'' e ``Avançar'' para buscar as ocorrências no documento, uma a uma. Clique em ``substituir'' para alterar a ocorrência selecionada e em ``Substituir tudo'' para modificar todas as aparições.

\begin{figure}[H]
    \centering
    \includegraphics[width=.39\textwidth]{/slides/editar/Imagem 2.png}
    \caption{}
\end{figure}
