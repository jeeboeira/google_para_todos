% sections/slides/inserir.tex
% !TeX root = ../../main.tex

\section{Inserir}

A \gls{aba} “Inserir”, no menu principal da barra superior, contém opções para adição de novos elementos aos slides, tais como: imagens, vídeos, formas, texto, tabelas, etc. Abaixo, segue uma breve explicação sobre como inserir cada um desses elementos:

\subsection{Imagem}

Utilizada para colar fotos e figuras no \gls{slide}. Pode ser inserida de diversas formas: através de \textbf{\gls{upload}} de arquivos do computador (mídias locais), pesquisas na \gls{web}, importação de outros \gls{aplicativos} em \gls{nuvem} como o \textbf{Google Fotos} e o \textbf{Google Drive}, utilizando a câmera em dispositivos compatíveis e até mesmo por \gls{url} (\textbf{\gls{link}}).

\begin{figure}[H]
    \centering
    \includegraphics[width=.6\textwidth]{images/slides/inserir/Imagem 1.png}
    \caption{Guia “Imagem”, mostrando as demais funções}
\end{figure}

\subsection{Caixa de texto}

Gera uma área retangular no \gls{slide} para digitação. Pode ser redimensionada pelo usuário, mas ajusta-se automaticamente ao escrever palavras, frases ou textos.

\begin{figure}[H]
    \centering
    \includegraphics[width=.7\textwidth]{/slides/inserir/Imagem 2.png}
    \caption{Guia “Caixa de Texto”}
\end{figure}

\subsection{Forma}

Permite posicionar figuras geométricas ou símbolos, como círculos, quadrados, setas, balões de fala, estrelas, entre outros. Pode ser utilizado para destacar partes do \gls{slide} ou montar desenhos simples.

\begin{figure}[H]
    \centering
    \includegraphics[width=.6\textwidth]{/slides/inserir/Imagem 3.png}
    \caption{Guia “Forma”, mostrando as demais opções de formas}
\end{figure}

\subsection{Diagrama}

Insere modelos de organização representativos: fluxogramas, hierarquias e processos, por exemplo. Muito utilizados para ilustrar projetos e esquematizar planejamentos diversos

\begin{figure}[H]
    \centering
    \includegraphics[width=.6\textwidth]{/slides/inserir/Imagem 4.png}
    \caption{Guia “Diagrama”, mostrando as demais opções de Diagrama}
\end{figure}

\subsection{Tabela}

\begin{figure}[H]
    \centering
    \includegraphics[width=.4\textwidth]{/slides/inserir/Imagem 5.png}
    \caption{Guia “Tabela”}
\end{figure}

\subsection{Gráfico}

Permite adicionar gráficos para ilustrar dados de maneira visual. Podem ser incluídos das seguintes formas: “Colunas”, “Linhas”, “Pizza” e “Barras”. Também é possível importar um gráfico do Google Planilhas (recomendamos a leitura do capítulo sobre essa ferramenta para mais informações).

\begin{figure}[H]
    \centering
    \includegraphics[width=.52\textwidth]{/slides/inserir/Imagem 6.png}
    \caption{Guia “Gráfico”, mostrando as demais opções de gráfico}
\end{figure}

\subsection{Word Art}

Gera textos com estilizações diferentes. É geralmente utilizada em títulos. O texto pode ter cores, bordas e formatos personalizados. Na \gls{aba} “Opções de formatação”, você pode ainda ajustar alguns atributos de sombreamento, reflexo e rotação, por exemplo.

\begin{figure}[H]
    \centering
    \includegraphics[width=.34\textwidth]{/slides/inserir/Imagem 7.png}
    \caption{Guia “Word Art”}
\end{figure}

\subsection{Linha}

Desenha linhas retas ou curvas no \gls{slide}. Há também algumas outras opções mais personalizadas, mas essas são as mais utilizadas. É um recurso interessante que pode ser usado para ligar elementos ou destacar algo no \gls{slide}.

\begin{figure}[H]
    \centering
    \includegraphics[width=.4\textwidth]{/slides/inserir/Imagem 8.png}
    \caption{Guia “Linha”. mostrando as demais opções de linha}
\end{figure}

\subsection{Vídeo}

Permite adicionar vídeos ao \gls{slide}. Podem ser importados através de uma pesquisa direta no YouTube, via \gls{url} ou Google Drive (vídeo salvo em \gls{nuvem} na sua conta Google).

\begin{figure}[H]
    \centering
    \includegraphics[width=.35\textwidth]{/slides/inserir/Imagem 9.png}
    \caption{Guia “Vídeo”}
\end{figure}

\subsection{\Gls{link}}

A partir de um texto previamente selecionado, cria um atalho (hiperlink) que redireciona para outro \gls{slide}, site ou documento.

\begin{figure}[H]
    \centering
    \includegraphics[width=.32\textwidth]{/slides/inserir/Imagem 10.png}
    \caption{Guia “Link”}
\end{figure}

\subsection{Animação}

Permite adicionar animações de movimento em textos, imagens ou formas, por exemplo: um texto que surge do canto inferior quando o \gls{slide} é apresentado.

\begin{figure}[H]
    \centering
    \includegraphics[width=.36\textwidth]{/slides/inserir/Imagem 11.png}
    \caption{Guia “Animação”}
\end{figure}

\subsection{Números de slide}

Adiciona numeração automática para as páginas de \gls{slide} criadas (1,2,3...).

\begin{figure}[H]
    \centering
    \includegraphics[width=.4\textwidth]{/slides/inserir/Imagem 12.png}
    \caption{Guia “Número o slide”}
\end{figure}

\subsection{Novo slide}

Também pode ser “encurtado” pelo comando \tecla{CTRL + M}. Adiciona um novo \gls{slide} vazio ao final da sua apresentação.

\begin{figure}[H]
    \centering
    \includegraphics[width=.39\textwidth]{/slides/inserir/Imagem 13.png}
    \caption{Guia “Novo Slide”}
\end{figure}

\subsection{Comentários}

Em um determinado \gls{slide}, permite adicionar observações sobre um ponto específico através de uma caixa de inserção de texto. O comentário será exibido ao lado do \gls{slide} toda vez que este for selecionado. Útil para quem está trabalhando no documento de forma colaborativa (em equipe).

\begin{figure}[H]
    \centering
    \includegraphics[width=.39\textwidth]{/slides/inserir/Imagem 14.png}
    \caption{Guia Comentários}
\end{figure}
