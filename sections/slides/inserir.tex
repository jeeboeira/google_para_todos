% sections/slides/inserir.tex
% !TeX root = ../../main.tex

\section{Inserir}

A aba “Inserir”, no menu principal da barra superior, contém opções para adição de novos elementos aos slides, tais como: imagens, vídeos, formas, texto, tabelas, etc. Abaixo, segue uma breve explicação sobre como inserir cada um desses elementos:

\subsection{Imagem}

Utilizada para colar fotos e figuras no slide. Pode ser inserida de diversas formas: através de \textbf{upload} de arquivos do computador (mídias locais), pesquisas na web, importação de outros aplicativos em nuvem como o \textbf{Google Fotos} e o \textbf{Google Drive}, utilizando a câmera em dispositivos compatíveis e até mesmo por URL (\textbf{link}).

\begin{figure}[H]
    \centering
    \includegraphics[width=.39\textwidth]{/slides/inserir/Imagem 1.png}
    \caption{}
\end{figure}

\subsection{Caixa de texto}

Gera uma área retangular no slide para digitação. Pode ser redimensionada pelo usuário, mas ajusta-se automaticamente ao escrever palavras, frases ou textos.

\begin{figure}[H]
    \centering
    \includegraphics[width=.39\textwidth]{/slides/inserir/Imagem 2.png}
    \caption{}
\end{figure}

\subsection{Forma}

Permite posicionar figuras geométricas ou símbolos, como círculos, quadrados, setas, balões de fala, estrelas, entre outros. Pode ser utilizado para destacar partes do slide ou montar desenhos simples.

\begin{figure}[H]
    \centering
    \includegraphics[width=.39\textwidth]{/slides/inserir/Imagem 3.png}
    \caption{}
\end{figure}

\subsection{Diagrama}

Insere modelos de organização representativos: fluxogramas, hierarquias e processos, por exemplo. Muito utilizados para ilustrar projetos e esquematizar planejamentos diversos

\begin{figure}[H]
    \centering
    \includegraphics[width=.39\textwidth]{/slides/inserir/Imagem 4.png}
    \caption{}
\end{figure}

\subsection{Tabela}

\begin{figure}[H]
    \centering
    \includegraphics[width=.39\textwidth]{/slides/inserir/Imagem 5.png}
    \caption{}
\end{figure}

\subsection{Gráfico}

Permite adicionar gráficos para ilustrar dados de maneira visual. Podem ser incluídos das seguintes formas: “Colunas”, “Linhas”, “Pizza” e “Barras”. Também é possível importar um gráfico do Google Planilhas (recomendamos a leitura do capítulo sobre essa ferramenta para mais informações).

\begin{figure}[H]
    \centering
    \includegraphics[width=.39\textwidth]{/slides/inserir/Imagem 6.png}
    \caption{}
\end{figure}

\subsection{Word Art}

Gera textos com estilizações diferentes. É geralmente utilizada em títulos. O texto pode ter cores, bordas e formatos personalizados. Na aba “Opções de formatação”, você pode ainda ajustar alguns atributos de sombreamento, reflexo e rotação, por exemplo.

\begin{figure}[H]
    \centering
    \includegraphics[width=.39\textwidth]{/slides/inserir/Imagem 7.png}
    \caption{}
\end{figure}

\subsection{Linha}

Desenha linhas retas ou curvas no slide. Há também algumas outras opções mais personalizadas, mas essas são as mais utilizadas. É um recurso interessante que pode ser usado para ligar elementos ou destacar algo no slide.

\begin{figure}[H]
    \centering
    \includegraphics[width=.39\textwidth]{/slides/inserir/Imagem 8.png}
    \caption{}
\end{figure}

\subsection{Vídeo}

Permite adicionar vídeos ao slide. Podem ser importados através de uma pesquisa direta no YouTube, via URL (link de um vídeo da internet) ou Google Drive (vídeo salvo em nuvem na sua conta Google).

\begin{figure}[H]
    \centering
    \includegraphics[width=.39\textwidth]{/slides/inserir/Imagem 9.png}
    \caption{}
\end{figure}

\subsection{Link}

A partir de um texto previamente selecionado, cria um atalho (hiperlink) que redireciona para outro slide, site ou documento.

\begin{figure}[H]
    \centering
    \includegraphics[width=.39\textwidth]{/slides/inserir/Imagem 10.png}
    \caption{}
\end{figure}

\subsection{Animação}

Permite adicionar animações de movimento em textos, imagens ou formas, por exemplo: um texto que surge do canto inferior quando o slide é apresentado.

\begin{figure}[H]
    \centering
    \includegraphics[width=.39\textwidth]{/slides/inserir/Imagem 11.png}
    \caption{}
\end{figure}

\subsection{Números de slide}

Adiciona numeração automática para as páginas de slide criadas (1,2,3...).

\begin{figure}[H]
    \centering
    \includegraphics[width=.39\textwidth]{/slides/inserir/Imagem 12.png}
    \caption{}
\end{figure}

\subsection{Novo slide}

Também pode ser “encurtado” pelo comando \tecla{CTRL + M}. Adiciona um novo slide vazio ao final da sua apresentação.

\begin{figure}[H]
    \centering
    \includegraphics[width=.39\textwidth]{/slides/inserir/Imagem 13.png}
    \caption{}
\end{figure}

\subsection{Comentários}

Em um determinado slide, permite adicionar observações sobre um ponto específico através de uma caixa de inserção de texto. O comentário será exibido ao lado do slide toda vez que este for selecionado. Útil para quem está trabalhando no documento de forma colaborativa (em equipe).

\begin{figure}[H]
    \centering
    \includegraphics[width=.39\textwidth]{/slides/inserir/Imagem 14.png}
    \caption{}
\end{figure}
