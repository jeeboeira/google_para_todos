% sections/slides/extensao.tex
% !TeX root = ../../main.tex

\section{Extensões}

Erros e potenciais melhorias são frequentemente explorados por usuários de um sistema. Esses testes e avaliações fazem com que alguns desenvolvedores criem ``Extensões'' compatíveis para integrar sistemas incompletos, fornecendo novas funcionalidades às aplicações, como no caso do \textbf{Google Apresentações}.

Para utilizar esses recursos, ainda no menu superior principal do \textbf{Google Apresentações}, clique sobre a \gls{aba} ``Extensões'' e, depois, em ``Complementos''. Serão exibidas três opções:

\begin{itemize}
  \item \textbf{Instalar complemento}: ao clicar nessa opção, será aberta uma nova janela. Você pode buscar pelo complemento desejado através da \gls{barrapesquisa} ``Pesquisar \gls{aplicativos}''. Ao encontrar a extensão desejada, basta clicar em ``Instalar'', na extremidade inferior do respectivo \textbf{\gls{card}}, abaixo do nome da extensão;
  \item \textbf{Gerenciar complementos}: nesta \gls{aba} você pode visualizar os complementos baixados e, para remover, basta selecionar os três pontos ao lado do item e clicar em ``Desinstalar'';
  \item \textbf{Ver complementos}: exibe os complementos disponíveis para utilização no respectivo documento.
\end{itemize}
