% sections/slides/dicas_e_boas_praticas.tex
% !TeX root = ../../main.tex

\section{Dicas e Boas Práticas}
\subsection{Principais Atalhos}

\begin{table}[h!]
\centering
\begin{tabular}{|l|l|}
\hline
Nova apresentação & Ctrl + N \\ \hline
Abrir apresentação & Ctrl + O \\ \hline
Salvar & Ctrl + S \\ \hline
Imprimir apresentação & Ctrl + P \\ \hline
Recortar & Ctrl + X \\ \hline
Copiar & Ctrl + C \\ \hline
Colar & Ctrl + V \\ \hline
Desfazer & Ctrl + Z \\ \hline
Refazer & Ctrl + Y ou Ctrl + Shift + Z \\ \hline
Inserir link & Ctrl + K \\ \hline
Copiar formatação & Ctrl + Alt + C \\ \hline
Colar formatação & Ctrl + Alt + V \\ \hline
Selecionar tudo & Ctrl + A \\ \hline
Pesquisar na apresentação & Ctrl + F \\ \hline
\end{tabular}
\end{table}

\subsection{Atalhos para Apresentação}

\begin{table}[h!]
\centering
\begin{tabular}{|l|l|}
\hline
Iniciar apresentação do início & Ctrl + F5 \\ \hline
Iniciar apresentação do slide atual & Ctrl + Shift + F5 \\ \hline
Sair da apresentação & Esc \\ \hline
Avançar/voltar slide & → / ← ou Page Down / Page Up \\ \hline
Acessar todos os atalhos & Ctrl + / \\ \hline
\end{tabular}
\end{table}

\subsection{Dicas e Boas Práticas}

\begin{itemize}
  \item Escolha um tema claro e minimalista para suas apresentações, isso permite que o texto - o conteúdo mais importante da sua apresentação - seja exibido de forma clara e acessível ao público;
  \item Use cores neutras no fundo (branco, cinza claro, bege) para dar o devido destaque à textos e imagens;
  \item Escreva pouco texto em cada slide: destaque apenas os pontos principais. Busque abordar os detalhes e aprofundamentos para a sua fala, tornando a leitura menos cansativa;
  \item Utilize fontes legíveis para clareza: Arial, Roboto ou Open Sans, por exemplo. O tamanho mínimo recomendado para a fonte é de 20 pt;
  \item Utilize alinhamento e espaçamento adequados: selecione o texto → guia “Formatar” → “Espaçamento entre linhas e parágrafos” → opção “1,5”;
  \item Lembre-se de sempre citar e referenciar a fonte de todas as imagens e citações utilizadas em sua apresentação.
  \item Utilize numeração nos slides para melhor organização e orientação.
  \item Verifique a ortografia da apresentação através do menu “Ferramentas” → “Verificação ortográfica”. Isso permite identificar erros gramaticais e corrigi-los rapidamente;
\end{itemize}
