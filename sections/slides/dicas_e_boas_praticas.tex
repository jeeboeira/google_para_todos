% sections/slides/dicas_e_boas_praticas.tex
% !TeX root = ../../main.tex

\section{Dicas e Boas Práticas}
\subsection{Principais Atalhos}

\begin{table}[!ht]
  \centering
  \begin{tabular}{@{}lll@{}}
    \toprule
    \textbf{Ação} & \textbf{Atalho (Windows)} & \textbf{Atalho (macOS)} \\ 
    \midrule
      Nova apresentação & Ctrl + N & \cmd + N \\
      Abrir apresentação & Ctrl + O & \cmd + O \\
      Salvar & Ctrl + S & \cmd + S \\
      Imprimir apresentação & Ctrl + P & \cmd + P \\
      Recortar & Ctrl + X & \cmd + X \\
      Copiar & Ctrl + C & \cmd + C \\
      Colar & Ctrl + V & \cmd + V \\
      Desfazer & Ctrl + Z & \cmd + Z \\
      Refazer & Ctrl + Y ou Ctrl + Shift + Z & \cmd + Y ou \cmd + Shift + Z \\
      Inserir link & Ctrl + K & \cmd + K \\
      Copiar formatação & Ctrl + Alt + C & \cmd + Option + C \\
      Colar formatação & Ctrl + Alt + V & \cmd + Option + V \\
      Selecionar tudo & Ctrl + A & \cmd + A \\
      Pesquisar na apresentação & Ctrl + F & \cmd + F \\
    \bottomrule
  \end{tabular}
  \caption{Principais atalhos do Google Apresentações.}
  \label{tab:atalhos_apresentacoes}
\end{table}

\subsection{Atalhos para Apresentação}

\begin{table}[!ht]
  \centering
  \begin{tabular}{@{}lll@{}}
    \toprule
    \textbf{Ação} & \textbf{Atalho (Windows)} & \textbf{Atalho (macOS)} \\ 
    \midrule
      Iniciar apresentação do início & Ctrl + F5 & \cmd + Shift + Enter \\
      Iniciar apresentação do slide atual & Ctrl + Shift + F5 & \cmd + Enter \\
      Sair da apresentação & Esc & Esc \\
      Avançar/voltar slide & → / ← ou Page Down / Up & → / ← ou Page Down / Up \\
      Acessar todos os atalhos & Ctrl + / & \cmd + / \\
    \bottomrule
  \end{tabular}
  \caption{Atalhos para o modo de apresentação.}
  \label{tab:atalhos_modo_apresentacao}
\end{table}

\subsection{Dicas e Boas Práticas}

\begin{itemize}
  \item Escolha um tema claro e minimalista para suas apresentações, isso permite que o texto - o conteúdo mais importante da sua apresentação - seja exibido de forma clara e acessível ao público;
  \item Use cores neutras no fundo (branco, cinza claro, bege) para dar o devido destaque à textos e imagens;
  \item Escreva pouco texto em cada slide: destaque apenas os pontos principais. Busque abordar detalhes e explicações em sua fala, tornando a leitura menos cansativa;
  \item Utilize fontes legíveis para clareza: Arial, Roboto ou Open Sans, por exemplo. O tamanho mínimo recomendado para a fonte é de 20 pt;
  \item Utilize alinhamento e espaçamento adequados: selecione o texto → guia “Formatar” → “Espaçamento entre linhas e parágrafos” → opção “1,5”;
  \item Lembre-se de sempre citar e referenciar a fonte de todas as imagens e citações utilizadas em sua apresentação;
  \item Utilize numeração nos slides para melhor organização e orientação;
  \item Verifique a ortografia da apresentação através do menu “Ferramentas” → “Verificação ortográfica”. Isso permite identificar erros gramaticais e corrigi-los rapidamente;
\end{itemize}
