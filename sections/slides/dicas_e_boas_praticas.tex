% sections/slides/dicas_e_boas_praticas.tex
% !TeX root = ../../main.tex

\section{Dicas e Boas Práticas}
\subsection{Principais Atalhos}

\begin{table}[h!]
\centering
\begin{tabular}{|l|l|}
\hline
Nova apresentação & \tecla{Ctrl + N} \\ \hline
Abrir apresentação & \tecla{Ctrl + O} \\ \hline
Salvar & \tecla{Ctrl + S} \\ \hline
Imprimir apresentação & \tecla{Ctrl + P} \\ \hline
Recortar & \tecla{Ctrl + X} \\ \hline
Copiar & \tecla{Ctrl + C} \\ \hline
Colar & \tecla{Ctrl + V} \\ \hline
Desfazer & \tecla{Ctrl + Z} \\ \hline
Refazer & \tecla{Ctrl + Y} ou \tecla{Ctrl + Shift + Z} \\ \hline
Inserir link & \tecla{Ctrl + K} \\ \hline
Copiar formatação & \tecla{Ctrl + Alt + C} \\ \hline
Colar formatação & \tecla{Ctrl + Alt + V} \\ \hline
Selecionar tudo & \tecla{Ctrl + A} \\ \hline
Pesquisar na apresentação & \tecla{Ctrl + F} \\ \hline
\end{tabular}
\end{table}

\subsection{Atalhos para Apresentação}

\begin{table}[h!]
\centering
\begin{tabular}{|l|l|}
\hline
Iniciar apresentação do início & \tecla{Ctrl + F5} \\ \hline
Iniciar apresentação do slide atual & \tecla{Ctrl + Shift + F5} \\ \hline
Sair da apresentação & \tecla{Esc} \\ \hline
Avançar/voltar slide & \tecla{→} / \tecla{←} ou \tecla{Page Down} / \tecla{Page Up} \\ \hline
Acessar todos os atalhos & \tecla{Ctrl + /} \\ \hline
\end{tabular}
\end{table}

\subsection{Dicas e Boas Práticas}

\begin{itemize}
  \item Escolha um tema claro e minimalista para suas apresentações, isso permite que o texto - o conteúdo mais importante da sua apresentação - seja exibido de forma clara e acessível ao público;
  \item Use cores neutras no fundo (branco, cinza claro, bege) para dar o devido destaque à textos e imagens;
  \item Escreva pouco texto em cada slide: destaque apenas os pontos principais. Busque abordar os detalhes e aprofundamentos para a sua fala, tornando a leitura menos cansativa;
  \item Utilize fontes legíveis para clareza: Arial, Roboto ou Open Sans, por exemplo. O tamanho mínimo recomendado para a fonte é de 20 pt;
  \item Utilize alinhamento e espaçamento adequados: selecione o texto → guia “Formatar” → “Espaçamento entre linhas e parágrafos” → opção “1,5”;
  \item Lembre-se de sempre citar e referenciar a fonte de todas as imagens e citações utilizadas em sua apresentação.
  \item Utilize numeração nos slides para melhor organização e orientação.
  \item Verifique a ortografia da apresentação através do menu “Ferramentas” → “Verificação ortográfica”. Isso permite identificar erros gramaticais e corrigi-los rapidamente;
\end{itemize}
