% sections/slides/organizar.tex
% !TeX root = ../../main.tex
\section{Menu Organizar}

O menu “Organizar” também está localizado na barra superior principal do Google Apresentações. Nele, você irá encontrar algumas funcionalidades úteis para adequar visualmente os elementos em seus slides.

\begin{figure}[H]
    \centering
    \includegraphics[width=.39\textwidth]{/slides/organizar/Imagem 1.png}
    \caption{}
\end{figure}

\subsection{Ordenar}

A função “Ordenar” controla a posição dos objetos na profundidade do slide, ou seja, os elementos que ficam na frente e atrás na visualização. É útil para revelar elementos importantes sobrepostos. 

\subsection{Alinhar}

A função “Alinhar” serve para posicionar dois ou mais objetos em uma mesma linha, vertical ou horizontalmente. É usada para deixar os elementos visualmente alinhados entre si ou em relação à área total do slide. Você seleciona um ou mais objetos e escolhe uma das opções de alinhamento (“Esquerda”, “Centralizar” e “Direita” ou “Início”, “Meio” e “Fim”), o sistema irá ajustar automaticamente a posição dos itens no slide com base na sua escolha.

\subsection{Distribuir}

A função “Distribuir” serve para igualar o espaçamento entre objetos, organizando-os de forma equilibrada no slide. Funciona apenas com dois ou mais objetos selecionados. Evita a presença de espaços irregulares quando você posiciona elementos manualmente. O sistema calcula automaticamente a distância entre os elementos e os distribui de maneira uniforme.

\subsection{Centro de página}

A função “Centro de página” é utilizada para posicionar um objeto no centro do slide automaticamente, sem a necessidade de ajustes manuais. Ela garante que o item fique centralizado vertical e/ou horizontalmente no slide.

\subsection{Girar}

O comando Girar ajusta a rotação e orientação do objeto. Pode ser aplicado a imagens, formas, caixas de texto e elementos gráficos. Pode-se, por exemplo, girar o elemento 90º graus à direita, 90º graus à esquerda ou inverter seu conteúdo verticalmente (de cabeça para baixo).

\begin{dica}
    Você também pode utilizar o ícone de “rotação” acima do objeto selecionado e arrastá-lo para a esquerda ou direita, rotacionando o objeto conforme desejado.
\end{dica}

\subsection{Imagem}

Após selecionar uma imagem, a função "Definir imagem como plano de fundo", dentro da aba de “Imagens”, em “Organizar”, permite definir a imagem como plano de fundo do slide.

\subsection{Agrupar}

A função Agrupar serve para unir dois ou mais objetos selecionados, como imagens, formas e caixas de texto em um único “bloco” ou “contêiner”. Após agrupados, todos os objetos podem ser movidos, redimensionados ou girados juntos, mantendo o posicionamento relativo entre eles.

\subsection{Desagrupar}

A função “Desagrupar” é o inverso da “Agrupar”: ela separa os objetos que estavam agrupados, permitindo manipulá-los individualmente outra vez.
