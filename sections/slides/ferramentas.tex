% sections/slides/ferramentas.tex
% !TeX root = ../../main.tex

\section{Ferramentas}

A aba ``Ferramentas'' contempla uma série de funções gerais que podem ser úteis para o seu projeto. Abaixo, segue descrição das mesmas:

\textbf{Ortografia}

Quando uma palavra é digitada incorretamente, ela aparecerá sublinhada em vermelho para o usuário. Ao clicar sobre essa palavra, o Google Apresentações exibirá sugestões de correção. Também é possível realizar a verificação manual acessando: Ferramentas -> Ortografia -> Verificação ortográfica.

\textbf{[Imagem 1]} ortografia - ferramentas.png

Nesse menu, o usuário pode \textbf{aceitar} ou \textbf{ignorar} as sugestões apresentadas:

\textbf{[Imagem 2]} ortografia - exemplo.png

A opção ``Mostrar sugestões de ortografia'' controla se as sugestões aparecerão automaticamente em um pequeno \textbf{popup} acima da palavra incorreta.

A função ``Dicionário pessoal'' abre uma janela que permite adicionar palavras personalizadas à uma lista, de modo que todas as suas ocorrências não sejam identificadas como erros, ou seja, sublinhadas em vermelho.

\textbf{[Imagem 3]} exemplo dicio.png

\textbf{Objetos Vinculados}

Ao inserir, por exemplo, um \textbf{gráfico vinculado a uma planilha}, o objeto permanece conectado ao arquivo original do Google Planilhas. Assim, qualquer \textbf{edição feita na planilha} poderá ser \textbf{atualizada automaticamente} na apresentação. 
Basta clicar sobre o gráfico e selecionar \textbf{``Atualizar''} para sincronizar os dados.

\textbf{[Imagem 4]} gráfico vinculado.png

\textbf{[Imagem 5]} exemplo atualizado.png

\textbf{Digitar as anotações do apresentador}

Com essa ferramenta, é possível ditar as anotações do slide usando o microfone. À medida que o usuário fala, o sistema converte a fala em texto. Ao clicar no ícone do microfone, ele ficará vermelho, indicando que já está gravando e reconhecendo sua voz.

\textbf{[Imagem 6]} digita voz.png

\textbf{Configuração de notificação}

Permite personalizar o tipo de notificação que o usuário deseja receber sobre comentários e atividades do projeto atual: ``Todos os comentários'', ``Comentários para você'' ou ``Nenhum''.

\textbf{[Imagem 7]} config notif.png

\textbf{Preferências}

\begin{itemize}
  \item \textbf{Gerais}: nesta seção, é possível ativar ou desativar determinadas opções de formatação e comportamento padrão do Google Apresentações.
\end{itemize}

\textbf{[Imagem 8]} pref geral.png

\begin{itemize}
  \item \textbf{Substituições}: nesta seção, o usuário pode alterar ou remover substituições pré-definidas --- por exemplo, converter automaticamente ``1/4'' em ``¼''.
\end{itemize}

\textbf{[Imagem 9]} pref substituicoes.png

\textbf{Acessibilidade}

Na aba ``Acessibilidade'', temos acesso a alguns recursos de acessibilidade, como o suporte a \textbf{braille}, compatibilidade com lupa e notificações quando alguém acessa ou sai do documento.

\textbf{Painel de atividades}

O ``Painel de atividades'' reúne informações detalhadas sobre o projeto, incluindo:

\begin{itemize}
  \item Com quem o arquivo está compartilhado e suas permissões.
  \item Tendências de visualização e comentários.
  \item Histórico de compartilhamento.
  \item Configurações de preferências de atividade.
\end{itemize}

\textbf{[Imagem 10]} painel atividade.png
