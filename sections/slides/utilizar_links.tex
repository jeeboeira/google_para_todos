% sections/slides/utilizar_links.tex
% !TeX root = ../../main.tex

\section{Utilizar Links}
\subsection{Adicionar Links}

Para adicionar \textbf{links} que direcionam para algum site ou endereço da web, crie uma caixa de texto (menu ``Inserir'' \textgreater{} ``Caixa de texto'') e cole o link dentro dela (botão direito do mouse \textgreater{} ``Colar'' ou \tecla{Ctrl + V}). Selecione o texto inserido e, na barra ``Inserir'', clique na opção ``Link'' - ou utilize o comando \tecla{Ctrl + K}. Clique sobre o texto e selecione o ícone de lápis ``Editar link''. Na primeira caixa de texto, insira como o \textbf{link} deverá aparecer para o usuário visualmente, por exemplo, ``Google''. Na caixa abaixo, insira o endereço web (\textbf{url}) ao qual o usuário será levado quando clicar no texto do \textbf{slide}, neste caso, ``https://www.google.com/''. Por fim, clique em ``Aplicar'' para salvar as modificações.

\textbf{IMAGEM 9 - INSERIR LINK E SUBSTITUIR}

Ao apresentar o slide em Modo de Apresentação, o link se torna clicável, direcionando-o automaticamente para a página ao executar a ação.

\subsection{Remover Links}

Para remover \textbf{links} de um texto, clique sobre o mesmo (que deve estar destacado em azul e sublinhado) e selecione a opção ``Excluir''. A formatação será alterada para o modo padrão e o texto será mantido, sem direcionamento para endereços \textbf{web}.

\subsection{Link entre slides}

Para criar links internos que alternam entre slides, é necessário, primeiramente, selecionar a imagem, palavra ou objeto que poderá ser clicado na apresentação. Depois disso, utilize o atalho \tecla{Ctrl + K} ou abra a guia ``Inserir'' e clique sobre ``Link''. Aberta a caixa de seleção, clique na opção ``Slides nesta apresentação'' e selecione a opção que possui o nome do \textbf{slide} para o qual deseja direcionar a apresentação (por padrão, ``Slide 1'', ``Slide 2'', etc.).

\textbf{IMAGEM 8 LINK ENTRE SLIDES}
