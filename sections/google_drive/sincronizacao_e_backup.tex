% sections/google_drive/sincronizacao_e_backup.tex
% !TeX root = ../../main.tex

\section{Backup e Sincronização}

\subsection{Backup}
O ato de realizar backups é muito importante para evitar perdas de arquivos, pois com ele, caso algo aconteça com o original, sempre terá uma “cópia de segurança”.

No Google Drive, há mais de uma maneira de realizá-lo. A mais simples é copiar o \gls{arquivo} desejado. Para isso, basta clicar nele com o botão direito, ou nos “três pontinhos”, e depois clicar na opção “Fazer uma cópia”, gerando um \gls{arquivo} idêntico ao original. Esta ação não pode ser feita em pastas.

A outra é utilizando o Google Drive para \gls{desktop}. Desta forma, pode fazer cópias dos arquivos de seu computador para o Drive, podendo acessar a cópia diretamente de seu computador, tanto no modo \textbf{Stream} quanto no \textbf{Mirror}, porém, com a diferença de que no \textbf{Stream} é necessária uma conexão com a internet, enquanto no \textbf{Mirror} não.

\subsection{Sincronização}
A sincronização de arquivos no Drive é feita por meio de sua versão \gls{desktop}, permitindo que acesse seus arquivos salvos no Drive diretamente de seu computador. Esses arquivos podem ser salvos em modo Stream (streaming) ou Mirror (espelhamento), esta diferença já foi explicada de forma resumida na seção sobre os diferenciais da versão \gls{desktop} do Google Drive. Para melhorar o entendimento de suas diferenças, confira a tabela comparativa na figura \ref{fig:sincronizacao_e_backup}.

\begin{figure}[!htbp]
	\centering
	\includegraphics[width=1\textwidth]{/google_drive/imagem_7.png}
	\caption{Tabela comparativa entre as opções de sincronização do Google Drive, Fonte: \cite{google2025streammirror}}
	\label{fig:sincronizacao_e_backup}
\end{figure}

Clicando no ícone de Configurações (“engrenagem”) e depois na \gls{aba} "Preferências", o usuário pode ver e alterar seu modo de sincronização ativado, clicando novamente em Configurações (dentro da janela de Preferências). É possível alterar configurações sobre o limite de uso do cache, escolher entre usar uma unidade de armazenamento ou uma pasta para a sincronização, alterar as configurações em relação ao Google Fotos, dentre outras opções.

Também é possível ver: seus arquivos disponíveis de forma off-line, lista de erros, página de ajuda, sobre, enviar \gls{feedback} ou sair da conta, ao clicar no ícone de Configurações na \Gls{homepage}.

Já na parte esquerda da tela, é possível ver o botão "Abrir a pasta do Drive" que, como o próprio nome já diz, abre a pasta correspondente a sincronização do Google Drive - isto é, o local onde você pode acessar seus arquivos do Drive no computador.

Além disso, é possível ver sua atividade de sincronização, ou seja, quais arquivos foram sincronizados, estão sendo sincronizados ou ainda não foram sincronizados. É possível também ver suas notificações. Tais como: atualizações e alterações em arquivos compartilhados.
