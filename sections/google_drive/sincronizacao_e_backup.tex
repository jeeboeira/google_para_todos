% sections/google_drive/sincronizacao_e_backup.tex
% !TeX root = ../../main.tex

\section{Backup e Sincronização}

\subsection{Backup}
A realização de backups é extremamente importante para evitar perdas de arquivos. 
Com ele você pode manter uma “cópia de segurança” dos seus arquivos, 
caso algo indesejado aconteça com os originais.

No Google Drive, há mais de uma maneira de realizá-lo. A mais simples é copiar o 
\gls{arquivo} desejado. Para isso, basta clicar nele com o botão direito ou no ícone dos 
“três pontinhos”. Depois, clique na opção “Fazer uma cópia”, gerando um \gls{arquivo} 
idêntico ao original. Esta ação não pode ser feita em pastas.

A outra forma é utilizando o Google Drive para \gls{desktop}. Desta forma, pode fazer cópias 
dos arquivos de seu computador para o Drive. Deste modo, você poderá acessar a cópia 
diretamente de seu computador, tanto no modo \textbf{Stream} quanto no modo \textbf{Mirror}, 
porém, com a diferença de que no \textbf{Stream} é necessária conexão com a internet, 
enquanto no \textbf{Mirror}, não. Leia a seção \textbf{3.14 Diferenciais da versão 
desktop do Google Drive}, para mais informações sobre estes modos. Uma tabela comparativa
entre os dois modos será apresentada na próxima seção.

\subsection{Sincronização}
A sincronização de arquivos no Google Drive é realizada por meio da versão \gls{desktop}, 
que permite acessar os arquivos do Drive diretamente pelo computador. Como apresentado
anteriormente, os arquivos podem ser armazenados de duas maneiras: Stream (transmissão) 
ou Mirror (espelhamento). Para facilitar a compreensão das diferenças entre eles, 
consulte a tabela comparativa mostrada na figura \ref{fig:sincronizacao_e_backup} 
a seguir.

\begin{figure}[!htbp]
	\centering
	\includegraphics[width=1\textwidth]{/google_drive/imagem_7.png}
	\caption{Tabela comparativa entre as opções de sincronização do Google Drive, Fonte: \cite{google2025streammirror}}
	\label{fig:sincronizacao_e_backup}
\end{figure}

Clicando no ícone de Configurações (“engrenagem”) e depois na \gls{aba} "\textbf{Preferências}, o 
usuário pode ver e alterar seu modo de sincronização ativado, basta clicar novamente em 
Configurações (dentro da janela de Preferências). É possível alterar configurações 
sobre o limite de uso do cache, escolher entre usar uma unidade de armazenamento 
ou uma pasta para a sincronização, alterar as configurações em relação ao Google 
Fotos, dentre outras funcionalidade.

Ao clicar no ícone de "\textbf{Configurações}" da \gls{homepage}, também é possível ver: seus 
arquivos disponíveis de forma off-line, lista de erros, página de ajuda, sobre, 
enviar \gls{feedback} ou sair da conta.

Já na parte esquerda da tela, é possível ver o botão "\textbf{Abrir a pasta do Drive}" que, como 
o próprio nome já diz, abre a pasta correspondente a sincronização do Google Drive, isto 
é, o local onde você pode acessar seus arquivos do Drive no computador.

Além disso, é possível ver sua \textbf{atividade de sincronização}, ou seja, quais 
arquivos "foram sincronizados", "estão sendo sincronizados" ou "ainda não foram 
sincronizados". É possível, também, ver suas \textbf{notificações}, tais como: 
atualizações e alterações em arquivos compartilhados.
