% sections/google_drive/introducao.tex
% !TeX root = ../../../main.tex

\section{Introdução ao Google Drive}
O Google Drive, ou apenas Drive, é uma ferramenta de armazenamento e 
compartilhamento de arquivos na \gls{nuvem} desenvolvida pela Google. Apresentado pela 
primeira vez em 2012, ele permite ao usuário salvar arquivos de qualquer tipo 
(documentos, imagens, vídeos, etc) e também acessá-los de qualquer lugar.

Com 15GB de armazenamento em sua versão gratuita (que são compartilhados com 
GMail e Google Fotos) possui planos de assinatura que permitem ao usuário obter 
maior espaço de armazenamento por meio de valores variados.

É disponibilizado não somente em versão de website, como também possui uma 
versão para \textit{\gls{desktop}} (Windows e MacOS), além de um aplicativo \gls{mobile} 
(iOS e Android).

\section{\Gls{homepage} e menu lateral}
Após fazer o \gls{login} em sua conta Google para entrar no Drive, você verá a 
seguinte página:

\begin{figure}[htbp]
    \centering
    \includegraphics[width=1\textwidth]{/google_drive/imagem_5.png}
    \caption{\Gls{homepage} do Google Drive, captura de tela/Vinícius Maurer}
\end{figure}
	
Da direita para a esquerda na parte superior da aplicação pode-se ver:

\begin{itemize}
    \item Ícone do perfil do usuário: este botão permite acessar sua conta ou trocá-la;
    \item Ícone do Google apps ou "quadrado de pontinhos": permite ao usuário acessar os outros \gls{aplicativos} da Google mais facilmente;
    \item Botão de configurações (a engrenagem);
    \item Botão de suporte;
    \item \Gls{barrapesquisa}, com ícone de pesquisa avançada ao lado direito.
\end{itemize}

Há ainda  a possibilidade de haver um botão para o Gemini, a inteligência artificial da Google, caso o mesmo esteja ativo na conta.

Já no lado esquerdo da tela pode-se observar:

\begin{itemize}
    \item Botão "Novo": para fazer \gls{upload}/criação de arquivos/pastas;
    \item Botão "Pessoal": leva para a \Gls{homepage} do Drive;
    \item Botão "Atividades": presente somente em contas Google \Gls{workspace}, mostra um registro do que foi feito/por quem foi, a respeito de alterações, permissões, etc.;
    \item Botão "Espaços de trabalho": presente somente em contas Google \Gls{workspace}, permite montar agrupamentos de arquivos/pastas sem alterar a localização real deles;
    \item Botão "Computadores": não presente na imagem, referente ao \gls{backup} e sincronização da versão \gls{desktop} do Google Drive; 
    \item Botão "Meu Drive": mostra o seu Drive, com todas as pastas e arquivos;
    \item Botão "Drives compartilhados": presente somente a contas Google \Gls{workspace}, mostra um drive comum de uma organização, que não pertence a um indivíduo;
    \item Botão "Compartilhados comigo": mostra arquivos e pastas que foram compartilhadas com o usuário;
    \item Botão "Recentes": mostra pastas e arquivos recentemente acessados;
    \item Botão "Com estrela": mostra os arquivos marcados com uma estrela;
    \item Botão "\Gls{spam}": mostra arquivos de \gls{spam};
    \item Botão "Lixeira": mostra os arquivos excluídos pelo usuário, mas ainda não de maneira permanente;
    \item Botão "Armazenamento": mostra uma relação do uso de seu armazenamento entre os \gls{aplicativos} Google.
\end{itemize}

Também é possível ver no lado direito, uma \gls{aba} com complementos da Google, por 
\textit{\gls{default}} Agenda, Contatos, \Gls{keep} e Tarefas, com um botão de cruz/”mais” para 
colocar complementos adicionais. Esta \gls{aba} pode ser minimizada e estendida ao 
clicar no botão em forma de seta na parte inferior.

\section{Upload de arquivos}
Para fazer o \gls{upload} de pastas ou arquivos, basta clicar no botão "Novo", localizado no menu lateral. Ao fazer isso, aparecerá um menu que permite ao usuário enviar arquivos, pastas, criar uma pasta no seu ambiente do drive, ou então criar arquivos dos \gls{aplicativos} padrões da Google, bem como de outros apps que o usuário tenha conectado.

\begin{figure}[H]
    \centering
    \includegraphics[width=.55\textwidth]{/google_drive/imagem_1.png}
    \caption{Menu de \gls{upload} do Google Drive, captura de tela/Vinícius Maurer}
\end{figure}

Basta então selecionar o \gls{arquivo} que se deseja subir e esperar o tempo necessário para o \gls{upload}. Caso o usuário utilize o Google Chrome ou Firefox como navegador, também é possível fazer o envio de arquivos e pastas arrastando-os para o ambiente do Drive e os soltando. Ainda é possível subir arquivos por meio da versão para \textit{\gls{desktop}} do Google Drive (\cite{google2025drive})

\section{Criar pastas}
O uso de pastas é essencial para a melhor utilização do sistema, pois “[...] a 
organização das informações permite que o processo de tomada de decisões seja 
eficaz e rápido, impedindo barreiras e atrasos em diversos processos.” (\cite{bertazzi2022digitais}).

Por isso é necessário saber como criá-las. Para isso, clique no botão "Novo" e no menu suspenso selecionar a opção "Nova pasta". Essa ação abrirá uma nova janela onde será preciso digitar um nome para pasta, com isso feito, clique em "Criar".

\section{Abrir arquivos}
Para abrir um \gls{arquivo}, basta clicá-lo duas vezes.

“Os arquivos criados com o Google apps são abertos no navegador ou no app para dispositivos móveis. Outros tipos de \gls{arquivo} na sua pasta do Drive são abertos nos apps correspondentes. Por exemplo, o Adobe Reader para arquivos \gls{pdf}.” (\cite{google2025adicionar})

O mesmo vale para entrar em uma pasta.

\section{Mover arquivos}
Para mover um arquivos ou pastas, clique no botão "Mais opções" (ícone de “três pontinhos”) ou com o botão direito do mouse no \gls{arquivo} que você deseja mover. Isto deverá abrir um menu com várias opções, clique ou passe o mouse sobre a opção "Organizar" e clique em "Mover".

\begin{figure}[H]
    \centering
    \includegraphics[width=.8\textwidth]{/google_drive/imagem_2.png}
    \caption{Menu de ações adicionais para um \gls{arquivo} do Google Drive, captura 
    de tela/Vinícius Maurer}
\end{figure}

Com isso é aberta uma janela sobreposta, nela você pode escolher o local para onde quer mover o \gls{arquivo}. Com o local selecionado, basta clicar no botão "Mover".

\section{Excluir arquivos}
Para excluir arquivos ou pastas, clique com o botão direito mouse ou no botão "Mais opções" (ícone de “três pontinhos”) para abrir as opções adicionais, e escolher a opção "Mover para a lixeira", e novamente "Mover para a lixeira" para excluir, ou "Cancelar" para cancelar. Alternativamente o usuário pode apertar a tecla \textbf{Delete} no teclado com o \gls{arquivo} selecionado, o que também abrirá uma janela pedindo confirmação para mover para a lixeira.

“O \gls{arquivo} permanecerá na lixeira por 30 dias antes de ser excluído  automaticamente.”(\cite{google2025adicionar}) Ao colocar o \gls{arquivo} na lixeira, se ele for seu ele será excluído do seu Drive e uma cópia será feita com quem você compartilhou, caso ele seja de outra pessoa, ele será apenas removido do seu Drive. (\cite{google2025adicionar})

Para excluir em definitivo, abra a lixeira clicando em "Lixeira" no menu lateral, nela  clique em "Esvaziar lixeira" para excluir definitivamente o que estiver lá, se quiser excluir um item em específico, clique nele, ou então nos “três pontinhos”, e clicar no ícone de "Excluir definitivamente". Há também a opção de "Restaurar" para restaurar um \gls{arquivo}.

\begin{figure}[H]
    \centering
    \includegraphics[width=1\textwidth]{/google_drive/imagem_3.png}
    \caption{Lixeira do Drive, captura de tela/Vinícius Maurer}
\end{figure}

\section{Baixar arquivos}
Para baixar um \gls{arquivo}/pasta, clique com o botão direito do mouse ou nos “três pontinhos” e selecione "Baixar", ou então clique diretamente no ícone de \gls{download} mostrado ao selecionar (ou passar o mouse por cima) o \gls{arquivo}.

\section{Renomear arquivos}
Para renomear um \gls{arquivo}/pasta, clique com o botão direito do mouse ou nos “três pontinhos” e selecione "Renomear", com isso, basta digitar o novo nome e confirmar clicando em "Ok".

Ou então clique diretamente no ícone de renomear (simbolizado por uma caneta), mostrado ao selecionar, ou ao passar o mouse por cima do \gls{arquivo}.

\section{Marcar arquivos com estrela}
Para marcar um \gls{arquivo}/pasta com estrela, clique com o botão direito do mouse ou nos “três pontinhos” e no menu "Organizar", selecione a opção \textbf{Adicionar a "Com estrela"}. Todo \gls{arquivo} que o usuário marcar com estrela será mostrado na \gls{aba} "Com estrela".

\section{Pesquisar arquivos}
Para pesquisar um \gls{arquivo} ou pasta, clique na \gls{barrapesquisa} localizada na parte superior da tela e digite o termo que deseja pesquisar. Alternativamente, você pode usar a pesquisa avançada, clicando no ícone no lado direito. Com ela você pode pesquisar pelas opções: "Tipo", "Proprietário", "Com as palavras", "Nome do Item", "Local", "Data de modificação", "Aprovações e assinaturas eletrônicas", "Compartilhado com" e "Acompanhamentos"

\begin{figure}[H]
    \centering
    \includegraphics[width=.9\textwidth]{/google_drive/imagem_4.png}
    \caption{Pesquisa avançada do Drive, captura de tela/Vinícius Maurer}
\end{figure}