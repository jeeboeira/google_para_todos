% sections/google_drive/introducao.tex
% !TeX root = ../../../main.tex

\section{Introdução ao Google Drive}
O Google Drive, ou apenas Drive, é uma ferramenta de armazenamento e 
compartilhamento de arquivos na \gls{nuvem} desenvolvida pela Google. Apresentado pela 
primeira vez em 2012, ele permite ao usuário salvar arquivos de qualquer tipo 
(documentos, imagens, vídeos, etc) e também acessá-los de qualquer lugar.

Com 15GB (gigabytes) de armazenamento em sua versão gratuita (que são compartilhados com 
Gmail e Google Fotos), possui planos de assinatura com valores variados que permitem ao 
usuário ampliar a capacidade máxima de armazenamento em nuvem.

É disponibilizado não somente como um site na web, mas também como uma 
aplicação para \textit{\gls{desktop}} (Windows e MacOS) e um aplicativo \gls{mobile} 
(iOS e Android).

\section{\gls{homepage} e menu lateral}
Após fazer o \gls{login} em sua conta Google para entrar no Drive, você verá a 
seguinte página:

\begin{figure}[htbp]
    \centering
    \includegraphics[width=1\textwidth]{/google_drive/imagem_5.png}
    \caption{Captura de tela exibindo a \gls{homepage} do Google Drive.}
\end{figure}
	
Da direita para a esquerda, na parte superior da aplicação, pode-se ver:

\begin{itemize}
    \item Ícone de \textbf{Perfil do Usuário}: ao clicar sobre este botão, serão exibidas algumas informações 
    da sua conta Google, contas alternativas para mudar o perfil ativo e botões para adicionar uma nova conta, sair da atual ou de todas;
    \item Ícone do \textbf{Google Apps} ou "quadrado de pontinhos": ao ser clicado, abre um atalho para os outros \gls{aplicativos} da Google, facitando o acesso;
    \item Botão de \textbf{Configurações} ou "engrenagem": ao clicar nele, abre um menu com opções de configurações do Drive, como temas, gerenciamento de 
    aplicativos conectados, configurações gerais e de notificação;
    \item Botão de \textbf{Suporte}: abre uma janela com algumas opções de ajuda e diretrizes do Drive;
    \item \textbf{\gls{barrapesquisa}}, com \textbf{ícone de pesquisa avançada} ao lado direito (similar a três barras de volume), que permite a
    aplicação de filtros diversos para pesquisas de arquivos e diretórios mais específicos em seu Drive.
    \item Pode-se encontrar, ainda, um botão em formato de estrela do \textbf{Gemini}, a inteligência artificial da Google, caso o mesmo esteja ativo na conta.
\end{itemize}

Já, no lado esquerdo da tela, pode-se observar:

\begin{itemize}
    \item Botão "\textbf{Novo}": utilizado para a criação de pastas, \gls{upload} de arquivos/pastas e criação de documentos com os aplicativos Google (Documentos, Planilhas, Apresentações, etc);
    \item Botão "\textbf{Pessoal}": direciona o usuário para a \gls{homepage} (página principal) do Drive;
    \item Botão "\textbf{Atividades}": presente somente em contas com assinatura Google \gls{workspace}. Mostra um registro do que foi feito e por quem foi feito, a respeito de alterações, permissões, etc.;
    \item Botão "\textbf{Espaços de trabalho}": presente somente em contas Google \gls{workspace}, permite montar agrupamentos de arquivos/pastas sem alterar a localização real deles;
    \item Botão "\textbf{Computadores}": não presente na imagem, referente ao \gls{backup} e sincronização de arquivos do computador na versão \gls{desktop} do Google Drive; 
    \item Botão "\textbf{Meu Drive}": mostra o seu Drive, com todas as pastas e arquivos carregados;
    \item Botão "\textbf{Drives compartilhados}": presente somente em contas Google \gls{workspace}. Trata-se de um drive comum para membros de um grupo ou organização. Não pertence a um único indivíduo;
    \item Botão "\textbf{Compartilhados comigo}": lista todos os arquivos e pastas que foram compartilhadas com o usuário;
    \item Botão "\textbf{Recentes}": mostra pastas e arquivos recentemente acessados em seu Drive;
    \item Botão "\textbf{Com estrela}": mostra os arquivos marcados com uma estrela;
    
    \begin{dica}
        Você pode adicionar um arquivo ou pasta aos favoritos (com estrela) clicando com o botão direito do mouse sobre o item desejado e selecionando a opção 'Organizar' e 'Adicionar a "Com Estrela"'. 
        Pode-se utilizar, ainda, o atalho \tecla{Ctrl} + \tecla{Alt} + \tecla{S}. Essa prática facilita o acesso rápido a arquivos importantes. 
    \end{dica}

    \item Botão "\textbf{\gls{spam}}": mostra os arquivos classificados como "\gls{spam}". Tratam-se de arquivos com conteúdos indesejados ou potencialmente maliciosos. 
    O Drive realiza essa classificação automaticamente.;
    \item Botão "\textbf{Lixeira}": lista os arquivos excluídos pelo usuário. 
    Para mais informações sobre como excluir arquivos, leia a seção \textbf{3.7 Excluir arquivos}. Estes arquivos serão excluídos permanentemente após 30 dias na lixeira;
    \item Botão "\textbf{Armazenamento}": mostra uma relação com a distribuição de uso do seu armazenamento em nuvem entre os \gls{aplicativos} Google.
\end{itemize}

Também é possível ver, ao lado direito, uma \gls{aba} com complementos da Google, por 
\textit{\gls{default}} (padrão): Agenda, Contatos, \gls{keep} e Tarefas, com um botão de "cruz" (Instalar complementos) para 
adicionar novas funcionalidades. Esta \gls{aba} pode ser minimizada e estendida ao 
clicar no botão em forma de seta na parte inferior da guia.

\section{Upload de arquivos}
Para fazer o \gls{upload} de pastas ou arquivos, basta clicar no botão "Novo", localizado no menu lateral. Ao fazer isso, aparecerá um menu que permite ao usuário enviar arquivos, pastas, criar uma pasta no seu ambiente do drive, ou então, criar arquivos dos \gls{aplicativos} padrões da Google, bem como de outros apps que já tenham sido vinculados à sua conta.

\begin{figure}[H]
    \centering
    \includegraphics[width=.55\textwidth]{/google_drive/imagem_1.png}
    \caption{Captura e tela do menu de \gls{upload} do Google Drive.}
\end{figure}

Basta, então, selecionar o \gls{arquivo} que deseja "subir" ao seu armazenamento em nuvem e esperar o tempo necessário para o \gls{upload}. Caso o usuário utilize o Google Chrome ou Firefox como navegador, também é possível fazer o envio de arquivos e pastas arrastando seus ícones, com o cursor do mouse, do local de origem para o ambiente do Drive, e os soltando. Ainda, é possível subir arquivos por meio da versão para \textit{\gls{desktop}} do Google Drive.

\section{Criar pastas}
O uso de pastas é essencial para a melhor utilização do sistema, pois “[...] a 
organização das informações permite que o processo de tomada de decisões seja 
eficaz e rápido, impedindo barreiras e atrasos em diversos processos.”.

Por isso, é necessário saber como criá-las. Para criar uma nova pasta, clique no botão "Novo" e, no menu suspenso, selecione a opção "Nova pasta". Essa ação abre uma nova janela onde será preciso digitar um nome para pasta. Com isso feito, clique em "Criar".

\section{Abrir arquivos}
Para abrir um \gls{arquivo} do Google Drive, basta clicar sobre seu ícone duas vezes. O mesmo vale para acessar um diretório (pasta). 

“Os arquivos criados com o Google apps são abertos no 
navegador ou no app para dispositivos móveis. Outros tipos de \gls{arquivo} na sua pasta do Drive são abertos nos apps correspondentes. 
Por exemplo, o Adobe Reader para arquivos \gls{pdf}.''

\section{Mover arquivos}
Para mover um ou mais arquivo(s) ou pasta(s), clique no botão "Mais opções" (ícone de “três pontinhos”) ou com o botão direito do mouse sobre o \gls{arquivo} que você deseja mover. 
Isto abrirá um menu com várias opções. Clique ou passe o mouse sobre a opção "Organizar" e clique em "Mover".

\begin{figure}[H]
    \centering
    \includegraphics[width=.8\textwidth]{/google_drive/imagem_2.png}
    \caption{Captura de tela do menu de ações adicionais para um \gls{arquivo} do Google Drive}
\end{figure}

Com isso, será aberta uma janela sobreposta. Nela você pode escolher o local para o qual deseja mover o \gls{arquivo}. Com o local selecionado, basta clicar no botão "Mover".

\section{Excluir arquivos}
Para excluir arquivos ou pastas, clique com o botão direito do mouse, ou clique no botão "Mais opções" (ícone de “três pontinhos”), sobre o arquivo
desejado, para abrir as opções adicionais. Escolha a opção "Mover para a lixeira" e, novamente, "Mover para a lixeira" para excluir, ou "Cancelar" para cancelar a ação. Alternativamente, o usuário pode apertar a tecla \textbf{Delete} no teclado com o \gls{arquivo} selecionado, o que também abrirá uma janela pedindo confirmação para mover para a lixeira.

“O \gls{arquivo} permanecerá na lixeira por 30 dias antes de ser excluído  automaticamente.'' Ao colocar o \gls{arquivo} na lixeira, caso ele seja de sua autoria, será excluído do seu Drive e uma cópia será feita para pessoas com as quais você o compartilhou. Caso ele seja de outra pessoa, será apenas removido do seu Drive. 

Para excluir um arquivo definitivamente, abra a lixeira clicando no botão "Lixeira" do menu lateral esquerdo. Nela, clique em "Esvaziar lixeira" para excluir definitivamente tudo que estiver lá. Se quiser excluir um item específico, clique sobre ele com o botão direito do mouse, ou então no ícone de “três pontinhos”, e selecione a opção "\textbf{Excluir definitivamente}". 
Há também a opção de "\textbf{Restaurar}" para restaurar um \gls{arquivo} a seu local de origem, ou seja, a pasta onde estava antes de ser excluído.

\begin{figure}[H]
    \centering
    \includegraphics[width=1\textwidth]{/google_drive/imagem_3.png}
    \caption{Lixeira do Drive, captura de tela/Vinícius Maurer}
\end{figure}

\section{Baixar arquivos}
Para baixar um \gls{arquivo} ou pasta, clique com o botão direito do mouse sobre ele, ou nos “três pontinhos”, e selecione "Baixar", ou então, clique diretamente no ícone de download mostrado ao selecionar ou passar o mouse por cima do \gls{arquivo}.

\section{Renomear arquivos}
Para renomear um \gls{arquivo} ou pasta, clique com o botão direito do mouse, ou nos “três pontinhos”, e selecione "Renomear". Com isso, basta digitar o novo nome e confirmar clicando em "Ok".

Outra alternativa é clicar diretamente no ícone de renomear (simbolizado por uma caneta), mostrado ao selecionar ou passar o mouse por cima do \gls{arquivo}.

\section{Marcar arquivos com estrela}
Para marcar um \gls{arquivo} ou pasta com estrela, clique com o botão direito do mouse sobre ele, ou nos “três pontinhos”, e, no menu "Organizar", selecione a opção "\textbf{Adicionar a 'Com estrela'}". Todo \gls{arquivo} que o usuário marcar "com estrela" será mostrado na \gls{aba} "Com estrela", localizada na guia lateral esquerda da tela - vide seção \textbf{3.2 Homepage e menu lateral}.

\section{Pesquisar arquivos}
Para pesquisar um \gls{arquivo} ou pasta, clique na \gls{barrapesquisa} localizada na parte superior da tela e digite o termo que deseja pesquisar. Adicionalmente, você pode usar a \textbf{pesquisa avançada}, clicando no ícone ao lado direito da \gls{barrapesquisa}. 
Com ela, você pode buscar pelas opções: "Tipo", "Proprietário", "Com as palavras", "Nome do Item", "Local", "Data de modificação", "Aprovações e assinaturas eletrônicas", "Compartilhado com" e "Acompanhamentos", especificando ainda mais o que deseja encontrar.

\begin{figure}[H]
    \centering
    \includegraphics[width=.9\textwidth]{/google_drive/imagem_4.png}
    \caption{Captura de tela de uma pesquisa avançada no Drive}
\end{figure}