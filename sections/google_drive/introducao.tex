% sections/google_drive/introducao.tex
% !TeX root = ../../../main.tex

\section{Introdução ao Google Drive}
O Google Drive, ou apenas Drive, é uma ferramenta de armazenamento e 
compartilhamento de arquivos na nuvem desenvolvida pela Google. Apresentado pela 
primeira vez em 2012, ele permite ao usuário salvar arquivos de qualquer tipo 
(documentos, imagens, vídeos, etc) e também acessá-los de qualquer lugar.

Com 15GB de armazenamento em sua versão gratuita (que são compartilhados com 
Gmail e Google Fotos) possui planos de assinatura que permitem ao usuário obter 
maior espaço de armazenamento por meio de valores variados.

É disponibilizado não somente em versão de website, como também possui uma 
versão para \textit{desktop} (Windows e MacOS), além de um aplicativo mobile 
(iOS e Android).

\section{Homepage e menu lateral}
Após fazer o login em sua conta Google para entrar no Drive, você verá a 
seguinte página:

\begin{figure}[htbp]
    \centering
    \includegraphics[width=1\textwidth]{/google_drive/imagem_5.png}
    \caption{Homepage do Google Drive, captura de tela/Vinícius Maurer}
\end{figure}
	
Da direita para a esquerda na parte superior da aplicação você pode ver: seu 
usuário (permite você acessar sua conta ou trocá-la); o ícone do Google apps ou 
”quadrado de pontinhos” (permite ao usuário acessar os outros aplicativos da 
Google mais facilmente); o botão de configurações (a engrenagem); o botão de 
suporte e a barra de pesquisa (bem como o ícone de pesquisa avançada). Há ainda 
a possibilidade de haver um botão para o Gemini (a inteligência artificial da 
Google), caso o mesmo esteja ativado.

Já no lado esquerdo da tela você pode observar: o botão ‘Novo’ (para fazer 
upload/criação de arquivos/pastas); a aba ‘Pessoal’ que leva de volta para a 
homepage; a aba de ‘Atividades’ presente somente em contas Google Workspace 
(mostra um registro do que foi feito/por quem foi, a respeito de alterações, 
permissões, etc); a aba ‘Espaços de trabalho’ presente somente em contas Google 
Workspace (permite montar agrupamentos de arquivos/pastas sem alterar a 
localização real deles); a aba ‘Computadores’, não presente na imagem, referente 
ao backup e sincronização da versão desktop do Google Drive; a aba ‘Meu Drive’ 
mostra, como o nome já diz, o seu Drive, com todas as pastas e arquivos; a aba 
‘Drives compartilhados’ presente somente a contas Google Workspace, mostra um 
drive comum de uma organização, que não pertence a um indivíduo; a aba 
‘Compartilhados comigo’ mostra arquivos e pastas que foram compartilhadas com o 
usuário; ‘Recentes’ mostra pastas e arquivos recentemente acessados; ‘Com 
estrela’ mostra os arquivos marcados com uma estrela; ‘Spam’ para arquivos de 
spam; ‘Lixeira’ que mostra os arquivos que você exclui, mas ainda não de maneira 
permanente; e ‘Armazenamento’ que mostra uma relação do uso de seu armazenamento 
entre os aplicativos Google.
	
Também é possível ver no lado direito, uma aba com complementos da Google, por 
\textit{default} Agenda, Contatos, Keep e Tarefas, com um botão de cruz/”mais” para 
colocar complementos adicionais. Esta aba pode ser minimizada e estendida ao 
clicar no botão em forma de seta na parte inferior.


\section{Upload de arquivos}
Para fazer o upload de pastas ou arquivos, basta clicar no botão ‘Novo’ 
(localizado no menu lateral), ao fazer isso, aparecerá um menu, que permite ao 
usuário enviar arquivos, pastas, criar uma pasta no seu ambiente do drive, ou 
então criar arquivos dos aplicativos padrões da Google, também de outros apps 
que o usuário tenha conectado.

\begin{figure}[H]
    \centering
    \includegraphics[width=.55\textwidth]{/google_drive/imagem_1.png}
    \caption{Menu de upload do Google Drive, captura de tela/Vinícius Maurer}
\end{figure}

Basta então selecionar o arquivo que você deseja subir e esperar o tempo 
necessário para o upload. Caso o usuário utilize o Google Chrome ou Firefox como 
navegador, também é possível fazer o envio de arquivos e pastas arrastando-os 
para o ambiente do Drive e os soltando. Ainda é possível subir arquivos por meio 
da versão para \textit{desktop} do Google Drive (\cite{google2025drive})

\section{Criar pastas}
O uso de pastas é essencial para a melhor utilização do sistema, pois “[...] a 
organização das informações permite que o processo de tomada de decisões seja 
eficaz e rápido, impedindo barreiras e atrasos em diversos processos.” (\cite{bertazzi2022digitais}).

Por isso é necessário saber como criá-las e, para isso, é preciso clicar no 
botão ‘Novo’, então é preciso colocar um nome para pasta, com isso, para criar, 
clique em ‘Criar’.

\section{Abrir arquivos}
Para abrir um arquivo, basta clicá-lo duas vezes.

“Os arquivos criados com o Google apps são abertos no navegador ou no app para 
dispositivos móveis. Outros tipos de arquivo na sua pasta do Drive são abertos 
nos apps correspondentes. Por exemplo, o Adobe Reader para arquivos PDF.” 
(\cite{google2025adicionar})

O mesmo vale para entrar em uma pasta.

\section{Mover arquivos}
Para mover um arquivos ou pastas, clique no ícone de “três pontinhos” ou com o 
botão direito no arquivo que você deseja mover. Isto deverá abrir um menu com 
várias opções, clique ou passe o mouse sobre a aba ‘Organizar’ e clique em 
‘Mover’.

\begin{figure}[H]
    \centering
    \includegraphics[width=.8\textwidth]{/google_drive/imagem_2.png}
    \caption{Menu de ações adicionais para um arquivo do Google Drive, captura 
    de tela/Vinícius Maurer}
\end{figure}

Com isso é aberta uma aba, nela você pode escolher o local para onde quer mover 
o arquivo, com o local selecionado, basta clicar em ‘Mover’.

\section{Excluir arquivos}
Para excluir arquivos (ou pastas), clique com o botão direito ou nos “três 
pontinho” para abrir as opções adicionais (alternativamente você pode apertar 
delete), e escolha ‘Mover para a lixeira’, e novamente ‘Mover para a lixeira’ 
para excluir, ou ‘Cancelar’ para cancelar.

“O arquivo permanecerá na lixeira por 30 dias antes de ser excluído  automaticamente.”(\cite{google2025adicionar}) Ao colocar o arquivo na lixeira, se ele for seu ele será excluído do seu Drive e uma cópia será feita com quem você compartilhou, caso ele seja de outra pessoa, ele será apenas removido do seu Drive. (\cite{google2025adicionar})

Para excluir em definitivo, abra a sua lixeira clicando em ‘Lixeira’ no menu 
lateral, nela você pode clicar em ‘Esvaziar lixeira’ para excluir 
definitivamente o que estiver lá, se quiser excluir um item em específico, você 
pode ou clicar nele, ou então nos “três pontinhos”, e clicar no ícone de 
‘Excluir definitivamente’, você também pode escolher ‘Restaurar’ para restaurar 
um arquivo.

\begin{figure}[H]
    \centering
    \includegraphics[width=1\textwidth]{/google_drive/imagem_3.png}
    \caption{Lixeira do Drive, captura de tela/Vinícius Maurer}
\end{figure}

\section{Baixar arquivos}
Para baixar um arquivo/pasta, clique com o botão direito ou nos “três pontinhos” 
e selecione ‘Baixar’, ou então clique diretamente no ícone de download mostrado 
ao selecionar (ou passar o mouse por cima) o arquivo.

\section{Renomear arquivos}
Para renomear um arquivo/pasta, clique com o botão direito ou nos “três 
pontinhos” e selecione ‘Renomear’, com isso, basta colocar o novo nome e 
confirmar clicando em ‘Ok’.

Ou então clique diretamente no ícone de renomear (simbolizado por uma caneta), 
mostrado ao selecionar (ou passar o mouse por cima) o arquivo.

\section{Adicionar uma estrela aos arquivos}
Para adicionar uma estrela a um arquivo/pasta, clique no ícone de adicionar 
estrela (simbolizado por uma estrela), mostrado ao selecionar (ou passar o mouse 
por cima) o arquivo.

Todo arquivo ao qual você adicionar uma estrela, será mostrado na aba ‘Com 
estrela’.

\section{Pesquisar arquivos}
Para pesquisar um arquivo, ou pasta, clique na barra de pesquisa, localizada na 
parte superior da tela, e digite o termo que deseja pesquisar. Alternativamente 
você pode usar a pesquisa avançada, clicando no ícone no lado direito, com ela 
você pode pesquisar pelas opções: ‘Tipo’, ‘Proprietário’, ‘Com as palavras’, 
‘Nome do Item’, ‘Local’, ‘Data de modificação’, ‘Aprovações e assinaturas 
eletrônicas’, ‘Compartilhado com’ e ‘Acompanhamentos’

\begin{figure}[H]
    \centering
    \includegraphics[width=.9\textwidth]{/google_drive/imagem_4.png}
    \caption{Pesquisa avançada do Drive, captura de tela/Vinícius Maurer}
\end{figure}