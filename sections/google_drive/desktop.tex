% sections/google_drive/\gls{desktop}.tex
% !TeX root = ../../main.tex

\section{Diferenciais da Versão Desktop do Google Drive}

Enquanto a versão \gls{web} do Google Drive permite acessar seus arquivos de qualquer lugar 
com conexão à internet, a versão para computador se integra ao sistema do seu 
dispositivo, facilitando o uso no dia a dia. Com ela, é possível abrir, mover e editar 
arquivos diretamente pelo explorador de arquivos do seu sistema (como o Windows 
Explorer ou o Finder, no Mac), sem precisar acessar o navegador.

Em vez de abrir um navegador, fazer \gls{login} e navegar pela interface \gls{web}, o aplicativo 
\gls{desktop} permite que o usuário traga seus arquivos da \gls{nuvem} como se eles estivessem 
salvos localmente. O usuário pode arrastar, soltar, copiar e colar arquivos e pastas 
diretamente para a pasta do Google Drive em seu computador e eles serão automaticamente 
sincronizados com a \gls{nuvem}. Isso elimina a necessidade de \gls{upload} manual e torna o 
processo de salvamento de arquivos mais rápido e intuitivo.

Outro grande diferencial é o controle sobre como e onde seus arquivos são armazenados 
no seu computador. Com a versão para \gls{desktop}, o usuário pode escolher entre "stream" 
e "mirror" de arquivos:

\begin{itemize}
	\item \textbf{Stream de arquivos}: Esta opção é ideal para economizar espaço em 
	disco. Os arquivos ficam na \gls{nuvem} e são baixados para o seu computador somente 
	quando precisar abri-los.
	\item \textbf{Mirror de arquivos}: Se precisar de acesso \gls{offline} a todos os seus 
	arquivos, esta opção mantém uma cópia local e outra na \gls{nuvem}, garantindo que você 
	tenha seus documentos sempre disponíveis, mesmo sem internet.
\end{itemize}

\section{Configurando o Google Drive para Computador}
Esta seção trata-se de um guia passo a passo, do \gls{download} à configuração inicial, para 
que o usuário possa começar a usar a ferramenta de forma rápida e eficiente.

\subsection{Como Baixar a Versão para Desktop}
\begin{enumerate}
	\item Acesse o site oficial do Google Drive em https://drive.google.com/drive/home.
	\item No canto superior direito da tela, clique na engrenagem e clique na opção: 
	“Use o Drive no computador” ou acesse o \gls{link} https://workspace.google.com/products/drive/\#\gls{download}.
	\item Clique no botão para iniciar o \gls{download} do \gls{arquivo} de instalação. O sistema 
	operacional do seu computador (Windows ou Mac) será detectado automaticamente.
\end{enumerate}

\subsection{Guia de Instalação e Configuração}
Após o \gls{download}, siga as instruções simples de instalação. A configuração inicial é o 
passo mais importante para definir como o Drive funcionará no seu computador. Após a 
instalação, o aplicativo solicitará que faça \gls{login} com sua conta do Google.
