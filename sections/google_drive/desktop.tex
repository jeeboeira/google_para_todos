% sections/google_drive/desktop.tex
% !TeX root = ../../main.tex

\section{Diferenciais da Versão Desktop do Google Drive}

Enquanto o Google Drive na web oferece acesso aos seus arquivos de qualquer lugar com internet, a versão para computador traz uma usabilidade que se integra diretamente ao seu fluxo de trabalho diário. A principal diferença é a sua capacidade de trabalhar com seus arquivos diretamente do explorador de arquivos do seu computador (Windows Explorer ou Finder no Mac).

Em vez de abrir um navegador, fazer login e navegar pela interface web, o aplicativo desktop permite que o usuário traga seus arquivos na nuvem como se eles estivessem salvos localmente. O usuário pode arrastar, soltar, copiar e colar arquivos e pastas diretamente para a pasta do Google Drive em seu computador e eles serão automaticamente sincronizados com a nuvem. Isso elimina a necessidade de uploads manuais e torna o processo de salvamento de arquivos rápido e intuitivo.

Outro grande diferencial é o controle sobre como e onde seus arquivos são armazenados no seu computador. Com a versão para desktop, o usuário pode escolher entre "stream" e "mirror" de arquivos.

\begin{itemize}
	\item Stream de arquivos: Esta opção é ideal para economizar espaço em disco. Os arquivos ficam na nuvem, e são baixados para o seu computador somente quando precisar abri-los.
	\item Mirror de arquivos: Se precisar de acesso offline a todos os seus arquivos, esta opção mantém uma cópia local e outra na nuvem, garantindo que tenha seus documentos sempre disponíveis, mesmo sem internet.
\end{itemize}

\section{Configurando o Google Drive para Computador}
Esta seção é um guia passo a passo, do download à configuração inicial, para que o usuário possa começar a usar a ferramenta de forma rápida e eficiente.

\subsection{Como Baixar a Versão para Desktop}
\begin{enumerate}
	\item Acesse o site oficial do Google Drive em https://drive.google.com/drive/home.
	\item No canto superior direito clique na engrenagem e clique na opção: “Use o Drive no computador” ou acesse o link https://workspace.google.com/products/drive/\#download.
	\item Clique no botão para iniciar o download do arquivo de instalação. O sistema operacional do seu computador (Windows ou Mac) será detectado automaticamente.
\end{enumerate}

\subsection{Guia de Instalação e Configuração}
Após o download, siga as instruções simples de instalação. A configuração inicial é o passo mais importante para definir como o Drive funcionará no seu computador.
Faça login na sua conta Google: Após a instalação, o aplicativo solicitará que faça login com sua conta do Google.
