% sections/google_drive/colaboracao.tex
% !TeX root = ../../main.tex

\section{Segurança}

Além das permissões de compartilhamento, o Google Drive oferece outras medidas de segurança, tais como o bloqueio de arquivos, que impede a edição ou comentário em documentos por indivíduos além do proprietário, mesmo que tenham tais permissões. Contudo, colaboradores “editores” ainda podem retirar o bloqueio. Para evitar que isto ocorra, clique com o botão direito do mouse, ou nos “três pontinhos”, e vá para a opção "Informações sobre o arquivo". Depois clique em "Bloquear". Por fim, basta confirmar o bloqueio.

Pode-se verificar se o arquivo está bloqueado se, ao lado do nome do usuário, estiver presente um ícone de cadeado, como na imagem abaixo:

\begin{figure}[htbp]
	\centering
	\includegraphics[width=1\textwidth]{/google_drive/imagem_10.png}
	\caption{Arquivo bloqueado no Drive, Captura de tela/Vinícius Maurer}
\end{figure}

Para desbloquear, siga os mesmos passos citados anteriormente, porém, agora clique em "Desbloquear" e, depois, confirme a ação.

O Drive também bloqueia automaticamente arquivos detectados como \textit{malware}, \textit{spam} ou \textit{phishing}, mas não busca por estes em arquivos maiores que 100MB, algo que geralmente é exibido por meio de um aviso. Estes arquivos devem aparecer na sua aba de "Spam", presente no menu lateral esquerdo.

\begin{figure}[htbp]
	\centering
	\includegraphics[width=1\textwidth]{/google_drive/imagem_11.png}
	\caption{Aviso de arquivo acima dos limites da verificação de segurança, Captura de tela/ Vinicius Maurer}
\end{figure}

Também é possível consultar os avisos de segurança de um arquivo ao clicar nele com o botão direito - ou nos “três pontinhos” -, ir para "Informações sobre o arquivo" e clicar em "Limitações de segurança".
