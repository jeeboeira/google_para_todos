% sections/google_drive/colaboracao.tex
% !TeX root = ../../main.tex

\section{Segurança}

Além das permissões de compartilhamento, o Google Drive oferece outras medidas de 
segurança, tais como: o \textbf{bloqueio de arquivos}, que impede a edição ou 
comentário em documentos por indivíduos além do proprietário, mesmo que estes tenham tais 
permissões. No entanto, colaboradores “editores” ainda podem retirar esse bloqueio. 
Para evitar que isto ocorra, clique com o botão direito do mouse, ou nos 
“três pontinhos”, e vá para a opção "Informações sobre o \gls{arquivo}". 
Depois clique em "Bloquear". Por fim, basta confirmar o bloqueio.

Pode-se verificar se o \gls{arquivo} está bloqueado se, ao lado do nome do usuário, 
estiver presente um ícone de cadeado, como na imagem abaixo:

\begin{figure}[htbp]
	\centering
	\includegraphics[width=1\textwidth]{/google_drive/imagem_10.png}
	\caption{Captura de tela exibindo \gls{arquivo} bloqueado no Google Drive}
\end{figure}

Para desbloquear, siga os mesmos passos citados acima, porém, agora clique 
em "Desbloquear". Depois, basta confirmar a ação.

O Drive também bloqueia automaticamente arquivos detectados como \textbf{\gls{malware}}, 
\textbf{\gls{spam}} ou \textbf{\gls{phishing}}, mas não busca por estes em arquivos 
com mais de 100MB, algo que geralmente é exibido por meio de um aviso. Estes arquivos devem 
aparecer na sua \gls{aba} de "\gls{spam}", presente no menu lateral esquerdo. A figura 3.12 ilustra
este comportamento.

\begin{figure}[htbp]
	\centering
	\includegraphics[width=1\textwidth]{/google_drive/imagem_11.png}
	\caption{Aviso de \gls{arquivo} acima dos limites da verificação de segurança do Google Drive}
\end{figure}

Também é possível consultar os avisos de segurança de um \gls{arquivo} ao clicar nele com o 
botão direito, ou nos “três pontinhos”, ir para "Informações sobre o \gls{arquivo}" e 
clicar em "Limitações de segurança".