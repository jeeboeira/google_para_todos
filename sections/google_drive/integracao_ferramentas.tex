% sections/google_drive/integracao_ferramentas.tex
% !TeX root = ../../main.tex

\section{Integração com outras ferramentas}

O Google Drive vai muito além de ser só um “lugar para guardar arquivos”. O que realmente faz a diferença é a forma como ele se conecta com outros aplicativos, que o transforma em uma espécie de hub central de organização, criação e colaboração. Essa flexibilidade permite que você simplifique tarefas, automatize processos e agilize a resolução de suas necessidades.

\subsection{Como conectar novos aplicativos}
A maneira mais prática de expandir as funções do seu Drive é utilizando o Google Workspace Marketplace. É lá que você encontra aplicativos de diferentes áreas que podem ser integrados diretamente à sua conta. Para instalar, siga o passo a passo:

\begin{enumerate}
	\item Abra o Google Drive no navegador;
	\item Clique em “Novo” (canto superior esquerdo);
	\item Vá em “Mais” $\rightarrow$ “Conectar mais aplicativos”;
	\item O Marketplace será aberto.
\end{enumerate}

\begin{figure}[htbp]
	\centering
	\includegraphics[width=1\textwidth]{/google_drive/imagem_9.png}
	\caption{Menu do marketplace}
\end{figure}

Com o Marketplace aberto, é só pesquisar os aplicativos que você necessita. Quando encontrar, clique em Instalar e o app será vinculado à sua conta automaticamente.

\begin{atencao}
	Nem todos os apps são gratuitos. Alguns exigem assinatura ou pagamento único para liberar todas as funções. Então, antes de instalar, vale dar uma olhada nos detalhes para não ser “pego de surpresa”.
\end{atencao}

\section{Exemplos de integrações úteis}
\subsection{Diagramas e fluxos de trabalho}

Ferramentas como Lucidchart, draw.io e Miro permitem criar fluxogramas, organogramas, mapas mentais e até modelos ER (Entidade-Relacionamento). Ao integrá-los com o Drive, todos os arquivos ficam salvos automaticamente, o que facilita a colaboração em equipe e evita a dispersão de diferentes versões por diretórios desordenados.

\begin{figure}[htbp]
	\centering
	\includegraphics[width=1\textwidth]{/google_drive/imagem_13.png}
	\caption{Captura de tela da ferramenta Miro}
\end{figure}

Exemplo Prático (Miro): Para trazer um documento do Drive para o seu quadro do Miro, basta criar um novo quadro em branco. Na barra de ferramentas do Miro, você pode selecionar a opção de inserir documentos do Google Drive. Ao utilizar o link público do documento (configurado com a permissão correta), o Miro exibe o conteúdo desse documento diretamente no seu quadro, facilitando as sessões de brainstorming e feedback visual sem que o usuário precise sair da ferramenta.

\subsection{Design e edição}

Se você curte design, o Canva pode ser um ótimo aliado. Ele se conecta ao Drive e permite importar imagens, editar projetos e salvar tudo na pasta certa. Para edição de PDFs e imagens, ferramentas como Lumin PDF e Pixlr também são muito práticas: você as abre diretamente do Drive, faz a edição e salva sem precisar baixar nada no computador.

\subsection{Programação e ciência de dados}
Para você que é da área da tecnologia e programação, o Google Colab é um dos melhores exemplos de integração. Ele permite criar e compilar notebooks (anotações + blocos de código) em Python (ou outras linguagens) diretamente da nuvem, com o armazenamento dos arquivos em seu Drive. Assim, dá para trabalhar com análise de dados, treinar modelos de machine learning ou colaborar em projetos de programação sem instalar nada localmente.

\begin{figure}[htbp]
	\centering
	\includegraphics[width=1\textwidth]{/google_drive/imagem_12.png}
	\caption{Captura de tela da ferramenta Google Colab}
\end{figure}

\subsection{Automação de tarefas}
Plataformas como Zapier e IFTTT conectam o Drive a centenas de outros aplicativos. Você pode criar “gatilhos” automáticos, como no exemplo prático demonstrado a seguir:
\begin{itemize}
	\item Gatilho: um anexo chega no seu Gmail.
	\item Ação: o arquivo é salvo automaticamente na pasta “Faturas” do Drive.
\end{itemize}
Isso economiza tempo, reduz erros e garante que ninguém fique de fora da informação.

\subsection{Assinaturas eletrônicas}
Ferramentas como DocuSign e PandaDoc também podem ser integradas ao Drive. Você manda um contrato direto da sua pasta, a pessoa assina digitalmente e a versão final já volta para o Drive. Simples e rápido, sem precisar imprimir nada.


