% sections/google_drive/integracao_ferramentas.tex
% !TeX root = ../../main.tex

\section{Integração com outras ferramentas}

O Google Drive vai muito além de ser só um “lugar para guardar arquivos”. O que realmente faz a diferença é a forma como ele se conecta com outros \gls{aplicativos}, que o transforma em uma espécie de hub central de organização, criação e colaboração. Essa flexibilidade permite simplificar tarefas, automatizar processos e agilizar a resolução de suas necessidades.

\subsection{Como conectar novos \gls{aplicativos}}
A maneira mais prática de expandir as funções do seu Drive é utilizando o Google \Gls{workspace} \Gls{marketplace}. Nele se encontram \gls{aplicativos} de diferentes áreas que podem ser integrados diretamente à sua conta. Para instalar, siga o passo a passo:

\begin{enumerate}
	\item Abra o Google Drive no navegador;
	\item Clique em “Novo” (canto superior esquerdo);
	\item Vá em “Mais” $\rightarrow$ “Conectar mais \gls{aplicativos}”;
	\item O \Gls{marketplace} será aberto.
\end{enumerate}

\begin{figure}[htbp]
	\centering
	\includegraphics[width=1\textwidth]{/google_drive/imagem_9.png}
	\caption{Menu do \Gls{marketplace}}
\end{figure}

Com o \Gls{marketplace} aberto, é só pesquisar os \gls{aplicativos} que você necessita. Quando encontrar, clique em Instalar e o app será vinculado à sua conta automaticamente.

\begin{atencao}
	Nem todos os apps são gratuitos. Alguns exigem assinatura ou pagamento único para liberar todas as funções. Então, antes de instalar, vale dar uma olhada nos detalhes para não ser “pego de surpresa”.
\end{atencao}

\section{Exemplos de integrações úteis}
\subsection{Diagramas e fluxos de trabalho}

Ferramentas como Lucidchart, draw.io e \Gls{miro} permitem criar fluxogramas, organogramas, mapas mentais e até \Gls{modeloer}. Ao integrá-los com o Drive, todos os arquivos ficam salvos automaticamente, o que facilita a colaboração em equipe e evita a dispersão de diferentes versões por diretórios desordenados.

\begin{figure}[htbp]
	\centering
	\includegraphics[width=1\textwidth]{/google_drive/imagem_13.png}
	\caption{Captura de tela da ferramenta \Gls{miro}}
\end{figure}

Exemplo Prático (\Gls{miro}): Para trazer um documento do Drive para o seu quadro do \Gls{miro}, basta criar um novo quadro em branco. Na barra de ferramentas do \Gls{miro}, há a possibilidade de selecionar a opção de inserir documentos do Google Drive. Ao utilizar o \gls{link} público do documento (configurado com a permissão correta), o \Gls{miro} exibe o conteúdo desse documento diretamente no seu quadro, facilitando as sessões de brainstorming e \gls{feedback} visual sem que o usuário precise sair da ferramenta.

\subsection{Design e edição}

Se busca uma ferramenta de design, o Canva pode ser um ótimo aliado. Ele se conecta ao Drive e permite importar imagens, editar projetos e salvar tudo na pasta certa. Para edição de PDFs e imagens, ferramentas como Lumin \gls{pdf} e Pixlr também são muito práticas: são abertas diretamente pelo Drive, fazem a edição e salvam sem precisar baixar nada no computador.

\subsection{Programação e ciência de dados}
Para quem é da área da tecnologia e programação, o Google Colab é um dos melhores exemplos de integração. Ele permite criar e compilar notebooks (anotações + blocos de código) em Python (ou outras linguagens) diretamente da \gls{nuvem}, com o armazenamento dos arquivos em seu Drive. Assim, dá para trabalhar com análise de dados, treinar modelos de \gls{machinelearning} ou colaborar em projetos de programação sem instalar nada localmente.

\begin{figure}[htbp]
	\centering
	\includegraphics[width=1\textwidth]{/google_drive/imagem_12.png}
	\caption{Captura de tela da ferramenta Google Colab}
\end{figure}

\subsection{Automação de tarefas}
Plataformas como \Gls{zapier} e \Gls{ifttt} conectam o Drive a centenas de outros \gls{aplicativos}. Podem ser criados “gatilhos” automáticos, como no exemplo prático demonstrado a seguir:
\begin{itemize}
	\item \Gls{gatilho}: um anexo chega no seu Gmail.
	\item Ação: o \gls{arquivo} é salvo automaticamente na pasta “Faturas” do Drive.
\end{itemize}
Isso economiza tempo, reduz erros e garante que ninguém fique de fora da informação.

\subsection{Assinaturas eletrônicas}
Ferramentas como \gls{docusign} e \Gls{pandadoc} também podem ser integradas ao Drive. Você manda um contrato direto da sua pasta, a pessoa assina digitalmente e a versão final já volta para o Drive. Simples e rápido, sem precisar imprimir nada.


