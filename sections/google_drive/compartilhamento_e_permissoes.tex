% sections/google_drive/compartilhamento_e_permissoes.tex
% !TeX root = ../../main.tex

\section{Benefícios e Riscos do Compartilhamento}
O compartilhamento de arquivos e pastas no Google Drive para computadores simplifica significativamente a colaboração. Ele permite que várias pessoas trabalhem no mesmo documento em tempo real, eliminando a necessidade de envio de arquivos por e-mail, sem ter que lidar com diferentes versões e problemas de compatibilidade. Com as permissões adequadas, se garante que as pessoas certas tenham o nível de acesso adequado, seja para visualizar, comentar ou editar o conteúdo, melhorando a produtividade e a organização do trabalho em equipe.

Apesar dos benefícios, o compartilhamento inadequado pode levar a riscos de segurança. Se não gerenciar as permissões corretamente, informações sensíveis podem ser acessadas, copiadas ou até mesmo alteradas por pessoas não autorizadas. Por exemplo, dar permissão de "Editor" a alguém que só precisa visualizar um documento pode permitir que essa pessoa compartilhe o \gls{arquivo} com terceiros sem o seu conhecimento. Por isso, é fundamental definir o nível de acesso cuidadosamente.

\section{Como Compartilhar Arquivos e Pastas}

Esta seção vai te guiar pelo processo de compartilhamento de arquivos e gerenciamento de permissões usando o Google Drive. Para compartilhar um \gls{arquivo} ou pasta diretamente do seu computador, siga os passos abaixo.

Primeiro, localize o \gls{arquivo} ou pasta desejado dentro da sua pasta do Google Drive ou no computador caso esteja usando o aplicativo \gls{desktop}. Clique com o botão direito no \gls{arquivo} ou pasta e selecione a opção "Compartilhar". Dentro do menu de compartilhamento, existem duas opções:

\begin{figure}[htbp]
	\centering
	\includegraphics[width=0.5\textwidth]{/google_drive/imagem_6.png}
	\caption{Menu de compartilhamento}
\end{figure}

\begin{itemize}
	\item Compartilhamento direto (com e-mail):
	\begin{itemize}
		\item Como funciona: O usuário adiciona o endereço de e-mail de uma ou mais pessoas diretamente no menu de compartilhamento;
		\item Vantagens: Oferece maior controle sobre quem pode acessar o \gls{arquivo}. Apenas as pessoas convidadas explicitamente podem ver o conteúdo. Se pode gerenciar as permissões de cada pessoa individualmente e revogar o acesso a qualquer momento;
		\item Quando usar: Ideal para documentos confidenciais ou quando se sabe exatamente com quem precisa compartilhar.
	\end{itemize}
	\item Compartilhamento com \gls{link}:
	\begin{itemize}
		\item Como funciona: Gera um \gls{link} para o \gls{arquivo} e define o nível de acesso geral, como "Qualquer pessoa com o \gls{link}" ou "Restrito";
		\item Vantagens: Extremamente prático para compartilhar com um grande número de pessoas. Não é necessário digitar e-mails e o \gls{link} pode ser facilmente enviado em mensagens ou documentos;
		\item Quando usar: Ótimo para compartilhar materiais públicos, como guias de estudo, portfólios ou outros documentos que não exigem controle rigoroso. No entanto, se o \gls{link} cair nas mãos erradas o acesso ao seu \gls{arquivo} pode ser comprometido, dependendo do nível de acesso concedido, conforme veremos a seguir.
	\end{itemize}
\end{itemize}

Após definido o método de compartilhamento, é necessário definir o nível de acesso. Ao lado do nome de cada pessoa ou \gls{link} criado, há a possibilidade de definir as permissões de acesso, através de um botão que abre um menu. As opções são:

\begin{itemize}
	\item Leitor: A pessoa pode apenas visualizar o conteúdo;
	\item Comentador: A pessoa pode visualizar e adicionar comentários;
	\item Editor: A pessoa pode editar, compartilhar com outras pessoas e fazer alterações no \gls{arquivo};
	\item Proprietário: A pessoa que criou o \gls{arquivo}/pasta tem permissão total sobre o mesmo. Este nível de acesso pode ser transferido para um novo usuário.
\end{itemize}

Após selecionar o nível de acesso, finalize o processo clicando em "Salvar” para compartilhar. Caso tenha escolhido via \gls{link}, não se esqueça de copiar o mesmo. Com o compartilhamento criado, existem algumas configurações adicionais que podem ser selecionadas a partir da engrenagem de configuração, presente dentro do menu de compartilhamento. São elas:

\begin{itemize}
	\item Permitir que os editores mudem as permissões e compartilhem;
	\item Pessoas que podem baixar, copiar e imprimir o conteúdo;
	\item Pessoas que comentaram e visualizaram.
\end{itemize}

Essas configurações servem para melhorar o controle de acessos e permissões, colaborando para a criação de um ambiente mais seguro para os usuários.