% sections/google_drive/compartilhamento_e_permissoes.tex
% !TeX root = ../../main.tex

\section{Benefícios e Riscos do Compartilhamento}
O compartilhamento de arquivos e pastas no Google Drive para computadores 
simplifica significativamente o trabalho colaborativo. Ele permite que várias pessoas 
trabalhem no mesmo documento simultaneamente, eliminando a necessidade de envio 
de arquivos por e-mail, problemas com diferentes versões e compatibilidade entre
aplicações. Com a atribuição adequada de permissões, garante-se que as pessoas 
certas terão o nível de acesso adequado aos arquivos, sejam visualizadores, 
comentadores ou editores do conteúdo. Dessa forma, é possível controlar 
privilégios, melhorando a produtividade e a organização de todo o grupo de colaboradores.

Apesar dos benefícios, o compartilhamento inadequado pode levar a riscos de segurança e
comprometimentos de dados severos. Se não gerenciar as permissões corretamente, 
informações sensíveis podem ser acessadas, copiadas e até mesmo alteradas por pessoas 
não autorizadas. Por exemplo, ao dar permissão de "Editor" a alguém que só precisa 
visualizar um documento, é possível que essa pessoa compartilhe o \gls{arquivo} com 
terceiros sem o seu conhecimento. Por isso, é fundamental definir 
o nível de acesso de cada colaborador cuidadosamente.

\section{Como Compartilhar Arquivos e Pastas}

Esta seção vai te guiar pelo processo de compartilhamento de arquivos e gerenciamento de permissões usando o Google Drive. Para compartilhar um \gls{arquivo} ou pasta diretamente do seu computador, siga os passos abaixo.

Primeiro, localize o \gls{arquivo} ou pasta desejado dentro da sua pasta do Google Drive ou no computador caso esteja usando o aplicativo \gls{desktop}. Clique com o botão direito no \gls{arquivo} ou pasta e selecione a opção "Compartilhar". Dentro do menu de compartilhamento, existem duas opções:

\begin{figure}[htbp]
	\centering
	\includegraphics[width=0.5\textwidth]{/google_drive/imagem_6.png}
	\caption{Menu de compartilhamento}
\end{figure}

\begin{itemize}
	\item \textbf{Compartilhamento direto} (com e-mail):
	\begin{itemize}
		\item \textbf{Como funciona}: o usuário adiciona o endereço de e-mail de uma ou mais pessoas diretamente no menu de compartilhamento;
		\item \textbf{Vantagens}: oferece maior controle sobre quem pode acessar o \gls{arquivo}. Apenas as pessoas convidadas explicitamente podem ver o conteúdo. Pode-se gerenciar as permissões de cada pessoa individualmente e revogar o acesso a qualquer momento;
		\item \textbf{Quando usar}: ideal para documentos confidenciais ou quando se sabe exatamente com quem precisa compartilhar.
	\end{itemize}
	\item \textbf{Compartilhamento com link}:
	\begin{itemize}
		\item \textbf{Como funciona}: gera um \gls{link} para o \gls{arquivo} e define o nível de acesso geral, como "Qualquer pessoa com o \gls{link}" ou "Restrito";
		\item \textbf{Vantagens}: extremamente prático para compartilhar com um grande número de pessoas. Não é necessário digitar e-mails e o \gls{link} pode ser facilmente enviado em mensagens ou documentos;
		\item \textbf{Quando usar}: ótimo para compartilhar materiais públicos, como guias de estudo, portfólios ou outros documentos que não exigem controle rigoroso. No entanto, se o \gls{link} cair nas "mãos erradas", o acesso ao seu \gls{arquivo} pode ser comprometido, de acordo com nível de acesso concedido, conforme veremos a seguir.
	\end{itemize}
\end{itemize}

Após definido o método de compartilhamento, é necessário definir o \textbf{nível de acesso}. Ao lado do nome de cada pessoa ou \gls{link} criado, há a possibilidade de definir as permissões de acesso, através de um botão que abre um menu. As opções são:

\begin{itemize}
	\item \textbf{Leitor}: A pessoa pode apenas visualizar o conteúdo;
	\item \textbf{Comentador}: A pessoa pode visualizar e adicionar comentários;
	\item \textbf{Editor}: A pessoa pode editar, compartilhar com outras pessoas e fazer alterações no \gls{arquivo};
	\item \textbf{Proprietário}: A pessoa que criou o \gls{arquivo} ou pasta tem permissão total sobre o mesmo. 
	Este nível de acesso pode ser transferido para um novo usuário.

	\begin{dica}
		Para conceder a permissão de "\textbf{Proprietário}" a outro usuário, é necessário que o mesmo possua uma conta Google. Além disso, ao transferir a propriedade, o antigo proprietário perde o controle total sobre o arquivo, podendo manter apenas permissões de editor, comentador ou leitor. Para realizar a ação, clique no menu de permissões ao lado do nome do usuário e selecione "Tornar proprietário".
	\end{dica}

\end{itemize}

Após selecionar o nível de acesso, finalize o processo clicando em "Salvar” para 
compartilhar. Caso tenha escolhido via \gls{link}, não se esqueça de copiar o mesmo. 
Com o compartilhamento criado, existem algumas configurações adicionais que podem 
ser selecionadas a partir da engrenagem de configuração, presente dentro do menu de 
compartilhamento. São elas:

\begin{itemize}
	\item \textbf{Permitir que os editores mudem as permissões e compartilhem};
	\item \textbf{Pessoas que podem baixar, copiar e imprimir o conteúdo};
	\item \textbf{Pessoas que comentaram e visualizaram}.
\end{itemize}

Essas configurações servem para melhorar o controle de acessos e permissões, 
colaborando para a criação de um ambiente mais seguro para todos os usuários.