% sections/documentos/guias_de_documentos.tex
% !TeX root = ../../main.tex

\section{Guias de Documentos: Estrutura de Tópicos}
O Google Documentos oferece a funcionalidade de estrutura de tópicos para seus arquivos. 
Essa ferramenta é muito útil para organizar o documento, facilitando pesquisas, 
construção de sumários, divisão de trabalho entre outros usuários e categorização de conteúdos. 
Costuma-se organizar em tópicos textos grandes, como produções acadêmicas, apostilas, manuais e até mesmo livros de receitas. 

% TODO: Adicionar referência: (encontrados na aba de ferramentas AQUI PODE REFERENCIAR A PARTE DA NOSSA APOSTILA QUE FALA DE ESTILOS DE TEXTO).
Para que seus tópicos sejam adicionados no guia, deve-se utilizar os estilos de título (encontrados na aba de ferramentas). 
Dessa forma, a organização do arquivo é interpretada automaticamente pelo Google Documentos e já é possível visualizar a 
formação de guias na parte lateral esquerda da tela:

\begin{figure}[H]
    \centering
    \includegraphics[width=1\textwidth]{images/documentos/guias_de_documentos/Imagem32.png}
    \caption{}
\end{figure}

Ao clicar nos itens de cada tópico, você é direcionado para a localização do respectivo título. 
Dessa forma, o leitor encontra facilmente o trecho que o interessa para leitura. Perceba, também, 
como o texto acinzentado que segue a formatação de subtítulo não aparece como um tópico no guia, 
ele apenas funciona como um organizador interno do tópico pai.