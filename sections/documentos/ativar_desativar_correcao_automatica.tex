% sections/documentos/ativar_desativar_correcao_automatica.tex
% !TeX root = ../../main.tex

\section{Ativar/Desativar Correção Ortográfica}
O Google Documentos conta com uma funcionalidade de correção ortográfica automática que funciona como um auxílio para identificar e sugerir 
modificações em palavras que são escritas de forma incorreta. Caso esse recurso esteja ativado, palavras incorretas serão sublinhadas 
em vermelho/azul e um pequeno menu te possibilitará realizar a correção daquele texto.
Para ativar essa funcionalidade, selecione o item “Ferramentas” na listagem de menus, em seguida passe o mouse sobre “Ortografia e Gramática”. 
Você irá visualizar as seguintes opções:

\begin{figure}[H]
    \centering
    \includegraphics[width=1\textwidth]{images/documentos/ativar_desativar_correcao_automatica/Imagem19.png}
    \caption{}
\end{figure}

A opção “Verificação ortográfica e gramatical” realiza uma varredura completa no documento e aponta onde foram encontrados erros. 
Você pode escolher ignorar a sugestão de mudança ou aceitar o ajuste automático. 
Na imagem a seguir apresenta-se um exemplo de correção de texto realizado pela verificação ortográfica e gramatical.

\begin{figure}[H]
    \centering
    \includegraphics[width=1\textwidth]{images/documentos/ativar_desativar_correcao_automatica/Imagem20.png}
    \caption{}
\end{figure}

% TODO: Achar o ícone de check
As opções “Mostrar sugestões de ortografia” e “Mostrar sugestões gramaticais” podem ser ativadas ou desativadas com um clique. 
Caso o ícone de check (✔) esteja ao lado da opção, tenha certeza de que ela está ativada. As sugestões de ortografia deixam as palavras 
incorretas sublinhadas em vermelho e indicam erros graves. Já as sugestões gramaticais sublinham as palavras em cor azul e indicam que 
melhorias de concordância, pontuação ou estilos podem ser realizadas.

Ao clicar sobre a palavra que contém erros, um pequeno menu de correção rápida irá aparecer. Você pode clicar na palavra sugerida para que a correção seja feita de forma automática. Também é possível selecionar a opção com o ícone de “x” que irá ignorar a sugestão. Por fim, ao selecionar os três pontos um menu mais extenso é aberto.

\begin{figure}[H]
    \centering
    \includegraphics[width=.9\textwidth]{images/documentos/ativar_desativar_correcao_automatica/Imagem21.png}
    \caption{}
\end{figure}
O mesmo menu estendido que é aberto ao clicar nos três pontos, aparece no momento em que se clica com o botão direito na palavra a ser corrigida.

\begin{figure}[H]
    \centering
    \includegraphics[width=.7\textwidth]{images/documentos/ativar_desativar_correcao_automatica/Imagem22.png}
    \caption{}
\end{figure}

Por fim, ao selecionar a opção de “Dicionário pessoal” no submenu de “Ferramentas”, uma tela em que pode-se adicionar novas palavras irá aparecer. As palavras adicionadas ali não são submetidas à correção ortográfica. Verifique que na imagem a seguir a palavra “exempo” foi adicionada no dicionário pessoal e em seguida essa palavra não é mais vista como incorreta pelo corretor.

\begin{figure}[H]
    \centering
    \includegraphics[width=.8\textwidth]{images/documentos/ativar_desativar_correcao_automatica/Imagem23.png}
    \caption{}
\end{figure}

\subsection{Listas}
O Google Documentos fornece a possibilidade de criar listas com diversas aparências. Esses itens são bastante interessantes no momento de destacar alguma informação, demonstrar um passo a passo ou organizar um conjunto de dados.
Você pode encontrar as possibilidades de lista ao clicar nos seguintes três ícones, cada um com seu submenu.

\begin{figure}[H]
    \centering
    \includegraphics[width=.5\textwidth]{images/documentos/ativar_desativar_correcao_automatica/Imagem24.png}
    \caption{}
\end{figure}

\begin{figure}[H]
    \centering
    \includegraphics[width=.5\textwidth]{images/documentos/ativar_desativar_correcao_automatica/Imagem25.png}
    \caption{}
\end{figure}

\begin{figure}[H]
    \centering
    \includegraphics[width=.5\textwidth]{images/documentos/ativar_desativar_correcao_automatica/Imagem26.png}
    \caption{}
\end{figure}

As listas de marcação contam com caixas individuais que podem ser marcadas como concluídas. Utilizando a tecla tab, é possível criar subitens em cada item da lista (filhos e netos). O primeiro item do submenu risca as tarefas já concluídas, enquanto no segundo, a mudança visual é apenas na caixa de seleção.

\begin{figure}[H]
    \centering
    \includegraphics[width=.48\textwidth]{images/documentos/ativar_desativar_correcao_automatica/Imagem27.png}
    \caption{}
\end{figure}

As listas com marcadores utilizam recursos visuais como pontos preenchidos, pontos vazados, caixas, setas e estrelas. É possível selecionar as combinações de ícones desejados no submenu. Alguns exemplos na imagem abaixo:

\begin{figure}[H]
    \centering
    \includegraphics[width=.48\textwidth]{images/documentos/ativar_desativar_correcao_automatica/Imagem28.png}
    \caption{}
\end{figure}

Por fim, ao selecionar a opção de listas numeradas, é possível organizar seus dados utilizando apenas números, números em conjunto com letras e algarismos romanos. Veja exemplos na imagem a seguir:

\begin{figure}[H]
    \centering
    \includegraphics[width=.48\textwidth]{images/documentos/ativar_desativar_correcao_automatica/Imagem29.png}
    \caption{}
\end{figure}