% sections/documentos/cabecalho_e_rodape.tex
% !TeX root = ../../main.tex

\section{Cabeçalho e Rodapé}

Os cabeçalhos e rodapés são áreas do documento que aparecem repetidamente em todas as páginas, localizadas no topo (cabeçalho) e na parte inferior (rodapé). Eles servem para adicionar informações fixas, como título, autor, número de página, data ou outros elementos que possam auxiliar na organização e visualização do documento.

\subsection{Inserindo Cabeçalho e Rodapé}

%\begin{passos}
    \begin{enumerate}
        \item Para adicionar um cabeçalho, clique duas vezes com o botão esquerdo do mouse, com o cursor próximo à linha superior de qualquer página do documento. Uma caixa de edição será aberta, permitindo digitar ou inserir o conteúdo desejado.
        \item Basta adicionar um texto ou imagem de preferência e, em seguida, clicar em qualquer outro ponto da página para fechar a caixa de edição. A informação inserida aparecerá automaticamente no topo de todas as páginas do documento.
        \item Para adicionar informações ao rodapé, o processo é o mesmo: clique duas vezes no espaço próximo ao limite inferior da página e insira o conteúdo desejado. Assim, o texto ou imagem será exibido na parte inferior de todas as páginas do documento.
    \end{enumerate}
%\end{passos}


\subsection{Formatação de Espaçamento }
	É possível personalizar o tamanho do cabeçalho e rodapé das páginas e alterar algumas configurações estruturais.
 
\begin{enumerate}
    \item Para alterar o espaçamento de cabeçalho ou rodapé, abra o menu \textbf{Formatar} e selecione a opção \textbf{Cabeçalhos e Rodapés}.
    \item Uma janela de configuração irá abrir com dois tipos de personalização, \textbf{Margens} e \textbf{Layout}.
    
    \begin{figure}[H]
        \centering
        \includegraphics[width=.7\textwidth]{images/documentos/cabecalho_e_rodape/Imagem1.png}
        \caption{Tela de dimensionamento do cabeçalho e rodapé.}
    \end{figure}

    \item Na seção Margens, é possível definir o tamanho dos cabeçalhos e rodapés do arquivo, ajustando a distância entre essas áreas e o conteúdo principal do documento.
    \item A seção Layout apresenta duas configurações adicionais:
\end{enumerate}

\begin{itemize}
    \item \textbf{Primeira página diferente:} ao ativar essa opção, o conteúdo do cabeçalho e rodapé não será exibido na primeira página, o que é útil para capas e folhas de rosto, por exemplo.
    \item \textbf{Diferentes em páginas pares e ímpares:} permite utilizar modelos distintos de cabeçalhos e rodapés para páginas pares e ímpares, recurso comum em livros, revistas e relatórios extensos.
\end{itemize}

