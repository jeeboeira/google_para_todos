% sections/documentos/download_imprimir.tex
% !TeX root = ../../main.tex

\section{Download e Imprimir}


\subsection{Baixar o arquivo}
Você pode fazer o download do arquivo em qualquer momento de sua produção. Depois de baixar e gerar um novo arquivo, 
nenhuma nova alteração realizada irá afetar o arquivo anteriormente criado.

A opção se encontra no menu \textbf{Arquivo}, com o nome \textbf{Baixar}. Segue uma breve explicação sobre os formatos mais frequentemente utilizados, 
caso surjam dúvidas na hora de escolher:

\begin{itemize}
    \item \textbf{.docx:} para abrir e editar o arquivo no Microsoft Word;
    \item \textbf{.odt:} para abrir e editar o arquivo no Libreoffice Writer (ferramenta gratuita);
    \item \textbf{.pdf:} se o documento estiver concluído e não serem necessárias novas edições;
\end{itemize}

\subsection{Imprimir o arquivo}
Para realizar a impressão do arquivo, é necessário que haja uma impressora conectada ao computador, via USB ou Wi-fi, por exemplo. 
Com a impressora ligada, clique na opção \textbf{Imprimir}, presente no menu \textbf{Arquivo}, ou utilize o atalho \tecla{Ctrl} + \tecla{P}. 

A tela de impressão irá abrir sobre a janela do arquivo. Certifique-se que a impressora correta está selecionada, por fim, clique em \textbf{Imprimir}.