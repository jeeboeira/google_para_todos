% sections/documentos/documentos.tex
% !TeX root = ../main.tex

\section{Introdução ao Documentos Google}
O Google Documentos é uma ferramenta de escrita e edição de texto produzida e 
mantida pela Google, seu uso é online e gratuito à todos os usuários de contas 
Google. O acesso à plataforma é realizado por qualquer navegador (Google Chrome,
 Microsoft Edge, Safari, etc…) através do link 
 \hyperlink{https://docs.google.com/}{https://docs.google.com/}, você precisará 
realizar o login na sua conta Google e já terá acesso à página com alguns 
modelos de documentos pré-prontos. 

\begin{figure}[H]
    \centering
    \includegraphics[width=.9\textwidth]{/documentos/Imagem 9.png}
    \caption{Ao clicar em “Documento em branco”, você será direcionado à um novo 
        arquivo vazio.}
\end{figure}

\begin{figure}[H]
    \centering
    \includegraphics[width=.9\textwidth]{/documentos/Imagem 10.png}
    \caption{}
\end{figure}


A seguir, serão apresentados alguns elementos disponíveis na tela para 
contextualizar algumas funcionalidades básicas do Google Documentos:

\subsubsection{Título do arquivo}
O título do arquivo fica no canto superior esquerdo da página. Para renomeá-lo é 
só clicar no campo e escrever o novo título.

\subsubsection{Menus}
Os menus ficam localizados logo abaixo do título do arquivo. As opções são: 
“Arquivo”, “Editar”, “Ver”, “Inserir”, “Formatar”, “Ferramentas”, “Extensões” e 
“Ajuda”.  Ao clicar nos menus, outros submenus irão aparecer.

\subsubsection{Barra de ferramentas}
A barra de ferramentas fica logo abaixo dos menus. Ao longo da apostila, as 
funcionalidades dos itens da barra de ferramentas serão explicados.


\begin{figure}[H]
    \centering
    \includegraphics[width=.9\textwidth]{/documentos/Imagem 11.png}
    \caption{}
\end{figure}


\subsubsection{Página}
A página do arquivo é o grande campo branco, nele você conseguirá construir o 
seu arquivo. Ela fica no centro da tela, é só clicar com o mouse e começar a 
escrever.

\subsubsection{Ajustar o zoom}
Caso a página esteja muito pequena ou muito grande em sua tela, é possível 
ajustar o zoom seguindo os passos:

\begin{enumerate}
    \item Clicar na opção “Zoom” da barra de ferramentas (É um menu com o texto 
    “100\%”);
    \item Selecionar a opção desejada. Aumente a porcentagem para aumentar o 
    zoom e vice versa.
\end{enumerate}

\begin{dica}
Ajustar o zoom aumenta a aparência das letras no seu computador, mas mantém o tamanho correto para a impressão.
\end{dica}


\subsubsection{Réguas}
As réguas ficam na lateral esquerda e superior da tela.

\begin{checagem}[title=Para exibir as réguas]
    Caso você não esteja vendo as réguas da página, você pode seguir os passos abaixo:
    \begin{enumerate}[leftmargin=*]
      \item Clicar sobre o menu “Ver”;
      \item Clicar sobre a opção “Exibir régua”;
    \end{enumerate}
\end{checagem}

Para exibir as réguas
Caso você não esteja vendo as réguas da página, você pode seguir os passos 
abaixo:
Clicar sobre o menu “Ver”;
Clicar sobre a opção “Exibir régua”;


\subsubsection{Ajustar as margens da página}
Ao mexer a seta azul da direita, você ajusta o tamanho da margem direita, ou 
seja, até onde o texto chega na página.

Ao clicar à esquerda da seta da esquerda, você consegue mover a margem esquerda 
da página inteira.

Ao mexer a seta azul da esquerda, você consegue ajustar a margem esquerda da 
linha que está com o cursor.

Ao mexer somente a barra horizontal azul, você passa para um ajuste mais fino, 
você ajusta a margem esquerda da segunda linha de texto, permitindo que você 
diferencie o começo dos parágrafos, por exemplo.

\begin{figure}[H]
    \centering
    \includegraphics[width=.9\textwidth]{/documentos/Imagem 12.png}
    \caption{}
\end{figure}

