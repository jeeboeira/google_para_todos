% sections/documentos/menu_ferramentas.tex
% !TeX root = ../../main.tex

\section{Menu Ferramentas}
\subsection{Ortografia e Gramática}

Na barra de menu localizada na parte superior da tela, logo abaixo do nome do documento, clique na opção "Ferramentas". No menu suspenso que é exibido, selecione a opção "Ortografia e gramática". Isso abrirá um submenu com as opções:

\begin{itemize}
    \item Verificação ortográfica e gramatical
    \item Mostrar sugestões de ortografia
    \item Mostrar sugestões gramaticais
    \item Dicionário pessoal
\end{itemize}

Será discutido a seguir sobre cada uma destas opções.

\begin{figure}[H]
    \centering
    \includegraphics[width=.9\textwidth]{/documentos/menu_ferramentas/pt1/Imagem1.png}
    \caption{Menu de Ferramentas com a opção "Ortografia e gramática" expandida}
\end{figure}


\subsubsection{Verificação ortográfica e gramatical}
Ao selecionar esta opção do submenu citado, será aberto um painel sobreposto no canto superior direito da página. Neste painel encontram-se as sugestões de ajustes de ortografia e gramática. Esta opção também é acessível através do atalho "Ctrl + Alt + X".

\begin{figure}[H]
    \centering
    \includegraphics[width=.9\textwidth]{/documentos/menu_ferramentas/pt1/Imagem2.png}
    \caption{Painel de verificação ortográfica e gramatical em destaque}
\end{figure}


\paragraph{Navegação entre as sugestões}
Use as setas ao lado do título do painel sobreposto para alternar entre os elementos textuais passíveis de revisão. A seta apontada para a direita avança para a seleção do próximo elemento no documento, enquanto a seta voltada à esquerda alterna a seleção para o elemento anterior.

\paragraph{Aceitar sugestões de alteração}
Para aceitar uma sugestão, basta clicar em uma das opções disponíveis no corpo do painel. Elas aparecerão com formato de botões arredondados logo acima dos botões "Ignorar" e "Aceitar". Caso tenha somente uma sugestão para o elemento selecionado, clicar no botão "Aceitar" terá o mesmo efeito que clicar na sugestão. Qualquer uma das duas formas de aceitação resultará na alteração do elemento selecionado e navegará para o próximo a ser revisado.

\paragraph{Ignorar sugestões de alteração}
Para ignorar a alteração no elemento selecionado, basta clicar no botão "Ignorar". Isso remove a marcação colorida abaixo do elemento e alterna para o próximo a ser revisado.

\paragraph{Mais opções de revisão gramatical e ortográfica}
\label{Mais opções de revisão gramatical e ortográfica}
Caso o texto do elemento selecionado apareça em mais lugares, abre-se a opção de alterar todas as suas ocorrências com uma única ação. Ao lado do botão "Concluir", há um ícone de reticências disposto na vertical. Ao clicar nele, será aberto um menu suspenso como exibido na imagem a seguir.

\begin{figure}[H]
    \centering
    \includegraphics[width=.9\textwidth]{/documentos/menu_ferramentas/pt1/Imagem3.png}
    \caption{Menu "Mais opções" do painel "Ortografia e gramática" aberto}
\end{figure}

\textbf{Note que abrem-se opções para:}

\begin{itemize}
    \item Aceitar a sugestão em todas as ocorrências: Altera todas as ocorrências para a sugestão selecionada;
    \item Ignorar todas as ocorrências: Ignora todas as ocorrências do texto criticado;
    \item Adicionar expressão ao dicionário pessoal: Adiciona item criticado ao dicionário pessoal;
    \item Ver dicionário pessoal: Abre uma janela sobreposta à página para gerir as palavras do dicionário pessoal do usuário, podendo livremente adicionar e remover itens.
\end{itemize}

\subsubsection{Mostrar sugestões de ortografia}
Esta opção serve como uma caixa de seleção para ativar/desativar sugestões de ortografia, que são as marcações em vermelho abaixo de palavras com erros ortográficos identificados.

\subsubsection{Mostrar sugestões gramaticais}
Esta opção permite ativar/desativar sugestões de gramática, que correspondem às marcações em azul abaixo dos elementos textuais passíveis de revisão gramatical.

\subsubsection{Dicionário pessoal}
% TODO: Colocar a referência do item
Acessa o mesmo componente discutido no item acima. Mais opções de revisão gramatical e ortográfica, leia sobre a opção "Ver dicionário pessoal" para mais informações.


/subsection{Comparar Documentos}
Caso o usuário guarde cópias de um mesmo documento e queira verificar quais as diferenças entre as diferentes versões, a opção "Comparar documentos" do menu "Ferramentas" pode ser útil. 

Acesse-a clicando na opção "Ferramentas" localizada na barra de menus na parte superior da página e, em seguida, clique na opção "Comparar documentos". Realizar esta ação abrirá a janela "Comparar documentos".

Clicar no botão "Meu Google Drive" nesta janela abre uma outra janela que permite ao usuário navegar tanto por pastas do seu Google Drive como também por itens compartilhados com ele, podendo somente selecionar outros arquivos do Google Documentos.

\begin{figure}[H]
    \centering
    \includegraphics[width=.9\textwidth]{/documentos/menu_ferramentas/pt1/Imagem4.png}
    \caption{Janela de seleção de arquivo do Google Drive, com um arquivo selecionado para comparação}
\end{figure}

Selecione um item compatível e clique no botão "Abrir". Realizar esta ação volta para a janela "Comparar documentos". Note que, agora, o botão "Meu Google Drive" mudou o seu texto para ser o título do documento selecionado.

Logo abaixo há uma caixa de texto que permite atribuir as diferenças entre os documentos para um nome qualquer. Caso queira atribuir as mudanças para outra pessoa, basta digitar o nome dela nesta caixa de texto.

Logo em seguida, há a opção de incluir comentários do documento selecionado na comparação. Marque a caixa de seleção se deseja incluir esses elementos.

\textbf{Na parte inferior da janela "Comparar documentos" existem os seguintes botões:}

\begin{itemize}
    \item \textbf{Saiba mais:} Abre uma nova aba no navegador com uma página de ajuda sobre a ferramenta, útil caso queira obter informações oficiais sobre a funcionalidade;
    \item \textbf{Cancelar:} Fecha a janela "Comparar documentos", cancelando a operação;
    \item \textbf{Comparar:} Gera a comparação entre os documentos. Ao clicar neste botão, aguarde até que a comparação fique pronta. Assim que ela estiver preparada, o texto da janela irá mudar, avisando sobre a disponibilidade de abertura da comparação. Clique em "Abrir" para abrir uma nova aba do navegador com a comparação gerada.
\end{itemize}

Com a aba de comparação de documentos aberta, as alterações entre o documento originalmente aberto e o selecionado para comparação aparecerão como comentários neste novo documento. O usuário pode tanto aceitar como rejeitar tais alterações.

\begin{figure}[H]
    \centering
    \includegraphics[width=.9\textwidth]{/documentos/menu_ferramentas/pt1/Imagem5.png}
    \caption{Comparação entre dois documentos. Note os comentários localizados na parte direita da página, sinalizando as diferenças entre os documentos}
\end{figure}


%---- PARTE II ----%


\subsection{Números de Linha}
Os números de linha são uma ferramenta que o Google Documentos disponibiliza para que cada linha seja numerada sequencialmente no início. Para utilizar esse recurso, você pode selecionar o item “Ferramentas” na listagem de menus, em seguida, clique na opção “Números de Linha”. O seguinte painel lateral será aberto:

\begin{figure}[H]
    \centering
    \includegraphics[width=.8\textwidth]{/documentos/menu_ferramentas/pt2/Imagem13.png}
    \caption{Painel lateral aberto com configurações da numeração de linhas}
\end{figure}

Ao selecionar a opção “Mostrar o número de linhas”, será possível observar que as linhas já aparecem com contagem no seu campo de texto. 

\begin{figure}[H]
    \centering
    \includegraphics[width=.9\textwidth]{/documentos/menu_ferramentas/pt2/Imagem14.png}
    \caption{Corpo do documento com as linhas numeradas até o 20}
\end{figure}

A partir desse ponto, você pode configurar a numeração das linhas como desejar:

\begin{itemize}
    \item \textbf{Modo de numeração de linhas:}
    \begin{itemize}
        \item \textbf{Continuar no documento:} A numeração seguirá um fluxo contínuo durante todo o documento.
        \item \textbf{Reiniciar em cada página:} A cada página, a contagem irá reiniciar do 1.
        \item \textbf{Reiniciar em cada seção:} A cada seção, a numeração das linhas voltará ao início.
    \end{itemize}
    \item \textbf{Aplicar a:}
    \begin{itemize}    
        \item \textbf{Documento inteiro:} Essa opção manterá todas as linhas do documento enumeradas.
        \item \textbf{Esta seção:} Por outro lado, essa opção mantém a configuração de numeração de linhas apenas para uma seção específica do seu documento.
    \end{itemize}
\end{itemize}

\subsection{Traduzir documento}
Essa funcionalidade do Google Documentos é responsável por criar uma cópia do seu atual documento em outra língua, sem alterar o conteúdo do documento original.

Você pode acessar esse recurso selecionando o item “Ferramentas” na listagem de menus. Em seguida, clique na opção “Traduzir documento”. A seguinte janela irá aparecer:

\begin{figure}[H]
    \centering
    \includegraphics[width=.9\textwidth]{/documentos/menu_ferramentas/pt2/Imagem15.png}
    \caption{Menu com as configurações de “Traduzir documento” aberto}
\end{figure}

No primeiro campo, você pode digitar o nome do novo documento criado. Já, no campo inferior, você pode selecionar a língua desejada para tradução.

Ao clicar em “Traduzir”, um novo documento será criado e aberto automaticamente em uma nova guia do seu navegador.

\subsection{Preferências}
As preferências são configurações que controlam correções automáticas, substituições de um texto por outro e outros facilitadores que podem lhe auxiliar na tarefa de digitação. Você pode acessar esse recurso selecionando o item “Ferramentas” na listagem de menus, em seguida, clique na opção “Preferências”. A seguinte janela será aberta:

\begin{figure}[H]
    \centering
    \includegraphics[width=.9\textwidth]{/documentos/menu_ferramentas/pt2/Imagem16.png}
    \caption{Menu com configurações de “Preferências” aberto na aba “Geral”}
\end{figure}

Você pode perceber que algumas opções já aparecem pré-selecionadas. Caso você deseje que elas sejam desativadas, basta apenas clicar na caixinha ao lado do respectivo título. Como a maioria das opções contam com títulos bastante explicativos, iremos esclarecer a configuração “Ativar Markdown”: o Markdown é uma linguagem de marcação bastante simples que, ao ser ativada no Google Documentos, permite que você escreva com a sintaxe dessa linguagem de forma que os símbolos sejam traduzidos visualmente, de forma automática.

Observe também a segunda aba da janela, chamada “Substituições”:

\begin{figure}[H]
    \centering
    \includegraphics[width=.9\textwidth]{/documentos/menu_ferramentas/pt2/Imagem16.png}
    \caption{Menu com configurações de “Preferências” aberto na aba “Substituições”}
\end{figure}

Nessa aba, é possível configurar uma combinação de caracteres que será substituída por outra combinação que você definir. Algumas configurações também já estão pré-definidas, mas você pode editá-las escrevendo nos campos ou apagá-las, clicando sobre o “x” que está  à direita de cada linha.


\subsection{Acessibilidade}
Essas configurações de acessibilidade são acessadas através do menu “Ferramentas” na listagem de menus, clicando sobre a opção “Acessibilidade”. Então, a tela abaixo irá aparecer:

\begin{figure}[H]
    \centering
    \includegraphics[width=.9\textwidth]{/documentos/menu_ferramentas/pt2/Imagem18.png}
    \caption{Tela com configurações de “Acessibilidade”}
\end{figure}

No primeiro item, juntamente com seus subitens, podemos ativar configurações que funcionam juntamente com leitores de tela e softwares terceiros de leitura em braille. Já o segundo item do menu possibilita a utilização de ferramentas terceiras de leitura com lupa, que garantem um zoom na área ao redor de onde o mouse está apontando.