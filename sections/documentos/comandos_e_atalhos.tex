% sections/documentos/comandos_e_atalhos.tex
% !TeX root = ../../main.tex

\section{Comandos e Atalhos}
Os comandos e atalhos do Google Docs são recursos que facilitam e agilizam o uso da ferramenta. Com eles, o usuário pode executar diversas ações, como formatar texto, inserir elementos ou navegar pelo documento, sem precisar recorrer constantemente ao mouse. Esses atalhos ajudam a otimizar o tempo e a produtividade, além de tornar a edição e a criação de documentos mais práticas. A seguir, você encontrará alguns dos comandos mais úteis do Documentos Google.


\subsection{Edição e Formatação}

\renewcommand{\arraystretch}{1.8}

\begin{center}
\begin{tabular}{@{}llp{0.55\textwidth}@{}}
\textbf{Comando} & \textbf{Função} & \textbf{Descrição} \\ \hline
\tecla{Ctrl} + \tecla{C} & Copiar & Copia o texto ou elemento selecionado para a área de transferência, permitindo colar em outro local. \\
\tecla{Ctrl} + \tecla{X} & Recortar & Remove o conteúdo selecionado e o copia para a área de transferência. \\
\tecla{Ctrl} + \tecla{V} & Colar & Insere o conteúdo copiado ou recortado no ponto onde o cursor está posicionado. \\
\tecla{Ctrl} + \tecla{Shift} + \tecla{V} & Colar sem formatação & Cola o conteúdo copiado removendo qualquer formatação (cor, tamanho ou estilo de fonte), mantendo apenas o texto simples. \\
\tecla{Ctrl} + \tecla{Z} & Desfazer & Reverte a última ação realizada, como exclusão, formatação ou digitação incorreta. \\
\tecla{Ctrl} + \tecla{Shift} + \tecla{Z} & Refazer & Refaz a ação que foi desfeita, restaurando o estado anterior do texto. \\
\tecla{Ctrl} + \tecla{B} & Negrito & Aplica o estilo negrito ao texto selecionado, destacando-o visualmente. \\
\tecla{Ctrl} + \tecla{I} & Itálico & Aplica o estilo itálico, deixando o texto levemente inclinado para enfatizar termos. \\
\tecla{Ctrl} + \tecla{U} & Sublinhado & Adiciona uma linha abaixo do texto selecionado, usada para destacar títulos e termos importantes. \\
\tecla{Ctrl} + \tecla{\textbackslash} & Limpar formatação & Remove todas as formatações aplicadas (cor, fonte, negrito, itálico, alinhamento etc.), deixando o texto no estilo padrão do documento. \\
\tecla{Ctrl} + \tecla{.} & Aumentar fonte & Aumenta o tamanho da fonte do texto selecionado. \\
\tecla{Ctrl} + \tecla{,} & Diminuir fonte & Reduz o tamanho da fonte do texto selecionado. \\
\end{tabular}
\end{center}


\subsection{Seleção e Texto}

\begin{center}
\begin{tabular}{@{}llp{0.45\textwidth}@{}}
\textbf{Comando} & \textbf{Função} & \textbf{Descrição} \\ \hline
\tecla{Ctrl} + \tecla{A} & Selecionar tudo & Seleciona todo o conteúdo do documento, incluindo texto, imagens e tabelas, de uma só vez. \\
\tecla{Shift} + \tecla{$\leftarrow$} / \tecla{$\rightarrow$} & Selecionar caracteres & Expande a seleção caractere por caractere, para ajustar a marcação com precisão. \\
\tecla{Ctrl} + \tecla{Shift} + \tecla{$\rightarrow$} / \tecla{$\leftarrow$} & Selecionar palavras & Permite selecionar palavras inteiras de forma rápida, sem precisar arrastar o mouse. \\
\tecla{Shift} + \tecla{$\uparrow$} / \tecla{$\downarrow$} & Selecionar linhas & Expande a seleção linha por linha, facilitando a marcação de parágrafos inteiros ou blocos de texto consecutivos. \\
\tecla{Ctrl} + \tecla{Home} & Ir para o início & Move o cursor imediatamente para o início do texto, sem precisar rolar a página. \\
\tecla{Ctrl} + \tecla{End} & Ir para o final & Leva o cursor diretamente ao final do documento, agilizando a navegação em arquivos extensos. \\
\tecla{Ctrl} + \tecla{Backspace} & Excluir palavra anterior & Apaga a palavra anterior ao cursor. \\
\tecla{Ctrl} + \tecla{Delete} & Excluir próxima palavra & Remove a palavra à frente do cursor. \\
\end{tabular}
\end{center}

\subsection{Organização e Listas}

\begin{center}
\begin{tabular}{@{}llp{0.5\textwidth}@{}}
\textbf{Comando} & \textbf{Função} & \textbf{Descrição} \\ \hline
\tecla{Ctrl} + \tecla{Shift} + \tecla{8} & Criar lista com marcadores & Inicia uma lista com marcadores em pontos. \\
\tecla{Ctrl} + \tecla{Shift} + \tecla{9} & Remover lista & Converte o item em texto normal, sem marcador. \\
\tecla{Tab} & Recuar item da lista & Aumenta o nível de recuo em uma lista. \\
\tecla{Shift} + \tecla{Tab} & Voltar item da lista & Diminui o nível de recuo em uma lista. \\
\tecla{Ctrl} + \tecla{Shift} + \tecla{E} & Centralizar texto & Centraliza o parágrafo atual. \\
\tecla{Ctrl} + \tecla{Shift} + \tecla{L} & Alinhar à esquerda & Alinha o texto à margem esquerda. \\
\tecla{Ctrl} + \tecla{Shift} + \tecla{R} & Alinhar à direita & Alinha o texto à margem direita. \\
\tecla{Ctrl} + \tecla{Shift} + \tecla{J} & Justificar texto & Distribui o texto uniformemente entre as margens. \\
\end{tabular}
\end{center}

\subsection{Navegação e Localização}

\begin{center}
\begin{tabular}{@{}llp{0.46\textwidth}@{}}
\textbf{Comando} & \textbf{Função} & \textbf{Descrição} \\ \hline
\tecla{Ctrl} + \tecla{F} & Localizar & Busca palavras ou frases no documento. \\
\tecla{Ctrl} + \tecla{H} & Localizar e substituir & Substitui uma palavra por outra em todo o texto. \\
\tecla{Ctrl} + \tecla{K} & Inserir ou editar link & Adiciona ou modifica um link no texto. \\
\tecla{Alt} + \tecla{Enter} & Abrir link selecionado & Abre o link ativo em nova aba. \\
\tecla{Ctrl} + \tecla{Enter} & Inserir quebra de página & Inicia uma nova página no documento. \\
\tecla{Ctrl} + \tecla{Alt} + \tecla{Shift} + \tecla{H} & Histórico de versões & Mostra as versões anteriores do arquivo. \\
\tecla{Ctrl} + \tecla{/} & Exibir lista de atalhos & Abre a janela com todos os atalhos disponíveis do Google Docs. \\
\end{tabular}
\end{center}

\subsection{Ferramentas e Recursos}

\begin{center}
\begin{tabular}{@{}llp{0.44\textwidth}@{}}
\textbf{Comando} & \textbf{Função} & \textbf{Descrição} \\ \hline
\tecla{Ctrl} + \tecla{Shift} + \tecla{C} & Contagem de palavras &
Abre uma janela que mostra o número total de palavras, caracteres e páginas do documento.
Também é possível marcar a opção para exibir a contagem em tempo real enquanto escreve, útil para manter limites de tamanho em trabalhos escolares ou artigos. \\
\tecla{Ctrl} + \tecla{Alt} + \tecla{Shift} + \tecla{I} & Dicionário &
Exibe um painel lateral com o significado de palavras, sinônimos e traduções. \\
\tecla{Ctrl} + \tecla{Alt} + \tecla{M} & Inserir comentário &
Adiciona um comentário no ponto onde o cursor está posicionado. \\
\tecla{Ctrl} + \tecla{Alt} + \tecla{Shift} + \tecla{A} & Comentários &
Abre a visualização de todos os comentários e sugestões feitos no documento, ideal para trabalhos colaborativos. \\
\tecla{Ctrl} + \tecla{Alt} + \tecla{Shift} + \tecla{S} & Esquemas do documento &
Mostra o painel lateral com os títulos e subtítulos (estrutura hierárquica do texto). \\
\tecla{Ctrl} + \tecla{Shift} + \tecla{F} & Modo compacto &
Oculta temporariamente a barra de menus e ferramentas, liberando mais espaço para digitar. \\
\end{tabular}
\end{center}

\subsection{Arquivo e Documento}

\begin{center}
\begin{tabular}{@{}llp{0.67\textwidth}@{}}
\textbf{Comando} & \textbf{Função} & \textbf{Descrição} \\ \hline
\tecla{Ctrl} + \tecla{S} & Salvar & Garante que todas as alterações sejam salvas imediatamente. \\
\tecla{Ctrl} + \tecla{P} & Imprimir & Abre as opções de impressão. \\
\tecla{Ctrl} + \tecla{O} & Abrir documento & Permite escolher outro arquivo para editar. \\
\tecla{Ctrl} + \tecla{N} & Novo documento & Cria um novo arquivo em branco. \\
\end{tabular}
\end{center}