% sections/documentos/edicao_colaborativa.tex
% !TeX root = ../../main.tex

\section{Edição Colaborativa}


\subsection{Compartilhamento e suas permissões}
Outra funcionalidade bastante útil que o Google Documentos nos oferece é o compartilhamento e permissionamento de arquivos. Através dessa função é possível fazer com que outra pessoa tenha acesso ao seu documento, permitindo comentários, edições ou outras alterações. Para visualizar essas opções, basta clicar no botão “Compartilhar” no canto superior direito da tela, em seguida, a seguinte tela irá aparecer:

\begin{figure}[H]
    \centering
    \includegraphics[width=1\textwidth]{images/documentos/edicao_colaborativa/Imagem1.png}
    \caption{Tela mostrando as configurações de compartilhamentofinidos}
\end{figure}

Você pode observar que existe uma lista de pessoas com acesso (inclusive seu próprio usuário), e cada um conta com uma permissão diferente na direita. Através do campo de inserção de texto você pode adicionar outro usuário e definir qual a sua permissão:

\begin{itemize}
    \item \textbf{Leitor:} Esse usuário poderá apenas visualizar o documento e ler seu conteúdo.
    \item \textbf{Comentador:} O usuário poderá ler o documento e adicionar comentários, como sugestões de melhorias.
    \item \textbf{Editor:} Esse usuário tem permissões plenas para escrever, apagar, ler e adicionar outros usuários também.
\end{itemize}

O usuário convidado receberá um email com o convite para visualizar o arquivo e, também, uma mensagem de convite caso você tenha escrito uma.

Observe que, na imagem acima, também existe o botão “Copiar Link”, que irá gerar um link para compartilhar com outras pessoas. O link gerado seguirá as configurações definidas na opção selecionada acima.


\subsection{Comentários}
Os comentários são uma funcionalidade bastante útil que permite selecionar um trecho do texto e abrir uma aba lateral com comentários. Outros usuários podem responder ou marcar a discussão como resolvida.
Para adicionar um comentário, selecione um intervalo de texto qualquer e, na lateral direita, clique sobre o primeiro ícone. Após isso, basta iniciar a escrita do comentário.

\begin{figure}[H]
    \centering
    \includegraphics[width=1\textwidth]{images/documentos/edicao_colaborativa/Imagem2.png}
    \caption{Exemplo onde o botão de adicionar comentário aparece quando o texto é selecionado}
\end{figure}

\begin{dica}
    Ao digitar @ e escrever o email do usuário, essa pessoa será marcada no comentário e receberá notificação da sua menção. É um recurso bastante útil para garantir que o usuário visualize aquela discussão e possa respondê-la.
\end{dica}
