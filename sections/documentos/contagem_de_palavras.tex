% sections/documentos/contagem_de_palavras.tex
% !TeX root = ../../main.tex

\section{Contagem de Palavras}
Além da contagem de palavras, o Google Docs também oferece a contagem de páginas e caracteres (com e sem espaços).

Para visualizar essas informações é possível acessar a tela “Contagem de palavras”, disponibilizada na aba “Ferramentas”, na guia do menu principal. Também é possível acessar essa tela utilizando o atalho Ctrl+Shift+C.

\begin{figure}[H]
    \centering
    \includegraphics[width=.55\textwidth]{images/documentos/contagem_de_palavras/Imagem1.png}
    \caption{Tela com as informações de contagem, destaque no botão Mostrar contagem de palavras ao digitar}
\end{figure}

Ao abrir a tela com determinada parte do texto selecionado, podemos ver as informações correspondentes à seleção e ao total do arquivo. Caso não haja texto selecionado, serão exibidas somente as informações de todo o documento.


\subsection{Para visualizar a quantidade de palavras ao digitar}
Por conveniência, é possível visualizar a quantidade de palavras em conjunto com o texto. Para isso, basta abrir a tela “Contagem de Palavras” (conforme tópico anterior) e marcar a caixa de seleção que contém a opção “Mostrar a contagem de palavras ao digitar”. 

\begin{dica}
    Ao selecionar parte do texto com essa opção marcada, é possível contar as palavras de intervalos de texto específicos.
\end{dica}

\begin{dica}
    Para acompanhar alguma das outras opções de contagem disponíveis (páginas, caracteres, caracteres sem espaço), é possível clicar sobre a caixinha que contém a contagem de palavras, localizada no canto inferior esquerdo e selecionar a opção desejada.

    \begin{figure}[H]
        \centering
        \includegraphics[width=.6\textwidth]{images/documentos/contagem_de_palavras/Imagem2.png}
        \caption{Evento com os valores definidos}
    \end{figure}
    
\end{dica}

%\begin{figure}[H]
%    \centering
%    \includegraphics[width=.39\textwidth]{images/documentos/contagem_de_palavras/Imagem2.png}
%    \caption{Recorte mostrando a contagem de palavras em telaidos}
%\end{figure}
