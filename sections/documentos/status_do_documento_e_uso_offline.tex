% sections/documentos/status_do_documento_e_uso_offline.tex
% !TeX root = ../../main.tex

% TODO: As imagens desse arquivo talvez não tenham sido revisadas

\section{Status do Documento e Uso Offline}


\subsection{Status do Documento}
Para verificar o status do documento, clique sobre o último ícone à direita do título. A faixa superior, em azul, traz uma descrição sobre o status on-line do documento, e logo abaixo uma sobre o status off-line.

\begin{figure}[H]
    \centering
    \includegraphics[width=.39\textwidth]{images/documentos/status_do_documento_e_uso_offline/Imagem1.png}
    \caption{Evento com os valores definidos}
\end{figure}

\begin{atencao}
    Caso uma aba seja fechada com o status \textbf{Salvando…} é possível que haja perda de conteúdo, certifique-se de que o status do documento tenha a descrição \textbf{Todas as alterações foram salvas no Drive}.
\end{atencao}

No Google Docs, ao perder a conexão com a internet, não é mais possível editar o documento até que a conexão seja restabelecida. No entanto, é possível utilizar a ferramenta de forma off-line para dar continuidade aos trabalhos.

Quando o modo off-line está ativo, as edições realizadas são sincronizadas automaticamente com o arquivo on-line e logo assim a conexão é restabelecida.


\subsection{Uso Offline}
Essa configuração pode estar pré-configurada. Para verificar, clique no ícone \textbf{Ver status do documento}, visível na imagem anterior. Se houver um botão azul escrito \textbf{Ativar}, será necessário realizar os passos a seguir.

Inicialmente, verifique se há conexão com a internet e espaço na memória do dispositivo para os arquivos em questão.

Comece o processo clicando no botão \textbf{Ativar} descrito anteriormente.

Caso apareça uma janela dizendo que é necessária a instalação de uma extensão no navegador, clique em \textbf{Instalar}. Na aba que for aberta, clique no botão \textbf{Usar no Chrome} ou \textbf{Obter}. Clique, ainda, em \textbf{Adicionar extensão}, caso apareça uma nova janela. Em seguida, você pode fechar a aba. Atualize a página do Google Docs e repita o processo de ativação descrito anteriormente.

Caso apareça uma janela solicitando se é desejado ativar o uso off-line para os arquivos do drive, clique em \textbf{ativar} e atualize a página.

\begin{atencao}
    Lembre de esperar o status do documento apresentar a mensagem “Este documento está pronto para ser utilizado off-line” antes de retirar a conexão com a internet.
\end{atencao}

\begin{dica}
    Se você estiver logado em um e-mail estudantil ou corporativo e verificar que a opção “Ativar” não está aparecendo, é provável que a organização tenha desabilitado essa função para e-mails vinculados.
\end{dica}


\subsection{Disponibilização dos Arquivos Off-line}
Também é possível definir que alguns arquivos sempre estejam disponíveis no modo off-line, basta reproduzir os passos a seguir.

Abra a tela inicial do Drive e clique no ícone de utilização off-line destacado na imagem abaixo. 

\begin{figure}[H]
    \centering
    \includegraphics[width=.39\textwidth]{images/documentos/status_do_documento_e_uso_offline/Imagem2.png}
    \caption{Evento com os valores definidos}
\end{figure}

Após habilitar o uso off-line no Drive, clique no ícone de três pontos no canto direito superior do arquivo e selecione a opção \textbf{Tornar disponível offline}. Repita essa operação para todos os arquivos desejados.

\begin{atencao}
    Lembre de esperar os arquivos baixarem completamente antes de retirar a conexão com a internet.
\end{atencao}

Na aba em que foi ativado o uso off-line, os arquivos disponíveis estarão destacados. É possível selecionar um dos arquivos e iniciar a edição.

\begin{figure}[H]
    \centering
    \includegraphics[width=.39\textwidth]{images/documentos/status_do_documento_e_uso_offline/Imagem3.png}
    \caption{Evento com os valores definidos}
\end{figure}

\paragraph{EXTRA: No celular}
Para acessar os seus arquivos de forma off-line pelo celular, é preciso ter o aplicativo do Google Drive ou do Google Documentos instalado. Depois de abrir o aplicativo, encontrar o arquivo desejado e clicar no ícone de três pontos, selecione a opção \textbf{Tornar disponível off-line}. Um processo de download será iniciado e, assim que finalizado, o arquivo estará disponível para leitura e edição no modo off-line. 

% TODO: Aqui pedia imagem 4, mas ela não existe nos arquivos

\begin{dica}
    Começar arquivos em branco pelo celular não requer conexão com a internet
\end{dica}
