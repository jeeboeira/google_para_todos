% sections/documentos/introducao_documentos.tex
% !TeX root = ../../main.tex

% TODO: Jessé, comentei a seção duplicada no final do Docs, já ajustei aqui. Porém,
% vi que a numeração não está aparecendo. Se puder, dê uma olhada nisso depois

\section{Introdução ao Google Documentos}
O Google Documentos é uma ferramenta de escrita e edição de texto produzida e mantida pela Google, seu uso é online e gratuito à todos 
os usuários de contas Google. O acesso à plataforma é realizado por qualquer navegador (Google Chrome, Microsoft Edge, Safari, etc…) 
através do link \hyperlink{https://docs.google.com/}{https://docs.google.com/}, você precisará realizar o login na sua conta Google e já 
terá acesso à página com alguns modelos de documentos pré-prontos. 

\begin{figure}[H]
    \centering
    \includegraphics[width=1\textwidth]{images/documentos/introducao_documentos/Imagem9.png}
    \caption{Tela inicial do Google Docs com alguns modelos disponibilizados}
\end{figure}

Ao clicar em “Documento em branco”, você será direcionado à um novo arquivo vazio.

\begin{figure}[H]
    \centering
    \includegraphics[width=1\textwidth]{images/documentos/introducao_documentos/Imagem10.png}
    \caption{Visualização de um documento recém criado}
\end{figure}

A seguir, serão apresentados alguns elementos disponíveis na tela para contextualizar algumas funcionalidades básicas do Google Documentos:

\paragraph{Título do arquivo}
O título do arquivo fica no canto superior esquerdo da página. Para renomeá-lo é só clicar no campo e escrever o novo título.

\paragraph{Menus}
Os menus ficam localizados logo abaixo do título do arquivo. As opções são: “Arquivo”, “Editar”, “Ver”, “Inserir”, “Formatar”, “Ferramentas”, “Extensões” e “Ajuda”.  Ao clicar nos menus, outros submenus irão aparecer.

\paragraph{Barra de ferramentas}
A barra de ferramentas fica logo abaixo dos menus. Ao longo da apostila, as funcionalidades dos itens da barra de ferramentas serão explicados.

\begin{figure}[H]
    \centering
    \includegraphics[width=1\textwidth]{images/documentos/introducao_documentos/Imagem11.png}
    \caption{Barra de Ferramentas do documento}
\end{figure}

\paragraph{Página}
A página do arquivo é o grande campo branco, nele você conseguirá construir o seu arquivo. Ela fica no centro da tela, é só clicar com o mouse e começar a escrever.

\paragraph{Ajustar o zoom }
Caso a página esteja muito pequena ou muito grande em sua tela, é possível ajustar o zoom seguindo os passos:

\begin{itemize}
    \item Clicar na opção \textbf{Zoom} da barra de ferramentas (É um menu com o texto 100\%”);
    \item Selecionar a opção desejada. Aumente a porcentagem para aumentar o zoom e vice versa;
\end{itemize}

\begin{dica}
    Ajustar o zoom aumenta a aparência das letras no seu computador, mas mantém o tamanho correto para a impressão.
\end{dica}

\paragraph{Réguas}
As réguas ficam na lateral esquerda e superior da tela.

\begin{passos}
    Caso você não esteja vendo as réguas da página, você pode seguir os passos abaixo:
    \begin{enumerate}[leftmargin=*]
      \item Clicar sobre o menu “Ver”;
      \item Clicar sobre a opção “Exibir régua”;
    \end{enumerate}
\end{passos}

\paragraph{Ajustar as margens da página}
Ao mexer a seta azul da direita, você ajusta o tamanho da margem direita, ou seja, até onde o texto chega na página.

Ao clicar à esquerda da seta da esquerda, você consegue mover a margem esquerda da página inteira.

Ao mexer a seta azul da esquerda, você consegue ajustar a margem esquerda da linha que está com o cursor.

Ao mexer somente a barra horizontal azul, você passa para um ajuste mais fino, você ajusta a margem esquerda da segunda linha de texto, permitindo que você diferencie o começo dos parágrafos, por exemplo.

\begin{figure}[H]
    \centering
    \includegraphics[width=1\textwidth]{images/documentos/introducao_documentos/Imagem12.png}
    \caption{Exemplo de texto com as margens ajustadas}
\end{figure}