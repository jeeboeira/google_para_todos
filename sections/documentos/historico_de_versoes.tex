% sections/documentos/historico_de_versoes.tex
% !TeX root = ../../main.tex

\section{Histórico de Versões}
Histórico de Versões é um recurso do Google Docs que registra automaticamente as alterações do documento e através desses registros permite visualizar, restaurar e gerenciar versões anteriores do arquivo. Essa ferramenta é particularmente útil em trabalhos colaborativos por possibilitar a identificação de quem realizou determinada modificação, quando ela foi feita e qual foi o conteúdo alterado.

O histórico também oferece segurança contra perdas acidentais de conteúdo. Com ele, é possível recuperar trechos apagados, comparar versões antigas e retornar o documento ao estado anterior com apenas alguns cliques. Dessa forma, o recurso garante maior organização, rastreabilidade e tranquilidade ao trabalhar em documentos longos ou compartilhados com outros usuários.

\subsection{Acessando o Histórico de Versões}

O Google Docs salva automaticamente as alterações feitas no documento, e cada versão anterior do documento pode ser acessada facilmente.

\begin{enumerate}
    \item No menu \textbf{Arquivo}, selecione a opção \textbf{Histórico de versões} e, em seguida, \textbf{Ver histórico de versões}.
    \item Uma aba lateral será exibida, mostrando as versões anteriores listadas por data e hora.
    
    \begin{figure}[H]
        \centering
        \includegraphics[width=1\textwidth]{images/documentos/historico_de_versoes/Imagem1.png}
        \caption{Tela de visualização de versões, com as versões à direita}
    \end{figure}

    \item Clique sobre uma das versões para visualizar as alterações feitas. As modificações aparecem destacadas no texto.
\end{enumerate}

\begin{dica}
    Se preferir, você também pode usar o atalho de teclado \tecla{Ctrl} + \tecla{Alt} + \tecla{Shift} + \tecla{H} para abrir o histórico diretamente.
\end{dica}


\subsection{Restaurando uma Versão Anterior}

É possível restaurar o documento para qualquer estado anterior dele. Essa ação substitui o conteúdo atual pelo da versão escolhida.

\begin{passos}
    \begin{enumerate}
        \item Com a aba de histórico de versões aberta, selecione a versão desejada.
        \item Nos três pontos da versão escolhida, clique na opção \textbf{Restaurar esta versão}.
        \item Confirme a ação. O documento voltará exatamente ao estado salvo naquela data.
    \end{enumerate}
\end{passos}

\begin{dica}
    Se quiser recuperar apenas uma parte específica, abra a versão desejada, copie o trecho necessário e cole manualmente no documento atual.
\end{dica}

\subsection{Renomeando e Gerenciando Versões}

Para manter o histórico organizado, é possível nomear versões específicas. Essa prática é útil para identificar marcos importantes no desenvolvimento do documento, como revisões ou versões finais.

\begin{passos}
    \begin{enumerate}
        \item Na aba de histórico, selecione a versão que deseja renomear.
        \item Nos três pontos da versão escolhida, clique na opção \textbf{Renomear esta versão}.
        \item Digite um nome descritivo (por exemplo, \textbf{Versão Final} ou \textbf{Antes das correções}).
    \end{enumerate}
\end{passos}

\begin{atencao}
    Versões nomeadas não são excluídas automaticamente pelo sistema, o que garante que fiquem armazenadas de forma permanente no histórico.
\end{atencao}

\subsection{Recuperando Trechos Apagados}

O Histórico de Versões também pode ser utilizado para recuperar textos ou imagens que foram apagadas.

\begin{passos}
    \begin{enumerate}
        \item Acesse o Histórico de Versões.
        \item Selecione uma versão anterior e localize o trecho removido.
        \item Copie o conteúdo desejado para a versão atual do documento.
    \end{enumerate}
\end{passos}

Essa é a forma mais segura de restaurar partes específicas sem alterar o restante do arquivo.