% sections/documentos/menu_inserir.tex
% !TeX root = ../main.tex]

\section{Menu Inserir}


\subsection{Imagens}
Para inserir imagens no arquivo, tudo depende de onde que a imagem está, quando 
trabalhando com imagens que estão na internet ou no computador muitas vezes 
copiar e colar pode ser o suficiente. Porém, quando essa opção não funciona, 
podemos utilizar o menu “Inserir”.

\begin{figure}[H]
    \centering
    \includegraphics[width=.9\textwidth]{/documentos/Imagem 1.png}
    \caption{}
\end{figure}


\subsubsection{Fazer upload do computador}
\begin{enumerate}
    \item Selecionar a opção correspondente no menu “Inserir”;
    \item Localizar a imagem no seu computador;
    \item Clicar em “Abrir”;
\end{enumerate}


\subsubsection{Pesquisar na Web}
\begin{enumerate}
    \item Selecionar a opção correspondente no menu “Inserir”;
    \item Utilizar palavras-chave para buscar a imagem desejada;
    \item Clicar na imagem escolhida;
    \item Clicar no botão “Inserir” que vai aparecer no canto inferior direito;
\end{enumerate}


\subsubsection{Google Drive}
\begin{enumerate}
    \item Selecionar a opção correspondente no menu “Inserir”;
    \item Localizar a imagem no seu drive;
    \item Clicar no botão “Inserir” que vai aparecer no canto inferior direito;
\end{enumerate}


\subsubsection{Google Fotos}
\begin{enumerate}
    \item Selecionar a opção correspondente no menu “Inserir”;
    \item Localizar a imagem no seu Google Fotos;
    \item Clicar no botão “Inserir” que vai aparecer no canto inferior direito;
\end{enumerate}


\subsubsection{Câmera}
\begin{enumerate}
    \item Essa opção poderá ser utilizada quando o aparelho possuir uma câmera.
    \item Selecionar a opção correspondente no menu “Inserir”;
    \item Caso apareça a opção no seu navegador, permitir que o mesmo utilize a 
    câmera para tirar a foto dessa vez;
    \item Capture a foto e clique no botão “Inserir”;
\end{enumerate}


\subsubsection{Por URL}
\begin{enumerate}
    \item Selecionar a opção correspondente no menu “Inserir”;
    \item Colar o endereço da imagem;
    \item Clicar no botão “Inserir” que vai aparecer no canto inferior direito;
\end{enumerate}

\begin{dica}
Caso o endereço que você copiou esteja dando erro, verifique que você selecionou 
a opção “Copiar endereço da imagem” para copiar o endereço correto. 
\end{dica}


\subsubsection{Para ajustar o tamanho da imagem:}
\begin{enumerate}
    \item Selecionar a imagem;
    \item Pressionar os oito quadrados azuis nos vértices e arestas da imagem, 
    e arrastar;
\end{enumerate}

\begin{figure}[H]
    \centering
    \includegraphics[width=.9\textwidth]{/documentos/Imagem 2.png}
    \caption{}
\end{figure}


\subsubsection{Para rotacionar a imagem:}
\begin{enumerate}
    \item Selecionar a imagem;
    \item Pressionar e arrastar o círculo azul que aparece acima da imagem;
\end{enumerate}


\subsubsection{Para cortar a imagem:}
\begin{enumerate}
    \item Clicar com o botão direito na imagem;
    \item Selecionar a opção “Cortar imagem”;
    \item As bordas pretas que aparecem são as margens da imagem, é possível 
        ajustar tanto o tamanho da imagem, como o de suas bordas;
    \item Clique fora da imagem para completar o ajuste;
\end{enumerate}

\begin{figure}[H]
    \centering
    \includegraphics[width=.8\textwidth]{/documentos/Imagem 3.png}
    \caption{}
\end{figure}


\subsubsection{Para remover as alterações de corte na imagem:}
\begin{enumerate}
    \item Clicar com o botão direito na imagem;
    \item Selecionar a opção “Redefinir imagem”;
\end{enumerate}


\subsubsection{Para adicionar texto alternativo:}
\begin{enumerate}
    \item Clicar com o botão direito na imagem;
    \item Selecionar a opção “Texto alternativo”;
    \item Escrever o texto na aba lateral direita que foi aberta;
\end{enumerate}


\subsubsection{Para acessar o mais opções de imagem:}
\begin{enumerate}
    \item Clicar com o botão direito na imagem;
    \item Selecionar a opção “Opções de imagem”;
    \item Abre um menu lateral com mais opções, possui descrição na própria 
        ferramenta;
\end{enumerate}

\begin{figure}[H]
    \centering
    \includegraphics[width=.8\textwidth]{/documentos/Imagem 4.png}
    \caption{}
\end{figure}

\begin{dica}
O texto alternativo serve para acessibilidade, se o seu documento será consumido 
por uma pessoa com deficiência visual que utiliza descrição de áudio, é 
interessante adicionar uma descrição da imagem para ajudar com contextualização.
\end{dica}


\subsection{Tabelas}
As tabelas podem ser adicionadas no documento para ajudar a organizar a informação. 

\subsubsection{Para realizar a inserção de uma tabela no documento:}
Acessar o menu Inserir > Tabela;
Selecionar o tamanho da tabela desejado no esquema de grade;

DICA: Os números abaixo de esquema de grade se refere ao número de [colunas] x [linhas]

[IMAGEM 5]

DICA: Sob o menu “Elemento Básicos”, a ferramenta disponibiliza também alguns modelos prontos. O usuário é incentivado a investigar esses modelos e utilizá-los em parte ou integralmente.

\subsubsection{Para acessar os menus de coluna e linha:}
Ao passar o mouse pelas laterais esquerda e superior da tabela, os menus irão aparecer;

[IMAGEM 6]

\subsubsection{Para inserir uma coluna ou linha:}
É possível adicionar uma coluna ou linha clicando no símbolo de mais [+] nos menus de coluna e linha. A linha será adicionada abaixo da linha que tem o menu aberto. A coluna será adicionada à direita da que tem o menu aberto.

\subsubsection{Para trocar a ordem das colunas e linhas:}
É possível trocar a posição das colunas e linhas ao clicar e segurar o símbolo de seis pontos cinza claro que aparece nos menus de coluna e linha.

\subsubsection{Para fixar e desafixar o cabeçalho da tabela:}
Ao clicar no botão com ícone de tachinha no menu da linha, é possível fixar e desafixar o cabeçalho. É possível fixar mais de uma linha.

DICA: Fixar o cabeçalho, significa que, em tabelas que ocupam mais de uma página, as linhas fixadas vão aparecer no topo de todas as páginas.

DICA: O cabeçalho vai estar fixado quando o ícone de tachinha com a barra transversal estiver aparecendo. De forma a simbolizar que a operação de desafixação vai ser realizada ao clicar novamente.

\subsubsection{Para ordenar a tabela:}
No menu da coluna que se deseja ordenar, clicar no botão com ícone de três barras paralelas;
Selecionar se a ordem deve ser crescente ou decrescente.

DICA: Ordenar uma coluna mantém os valores da linha agrupados. Não edita a tabela, apenas organiza as informações.

\subsubsection{Para alterar as bordas da tabela:}
Ao passar o mouse por uma célula, aparece um botão com ícone de seta no canto superior direito. 
Ao clicar no botão é possível alterar a borda de uma célula ou grupo de células;

DICA: Para ajustes de cor e tamanho da borda, acessar mais configurações da tabela, e abrir o menu “Opções da tabela”. As opções estarão disponíveis sob o título “Cor”.

\subsubsection{Para acessar mais configurações da tabela:}
O menu oferece opções mais específicas de inserção e deleção de linhas e colunas. Assim como as funções também disponibilizadas nos menus de linha e coluna.
Clicar com o botão direito na tabela. Algumas operações são realizadas com relação à célula da tabela que foi clicada.

[IMAGEM 7]

\subsubsection{Descrição das outras opções do menu:}
Inserir linha de título: insere uma linha com uma única célula sobre a tabela;
Dividir célula: permite criar subdivisões de colunas e linhas em uma célula, o texto permanece na célula superior à esquerda.
Mesclar células: aparece na posição do botão anterior, quando mais de uma célula é selecionada.
Desfazer mesclagem de células: aparece no menu após uma mesclagem, reverte a operação;
Definir tipo de coluna: permite que o usuário defina um formato de informação, todos os valores naquela coluna devem ser preenchidos nesse formato.
Distribuir linhas e Distribuir colunas: Após ajustar o tamanho lateral e vertical total desejado, você pode utilizar essas opções para distribuir os espaços da tabela igualmente. É possível também selecionar um grupo de linhas ou colunas para distribuir o espaço.
Opções da tabela: Abre um menu lateral com mais opções, das quais a maioria está bem descrita dentro da própria ferramenta, descrições pertinentes a seguir;
Tabela > Estilo: a primeira opção mantém a tabela como único elemento em uma linha, o texto fica acima e abaixo da tabela. A segunda opção permite que o texto também apareça ao lado da tabela quando a tabela for menos larga que a página.

[IMAGEM 8]

Descrição imagens:
Ressaltar o menu Inserir > Imagem, eu pensei em um quadrado ao redor
 
Ressaltar as opções para imagem (segundo bloco)
Uma seta apontando para o menu lateral
Ressaltar o menu Inserir > Tabela
Duas setas apontando para os menus de linha e coluna
Ressaltar as opções para tabela (de inserir até opções de tabela)
Uma seta apontando para o menu lateral
