% sections/documentos/menu_inserir.tex
% !TeX root = ../../main.tex

\section{Menu Inserir}


\subsection{Imagens}
Para inserir imagens no \gls{arquivo}, tudo depende de onde que a imagem está, quando 
trabalhando com imagens que estão na internet ou no computador muitas vezes 
copiar e colar pode ser o suficiente. Porém, quando essa opção não funciona, 
podemos utilizar o menu “Inserir”.

\begin{figure}[H]
    \centering
    \includegraphics[width=0.6\textwidth]{/documentos/menu_inserir/pt1/Imagem1.png}
    \caption{}
\end{figure}


\subsubsection{Fazer \gls{upload} do computador}
\begin{enumerate}
    \item Selecionar a opção correspondente no menu “Inserir”;
    \item Localizar a imagem no seu computador;
    \item Clicar em “Abrir”;
\end{enumerate}


\subsubsection{Pesquisar na \gls{web}}
\begin{enumerate}
    \item Selecionar a opção correspondente no menu “Inserir”;
    \item Utilizar palavras-chave para buscar a imagem desejada;
    \item Clicar na imagem escolhida;
    \item Clicar no botão “Inserir” que vai aparecer no canto inferior direito;
\end{enumerate}


\subsubsection{Google Drive}
\begin{enumerate}
    \item Selecionar a opção correspondente no menu “Inserir”;
    \item Localizar a imagem no seu drive;
    \item Clicar no botão “Inserir” que vai aparecer no canto inferior direito;
\end{enumerate}


\subsubsection{Google Fotos}
\begin{enumerate}
    \item Selecionar a opção correspondente no menu “Inserir”;
    \item Localizar a imagem no seu Google Fotos;
    \item Clicar no botão “Inserir” que vai aparecer no canto inferior direito;
\end{enumerate}


\subsubsection{Câmera}
\begin{enumerate}
    \item Essa opção poderá ser utilizada quando o aparelho possuir uma câmera.
    \item Selecionar a opção correspondente no menu “Inserir”;
    \item Caso apareça a opção no seu navegador, permitir que o mesmo utilize a 
    câmera para tirar a foto dessa vez;
    \item Capture a foto e clique no botão “Inserir”;
\end{enumerate}


\subsubsection{Por \gls{url}}
\begin{enumerate}
    \item Selecionar a opção correspondente no menu “Inserir”;
    \item Colar o endereço da imagem;
    \item Clicar no botão “Inserir” que vai aparecer no canto inferior direito;
\end{enumerate}

\begin{dica}
Caso o endereço que você copiou esteja dando erro, verifique que você selecionou 
a opção “Copiar endereço da imagem” para copiar o endereço correto. 
\end{dica}


\subsubsection{Para ajustar o tamanho da imagem:}
\begin{enumerate}
    \item Selecionar a imagem;
    \item Pressionar os oito quadrados azuis nos vértices e arestas da imagem, 
    e arrastar;
\end{enumerate}

\begin{figure}[H]
    \centering
    \includegraphics[width=.7\textwidth]{/documentos/menu_inserir/pt1/Imagem2.png}
    \caption{}
\end{figure}


\subsubsection{Para rotacionar a imagem:}
\begin{enumerate}
    \item Selecionar a imagem;
    \item Pressionar e arrastar o círculo azul que aparece acima da imagem;
\end{enumerate}


\subsubsection{Para cortar a imagem:}
\begin{enumerate}
    \item Clicar com o botão direito na imagem;
    \item Selecionar a opção “Cortar imagem”;
    \item As bordas pretas que aparecem são as margens da imagem, é possível 
        ajustar tanto o tamanho da imagem, como o de suas bordas;
    \item Clique fora da imagem para completar o ajuste;
\end{enumerate}

\begin{figure}[H]
    \centering
    \includegraphics[width=.5\textwidth]{/documentos/menu_inserir/pt1/Imagem3.png}
    \caption{}
\end{figure}


\subsubsection{Para remover as alterações de corte na imagem:}
\begin{enumerate}
    \item Clicar com o botão direito na imagem;
    \item Selecionar a opção “Redefinir imagem”;
\end{enumerate}


\subsubsection{Para adicionar texto alternativo:}
\begin{enumerate}
    \item Clicar com o botão direito na imagem;
    \item Selecionar a opção “Texto alternativo”;
    \item Escrever o texto na \gls{aba} lateral direita que foi aberta;
\end{enumerate}


\subsubsection{Para acessar o mais opções de imagem:}
\begin{enumerate}
    \item Clicar com o botão direito na imagem;
    \item Selecionar a opção “Opções de imagem”;
    \item Abre um menu lateral com mais opções, possui descrição na própria 
        ferramenta;
\end{enumerate}

\begin{figure}[H]
    \centering
    \includegraphics[width=.8\textwidth]{/documentos/menu_inserir/pt1/Imagem4.png}
    \caption{}
\end{figure}

\begin{dica}
O texto alternativo serve para acessibilidade, se o seu documento será consumido 
por uma pessoa com deficiência visual que utiliza descrição de áudio, é 
interessante adicionar uma descrição da imagem para ajudar com contextualização.
\end{dica}


\subsection{Tabelas}
As tabelas podem ser adicionadas no documento para ajudar a organizar a informação. 

\subsubsection{Para realizar a inserção de uma tabela no documento:}
Acessar o menu Inserir > Tabela;
Selecionar o tamanho da tabela desejado no esquema de grade;

\begin{dica}
    Os números abaixo de esquema de grade se refere ao número de [colunas] x [linhas]
\end{dica}

\begin{figure}[H]
    \centering
    \includegraphics[width=.8\textwidth]{/documentos/menu_inserir/pt1/Imagem5.png}
    \caption{}
\end{figure}

\begin{dica}
    Sob o menu “Elemento Básicos”, a ferramenta disponibiliza também alguns modelos prontos. O usuário é incentivado a investigar esses modelos e utilizá-los em parte ou integralmente.
\end{dica}

\subsubsection{Para acessar os menus de coluna e linha:}
Ao passar o mouse pelas laterais esquerda e superior da tabela, os menus irão aparecer;

\begin{figure}[H]
    \centering
    \includegraphics[width=.8\textwidth]{/documentos/menu_inserir/pt1/Imagem6.png}
    \caption{}
\end{figure}

\subsubsection{Para inserir uma coluna ou linha:}
É possível adicionar uma coluna ou linha clicando no símbolo de mais [+] nos menus de coluna e linha. A linha será adicionada abaixo da linha que tem o menu aberto. A coluna será adicionada à direita da que tem o menu aberto.

\subsubsection{Para trocar a ordem das colunas e linhas:}
É possível trocar a posição das colunas e linhas ao clicar e segurar o símbolo de seis pontos cinza claro que aparece nos menus de coluna e linha.

\subsubsection{Para fixar e desafixar o cabeçalho da tabela:}
Ao clicar no botão com ícone de tachinha no menu da linha, é possível fixar e desafixar o cabeçalho. É possível fixar mais de uma linha.

\begin{dica}
    Fixar o cabeçalho, significa que, em tabelas que ocupam mais de uma página, as linhas fixadas vão aparecer no topo de todas as páginas.
\end{dica}

\begin{dica}
    O cabeçalho vai estar fixado quando o ícone de tachinha com a barra transversal estiver aparecendo. De forma a simbolizar que a operação de desafixação vai ser realizada ao clicar novamente.
\end{dica}

\subsubsection{Para ordenar a tabela:}
No menu da coluna que se deseja ordenar, clicar no botão com ícone de três barras paralelas;
Selecionar se a ordem deve ser crescente ou decrescente.

\begin{dica}
    Ordenar uma coluna mantém os valores da linha agrupados. Não edita a tabela, apenas organiza as informações.
\end{dica}

\subsubsection{Para alterar as bordas da tabela:}
Ao passar o mouse por uma célula, aparece um botão com ícone de seta no canto superior direito. 
Ao clicar no botão é possível alterar a borda de uma célula ou grupo de células;

\begin{dica}
    Para ajustes de cor e tamanho da borda, acessar mais configurações da tabela, e abrir o menu “Opções da tabela”. As opções estarão disponíveis sob o título “Cor”.
\end{dica}

\subsubsection{Para acessar mais configurações da tabela:}
O menu oferece opções mais específicas de inserção e deleção de linhas e colunas. Assim como as funções também disponibilizadas nos menus de linha e coluna.
Clicar com o botão direito na tabela. Algumas operações são realizadas com relação à célula da tabela que foi clicada.

\begin{figure}[H]
    \centering
    \includegraphics[width=.8\textwidth]{/documentos/menu_inserir/pt1/Imagem7.png}
    \caption{}
\end{figure}

\subsubsection{Descrição das outras opções do menu:}
\begin{itemize}
    \item \textbf{Inserir linha de título:} insere uma linha com uma única célula sobre a tabela;
    \item \textbf{Dividir célula:} permite criar subdivisões de colunas e linhas em uma célula, o texto permanece na célula superior à esquerda.
    \item \textbf{Mesclar células:} aparece na posição do botão anterior, quando mais de uma célula é selecionada.
    \item \textbf{Desfazer mesclagem de células:} aparece no menu após uma mesclagem, reverte a operação;
    \item \textbf{Definir tipo de coluna:} permite que o usuário defina um formato de informação, todos os valores naquela coluna devem ser preenchidos nesse formato.
    \item \textbf{Distribuir linhas e Distribuir colunas:} Após ajustar o tamanho lateral e vertical total desejado, você pode utilizar essas opções para distribuir os espaços da tabela igualmente. É possível também selecionar um grupo de linhas ou colunas para distribuir o espaço.
    \item \textbf{Opções da tabela:} Abre um menu lateral com mais opções, das quais a maioria está bem descrita dentro da própria ferramenta, descrições pertinentes a seguir;
    \item \textbf{Tabela > Estilo:} a primeira opção mantém a tabela como único elemento em uma linha, o texto fica acima e abaixo da tabela. A segunda opção permite que o texto também apareça ao lado da tabela quando a tabela for menos larga que a página.
\end{itemize}

\begin{figure}[H]
    \centering
    \includegraphics[width=.8\textwidth]{/documentos/menu_inserir/pt1/Imagem8.png}
    \caption{}
\end{figure}


\section{Gráficos}
Esta seção irá apresentar as formas de criar gráficos no Google Docs, as opções de ajustes visuais disponíveis e algumas 
informações sobre compatibilidade e uso. 

\subsection{Inserção de Gráficos}

\subsubsection{Inserção Diretamente no Google Docs}
Neste método de inserção, os gráficos são inseridos diretamente no arquivo, permitindo que o usuário visualize a estrutura do 
gráfico antes de trabalhar com seus dados. 

\begin{itemize}
    \item Na aba do menu Inserir, selecione a opção “Gráfico”. Depois clique em um dos tipos de gráfico disponíveis (Colunas, Barras, Pizza ou Linhas).
    \item O gráfico será inserido no documento, contendo dados de exemplo. Assim que isso acontecer, um arquivo do Google Planilhas é automaticamente criado e vinculado a ele. Todo gerenciamento de dados do gráfico será feito através desse arquivo do Planilhas.
    \item Para acessar esse documento, selecione o gráfico recém criado. Clique nos três pontos da aba “Gráfico vinculado” e clique na opção “Abrir documento original”. O arquivo de planilhas, com o seu gráfico vinculado, será aberto em outra aba do navegador.
\end{itemize}

\begin{figure}[H]
    \centering
    \includegraphics[width=.8\textwidth]{/documentos/menu_inserir/pt2/Imagem1.png}
    \caption{}
\end{figure}

\textbf{Observação:} 

Os únicos modelos de gráfico disponíveis através deste tipo de inserção são os quatro exibidos no menu Inserir. Para ter acesso a todos os modelos de gráfico disponíveis, siga as instruções do método de Inserção de Gráficos do Google Planilhas.

\subsubsection{Inserção através do Google Planilhas}
Neste método de inserção, é possível adicionar qualquer um dos modelos de gráfico do Google Planilhas. Para isso, é necessário ter salvo no seu Drive um arquivo (do Planilhas) que tenha um gráfico. Caso precise de instruções sobre como criar um arquivo com gráficos no Google Planilhas, veja a seção correspondente desta apostila.

\begin{itemize}
    \item Na aba do menu Inserir, selecione a opção “Gráfico”. Depois clique na opção “Do planilhas”.
    \item Uma tela de inserção de gráfico irá abrir. Escolha uma das abas que correspondem ao local em que o arquivo de planilhas está: Recente (refere-se a arquivos recém utilizados), Meu Drive (refere-se à arquivos do seu ambiente pessoal de armazenamento) e Compartilhado comigo (refere-se a arquivos de outros usuários que você possui acesso). Após localizar o arquivo desejado, selecione-o e clique no botão Inserir. 
    \item Na aba que se abriu, selecione o gráfico escolhido dentre os exibidos e clique em importar. O gráfico selecionado será inserido no seu arquivo Docs, com as mesmas configurações prévias do Google Planilhas.
\end{itemize}

\begin{figure}[H]
    \centering
    \includegraphics[width=.5\textwidth]{/documentos/menu_inserir/pt2/Imagem2.png}
    \caption{}
\end{figure}

\textbf{Observação:}

Qualquer alteração nos valores, fórmulas ou títulos do gráfico deve ser feita no Google Planilhas, pois não é possível editar diretamente os dados dentro do Docs.

Se o arquivo do Planilhas for movido, excluído ou se o acesso a ele for perdido, o gráfico no Docs permanecerá visível, mas ficará estático e não poderá mais ser atualizado automaticamente.

\subsubsection{Inserção através de outros arquivos não Google}
Este método permite inserir gráficos de arquivos que não fazem parte das ferramentas Google em um documento do Google Docs. 

\begin{itemize}
    \item Abra o arquivo de edição que contém o gráfico (Excel ou LibreOffice, por exemplo). Selecione o gráfico e o copie para a área de transferência. 
    \item Abra o documento do Google Docs onde deseja inserir o gráfico e cole o conteúdo da área de transferência.
\end{itemize}

\textbf{Observação:} 

No Docs, gráficos copiados de arquivos que não fazem parte do Google Workspace funcionam apenas como tabelas ou imagens estáticas. Ou seja, não é possível editar os dados ou atualizar o gráfico automaticamente. Nesse caso, qualquer alteração deve ser feita no arquivo original e o gráfico copiado novamente.


\subsection{Atualização de Dados}
É importante destacar que a manipulação de dados e variantes de um gráfico é feita diretamente no Google Planilhas, que é a ferramenta do Google Workspace certa para isso. Para instruções detalhadas sobre criação e manipulação do material informativo de gráficos, consulte a seção da apostila correspondente às planilhas do Google. Portanto, este tópico aborda apenas a atualização do gráfico, referente ao arquivo do Planilhas. 

\begin{itemize}
    \item Atualize os dados do seu gráfico no arquivo do Google Planilhas da maneira que quiser. Mudar a informação da célula e apertar a tecla “Enter” será o suficiente, visto que a ferramenta salva arquivos de forma automática.
    \item No arquivo Docs, selecione o gráfico já inserido no documento, e clique no botão “Atualizar”, como mostra a imagem.
\end{itemize}

\begin{figure}[H]
    \centering
    \includegraphics[width=.8\textwidth]{/documentos/menu_inserir/pt2/Imagem3.png}
    \caption{}
\end{figure}


\subsection{Ajustes visuais}

Apenas o Google Planilhas permite a manipulação completa das configurações de gráficos. No entanto, no Docs ainda é possível realizar alguns ajustes, como posicionamento, tamanho, alinhamento, quebra de texto e algumas opções básicas de cor ou borda.

\subsubsection{Posicionamento do gráfico}
No Google Docs, é possível arrastar o gráfico livremente ou ajustá-lo para obter o alinhamento mais adequado.

\begin{enumerate}
    \item Clique sobre o gráfico para selecioná-lo.
    \item Segure o botão esquerdo do mouse e arraste o gráfico até o local desejado na página.
    \item Solte o mouse para fixar o novo posicionamento.
\end{enumerate}

\subsection{Redimensionamento do gráfico}
	O redimensionamento permite aumentar ou diminuir a largura e a altura do gráfico, sem alterar os dados exibidos.

\begin{enumerate}
    \item Clique no gráfico para selecioná-lo.
    \item Observe que pequenos quadradinhos (alças de redimensionamento) aparecerão nas bordas do gráfico.
    \item Clique e arraste uma das alças:
    \item Solte o mouse quando atingir o tamanho desejado.
\end{enumerate}


\subsection{Quebra de texto do gráfico}
A quebra de texto define como o conteúdo textual do documento interage com o gráfico. 

\begin{enumerate}
    \item Clique no gráfico para selecioná-lo.
    \item Na barra inferior do gráfico, estão exibidas as opções de “quebra de texto”. Clique na opção desejada para mudar a interação do gráfico com o texto.
\end{enumerate}

\paragraph{Opções de quebra de texto:}
\begin{itemize}
    \item Em linha com o texto: o gráfico é tratado como se fosse uma palavra dentro da linha.
    \item Ajustar texto: o texto contorna o gráfico, ocupando os espaços livres ao redor dele, mas não força o gráfico a ficar em linha com o texto.
    \item Quebrar texto: o texto se ajusta ao redor do gráfico, ocupando os espaços livres.
    \item Atrás do texto: o gráfico fica posicionado no fundo, com o texto sobreposto.
    \item Na frente do texto: o gráfico aparece sobre o texto, ocultando-o parcialmente.
\end{itemize}

\begin{figure}[H]
    \centering
    \includegraphics[width=.8\textwidth]{/documentos/menu_inserir/pt2/Imagem4.png}
    \caption{}
\end{figure}

\subsubsection{Cores do gráfico}
No Google Docs, não é possível definir as cores de cada um dos elementos de um gráfico. Entretanto, é possível alterar a paleta de cores de um gráfico a partir de uma lista predefinida.

\begin{enumerate}
    \item Selecione o gráfico. Clique com o botão direito sobre ele e selecione “Opções de Imagem”.
    \item Um painel lateral abrirá. Selecione a opção “Novas Cores” e então clique na caixa  intitulada “Nenhuma cor nova”. Através dessa caixa é possível escolher uma entre várias opções de cores para o seu gráfico.
\end{enumerate}

\begin{figure}[H]
    \centering
    \includegraphics[width=.8\textwidth]{/documentos/menu_inserir/pt2/Imagem5.png}
    \caption{}
\end{figure}

\textbf{Observação:} 

Para ter maior controle sobre as cores do diagrama, recomenda-se fazê-lo nas configurações do arquivo do Google Planilhas, que possui as ferramentas necessárias para manipular a cor de todos os elementos do gráfico de forma individual.

\section{Símbolos}
Esta seção apresentará as formas de inserir caracteres especiais e emojis no Google Docs, destacando suas limitações e algumas opções de formatação.

\subsection{Inserção de Emojis}

\begin{enumerate}
    \item Na aba do menu Inserir, siga esta ordem de seleção: “Símbolos” → “Emoji”.
    \item Uma aba será aberta abaixo do cursor de texto, disponibilizando uma biblioteca com diversas opções de emojis.
\end{enumerate}


 
Os emojis inseridos são tratados como caracteres de texto, portanto podem ser redimensionados.

\subsection{Inserção de Caracteres Especiais}

\begin{enumerate}
    \item Na aba do menu Inserir, siga esta ordem de seleção: “Símbolos” → “Caracteres Especiais”.
    \item Uma tela será aberta, dispondo uma vasta biblioteca de caracteres especiais. Para facilitar a busca, é possível digitar o nome do caractere na caixa de pesquisa ou desenhar o símbolo desejado na área de desenho.
    \item Clique no caractere de escolha que ele será inserido no documento, no lugar em que o cursor de texto estiver.
\end{enumerate}

\textbf{Observação:}

Os caracteres especiais são inseridos como texto, podendo ser redimensionados, coloridos e estilizados com as opções de fonte.