% sections/documentos/trabalhando_com_estilos.tex
% !TeX root = ../../main.tex

\section{Trabalhando com Estilos}
Um estilo salva as definições de espaçamento, alinhamento, recuo, fonte da letra, tamanho e cor, bem como configurações em negrito, 
itálico e sublinhado. A seleção de estilos e atribuições relacionadas estão realçadas na figura a seguir. 

Um arquivo recém-criado já acompanha alguns estilos-base que podem ser aplicados pelo usuário para dar estrutura ao material. 
A aparência que o estilo possui corresponde à aparência visível no nome do estilo, no menu de estilos.

\begin{figure}[H]
    \centering
    \includegraphics[width=.9\textwidth]{/documentos/trabalhando_com_estilos/pt1/Imagem1.png}
    \caption{Menu das ferramentas de estilo do Docs, estão destacadas as mais comuns.}
\end{figure}

A utilização dos estilos permite a alteração da aparência de todos os blocos de texto que possuem o estilo correspondente atribuído, auxiliando na organização do arquivo e permitindo que qualquer adição de texto seja rapidamente formatada seguindo o padrão adotado.

\begin{atencao}
    Ao utilizar a ferramenta “Limpar formatação”, são removidas apenas as alterações que diferem do estilo definido
\end{atencao}

\begin{figure}[H]
    \centering
    \includegraphics[width=.9\textwidth]{/documentos/trabalhando_com_estilos/pt1/Imagem2.png}
    \caption{Ícone da ferramenta Limpar Formatação}
\end{figure}


\subsection{Como aplicar um estilo}
Para aplicar um estilo, é necessário apenas selecionar o texto que se deseja formatar e selecionar a opção de estilo na barra de ferramentas. 

\begin{atencao}
    Cada parágrafo de texto pode ter apenas um estilo. Se o estilo aplicado estiver afetando mais texto do que o desejado, certifique-se que o texto está separado por uma quebra de linha (ENTER).
\end{atencao}

\subsection{Como definir os estilos}
Para alterar um dos estilos definidos, primeiro faça as alterações em uma parte do texto, até estar satisfeito com a aparência. Em seguida, com o texto selecionado, clique no menu de estilos, coloque o mouse sobre o estilo que será substituído, e clique na opção “Usar essa formatação para (nome do estilo)”.

\begin{dica}
    Utilizar a opção de Desfazer \tecla{Ctrl} + \tecla{Z} também desfaz as definições de estilo.
\end{dica}

\subsection{Como retornar aos estilos base}
No menu de estilos, coloque o mouse sobre \textbf{Opções}, e selecione a opção \textbf{Redefinir estilos}.

\subsection{Estilos padrão}
Para utilizar os mesmos estilos em mais de um arquivo diferente, é possível definir estilos padrão. É possível ter um único conjunto de estilos padrão relacionados ao seu perfil do Google. As opções referentes aos estilos padrão podem ser encontradas no menu de estilos, na seção “Opções”.

\begin{dica}
    Para salvar um novo estilo sem perder os estilos salvos, primeiro selecione \textbf{Usar meus estilos-padrão}, faça a alteração e depois selecione \textbf{Salvar como meus estilos padrão}. 
\end{dica}

%---- PARTE II ----%

\subsection{Divisão do Texto em Colunas}
É possível estruturar o texto em duas colunas ou mais, além de configurar algumas opções de formatação.

\begin{enumerate}
    \item Primeiramente, para dividir apenas parte do texto do arquivo em colunas, selecione o trecho desejado com o cursor do mouse. 
    \item Na aba do menu Formatar, selecione a opção “Colunas”. Então selecione uma das opções apresentadas: 1, 2 ou 3 colunas.
    \item Para configurar o espaçamento entre as colunas, clique em “Formatar” → “Colunas” → “Mais opções”. Na tela que se abrir, defina a distância entre as colunas, na aba de Espaçamento.
\end{enumerate}

\begin{dica}
    Para dividir todo conteúdo do documento em colunas, não selecione nada.
\end{dica}

Na tela de “Mais opções” também é possível marcar a opção de adicionar uma linha vertical entre as colunas.


\subsection{Texto sem Divisão de Páginas}

O recurso de Texto sem Divisão de Páginas permite que o documento seja estruturado como um fluxo contínuo, sem quebras de página automáticas. Essa configuração é útil para arquivos que serão visualizados digitalmente ou para materiais que não precisam de separação física entre páginas.

\begin{enumerate}
    \item Na aba do menu Formatar, selecione a opção “Mudar para o formato sem páginas”. Ao habilitar esta opção, o conteúdo do documento será exibido em um fluxo contínuo, permitindo rolagem sem interrupções de página.
    \item Para desativar, basta voltar ao mesmo menu e marcar a opção “Mudar para o formato de Páginas”.
\end{enumerate}

\begin{atencao}
    Essa configuração altera apenas a visualização do documento; margens, cabeçalhos e rodapés permanecem configuráveis normalmente.
\end{atencao}

\subsection{Numerar Páginas}

A ferramenta de Numeração de Páginas permite inserir números em cada página do documento, facilitando a organização e referência, principalmente em trabalhos e relatórios acadêmicos.

\begin{enumerate}
    \item Para numerar as páginas do seu documento, abra o menu Formatar e selecione a opção “Números de página”. A tela de configuração de numeração de páginas será aberta, como ilustra a Figura 1.31.
    
%    \begin{figure}[H]
%        \centering
%        \includegraphics[width=.9\textwidth]{/documentos/trabalhando_com_estilos/pt2/Imagem1.png}
%        \caption{Tela de configuração da numeração de páginas.}
%    \end{figure}
    
    \item Na seção “Posições”, escolher a opção “Cabeçalho” ou "Rodapé" determina onde os números aparecerão na página. Caso a opção “Mostrar na primeira página” não esteja selecionada, o número da primeira página é omitido.
    \item Na seção “Numeração”, a opção “Iniciar em”, define a partir de qual número começará a numeração das páginas. Caso queira que as páginas mantenham a numeração normalmente, desconsidere essa opção.
    \item Após definir as preferências de implementação, basta clicar no botão “Aplicar”, na parte inferior da aba, que as páginas serão preenchidas por números.
\end{enumerate}

\begin{figure}[H]
        \centering
        \includegraphics[width=.5\textwidth]{/documentos/trabalhando_com_estilos/pt2/Imagem1.png}
        \caption{}
    \end{figure}


%---- PARTE III ----%

\subsection{Função colar formatação}
Utilizado para copiar o estilo visual de um trecho do texto para aplicá-lo em outro. Replica fonte, cor, tamanho, formatação em negrito, itálico, sublinhado ou tachado do texto original.

Para utilizar a função de colar formatação, você deve selecionar o texto que possui a formatação que deseja copiar e, em seguida, na barra de ferramentas da parte superior da tela, clicar no ícone “Pintar formatação” (ícone de rolo de pintura).

\begin{figure}[H]
    \centering
    \includegraphics[width=1\textwidth]{/documentos/trabalhando_com_estilos/pt3/Imagem1.png}
    \caption{Utilização da ferramenta pintar formatação para copiar as características do texto exemplo}
\end{figure}

Em seguida, você deve selecionar o trecho que deseja aplicar a formatação pressionando o botão esquerdo do mouse e soltando ao final da seleção. Então, a formatação será automaticamente aplicada para o trecho selecionado.

\begin{figure}[H]
    \centering
    \includegraphics[width=.8\textwidth]{/documentos/trabalhando_com_estilos/pt3/Imagem2.png}
    \caption{Texto com nova formatação}
\end{figure}


\subsection{Função limpar formatação}
Remover os estilos aplicados no trecho. Redefine cor, tamanho e formatações em negrito e/ou itálico, por exemplo. Retorna à estilização padrão do documento.

Para utilizar a função remover formatação, você deve selecionar o trecho em que deseja remover o estilo de texto e, em seguida, na parte superior da tela, acessar o campo “Formatar” e selecionar a opção “Limpar formatação”.

\begin{figure}[H]
    \centering
    \includegraphics[width=.7\textwidth]{/documentos/trabalhando_com_estilos/pt3/Imagem3.png}
    \caption{Opção de limpar formatação no menu Formatar.}
\end{figure}


\section{Alinhamento e Recuo}
\subsection{Alinhamento}

Define como o texto será ajustado dentro da margem do documento, as opções que podem ser utilizadas são:

\begin{itemize}
    \item \textbf{Esquerda:} O texto será alinhado na margem esquerda da página do documento.
    \item \textbf{Centralizado:} O texto será alinhado no meio da página.
    \item \textbf{Direita:} O texto será alinhado na margem direita da página.
    \item \textbf{Justificado:} O texto será distribuído uniformemente nas margens esquerda e direita da página, ajustando adequadamente o espaço entre as palavras.
\end{itemize}

Para aplicar o alinhamento, você deve acessar o ícone “Alinhar” na barra de ferramentas localizada na parte superior da tela e escolher uma das opções de alinhamento.

\begin{figure}[H]
    \centering
    \includegraphics[width=.9\textwidth]{/documentos/trabalhando_com_estilos/pt3/Imagem4.png}
    \caption{Menu de alinhamento e recuo do texto.}
\end{figure}


\subsection{Recuo}
Define a margem do parágrafo, ou seja, o espaço em que o parágrafo irá recuar em relação à margem principal do documento. As ferramentas de recuo que podem ser utilizadas são:

\begin{itemize}
    \item \textbf{Diminuir recuo:} Todo o parágrafo será deslocado “para fora” em relação à margem do texto (da esquerda para a direita).
    \item \textbf{Aumentar recuo:} Todo o parágrafo será deslocado “para dentro” em relação à margem do texto (da direita para a esquerda).
    \item \textbf{Recuo especial:} Permite definir um espaçamento somente para a primeira linha do parágrafo. Muito útil para a escrita de algumas modalidades de artigos acadêmicos, por exemplo.
\end{itemize}

Assim como as demais ferramentas citadas, para aumentar ou diminuir o recuo do parágrafo, é possível utilizar os ícones correspondentes na barra de ferramentas encontrada na parte superior da tela.

\begin{figure}[H]
    \centering
    \includegraphics[width=.9\textwidth]{/documentos/trabalhando_com_estilos/pt3/Imagem5.png}
    \caption{Barra de recuo, que dimensiona os parágrafos da página.}
\end{figure}

Para utilizar a parte de recuo especial ou para ter acesso a todas as formatações de alinhamento e recuo, você deve acessar, também na parte superior da tela, o campo “Formatar” e, em seguida, “Alinhar e recuar”. Para ter acesso ao campo de recuo especial, no mesmo submenu, clique sobre a opção “Opções de recuo”.

\begin{figure}[H]
    \centering
    \includegraphics[width=.9\textwidth]{/documentos/trabalhando_com_estilos/pt3/Imagem6.png}
    \caption{Opções de alinhamento e recuo, no menu Formatar.}
\end{figure}

Essa configuração permite ao usuário alterar a quantidade de centímetros de recuo de parágrafos à esquerda e à direita da página, além da possibilidade de habilitar o recuo especial, que conta com as opções “Primeira linha” e “Deslocamento”.

\begin{figure}[H]
    \centering
    \includegraphics[width=.7\textwidth]{/documentos/trabalhando_com_estilos/pt3/Imagem7.png}
    \caption{Tela de opções adicionais de recuo.}
\end{figure}

\subsection{Espaçamento entre linhas}
Para alterar o espaçamento entre linhas e parágrafos você deve acessar, na barra de ferramentas na parte superior da tela, o ícone de “Espaçamento entre linhas e parágrafos”.

Esse recurso possibilita configurar a distância vertical entre as linhas dentro de um parágrafo ou entre os parágrafos. Possui algumas opções de espaçamento pré-definidas: “Simples”, “1,15”, “1,5” e “Duplo”. Também é possível adicionar um espaço antes ou depois de um parágrafo. Ainda, é possível utilizar o “espaçamento personalizado”, definindo a medida dos espaços de acordo com a sua preferência.

Além dessas configurações, também estão disponíveis as seguintes opções:

\begin{itemize}
    \item \textbf{Manter com o próximo:} Com essa opção ativada, o Docs impede que o parágrafo selecionado separe-se do parágrafo seguinte por uma quebra de página;
    \item \textbf{Manter as linhas juntas:} Impede que linhas dentro de um mesmo parágrafo se separem entre páginas;
    \item \textbf{Evitar linhas únicas:} Essa opção evita que a última linha de um parágrafo esteja “sozinha” no topo da próxima página ou que a primeira linha de um parágrafo esteja “sozinha” no final de uma página anterior;
    \item \textbf{Adicionar quebra de página antes do texto:} Essa opção insere automaticamente uma quebra de linha antes do parágrafo selecionado.
\end{itemize}

\begin{figure}[H]
    \centering
    \includegraphics[width=1\textwidth]{/documentos/trabalhando_com_estilos/pt3/Imagem8.png}
    \caption{Menu de configuração do espaçamento entre linhas.}
\end{figure}