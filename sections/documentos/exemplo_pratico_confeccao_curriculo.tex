% sections/documentos/exemplo_pratico_confeccao_curriculo.tex
% !TeX root = ../../main.tex

\section{Exemplo Prático: Confecção de Currículo}

Como forma de exercitarmos o aprendizado adquirido sobre Google Documentos, iremos realizar a confecção de um Currículo. 
Se preferir, é possível utilizar os modelos já disponibilizados pelo Google Documentos e visualizar as opções através da galeria de \textbf{Modelos}. 
Aqui, na apostila, iremos construir o exemplo prático todo manualmente.

Toda a estrutura será construída utilizando os estilos de título disponíveis. As seções serão organizadas assim:

\begin{itemize}
    \item Dados pessoais
    \item Competências
    \item Experiência profissional
    \item Educação
    \item Idiomas/Premiações/Outras informações 
\end{itemize}

Iniciando pela seção de Dados Pessoais, o seu nome deve estar com a estilização Título 1 e alinhado à esquerda. 
Essa configuração já irá garantir grande destaque para os dados pessoais. Sua profissão pode estar na estilização \textbf{Subtítulo}, 
enquanto o resto dos dados pessoais (endereço, data de nascimento, email, LinkedIn, telefone) deve estar estilizada como \textbf{Texto normal}.

\begin{figure}[H]
    \centering
    \includegraphics[width=1\textwidth]{images/documentos/exemplo_pratico_confeccao_curriculo/Imagem33.png}
    \caption{Seção de Dados Pessoais do currículo}
\end{figure}

Tratando das outras seções, todos os títulos devem estar na estilização Título 2, podendo estar com todas as letras maiúsculas ou apenas a primeira.
Quando houver uma listagem de itens internamente, no conteúdo das seções aos títulos de itens deve ser atribuída a estilização Título 3. 
Dessa forma, uma ordem hierárquica é construída visualmente. O texto restante pode estar com estilo \textbf{Texto normal}. 
Os recursos de itálico, negrito e lista com marcadores podem ser utilizados livremente para manter a organização do conteúdo das seções.

\begin{figure}[H]
    \centering
    \includegraphics[width=1\textwidth]{images/documentos/exemplo_pratico_confeccao_curriculo/Imagem34.png}
    \caption{Exemplo de hierarquia de títulos}
\end{figure}

Na última seção, fique livre para abordar um assunto que apresente algum ponto forte seu que seja interessante para a vaga desejada. 
Recomenda-se enumerar prêmios recebidos pelos trabalhos, proficiência em outras línguas, objetivos e interesses. 
No nosso exemplo iremos apresentar conhecimentos de idioma.

Agora que a estrutura do currículo já está concluída, você pode adicionar leves elementos de personalidade. 
Evite cores muito extravagantes ou fontes de difícil legibilidade, mas adicione itens que encaixem com leveza no que construímos. 
No exemplo, optamos por mudar as cores dos títulos das seções para um azul escuro, e alteramos a cor acinzentada dos subtítulos para preto:

\begin{figure}[H]
    \centering
    \includegraphics[width=.8\textwidth]{images/documentos/exemplo_pratico_confeccao_curriculo/Imagem35.png}
    \caption{Visualização do documento depois da estilização, começo}
\end{figure}

\begin{figure}[H]
    \centering
    \includegraphics[width=.8\textwidth]{images/documentos/exemplo_pratico_confeccao_curriculo/Imagem36.png}
    \caption{Visualização do documento depois da estilização, final}
\end{figure}