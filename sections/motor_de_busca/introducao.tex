% !TeX root = ../main.tex
% sections/motor_de_busca/motor_de_busca.tex

\section{Introdução ao Motor de Busca}
O motor de busca do Google, conhecido mundialmente como \textbf{Google Search}, é a ferramenta central acessada ao visitar o site \href{https://google.com}{google.com}. Mais do que um simples buscador, ele se tornou sinônimo de pesquisa na internet, sendo um dos produtos digitais mais influentes e utilizados da história.

A história dessa ferramenta começa em 1995, quando Larry Page e Sergey Brin, ainda estudantes da Universidade de Stanford, criaram um projeto chamado \textbf{Backrub}. A ideia inicial era analisar a relevância de cada página com base nos links que recebia, criando assim uma forma de classificação que se tornaria o algoritmo conhecido como PageRank (\cite{pagePageRankCitationRanking1999}). Esse mecanismo revolucionou a forma de navegar na internet, tornando os resultados muito mais úteis e organizados. Pouco tempo depois, o projeto recebeu o nome de \textbf{Google}, inspirado no termo matemático “\textit{googol}” (o número 1 seguido de 100 zeros), representando a missão ambiciosa de seus criadores: “\textit{organizar a informação do mundo e torná-la universalmente acessível e útil}” (\cite{googleHowWeStarted}).

Desde então, o Google Search não apenas manteve sua relevância, mas evoluiu continuamente. Novos recursos foram incorporados ao longo dos anos, como a pesquisa por imagens, notícias, vídeos, mapas, compras e até comandos por voz. Hoje, a ferramenta consegue interpretar intenções de busca, corrigir erros ortográficos automaticamente, sugerir alternativas e até responder diretamente a perguntas simples por meio de caixas de destaque conhecidas como \textit{featured snippets} (\cite{googleHowWeStarted}).

Mesmo após quase três décadas de sua criação, o Google Search permanece indispensável para usuários de todas as idades e perfis. Segundo dados fornecidos pela empresa e apresentados por \citeauthor{srinivasanAIPersonalizationFuture2025} (\citeyear{srinivasanAIPersonalizationFuture2025}) em uma postagem no blog para anunciantes, são realizadas mais de 5 trilhões de pesquisas por ano em todo o mundo, abrangendo desde dúvidas cotidianas até pesquisas acadêmicas e profissionais. Esse alcance gigantesco mostra como a ferramenta se consolidou como um elemento essencial da vida digital.

Por isso, compreender como utilizá-la de forma estratégica é cada vez mais importante. Saber aplicar filtros, operadores de busca e recursos avançados pode facilitar o acesso a informações específicas, economizar tempo e melhorar a precisão dos resultados. Em um cenário onde a quantidade de dados disponíveis cresce, dominar o uso do Google Search é, sem dúvida, uma habilidade fundamental para navegar com eficiência no mundo conectado em que vivemos.
