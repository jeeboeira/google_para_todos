% sections/motor_de_busca/motor_de_busca.tex
% !TeX root = ../main.tex

\section{Operadores}

A ferramenta permite a pesquisa digitando qualquer termo na barra de pesquisa, porém, para melhor utilizar o poder dela, é necessário aprender sobre os operadores de pesquisa. Eles são termos específicos que, ao serem digitados, permitem alterar o comportamento da ferramenta como um todo, podendo aumentar a precisão dos resultados.

O operador mais simples que se pode usar é as aspas. Ao colocar um termo entre-as é possível filtrar todos os sites cujo a palavra aparece exatamente como escrita, evitando o uso de sinônimos ou palavras similares.

Perceba na figura \ref{fig:operadores1}, como os resultados apresentam variações da palavra “desenhada”, porém, não apresentam ela nos resultados. Já na \ref{fig:operadores2} é possível perceber que a palavra está presente nas páginas dos resultados.

\begin{figure}[h]
	\centering
	\includegraphics[width=.9\textwidth]{images/motor_de_busca/operadores1.png}
	\caption{Resultado da pesquisa Maçã Desenhada.}
	\label{fig:operadores1}
\end{figure}

%\vspace*{\fill}
%\newpage

\begin{figure}[h]
	\centering
	\includegraphics[width=.9\textwidth]{images/motor_de_busca/operadores2.png}
	\caption{Resultado da pesquisa Maçã “Desenhada”}
	\label{fig:operadores2}
\end{figure}


Combinado com as aspas, existem outros dois operadores importantes para filtrar ainda mais os resultados, são eles: \textbf{AND} e \textbf{OR}. Estas são palavras no inglês que significam E e OU respectivamente. Assim podemos criar uma pesquisa onde os resultados serão de maçãs ou bananas e estas especificamente serão desenhadas.

\begin{figure}[h]
	\centering
	\includegraphics[width=.9\textwidth]{images/motor_de_busca/operadores3.png}
	\caption{Resultado da pesquisa (Maçã OR Banana) AND “Desenhada”}
	\label{fig:operadores3}
\end{figure}


Olhares atentos a figura \ref{fig:operadores3} podem perceber a presença de outro elemento na pesquisa, os parênteses. Com eles é possível agrupar termos e pesquisas para facilitar a leitura e filtrar os resultados ainda mais. Sem a presença deles a pesquisa seria a seguinte: maçã OR banana AND “desenhada”, nela o motor iria trazer dados sobre maçãs ou bananas desenhadas, porém, com os parêntesis traremos resultados sobre maçãs ou bananas desenhadas.

Agora imaginemos que não queremos trazer os dados sobre maçãs ou bananas verdes, pois precisamos filtrar apenas por imagens delas maduras. Assim surge o operador -, com ele é possível remover um termo da pesquisa, como demonstrado na figura \ref{fig:operadores4}.

\begin{figure}[h]
	\centering
	\includegraphics[width=.9\textwidth]{images/motor_de_busca/operadores4.png}
	\caption{Resultado da pesquisa (Maçã OR Banana) AND “Desenhada” -verde}
	\label{fig:operadores4}
\end{figure}

Estes são os operadores mais simples que o motor de pesquisa do Google oferece, porém não são todos, existem ainda os operadores complexos que irão filtrar ainda mais os resultados. São eles palavras específicas que serão utilizadas com : no final, ex.: site, inurl, filetype, define, before, after, entre outros.

Para explicar estes operadores vamos imaginar o seguinte caso de uso: um edital no site do IFRS Farroupilha que vimos, nós acessamos ele em alguma data no mês de Agosto de 2025. Sabemos que ele se referia a Certificação de Conhecimento. Com isso em mente podemos formular uma pesquisa direcionada:

\codeblock{"certificação de conhecimentos" AND site:ifrs.edu.br AND inurl:farroupilha AND after:2025-08-1}

Esta pesquisa já é mais complexa que as outras, por isso é necessário explicá-la. Começamos com aquilo que sabemos com toda a certeza: o edital é sobre certificação de conhecimento, e mais, sabemos que este trecho estará presente no site, por isso, colocamos em aspas. Em seguida adicionamos o filtro \textbf{site}, com ele podemos especificar em qual site queremos pesquisar, no nosso caso no site do IFRS. Na sequência adicionamos o inurl, que irá filtrar pelo termo presente no link do site, no nosso caso, ele irá pegar o farroupilha presente no link. Por fim, adicionamos o after que irá filtrar por todos os resultados que o Google achar após a data de 1 de Agosto de 2025.

Ao observar a nossa pesquisa direcionada, demonstrada na  figura \ref{fig:operadores4}, é possível observar que o Google trouxe apenas um resultado que é justamente o que queríamos encontrar.

\begin{figure}[h]
	\centering
	\includegraphics[width=.9\textwidth]{images/motor_de_busca/operadores5.png}
	\caption{Resultado da pesquisa direcionada}
	\label{fig:operadores5}
\end{figure}

Um detalhe importante a se observar é o formato que a data foi colocada. Na pesquisa foi utilizado a seguinte formatação: 2025-08-01. Este formato de data é o formato da ISO 8601 (\cite{ISOISO86011988}), ele especifica que a formatação deve seguir a seguinte ordem: ano, mês e dia separados por hífen.

Agora que você já foi apresentado a como funcionam vários operadores e suas diferentes especificidades, você já é capaz de olhar a tabela 1 e destrinchar pesquisas para sua necessidade. Porém, se ainda restar alguma dúvida o próximo capítulo irá apresentar diferentes situações e pesquisas que irão atender a necessidade.

\begin{table}[h]
	\centering
\begin{tabular}{>{\ttfamily}p{3cm} p{6cm} p{5cm}}
		\toprule
		\textnormal{\textbf{Operador}} & \textbf{Explicação} & \textbf{Uso} \\
		\midrule
		``...'' & Pesquisa pelo termo exato & ``maçã'' \\
		\addlinespace
		OR & Entre um ou outro & maçã OR banana \\
		\addlinespace
		AND & Ambos & maçã AND banana \\
		\addlinespace
		-... & Remove os resultados cujo termo aparece & -verde \\
		\addlinespace
		(...) & Agrupa termos de pesquisa & (maçã OR morango) AND vermelho \\
		\addlinespace
		filetype & Pesquisa pelo tipo de documento & filetype:pdf \\
		\addlinespace
		site & Pesquisa dentro de um site & site:ifrs.edu.br \\
		\addlinespace
		related & Pesquisa por sites relacionados ao site especificado & related:ifrs.edu.br \\
		\addlinespace
		intitle & Pesquisa pelo termo presente no título do site & intitle:educação \\
		\addlinespace
		inurl & Pesquisa pelo termo presente na URL & inurl:farroupilha \\
		\addlinespace
		intext & Pesquisa pelo termo presente no texto do site & intext:certificação \\
		\addlinespace
		before & Pesquisa por resultados catalogados antes de X data & before:2020-01-01 \\
		\addlinespace
		after & Pesquisa por resultados catalogados depois de X data & after:2020-01-01 \\
		\bottomrule
	\end{tabular}
	\caption{Operadores de Pesquisa}
	\label{tab:operador1}
\end{table}

Ainda existem outros operadores de pesquisa, porém eles são considerados inconsistentes, pois funcionam de maneiras imprevisíveis e por isso não serão abordados nesta apostila.

