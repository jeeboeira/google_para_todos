% sections/motor_de_busca/motor_de_busca.tex
% !TeX root = ../../main.tex

\section{Ideias de uso}

Nesta seção, serão apresentados alguns exemplos práticos de como utilizar os operadores de pesquisa do Google para atender a diferentes necessidades. Cada exemplo incluirá uma breve descrição do cenário, a pesquisa formulada e uma explicação de como os operadores foram aplicados para refinar os resultados.

\textbf{Cenário 1: }Você está estudando e precisa encontrar artigos acadêmicos ou documentos oficiais sobre o impacto ambiental da mineração no Brasil, evitando notícias de jornais ou blogs de opinião.

\codeblock{"impacto ambiental" mineração (site:.gov.br OR site:.edu.br) filetype:pdf -notícias}

Essa pesquisa utiliza as aspas para focar no tópico a ser pesquisado, adiciona a palavra chave mineração, mas sem o uso das aspas, permitindo aparecer sinônimos e variações dela. Além disso adiciona filtros de sites, podendo ser do governo (.gov.br) ou sites de educação (.edu.br). Para trazer mais detalhes, foi filtra apenas por PDFs e para fechar, remove as notícias utilizando o operador \textbf{-} com o termo “notícias”.

\hseparator

\textbf{Cenário 2:} Você deseja ver como a inteligência artificial (IA) está evoluindo, porém com o recente aumento da popularidade do tema fica difícil de encontrar notícias antigas.

\codeblock{"inteligência artificial" (site:.com OR site:.org) before:2015-01-01}

Nesta pesquisa são usadas aspas em “inteligência artificial” para obter o termo exato. O operador \textbf{site}: limitou os resultados a domínios confiáveis e abrangentes (.com e .org). Já o \textbf{before} restringiu a pesquisa a conteúdos publicados antes de 2015, permitindo localizar artigos e notícias mais antigas. Assim, conseguindo visualizar como o tema era abordado antes da popularização atual. 

\hseparator

\textbf{Cenário 3: }Você está planejando uma viagem para o Japão, mas os guias turísticos apresentam informações que você já cansou de ler, por isso resolve procurar por experiências de viajantes em fóruns.

\codeblock{"viagem ao Japão" (site:reddit.com OR site:quora.com OR site:tripadvisor.com/forum)}

As aspas em “viagem ao Japão” fixam a expressão exata. O uso do operador \textbf{site}: restringe a busca a sites que funcionam como fóruns ou espaços de discussão (Reddit, Quora e o fórum do TripAdvisor). Isso elimina guias comerciais e blogs promocionais, priorizando relatos e discussões de pessoas reais que já viajaram.

\hseparator

\textbf{Cenário 4:} Você quer acessar livros de filosofia que já estejam em domínio público, preferencialmente em bibliotecas digitais.

\codeblock{"filosofia"AND "domínio público" (site:archive.org OR site:dominiopublico.gov.br) filetype:pdf}

As aspas fixam as expressões “filosofia” e “domínio público”. O operador \textbf{site}: restringe a bibliotecas digitais conhecidas, como Archive.org e Domínio Público. O \textbf{filetype:pdf} traz versões digitais completas das obras. Assim, evita-se material pago ou restrito.

\hseparator

\textbf{Cenário 5:} Você escreveu um trabalho excelente, porém o professor comentou que por mais que o conteúdo esteja bom ele não segue as normas ABNT. Então você decide que quer localizar versões digitais das normas da ABNT sobre formatação de trabalhos.

\codeblock{"ABNT" Formatação de Trabalhos (site:abnt.org.br OR site:.edu.br) filetype:pdf}

As aspas garantem que a palavra ABNT apareça e a pesquisa por formatação de trabalhos ajuda a ferramenta a direcionar o conteúdo. O \textbf{site:} limita a sites oficiais e acadêmicos, como a ABNT e universidades. O \textbf{filetype:pdf} busca documentos em formato pronto para consulta.

\hseparator

\textbf{Cenário 6:} Você quer encontrar receitas veganas de sobremesas que possam ser feitas em menos de 30 minutos. Porém, está cansado de bolo e está sem chocolate em casa.

\codeblock{"receita vegana" sobremesa "até 30 minutos" -chocolate -bolo}

As aspas fixam “receita vegana” e “até 30 minutos”. A palavra sem aspas (sobremesa) permite variações de termos. O operador - exclui resultados com chocolate ou bolo, filtrando para outras opções de sobremesa rápida.