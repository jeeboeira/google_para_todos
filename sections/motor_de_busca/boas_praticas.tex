% sections/motor_de_busca/motor_de_busca.tex
% !TeX root = ../main.tex


\newpage
\section{Boas práticas}

Depois de entendermos como realizar pesquisas básicas no Google, é importante entender sobre como podemos torná-las mais eficientes e confiáveis. Nem sempre os primeiros resultados apresentados serão os mais relevantes, e é justamente aí que entram as boas práticas de busca. Ao aplicar pequenas estratégias, como escolher palavras-chave específicas, utilizar operadores avançados e analisar criticamente as fontes encontradas, é possível economizar tempo e alcançar informações mais precisas. Por isso, é necessário se atentar a algumas recomendações gerais para fazer a sua pesquisa ser o mais precisa possível.

Inicialmente, uma das formas mais eficazes de melhorar os resultados de uma pesquisa no Google é ser específico na escolha das palavras-chave. Pesquisas muito amplas costumam trazer milhares de páginas irrelevantes, enquanto termos mais direcionados ajudam a filtrar a informação. Por exemplo, ao invés de procurar apenas por: \textbf{história Brasil}.

\begin{figure}[h]
	\centering
	\includegraphics[width=.9\textwidth]{images/motor_de_busca/boas_praticas1.png}
	\caption{Resultado da pesquisa história Brasil.}
	\label{fig:boas_praticas1}
\end{figure}

É possível refinar a pesquisa e escrever da seguinte maneira: \textbf{história do Brasil período colonial economia açúcar}. Assim, o motor de busca entende com mais clareza o que se deseja encontrar, entregando resultados mais próximos da necessidade real.

\newpage
\begin{figure}[h]
	\centering
	\includegraphics[width=.9\textwidth]{images/motor_de_busca/boas_praticas2.png}
	\caption{Resultado da pesquisa história do Brasil período colonial economia açúcar.}
	\label{fig:boas_praticas2}
\end{figure}

Após pensar em uma estrutura clara e específica, podemos começar a utilizar os operadores. Por exemplo as aspas, explicadas nos capítulos anteriores, são extremamente poderosas. Essa prática é especialmente útil em trabalhos acadêmicos ou quando se sabe que um dos termos chaves são tão importantes que precisam aparecer da maneira que foram escritos. Por exemplo, a busca por: \textbf{bolo de “chocolate”} garante que os resultados tragam exatamente esse termo, evitando páginas que falem apenas de bolos em geral.

\begin{figure}[h]
	\centering
	\includegraphics[width=.9\textwidth]{images/motor_de_busca/boas_praticas3.png}
	\caption{Resultado da pesquisa bolo de “chocolate”.}
	\label{fig:boas_praticas3}
\end{figure}

Depois de refinar os termos podemos utilizar operadores mais avançados aumentando o controle sobre os resultados exibidos. Por isso, ao escrever a sua pesquisa use como referência a tabela dos operadores apresentada no capítulo anterior. E, caso se perder, pegue como base um dos casos apresentados e mude a pesquisa para que se adapte ao que quer procurar.

Se, ainda assim, os resultados não estiverem muito bons, comece a refinar os termos, removendo tudo aquilo que não é um termo chave para a pesquisa. Assim, ao invés de digitar: \textbf{quais são os sintomas da dengue}, é mais eficiente escrever: \textbf{sintomas dengue}.

\begin{figure}[h]
	\centering
	\includegraphics[width=.9\textwidth]{images/motor_de_busca/boas_praticas4.png}
	\caption{Resultado da pesquisa sintomas dengue.}
	\label{fig:boas_praticas4}
\end{figure}

Por fim, nenhuma pesquisa será realmente eficaz se não houver atenção à credibilidade da fonte. O Google pode trazer resultados diferentes de acordo com a localização, idioma ou histórico de navegação, e nem sempre os primeiros links são os mais confiáveis. Muito menos o resultado da inteligência artificial produzida ao pesquisar, como demonstrado na figura \ref{fig:boas_praticas5}, onde destacado pelo texto “Visão geral criada por IA”, pode trazer reflexões rápidas e diretas, porém pode ser acompanhada de informações falsas e alucinações. Por isso, é fundamental avaliar se o site tem autoridade no tema, se apresenta dados atualizados e se não há indícios de informação duvidosa. Em uma pesquisa acadêmica sobre saúde, por exemplo, é muito mais confiável utilizar resultados de portais como a OMS, em vez de blogs pessoais sem referências científicas.

\vfill{1cm}
\begin{figure}[h]
	\centering
	\includegraphics[width=.9\textwidth]{images/motor_de_busca/boas_praticas5.png}
	\caption{Resultado da pesquisa sintomas dengue.}
	\label{fig:boas_praticas5}
\end{figure}
