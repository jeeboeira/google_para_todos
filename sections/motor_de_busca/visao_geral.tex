% sections/motor_de_busca/motor_de_busca.tex
% !TeX root = ../../main.tex

\section{O Motor de Busca}
Ao acessar a página \href{https://google.com}{google.com} com um navegador de internet, chega-se a tela apresentada na figura \ref{fig:o_motor_de_busca1}. Ela é composta por 4 elementos principais, que seguindo a ordem apresentada na imagem são: a barra de busca, o botão para iniciar a pesquisa, o botão "Estou com sorte" e o menu de atalhos.

\begin{figure}[!ht]
	\centering
	\includegraphics[width=.9\textwidth]{images/motor_de_busca/o_motor_de_busca1.png}
	\caption{Interface do motor de busca.}
	\label{fig:o_motor_de_busca1}
\end{figure}

O elemento mais importante desta tela é a barra de busca, nela será possível digitar os termos que se quer procurar na base de dados do Google, em seguida o botão “Pesquisa Google” tem o mesmo efeito prático que clicar Enter no teclado, que será enviar a busca e retornar os dados, em seguida temos o botão “Estou com sorte”, ele ao invés de procurar e mostrar os resultados irá acessar o primeiro resultado da pesquisa diretamente e por fim há o menu de atalhos, por ali é possível acessar diversas outras ferramentas da empresa, muitas delas já apresentadas nesta apostila, como o Google Docs, Agenda, Drive, entre outros.

Além dos elementos realçados na figura \ref{fig:o_motor_de_busca1} há outros links menos utilizados, como a localização atual, dados de privacidade, referência de como funciona a pesquisa, termos de uso, configurações gerais e atalhos para o \Gls{googleads} e \Gls{googlebusiness}.

Voltando para o elemento principal, a \gls{barrapesquisa}, nela há 3 botões, sendo eles, da esquerda para a direita: o teclado virtual, o microfone para digitação por voz e um atalho para o Google Lens, que permite uma pesquisa inversa de imagem, algo que será explorado nas seções seguintes.

O uso da ferramenta é simples. Na caixa de pesquisa digita-se aquilo que quer procurar, como: “maçã”. E o Google em seguida irá mostrar os resultados para o termo pesquisado, como demonstrado na figura \ref{fig:o_motor_de_busca2}.

\begin{figure}[!ht]
	\centering
	\includegraphics[width=.9\textwidth]{images/motor_de_busca/o_motor_de_busca2.png}
	\caption{Pesquisa no motor de busca.}
	\label{fig:o_motor_de_busca2}
\end{figure}

Ao realizar uma pesquisa simples, como no exemplo dado com o termo “maçã”, o Google irá apresentar uma página de resultados organizada em diferentes blocos de informação. Estes blocos não se limitam apenas a links para outros sites, mas também podem incluir imagens, vídeos, notícias e mapas.

Além dos resultados orgânicos, isto é, aqueles que aparecem de forma natural conforme os critérios do \gls{algoritmo}, o Google também apresenta resultados patrocinados. Estes são anúncios pagos por empresas que desejam aparecer com maior destaque para determinadas palavras-chave. Normalmente, esses links são identificados pela palavra “Patrocinado” ou “Anúncio” logo abaixo do título, permitindo que o usuário saiba que se trata de publicidade.

Outro recurso importante é a seção de pesquisa relacionada, como demonstrado na figura \ref{fig:o_motor_de_busca3}, que aparece ao final da página. Ali o usuário encontra sugestões de outros termos próximos ao que ele digitou, o que pode ser útil para refinar a busca e encontrar resultados mais específicos.

\begin{figure}[!ht]
	\centering
	\includegraphics[width=.9\textwidth]{images/motor_de_busca/o_motor_de_busca3.png}
	\caption{Pesquisa no motor de busca.}
	\label{fig:o_motor_de_busca3}
\end{figure}

Na parte superior da tela de resultados, logo abaixo da barra de busca, o usuário também pode escolher filtros de pesquisa. Estes filtros permitem restringir a busca a categorias específicas, como “Imagens”, “Vídeos”, “Notícias”, “Shopping” e “Mapas”. Assim, em vez de ter acesso apenas a páginas da \gls{web}, o usuário pode rapidamente encontrar aquilo que procura em diferentes formatos.

Ainda pode se perguntar como o Google decide qual \gls{link} é mais relevante na pesquisa, para isso ele considera centenas de fatores. Entre os principais estão a relevância do conteúdo em relação ao termo pesquisado, a qualidade do site, a autoridade conquistada por meio de links de outros sites confiáveis (conhecido como \textit{PageRank}), além da experiência do usuário, como tempo de carregamento e adaptação a dispositivos móveis. Assim, quanto mais útil, confiável e bem estruturada for uma página, maiores são as chances de ela aparecer nas primeiras posições dos resultados (\cite{pagePageRankCitationRanking1999}).

Por fim, é importante notar que a experiência de busca no Google também é influenciada pela personalização dos resultados, que leva em conta fatores como a localização geográfica do usuário, o idioma configurado no navegador, o histórico de navegação e até interesses pessoais identificados por meio de pesquisas anteriores. Isso significa que duas pessoas que pesquisam o mesmo termo, em locais diferentes ou com perfis distintos, podem receber resultados diferentes. Essa personalização busca tornar a experiência mais relevante e prática, aproximando o usuário das informações que ele provavelmente considera mais úteis. 


