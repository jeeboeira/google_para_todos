% sections/exemplo/exemplo.tex
% !TeX root = ../main.tex

\chapter{Modelo de Capítulo — Lorem Ipsum}
\label{chap:lorem}

\begin{objetivos}
Ao final, você será capaz de:
\begin{itemize}
  \item Descrever o propósito do módulo;
  \item Executar um fluxo básico de tarefa;
  \item Identificar e resolver problemas comuns.
\end{itemize}
\end{objetivos}

\section{O que é e por que usar}
\produto{Lorem ipsum} dolor sit amet, consectetur adipiscing elit. Integer
aliquet, mauris non feugiat porta, ante massa gravida nibh, in venenatis lorem
nibh non dolor. Use \tecla{Ctrl+F} para buscar rapidamente no documento.

\section{Conceitos-chave}
\begin{itemize}
  \item Termo A — definição curta e objetiva;
  \item Termo B — quando usar e limitações;
  \item Termo C — relação com os demais conceitos.
\end{itemize}

\section{Passo a passo essencial}
\begin{passos}
\begin{enumerate}
  \item Acesse o sistema e faça login.
  \item Crie um recurso \emph{white} e nomeie segundo o padrão:
        \texttt{AAAA-MM-DD\_Projeto\_Descrição}.
  \item Realize a ação principal e valide o resultado esperado.
  \item Compartilhe com permissões mínimas necessárias.
\end{enumerate}
\end{passos}

\begin{dica}
Use a busca avançada para localizar rapidamente itens por tipo, proprietário e
data de modificação.
\end{dica}

\section{Exemplo de figura}
\begin{figure}[!ht]
  \centering
  % Placeholder de imagem (substitua por \includegraphics[width=.9\textwidth]{figs/...})
  \rule{0.9\textwidth}{6cm}
  \caption{Área reservada para imagem de exemplo.}
  \label{fig:exemplo}
\end{figure}

Como mostrado na \autoref{fig:exemplo}, mantenha as capturas com boa legibilidade.

\section{Exemplo de tabela}
\begin{table}[!ht]
\centering
\begin{tabular}{@{}ll@{}}
\toprule
Recurso & Descrição \\
\midrule
Item A  & Explicação resumida do item A. \\
Item B  & Explicação resumida do item B. \\
\bottomrule
\end{tabular}
\caption{Tabela de exemplo com \texttt{booktabs}.}
\label{tab:exemplo}
\end{table}

Consulte a \autoref{tab:exemplo} para o resumo dos itens.

\section{Atalhos e truques úteis}
\begin{itemize}
  \item \tecla{Ctrl+K} → inserir link;
  \item \tecla{Ctrl+Shift+C} → copiar formatação (exemplo);
  \item Buscas salvas para reutilizar filtros frequentes.
\end{itemize}

\section{Problemas comuns e soluções rápidas}
\begin{atencao}
\textbf{Não consigo compartilhar:} confirme o e-mail e o papel do usuário.\\
\textbf{Arquivo muito grande:} compacte ou use upload via aplicativo de desktop.\\
\textbf{Conflitos de edição:} use comentários e modo de \emph{sugestões}.
\end{atencao}

\section{Checklist de domínio}
\begin{checagem}
Marque mentalmente o que você faz sem consultar:
\begin{itemize}[leftmargin=*]
  \item \texttt{[ ]} Acessa e navega pelas áreas principais;
  \item \texttt{[ ]} Executa a tarefa essencial do módulo;
  \item \texttt{[ ]} Configura compartilhamento/segurança corretamente;
  \item \texttt{[ ]} Resolve um problema comum.
\end{itemize}
\end{checagem}
