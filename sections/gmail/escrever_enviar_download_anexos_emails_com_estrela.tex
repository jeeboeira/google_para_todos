% sections/gmail/escrever_enviar_download_anexos_emails_com_estrela.tex
% !TeX root = ../../main.tex


\section{Como Escrever um e-mail}
Este guia irá demonstrar o processo completo para compor e enviar uma mensagem 
de forma eficaz. Para tornar o aprendizado mais prático, vamos simular uma 
situação real: “O envio de um currículo para uma oportunidade de emprego”. 
Acompanhe como construir o e-mail, anexar arquivos, inserir links importantes, 
acessar os rascunhos e agendar o envio do e-mail.


\subsection{Iniciando um novo e-mail}
Clique no botão “Escrever” no canto superior esquerdo da tela. Uma janela de 
“Nova mensagem” será aberta no canto inferior direito, assim você poderá começar 
a compor seu e-mail.

\begin{figure}[H]
    \centering
    \includegraphics[width=.39\textwidth]{/gmail/escrever_enviar_download_anexos_emails_com_estrela/Imagem1.png}
    \caption{}
\end{figure}

Para uma melhor visualização, você pode maximizar essa janela clicando no ícone de tela cheia.

\begin{figure}[H]
    \centering
    \includegraphics[width=.39\textwidth]{/gmail/escrever_enviar_download_anexos_emails_com_estrela/Imagem2.png}
    \caption{}
\end{figure}

\subsection{Preenchendo os campos da mensagem}
Agora, preencha os campos essenciais do seu e-mail:

\begin{itemize}
    \item Destinatários / Para: O endereço de e-mail do destinatário.
    \item Assunto: Escreva um título claro e objetivo para a sua mensagem.
    \item Corpo do e-mail: Escreva o texto da sua mensagem no espaço principal.
\end{itemize}

\begin{figure}[H]
    \centering
    \includegraphics[width=.39\textwidth]{/gmail/escrever_enviar_download_anexos_emails_com_estrela/Imagem3.png}
    \caption{}
\end{figure}


\subsubsection{Anexando um Arquivo}
Para incluir um documento, como um currículo (PDF) ou uma foto, clique no ícone 
de \textbf{clipe de papel} na barra de ferramentas inferior. Uma janela do seu computador 
será aberta para que você possa navegar e selecionar o arquivo desejado, depois 
de localizar o arquivo clique em “Abrir”.

\begin{figure}[H]
    \centering
    \includegraphics[width=.39\textwidth]{/gmail/escrever_enviar_download_anexos_emails_com_estrela/Imagem4.png}
    \caption{}
\end{figure}

Após selecionar o arquivo e clicar em “Abrir”, ele aparecerá na parte inferior 
da sua mensagem, mostrando uma barra de carregamento, após essa barra ser 
completa, o arquivo está pronto para ser enviado.

\begin{figure}[H]
    \centering
    \includegraphics[width=.39\textwidth]{/gmail/escrever_enviar_download_anexos_emails_com_estrela/Imagem5.png}
    \caption{}
\end{figure}

Caso precise adicionar um link para um site ou perfil de rede social, como o 
LinkedIn, selecione o texto que deseja transformar em link e clique no ícone de 
\textbf{corrente} barra de ferramentas. Uma caixa de diálogo aparecerá para você, apenas 
cole a URL e clique em “Aplicar”.

\begin{figure}[H]
    \centering
    \includegraphics[width=.39\textwidth]{/gmail/escrever_enviar_download_anexos_emails_com_estrela/Imagem6.png}
    \caption{}
\end{figure}


\subsection{Acessando os Rascunhos}
Em caso de você acabar fechando a mensagem, não se preocupe em perder seu 
trabalho. O Gmail salva automaticamente seu progresso, se isso acontecer poderá 
encontrar seu e-mail salvo na pasta “Rascunhos”, localizada no menu à esquerda, 
assim podendo continuar de onde você parou.

\begin{figure}[H]
    \centering
    \includegraphics[width=.39\textwidth]{/gmail/escrever_enviar_download_anexos_emails_com_estrela/Imagem7.png}
    \caption{}
\end{figure}


\subsubsection{Enviando ou programando o e-mail}
Para enviar o e-mail imediatamente, basta clicar no botão “Enviar”.

\begin{figure}[H]
    \centering
    \includegraphics[width=.39\textwidth]{/gmail/escrever_enviar_download_anexos_emails_com_estrela/Imagem8.png}
    \caption{}
\end{figure}

Ao invés de enviar o e-mail imediatamente, você pode agendar o envio para um 
horário mais apropriado, para isso, ao lado do botão “Enviar”, clique na 
\textbf{seta para baixo} e selecione a opção “Programar envio”.

\begin{figure}[H]
    \centering
    \includegraphics[width=.39\textwidth]{/gmail/escrever_enviar_download_anexos_emails_com_estrela/Imagem8.1.png}
    \caption{}
\end{figure}

Uma janela aparecerá com sugestões de data e hora ou a opção “Escolher data e 
hora”, para que você personalize o envio.

\begin{figure}[H]
    \centering
    \includegraphics[width=.39\textwidth]{/gmail/escrever_enviar_download_anexos_emails_com_estrela/Imagem9.png}
    \caption{}
\end{figure}


\subsubsection{Gerenciando e-mails Programados}
Após programar o envio do e-mail, ele não irá aparecer na caixa de “Enviados”, 
mas sim em uma caixa chamada “Programados”. Você pode acessá-la no menu à 
esquerda para conferir os e-mails que estão aguardando o momento do envio.

\begin{figure}[H]
    \centering
    \includegraphics[width=.39\textwidth]{/gmail/escrever_enviar_download_anexos_emails_com_estrela/Imagem10.png}
    \caption{}
\end{figure}

Ao clicar no e-mail dentro da pasta “Programados”, você pode revisar seu 
conteúdo e o horário agendado.

\begin{figure}[H]
    \centering
    \includegraphics[width=.39\textwidth]{/gmail/escrever_enviar_download_anexos_emails_com_estrela/Imagem11.png}
    \caption{}
\end{figure}

Para cancelar o envio de um e-mail programado, você deve abrir novamente a pasta 
“Programados”, clicar sobre o e-mail que deseja cancelar, e clicar no botão 
“Cancelar envio”.

\begin{figure}[H]
    \centering
    \includegraphics[width=.39\textwidth]{/gmail/escrever_enviar_download_anexos_emails_com_estrela/Imagem11.1.png}
    \caption{}
\end{figure}


\subsection{Verificando e-mails Enviados}
Para acessar os e-mails enviados por você, basta clicar na pasta “Enviados”, 
localizada no menu à esquerda.

\begin{figure}[H]
    \centering
    \includegraphics[width=.39\textwidth]{/gmail/escrever_enviar_download_anexos_emails_com_estrela/Imagem12.png}
    \caption{}
\end{figure}


\section{Como realizar o download dos anexos}
Aprenda a baixar facilmente os arquivos que você recebe por e-mail, como 
documentos, fotos ou planilhas, diretamente para seu computador.


\subsection{Localize e Abra o e-mail}
Identifique na pasta “Caixa de Entrada”, ou em qualquer outra pasta, a mensagem 
que contém o anexo que deseja baixar. Para abrir o anexo pode ser seguido de 
duas maneiras, uma sendo normalmente exibido abaixo do “Assunto” e do resumo do 
corpo do email.

\begin{figure}[H]
    \centering
    \includegraphics[width=.39\textwidth]{/gmail/escrever_enviar_download_anexos_emails_com_estrela/Imagem13.png}
    \caption{}
\end{figure}

Ou também clicando na mensagem para abri-la e rolando a página até o final. A 
seção de anexos estará visível logo abaixo do corpo do e-mail.

\begin{figure}[H]
    \centering
    \includegraphics[width=.39\textwidth]{/gmail/escrever_enviar_download_anexos_emails_com_estrela/Imagem14.png}
    \caption{}
\end{figure}


\section{Como Usar a Função ``Com Estrela''}
A função “Com estrela” é uma forma rápida de destacar e-mails importantes para 
que você possa encontrá-los facilmente mais tarde, funcionando como uma lista de 
favoritos.


\subsubsection{Como Marcar e Desmarcar um e-mail com Estrela}
Identifique na pasta “Caixa de Entrada”, ou em qualquer outra pasta, a mensagem 
que deseja destacar. Ao lado esquerdo do nome do remetente, você verá um ícone 
de estrela vazia.

\begin{figure}[H]
    \centering
    \includegraphics[width=.39\textwidth]{/gmail/escrever_enviar_download_anexos_emails_com_estrela/Imagem13.1.png}
    \caption{}
\end{figure}

Para marcar o e-mail, basta clicar no ícone da \textbf{estrela}. Para desmarcar, basta 
clicar na estrela novamente.


\subsubsection{Como acessar seus e-mails marcados com Estrela}
Para visualizar de maneira separada todos os e-mails que você marcou com 
estrela, clique na pasta “Com estrela”, localizada no menu à esquerda.

\begin{figure}[H]
    \centering
    \includegraphics[width=.39\textwidth]{/gmail/escrever_enviar_download_anexos_emails_com_estrela/Imagem15.png}
    \caption{}
\end{figure}