% sections/gmail/configuracoes_avancadas.tex
% !TeX root = ../../main.tex

\section{Configurações de Organização Avançadas}
Apresenta como o usuário pode personalizar sua experiência no Gmail, inclui 
ajustes de idioma, aparência, ferramentas, organização da caixa de entrada e 
integração com outros serviços do Google.


\subsection{Como acessar as configurações}
Para acessar as configurações do Gmail, clique no ícone de engrenagem localizado 
no canto superior direito da tela.

\begin{figure}[H]
    \centering
    \includegraphics[width=.39\textwidth]{/gmail/configuracoes_avancadas/Imagem1.png}
    \caption{}
\end{figure}

\textbf{Ao abrir o menu, algumas opções de acesso rápido já são exibidas:}
\begin{itemize}
    \item \textbf{Apps do Gmail:} Permite escolher se os aplicativos integrados, 
    como Chat e Meet, serão exibidos ou não.
    \item \textbf{Densidade:} Define como os e-mails aparecem na lista, podendo 
    escolher entre compacto, confortável ou padrão, dependendo da preferência do 
    usuário e do tamanho da tela.
    \item \textbf{Tema:} Permite personalizar a aparência do Gmail, escolhendo 
    imagens de fundo oferecidas pela plataforma ou enviando imagens próprias.
    \item \textbf{Tipo de caixa de entrada:} Define a forma de priorização das 
    mensagens, permitindo destacar e-mails importantes ou organizar conforme 
    categorias específicas.
    \item \textbf{Painel de leitura:} Dá ao usuário a opção de abrir e-mails 
    diretamente na lista, à direita ou abaixo da caixa de entrada, facilitando a 
    leitura.
    \item \textbf{Conversa por e-mail:} Agrupa mensagens com o mesmo assunto em 
    uma sequência de conversa, tornando mais fácil acompanhar o histórico de 
    respostas.
\end{itemize}

Para acessar todas as opções de configuração, clique em “Mostrar todas as 
configurações”.

\begin{figure}[H]
    \centering
    \includegraphics[width=.39\textwidth]{/gmail/configuracoes_avancadas/Imagem2.png}
    \caption{}
\end{figure}


\subsection{Configuração geral}
Na Configuração geral, o usuário pode ajustar diversas preferências como:

% TODO: Colocar em negrito os nomes dos itens da lista
\begin{enumerate}
    \item Idioma: Define o idioma da interface e permite ajustar configurações de outros produtos Google. Também oferece suporte a ferramentas de inserção de texto para múltiplos idiomas.
    \item Tamanho máximo da página: Permite escolher quantas conversas ou contatos serão exibidos por página.
    \item Cancelar envio: Possibilita desfazer o envio de um e-mail por um período definido, evitando envios acidentais.
    \item Comportamento de resposta padrão: Permite definir se ao responder um e-mail, a resposta será enviada apenas ao remetente ou a todos os destinatários.
    \item Ações de passar o cursor: Habilita ou desabilita ícones de ação rápida que aparecem ao passar o mouse sobre um e-mail.
    \item Enviar e arquivar: Adiciona um botão que envia a resposta e arquiva a mensagem ao mesmo tempo.
    \item Estilo de texto padrão: Permite definir a fonte, tamanho e cor do texto usado nos e-mails.
    \item Imagens: Escolhe se as imagens nos e-mails são carregadas automaticamente ou se a exibição deve ser solicitada.
    \item E-mail dinâmico: Habilita recursos interativos dentro dos e-mails, como formulários e botões.
    \item Gramática: Oferece sugestões automáticas de correção gramatical enquanto o usuário digita um e-mail.
    \item Ortografia: Verifica e destaca erros ortográficos, sugerindo alterações para manter a escrita correta.
    \item Correção automática: Aplica automaticamente as correções ortográficas sugeridas pelo Gmail.
    
    \begin{figure}[H]
        \centering
        \includegraphics[width=.9\textwidth]{/gmail/configuracoes_avancadas/Imagem3.png}
        \caption{}
    \end{figure}

    \item Escrita inteligente: Sugere palavras ou frases enquanto o usuário digita.
    \item Personalização da Escrita inteligente: Ajusta as sugestões com base no histórico de mensagens do usuário.
    \item Visualização de conversas: Agrupa mensagens com o mesmo assunto, facilitando a leitura do histórico completo de uma conversa.
    \item Alertas: Sugere e-mails que podem ter sido esquecidos ou que precisam de acompanhamento.
    \item Resposta inteligente: Oferece respostas curtas e prontas com base no conteúdo do e-mail recebido.
    \item Painel de visualização: Permite dividir a tela em painéis para visualizar a lista de e-mails e o conteúdo de uma mensagem simultaneamente.
    \item Recursos inteligentes: Permite que informações do Gmail sejam usadas em outros aplicativos Google, como Chat e Meet.
    \item Recursos inteligentes do Google Workspace: Permite que informações de e-mails sejam utilizadas em outros serviços do Google Workspace.
    \item Notificações na área de trabalho: Ativa alertas de e-mail diretamente no computador, avisando sobre novas mensagens recebidas.
    \item Estrelas: Permite personalizar ícones de marcação para destacar mensagens importantes.
    \item Atalhos do teclado: Habilita comandos rápidos via teclado para realizar ações rápidas.
    \item Marcadores de botão: Define se os botões da interface exibem apenas ícones ou ícones acompanhados de texto.
    
    \begin{figure}[H]
        \centering
        \includegraphics[width=.9\textwidth]{/gmail/configuracoes_avancadas/Imagem4.png}
        \caption{}
    \end{figure}

    \item Minha foto: Permite adicionar ou alterar a foto de perfil do usuário.
    \item Criar contatos para preenchimento automático: Define se novos endereços de e-mail devem ser salvos automaticamente ao enviar mensagens.
    \item Assinatura: Permite criar assinaturas personalizadas que aparecem automaticamente no final do e-mail enviado.
    \item Indicadores de nível pessoal: Exibe indicadores para mostrar se o e-mail foi enviado diretamente ao usuário ou para um grupo.
    \item Snippets: Exibe um trecho do conteúdo do e-mail logo abaixo do assunto na caixa de entrada.
    \item Resposta automática de férias: Permite programar respostas automáticas para um período específico.
\end{enumerate}

\begin{atencao}
Após realizar alterações, é necessário clicar no botão “Salvar alterações” na 
parte inferior da tela para que todas as configurações sejam aplicadas.
\end{atencao}

\begin{figure}[H]
    \centering
    \includegraphics[width=.39\textwidth]{/gmail/configuracoes_avancadas/Imagem5.png}
    \caption{}
\end{figure}

\subsection{Configuração de Marcadores}
A aba Configuração de Marcadores permite ao usuário gerenciar e personalizar os 
marcadores existentes e criar novos, facilitando a organização da caixa de 
entrada.\newline

Marcadores funcionam como etiquetas que classificam e-mails, ajudando a 
encontrar mensagens importantes de forma rápida. Alguns marcadores vem como 
padrão, como Com estrela, Adiados, Spam, Lixeira, entre outros. Mas o usuário 
pode criar marcadores personalizados conforme suas necessidades.

\begin{passos}
    \begin{enumerate}
        \item Acesse a aba Marcadores nas configurações.
        \item Selecione quais marcadores deseja exibir na barra lateral 
            esquerda.
        \item Para criar um novo marcador, clique em “Criar novo marcador”, 
            insira o nome desejado e, se necessário, defina um marcador pai para 
            hierarquia e organização.
    \end{enumerate}
\end{passos}

\begin{figure}[H]
    \centering
    \includegraphics[width=.39\textwidth]{/gmail/configuracoes_avancadas/Imagem6.png}
    \caption{}
\end{figure}

\subsection{Configuração da Caixa de entrada}
A aba Configuração da Caixa de entrada permite personalizar a forma como as 
mensagens são exibidas.

\begin{enumerate}
    \item Tipo de Caixa de entrada: Permite definir quais e-mails serão exibidos primeiro.
    \item Categorias: Ativa abas como Principal, Social, Promoções, Atualizações e Fóruns, que organizam automaticamente as mensagens conforme o conteúdo.
    \item Painel de leitura: Permite dividir a tela, visualizando simultaneamente a lista de e-mails e o conteúdo de uma mensagem.
    \item Marcadores de importância: Exibe ícones para destacar e-mails considerados importantes, com base no histórico de uso do usuário e nos critérios do Gmail.
    \item E-mails filtrados: Mostra ou oculta mensagens que já foram direcionadas automaticamente por filtros criados pelo usuário.
\end{enumerate}

\begin{atencao}
Após realizar alterações, clique no botão “Salvar alterações” na parte inferior 
da tela para aplicá-las.
\end{atencao}

\begin{figure}[H]
    \centering
    \includegraphics[width=.39\textwidth]{/gmail/configuracoes_avancadas/Imagem7.png}
    \caption{}
\end{figure}


\subsection{Configuração de Contas e importação}
Essa aba permite gerenciar contas vinculadas e importar mensagens de outros 
e-mails.

\begin{enumerate}
    \item Alterar configurações da conta: Permite modificar informações básicas da conta, como senha, métodos de recuperação e outras preferências de segurança.
    \item Enviar e-mail como: Permite configurar endereços alternativos de envio.
    \item Verificar o e-mail de outras contas: Importa mensagens de outras contas de e-mail para centralizar tudo na mesma caixa de entrada.
    \item Permitir acesso à sua conta: Dá permissão a outros usuários para ler, responder e organizar e-mails em seu nome, sem compartilhar a senha.
\end{enumerate}

\begin{figure}[H]
    \centering
    \includegraphics[width=.39\textwidth]{/gmail/configuracoes_avancadas/Imagem8.png}
    \caption{}
\end{figure}


\subsubsection{Configuração de Filtros e endereços bloqueados}
Essa aba ajuda o usuário a organizar e proteger a caixa de entrada.

\begin{itemize}
    \item Permite criar filtros personalizados que executam ações 
        automaticamente, como arquivar, marcar como lido, aplicar marcadores, 
        encaminhar ou excluir e-mails.
    \item Também possibilita bloquear endereços específicos, garantindo que 
        mensagens de remetentes indesejados não cheguem à caixa de entrada.
\end{itemize}

\begin{figure}[H]
    \centering
    \includegraphics[width=.39\textwidth]{/gmail/configuracoes_avancadas/Imagem9.png}
    \caption{}
\end{figure}

\subsubsection{Configuração de Encaminhamento}
A aba Encaminhamento permite redirecionar automaticamente e-mails recebidos para 
outra conta e definir regras de encaminhamento com base em filtros.

\begin{atencao}
Após qualquer alteração, clique em “Salvar alterações” para que as configurações 
sejam aplicadas.
\end{atencao}

\begin{figure}[H]
    \centering
    \includegraphics[width=.39\textwidth]{/gmail/configuracoes_avancadas/Imagem10.png}
    \caption{}
\end{figure}

\subsection{Configuração de Complemento}
A aba Complemento permite ao usuário instalar e gerenciar extensões e 
integrações no Gmail.

\begin{itemize}
    \item É possível adicionar ferramentas de terceiros, como gerenciadores de 
        tarefas, CRMs ou apps de produtividade.
    \item Os complementos ajudam a automatizar processos, centralizar 
        informações e melhorar a organização.
\end{itemize}

\begin{figure}[H]
    \centering
    \includegraphics[width=.39\textwidth]{/gmail/configuracoes_avancadas/Imagem11.png}
    \caption{}
\end{figure}

\subsection{Configuração de Chat e Meet}
Esta seção integra os serviços de comunicação do Google no Gmail, permitindo 
configurações detalhadas para Chat e Meet.

\begin{enumerate}
    \item Chat: Ativa ou desativa o Google Chat no Gmail.
    \item Configurações do Chat: Ajusta a posição do Chat na tela e permite 
        ativar ou desativar o histórico de conversas.
    \item Configurações do Meet: Permite gerenciar preferências de reuniões, 
        como disponibilidade e integração de convites, diretamente pelo Gmail.
    \item Meet: Mostra ou oculta a seção do Google Meet na barra lateral.
\end{enumerate}

\begin{atencao}
Após realizar alterações, clique no botão “Salvar alterações” para aplicá-las.
\end{atencao}

\begin{figure}[H]
    \centering
    \includegraphics[width=.39\textwidth]{/gmail/configuracoes_avancadas/Imagem12.png}
    \caption{}
\end{figure}

\subsection{Configurações Avançadas}
A aba Configurações Avançadas disponibiliza funcionalidades adicionais que podem 
ser habilitadas ou desabilitadas conforme a necessidade do usuário.

\begin{enumerate}
    \item Avanço automático: Após realizar uma ação em um e-mail, o Gmail abre 
        automaticamente o próximo da lista.
    \item Modelos: Permite criar e utilizar mensagens pré-formatadas, 
        facilitando o envio de e-mails frequentes ou repetitivos.
    \item Atalhos de teclado personalizados: Habilita a criação de combinações 
        de teclas personalizadas, aumentando a produtividade para usuários que 
        preferem atalhos.
    \item Ícone de mensagem não lida: Exibe no ícone da aba do navegador, a 
        quantidade de e-mails não lidos.
\end{enumerate}

\begin{atencao}
Após realizar alterações, clique no botão “Salvar alterações” para aplicá-las.
\end{atencao}

\begin{figure}[H]
    \centering
    \includegraphics[width=.39\textwidth]{/gmail/configuracoes_avancadas/Imagem13.png}
    \caption{}
\end{figure}

\subsubsection{Configurações Off-line}
Esta aba permite ao usuário utilizar o Gmail mesmo sem conexão com a internet.

\begin{itemize}
    \item Ideal para situações em que o usuário não tem internet constante.
    \item As mensagens enviadas offline ficam na caixa de saída até que o Gmail 
    sincronize com o servidor ao reconectar.
\end{itemize}

\begin{figure}[H]
    \centering
    \includegraphics[width=.39\textwidth]{/gmail/configuracoes_avancadas/Imagem14.png}
    \caption{}
\end{figure}

\subsubsection{Configurações de Temas}
A aba Temas permite ao usuário personalizar a aparência visual do Gmail, 
ajustando cores, fundos e estilos da interface.

\begin{itemize}
    \item É possível escolher entre imagens de fundo fornecidas pelo Gmail ou 
    enviar imagens próprias.
\end{itemize}

\begin{figure}[H]
    \centering
    \includegraphics[width=.39\textwidth]{/gmail/configuracoes_avancadas/Imagem15.png}
    \caption{}
\end{figure}
