% sections/gmail/responder_encaminhar_marcar_emails.tex
% !TeX root = ../../../main.tex

\section{Como responder e encaminhar e-mails}
Após receber e ler os seus e-mails, as ações mais comuns são continuar a 
conversa ou compartilhar a informação com outras pessoas. Para isso, o Gmail 
oferece as funções “Responder” e “Encaminhar”. Aprender a utilizar essas 
ferramentas é fundamental para uma comunicação digital eficaz.


\subsubsection{Como responder a um e-mail}
Use a função de responder para enviar uma mensagem de volta ao remetente 
original, assim, poderá continuar a conversa.\newline

Para isso, abra o e-mail que você deseja responder. Na parte inferior da 
mensagem você encontrará o botão "Responder". Em seguida, uma nova caixa de 
diálogo aparecerá abaixo do e-mail. Digite sua resposta nesse novo espaço. 
Após a conclusão do texto da resposta, clique no botão Enviar.

\begin{figure}[H]
    \centering
    \includegraphics[width=.39\textwidth]{/gmail/responder_encaminhar_marcar_emails/Imagem1.png}
    \caption{}
\end{figure}

\subsubsection{Como encaminhar um e-mail}
Use a função de encaminhar quando for necessário enviar uma cópia exata de um 
e-mail que você recebeu para uma pessoa que não estava na conversa 
originalmente.\newline

Para isso, abra o e-mail que você deseja encaminhar e clique no botão 
"Encaminhar", localizado ao lado do botão de resposta. Na sequência, uma nova caixa 
de diálogo será aberta. No campo “Para”, digite o endereço de e-mail do 
destinatário. Se desejar, você pode escrever uma mensagem adicional acima do 
conteúdo original que está sendo encaminhado.

\begin{figure}[H]
    \centering
    \includegraphics[width=.39\textwidth]{/gmail/responder_encaminhar_marcar_emails/Imagem2.png}
    \caption{}
\end{figure}


\section{E-mails como importante}
O marcador "Importante" do Gmail é uma ferramenta inteligente que ajuda você a 
priorizar suas mensagens. O próprio Gmail tenta prever o que é importante para o 
usuário, mas você também pode marcar e desmarcar e-mails manualmente. Isso ajuda 
a organizar sua caixa de entrada e a encontrar rapidamente as mensagens que 
exigem mais atenção.


\subsubsection{Como marcar um e-mail como importante}
Para sinalizar manualmente que uma mensagem é uma prioridade, estando na sua 
Caixa de Entrada, selecione o e-mail desejado clicando na caixa de seleção. Em 
seguida, clique no ícone de \textbf{três pontos} na barra de ferramentas 
superior. Após abrir o menu de opções, selecione "Marcar como importante".

\begin{figure}[H]
    \centering
    \includegraphics[width=.39\textwidth]{/gmail/responder_encaminhar_marcar_emails/Imagem3.png}
    \caption{}
\end{figure}


\subsubsection{Como desmarcar um e-mail como importante}
Todos os e-mails marcados como importantes, seja por você ou pelo Gmail, são 
reunidos em uma pasta específica para fácil acesso. Para visualizá-los, basta 
clicar na pasta "Importante", localizada no menu lateral esquerdo da tela. Com 
isso, você consegue marcar o e-mail como não importante repetindo as mesmas ações: 
clicar na caixa de seleção, no ícone de três pontos na barra de ferramentas 
superior e, por fim, na opção "Marcar como não importante".

\begin{figure}[H]
    \centering
    \includegraphics[width=.39\textwidth]{/gmail/responder_encaminhar_marcar_emails/Imagem4.png}
    \caption{}
\end{figure}


\section{Como Lidar com E-mails de Spam}
\subsubsection{O que é Spam?}
Spam são e-mails indesejados, geralmente enviados em massa para um grande número 
de destinatários sem o consentimento deles. Na maioria das vezes, são mensagens 
de publicidade, mas também podem ser perigosas, contendo tentativas de golpe 
(phishing), vírus ou links maliciosos. Nunca clique em links ou baixe anexos de 
e-mails de spam.


\subsubsection{Como Marcar um E-mail como Spam (Denunciar Spam)}
Ao receber um e-mail de spam em sua caixa de entrada, você pode denunciá-lo para 
removê-lo e ajudar o filtro do Gmail. Para isso, basta selecionar a mensagem 
suspeita na sua caixa de entrada (sem a necessidade de abri-la) e clicar no ícone 
de atenção localizado no menu superior.

\begin{figure}[H]
    \centering
    \includegraphics[width=.39\textwidth]{/gmail/responder_encaminhar_marcar_emails/Imagem5.png}
    \caption{}
\end{figure}


\subsubsection{Como Acessar e Gerenciar a Caixa de Spam}
Como um e-mail legítimo pode ser classificado como spam por engano, é uma boa 
prática verificar esta pasta periodicamente. Para acessá-la, clique na pasta "Spam" no menu 
lateral esquerdo. Caso encontre um e-mail que não é spam, basta selecioná-lo e 
clicar no botão "Não é spam" no topo da tela para que ele retorne à sua Caixa de 
Entrada.

\begin{figure}[H]
    \centering
    \includegraphics[width=.39\textwidth]{/gmail/responder_encaminhar_marcar_emails/Imagem6.png}
    \caption{}
\end{figure}