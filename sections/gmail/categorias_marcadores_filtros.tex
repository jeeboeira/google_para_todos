% sections/gmail/categorias_marcadores_filtros.tex
% !TeX root = ../../../main.tex

\section{Categorias}
\subsection{O que são as Categorias}
As categorias são elementos utilizados para organização das mensagens da caixa 
de entrada. Elas separam os conteúdos em diferentes grupos. O grupo “Principal” 
é definido como a caixa de entrada padrão para mensagens no Gmail. Há, também, 
as categorias “Social”, “Atualizações”, “Fóruns” e “Promoções”, que podem ser 
selecionadas para filtragem das mensagens.

\subsection{Como acessar o campo de categorias}
O campo “Categorias” pode ser acessado ao clicar no “Menu Principal” (ícone de 
três barrinhas horizontais), no canto superior esquerdo da tela, caso a barra 
lateral não esteja sendo exibida. Após, basta selecionar o item “Mais” para 
exibir os demais filtros de pesquisa. Por fim, clique sobre o item “Categoria”.

\begin{figure}[H]
    \centering
    \includegraphics[width=.39\textwidth]{/gmail/categorias_marcadores_filtros/Imagem1.png}
    \caption{}
\end{figure}

Após abrir a opções de categorias, serão exibidas as seguintes opções:
\begin{itemize}
    \item Social: tratam-se dos e-mails que contêm mensagens de redes sociais e 
    sites de compartilhamento de mídia;
    \item Atualizações: agrupa os e-mails que contêm confirmações, notificações, 
    extratos e lembretes que não precisam de atenção imediata;
    \item Fóruns: corresponde aos e-mails com mensagens de grupos online, fóruns 
    de discussão e listas de e-mails;
    \item  Promoções: tratam-se dos e-mails que contêm informações de descontos, 
    ofertas e promoções;
\end{itemize}

\begin{figure}[H]
    \centering
    \includegraphics[width=.6\textwidth]{/gmail/categorias_marcadores_filtros/Imagem2.png}
    \caption{}
\end{figure}

\subsection{Como adicionar os campos de categoria na caixa de entrada}
Para adicionar qualquer uma dessas 4 abas em sua caixa de entrada, comece 
acessando a aba “Configurações”, no canto superior direito da tela.\newline

Em seguida, selecione a opção “Mostrar todas as configurações”.

\begin{figure}[H]
    \centering
    \includegraphics[width=.55\textwidth]{/gmail/categorias_marcadores_filtros/Imagem3.png}
    \caption{}
\end{figure}

Após, na parte superior da tela, selecione a opção “Caixa de entrada”.

\begin{figure}[H]
    \centering
    \includegraphics[width=.6\textwidth]{/gmail/categorias_marcadores_filtros/Imagem4.png}
    \caption{}
\end{figure}

Por padrão, a categoria “Principal” já estará selecionada, e não pode ser 
removida. Você pode adicionar os demais filtros (“Promoções”, “Social”, 
“Atualizações” e “Fóruns”) clicando sobre as caixinhas ao lado de seus nomes.

\begin{figure}[H]
    \centering
    \includegraphics[width=.6\textwidth]{/gmail/categorias_marcadores_filtros/Imagem5.png}
    \caption{}
\end{figure}

Após as alterações, role a página para baixo (com o scroll do mouse ou a barra 
lateral da janela) e clique no botão “Salvar alterações”, para definir as 
modificações.

\begin{figure}[H]
    \centering
    \includegraphics[width=.6\textwidth]{/gmail/categorias_marcadores_filtros/Imagem6.png}
    \caption{}
\end{figure}

Após realizar as alterações, você irá retornar automaticamente para a caixa de 
entrada, onde estarão sendo exibidos os campos de categoria selecionados na 
parte superior da tela.

\begin{figure}[H]
    \centering
    \includegraphics[width=.6\textwidth]{/gmail/categorias_marcadores_filtros/Imagem7.png}
    \caption{}
\end{figure}

Para desfazer essas alterações, basta reproduzir os passos anteriores e 
desmarcar as caixinhas ao lado do nome das opções de categorias que não gostaria 
mais de exibir.

\section{Marcadores}
\subsection{Para que servem os marcadores}
Os marcadores são uma forma eficiente de organizar suas mensagens. Com eles, 
você pode adicionar uma etiqueta ao e-mail, tornando mais fácil a tarefa de 
encontrar seus emails de maior relevância.

\subsection{Como criar um novo marcador}
Para acessar o campo de marcadores você deve ir na aba esquerda do “Menu 
Principal” e selecionar a opção “Mais”. Em seguida, clique sobre a “Criar novo 
marcador”.

\begin{figure}[H]
    \centering
    \includegraphics[width=.6\textwidth]{/gmail/categorias_marcadores_filtros/Imagem8.png}
    \caption{}
\end{figure}

Após selecionar a opção, será aberta uma janela, onde você poderá adicionar um 
nome ao seu novo marcador. Se já tiver algum marcador com a opção abaixo, você 
pode definir sub-marcadores para organização.

\begin{figure}[H]
    \centering
    \includegraphics[width=.6\textwidth]{/gmail/categorias_marcadores_filtros/Imagem9.png}
    \caption{}
\end{figure}

Os marcadores criados serão exibidos abaixo da opção “Marcadores”.

\begin{figure}[H]
    \centering
    \includegraphics[width=.6\textwidth]{/gmail/categorias_marcadores_filtros/Imagem10.png}
    \caption{}
\end{figure}

\subsection{Como aplicar marcadores aos e-mails}
Para aplicar um marcador a um email, você deve clicar sobre a mensagem a qual 
você quer aplicar o marcador e selecionar o símbolo de marcadores na parte 
superior da tela.

\begin{figure}[H]
    \centering
    \includegraphics[width=.6\textwidth]{/gmail/categorias_marcadores_filtros/Imagem11.png}
    \caption{}
\end{figure}

Ao selecionar o símbolo de marcadores, será aberta uma janela onde você pode 
adicionar o e-mail a um dos seus marcadores criados ou atribuir um ou mais 
campos de categoria.

\begin{figure}[H]
    \centering
    \includegraphics[width=.6\textwidth]{/gmail/categorias_marcadores_filtros/Imagem12.png}
    \caption{}
\end{figure}

Após adicionar o e-mail ao marcador definido, você terá acesso àquele e-mail em 
sua caixa de entrada ou, também, através da aba de “Marcadores”.

\begin{figure}[H]
    \centering
    \includegraphics[width=.6\textwidth]{/gmail/categorias_marcadores_filtros/Imagem13.png}
    \caption{}
\end{figure}

\subsection{Como gerenciar marcadores}
Para gerenciar os marcadores, você pode ir até a aba esquerda do “Menu 
Principal” e selecionar a opção “Mais”.\newline

Selecione a opção “Gerenciar Marcadores”

\begin{figure}[H]
    \centering
    \includegraphics[width=.6\textwidth]{/gmail/categorias_marcadores_filtros/Imagem14.png}
    \caption{}
\end{figure}

Será exibida uma tela que apresenta alguns marcadores-padrão definidos pelo 
sistema. Os marcadores criados pelo usuário serão exibidos na parte inferior 
desta tela. Com isso, você tem a possibilidade de “Criar novo marcador”, 
“mostrar”, “ocultar”, “remover” ou “editar” cada um dos marcadores existentes.

\begin{figure}[H]
    \centering
    \includegraphics[width=.6\textwidth]{/gmail/categorias_marcadores_filtros/Imagem15.png}
    \caption{}
\end{figure}

\section{Filtrar e-mails}
Com a ferramenta para filtro de e-mails, você pode encontrar e-mails enviados 
para alguém, e-mails enviados a você, ou até definir critérios para procura de 
e-mails de forma fácil e rápida. Para acessá-la, você deve clicar sobre o ícone 
“Mostrar opções de pesquisa”, que se encontra ao lado da barra de busca 
“Pesquisar e-mail”, na parte superior da tela em sua caixa de entrada.

\begin{figure}[H]
    \centering
    \includegraphics[width=.6\textwidth]{/gmail/categorias_marcadores_filtros/Imagem16.png}
    \caption{}
\end{figure}

\subsection{Tipos de filtros e suas respectivas funcionalidades}
\begin{itemize}
    \item “De” --- Com este filtro você pode procurar e-mails que uma pessoa 
    específica enviou para você.
    \item “Para” --- Com ele você procura e-mails que enviou para um destinatário 
    específico.
    \item “Assunto“ ---  Este filtro permite que você encontre e-mails através da 
    busca pelo campo “Assunto” da mensagem (o popular “título do e-mail”).
    \item “Contém as palavras” --- Com ele você pode adicionar uma ou mais 
    palavras que estejam no corpo do texto do(s) e-mail(s) procurado(s).                            % chktex 36
    \item “Não tem” --- Com ele você pode adicionar uma ou mais palavras as quais 
    não gostaria que estivessem incluídas no corpo do texto dos e-mails 
    procurados.
    \item “Tamanho” --- Com este filtro você encontra emails por espaço de 
    armazenamento ocupado, podendo, inclusive, definir a busca por e-mails 
    menores ou maiores que a quantidade especificada. Também é possível 
    selecionar se a medição será realizada em MB (megabytes), KB (kilobytes) ou 
    bytes.
    \item “Data entre” --- Pode utilizá-lo para definir datas iniciais e finais de 
    envio para procura de um e-mail, estipulando um intervalo de tempo ou um dia 
    específico.
    \item “Pesquisar” --- Com ele você pode definir um parâmetro específico para 
    procurar um e-mail, como, por exemplo, filtrar se ele encontra-se na caixa 
    de entrada, se possui algum filtro ou marcador específico, ou se está na 
    lixeira.
    \item Campo “Com anexo” --- Com esta opção selecionada, serão exibidos apenas 
    os e-mails que possuem ao menos um arquivo anexo a eles.
\end{itemize}

\begin{figure}[H]
    \centering
    \includegraphics[width=.6\textwidth]{/gmail/categorias_marcadores_filtros/Imagem17.png}
    \caption{}
\end{figure}

Com uma ou mais opções de busca definidas, você poderá utilizar a função “Criar 
um filtro”, que exibirá uma série de ações que podem ser atribuídas às mensagens 
encontradas com os parâmetros definidos. 

\begin{figure}[H]
    \centering
    \includegraphics[width=.6\textwidth]{/gmail/categorias_marcadores_filtros/Imagem18.png}
    \caption{}
\end{figure}

\begin{figure}[H]
    \centering
    \includegraphics[width=.6\textwidth]{/gmail/categorias_marcadores_filtros/Imagem19.png}
    \caption{}
\end{figure}

\subsubsection{As ações disponíveis são}
\begin{itemize}
    \item “Ignorar a caixa de entrada (arquivar)” --- Os emails irão diretamente 
        a aba “Arquivar”, ou seja, não irão mais aparecer na caixa de entrada.
    \item “Marcar como lida” --- As mensagens serão marcadas automaticamente 
        como “lidas”.
    \item “Marcar como estrela” --- Os e-mails serão marcados automaticamente 
        “com estrela”.
    \item “Aplicar o marcador:” --- Com ele você pode definir um marcador para 
        atribuir aos emails.
    \item “Encaminhar” --- Você poderá adicionar um email para que os 
        respectivos e-mails filtrados sejam encaminhados automaticamente.
    \item “Excluir” --- Os emails buscados serão excluídos automaticamente.
    \item “Nunca enviar para Spam” --- Os emails não serão atribuídos a aba 
        “Spam”.
    \item “Sempre marcar como importante” --- Irá marcá-los automaticamente como 
        “importante”.
    \item “Nunca marcar como importante” --- Os emails em questão não serão 
        atribuídos à aba “importante”.
    \item “Categorizar como:” --- Permite atribuir categorias automaticamente 
        aos e-mails buscados, sendo elas: “Principal”, “Social”, “Atualizações”, 
        “Fóruns” e “Promoções”.
    \item “Também aplicar filtros a conversas correspondentes” --- Irá aplicar 
        aos emails já existentes a opção desejada.
\end{itemize}

Por fim, basta clicar sobre o botão “Criar filtro”

\begin{figure}[H]
    \centering
    \includegraphics[width=.6\textwidth]{/gmail/categorias_marcadores_filtros/Imagem20.png}
    \caption{}
\end{figure}

Para apagar um dos filtros criados, é necessário entrar nas “Configurações” 
novamente.\newline

Em seguida, selecione a opção “Mostrar todas as configurações”.

\begin{figure}[H]
    \centering
    \includegraphics[width=.6\textwidth]{/gmail/categorias_marcadores_filtros/Imagem3.png}
    \caption{}
\end{figure}

Clique sobre a aba “Filtros e endereços bloqueados”

\begin{figure}[H]
    \centering
    \includegraphics[width=.6\textwidth]{/gmail/categorias_marcadores_filtros/Imagem21.png}
    \caption{}
\end{figure}

Nesta aba serão exibidas opções para criar novos filtros, bem como serão 
mostrados todos os filtros já existentes, que podem ser removidos ou editados.

\begin{figure}[H]
    \centering
    \includegraphics[width=.6\textwidth]{/gmail/categorias_marcadores_filtros/Imagem22.png}
    \caption{}
\end{figure}