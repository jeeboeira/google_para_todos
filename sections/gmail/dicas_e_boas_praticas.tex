% sections/gmail/dicas_e_boas_praticas.tex
% !TeX root = ../../main.tex

\section{Dicas e Boas Práticas do Gmail}

\begin{itemize}
    \item \textbf{Evite realizar operações de arquivar, excluir, aplicar etiquetas ou mover 
    muitos e-mails ao mesmo tempo}, pois isso pode causar instabilidade no sistema e 
    tornar o processo mais lento, o recomendável é realizar essas ações com menos de 
    mil mensagens por vez.
    \item \textbf{Utilize a pesquisa e filtros do Gmail}. Com apenas algumas palavras-chave é 
    possível encontrar e-mails específicos em toda sua conta, e com o auxílio dos filtros pode-se 
    classificar e-mails por remetente, assunto, aplicar etiquetas, marcar como lido 
    e encaminhar para outra conta. Isso torna seu uso mais eficiente.
    \item Para \textbf{reduzir a quantidade de mensagens indesejadas}, como \gls{spam} ou promoções que 
    você não deseja mais receber, basta abrir o e-mail e clicar na opção \textbf{“Cancelar 
    inscrição”}. Essa ação interrompe o envio de novas mensagens daquele remetente, 
    ajudando a manter a caixa de entrada mais limpa.
    \item O Gmail permite que o usuário gerencie \textbf{várias contas em um único lugar}, podendo 
    alternar entre elas sem a necessidade de sair e entrar novamente.
    \item Ativar as configurações de \textbf{resposta automática} ou \textbf{modelos de mensagens} é útil 
    para enviar mensagens frequentes com respostas semelhantes, isso ajuda a 
    economizar tempo e padronizar a comunicação.
    \item Para manter sua conta segura é recomendável aplicar a \textbf{autenticação em duas 
    etapas}, adicionando mais uma camada de proteção no \gls{login}.
    \item É importante saber \textbf{identificar sinais de \gls{phishing}}, como remetentes estranhos, 
    links suspeitos ou erros ortográficos. Nunca clique em links ou baixe anexos de 
    remetentes desconhecidos sem verificar a autenticidade.
    \item O Gmail permite configurar \textbf{assinaturas personalizadas}, que são inseridas ao 
    final do e-mail enviado, tornando suas mensagens mais profissionais.
    \item O Gmail é integrado ao \textbf{Google Agenda}, permitindo que o usuário crie eventos 
    diretamente a partir de um e-mail recebido. Basta clicar nos três pontos no 
    canto superior da mensagem e selecionar “Criar evento”. Isso ajuda a organizar 
    reuniões, prazos e lembretes sem sair da caixa de entrada.
\end{itemize}

\subsection{Comandos}
Ao ativar a opção de atalhos de teclado nas configurações do Gmail, é possível 
executar tarefas de forma muito mais rápida. Esses comandos ajudam a agilizar 
ações como abrir a caixa de entrada, escrever novos e-mails, arquivar ou excluir 
mensagens e navegar entre diferentes seções.\newline

\textbf{Principais atalhos:}\newline
\begin{itemize}
    \item \tecla{c} → Compor novo e-mail;
    \item \tecla{d} → Abrir composição em nova janela;
    \item \tecla{/} → Colocar o \gls{cursor} na \gls{barrapesquisa};
    \item \tecla{e} → Arquivar e-mail selecionado;
    \item \tecla{\#} → Excluir e-mail selecionado;
    \item \tecla{r} → Responder e-mail;
    \item \tecla{a} → Responder a todos;
    \item \tecla{f} → Encaminhar e-mail;
    \item \tecla{Shift} + \tecla{i} → Marcar como lido;
    \item \tecla{Shift} + \tecla{u} → Marcar como não lido;
    \item \tecla{g} + \tecla{i} → Ir para a caixa de entrada;
    \item \tecla{g} + \tecla{s} → Ir para e-mails com estrela;
    \item \tecla{g} + \tecla{t} → Ir para e-mails enviados;
    \item \tecla{*} + \tecla{a} → Selecionar todos os e-mails;
    \item \tecla{*} + \tecla{n} → Desmarcar todos os e-mails;
    \item \tecla{x} → Selecionar conversa atual;
    \item \tecla{z} → Desfazer última ação;
    \item \tecla{k} → Ir para a conversa mais nova;
    \item \tecla{j} → Ir para a conversa mais antiga;
    \item \tecla{o} ou \tecla{Enter} → Abrir conversa selecionada;
    \item \tecla{n} → Ir para a próxima mensagem em uma conversa;
    \item \tecla{p} → Voltar para a mensagem anterior em uma conversa;
    \item \tecla{s} → Marcar ou desmarcar estrela na conversa;
    \item \tecla{y} → Remover a conversa da visualização atual;
    \item \tecla{Shift} + \tecla{t} → Adicionar conversa as tarefas;
    \item \tecla{Shift} + \tecla{\#} → Mover conversa para a lixeira;
    \item \tecla{Ctrl} + \tecla{i} → Colocar texto em itálico;
    \item \tecla{Ctrl} + \tecla{b} → Colocar texto em negrito;
    \item \tecla{Ctrl} + \tecla{u} → Colocar texto Sublinhado;
    \item \tecla{Alt} + \tecla{Shift} + \tecla{5} → Colocar texto riscado;\newline
\end{itemize}




