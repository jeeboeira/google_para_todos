% sections/gmail/dicas_e_boas_praticas.tex
% !TeX root = ../../main.tex

\section{Dicas e Boas Práticas do Gmail}
Evite realizar operações de arquivar, excluir, aplicar etiquetas ou mover 
muitos e-mails ao mesmo tempo, pois isso pode causar instabilidade no sistema e 
tornar o processo mais lento, o recomendável é realizar essas ações com menos de 
mil mensagens por vez.\newline

Utilize a pesquisa e filtros do Gmail. Com apenas algumas palavras-chave é 
possível encontrar e-mails específicos em toda sua conta, e com o auxílio dos filtros pode-se classificar e-mails por remetente, assunto, aplicar etiquetas, marcar 
como lido e encaminhar para outra conta. Isso torna seu uso mais 
eficiente.

Para reduzir a quantidade de mensagens indesejadas, como spam ou promoções que 
você não deseja mais receber, basta abrir o e-mail e clicar na opção “Cancelar 
inscrição”. Essa ação interrompe o envio de novas mensagens daquele remetente, 
ajudando a manter a caixa de entrada mais limpa.\newline

O Gmail permite que o usuário gerencie várias contas em um único lugar, podendo 
alternar entre elas sem a necessidade de sair e entrar novamente.\newline

Ativar as configurações de resposta automática ou modelos de mensagens é útil 
para enviar mensagens frequentes com respostas semelhantes, isso ajuda a 
economizar tempo e padronizar a comunicação.\newline

Para manter sua conta segura é recomendável aplicar a autenticação em duas 
etapas, adicionando mais uma camada de proteção no login.\newline

É importante saber identificar sinais de phishing, como remetentes estranhos, 
links suspeitos ou erros ortográficos. Nunca clique em links ou baixe anexos de 
remetentes desconhecidos sem verificar a autenticidade.

O Gmail permite configurar assinaturas personalizadas, que são inseridas ao 
final do e-mail enviado, tornando suas mensagens mais profissionais.\newline

O Gmail é integrado ao Google Agenda, permitindo que o usuário crie eventos 
diretamente a partir de um e-mail recebido. Basta clicar nos três pontos no 
canto superior da mensagem e selecionar “Criar evento”. Isso ajuda a organizar 
reuniões, prazos e lembretes sem sair da caixa de entrada.\newline


\subsection{Comandos}
Ao ativar a opção de atalhos de teclado nas configurações do Gmail, é possível 
executar tarefas de forma muito mais rápida. Esses comandos ajudam a agilizar 
ações como abrir a caixa de entrada, escrever novos e-mails, arquivar ou excluir 
mensagens e navegar entre diferentes seções.

\newpage

\textbf{Principais atalhos:}\newline
\begin{itemize}
    \item \tecla{c} → Compor novo e-mail;\newline
    \item \tecla{d} → Abrir composição em nova janela;\newline
    \item \tecla{/} → Colocar o cursor na barra de pesquisa;\newline
    \item \tecla{e} → Arquivar e-mail selecionado;\newline
    \item \tecla{\#} → Excluir e-mail selecionado;\newline
    \item \tecla{r} → Responder e-mail;\newline
    \item \tecla{a} → Responder a todos;\newline
    \item \tecla{f} → Encaminhar e-mail;\newline
    \item \tecla{Shift} + \tecla{i} → Marcar como lido;\newline
    \item \tecla{Shift} + \tecla{u} → Marcar como não lido;\newline
    \item \tecla{g} + \tecla{i} → Ir para a caixa de entrada;\newline
    \item \tecla{g} + \tecla{s} → Ir para e-mails com estrela;\newline
    \item \tecla{g} + \tecla{t} → Ir para e-mails enviados;\newline
    \item \tecla{*} + \tecla{a} → Selecionar todos os e-mails;\newline
    \item \tecla{*} + \tecla{n} → Desmarcar todos os e-mails;\newline
    \item \tecla{x} → Selecionar conversa atual;\newline
    \item \tecla{z} → Desfazer última ação;\newline
    \item \tecla{k} → Ir para a conversa mais nova;\newline
    \item \tecla{j} → Ir para a conversa mais antiga;\newline
    \item \tecla{o} ou \tecla{Enter} → Abrir conversa selecionada;\newline
    \item \tecla{n} → Ir para a próxima mensagem em uma conversa;\newline
    \item \tecla{p} → Voltar para a mensagem anterior em uma conversa;\newline
    \item \tecla{s} → Marcar ou desmarcar estrela na conversa;\newline
    \item \tecla{y} → Remover a conversa da visualização atual;\newline
    \item \tecla{Shift} + \tecla{t} → Adicionar conversa as tarefas;\newline
    \item \tecla{Shift} + \tecla{\#} → Mover conversa para a lixeira;\newline
    \item \tecla{Ctrl} + \tecla{i} → Colocar texto em itálico;\newline
    \item \tecla{Ctrl} + \tecla{b} → Colocar texto em negrito;\newline
    \item \tecla{Ctrl} + \tecla{u} → Colocar texto Sublinhado;\newline
    \item \tecla{Alt} + \tecla{Shift} + \tecla{5} → Colocar texto riscado;\newline
\end{itemize}




