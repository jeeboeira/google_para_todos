% sections/gmail/exclusao_reg_leitura_adiar.tex
% !TeX root = ../../../main.tex


\section{Como Excluir e Recuperar e-mails excluídos}
Este guia irá demonstrar o processo completo para manter uma caixa de entrada 
limpa. Para isso, vamos explorar as funções de excluir, recuperar e excluir 
definitivamente e-mails indesejados.


\subsubsection{Excluindo um e-mail}
Existem duas maneiras possíveis de excluir um e-mail indesejado: diretamente 
pela caixa de entrada ou estando com a mensagem aberta. Para excluir a partir da 
caixa de entrada, basta marcar a caixa de seleção ao lado esquerdo do e-mail --- 
podem ser marcadas mais de uma caixa de seleção simultaneamente --- em seguida, 
um menu de ações aparecerá no topo da página. Apenas clique no botão com o ícone 
de \textbf{lixeira}.

\begin{figure}[H]
    \centering
    \includegraphics[width=.39\textwidth]{/gmail/exclusao_reg_leitura_adiar/Imagem1.png}
    \caption{}
\end{figure}

Para excluir o e-mail com a mensagem aberta, basta clicar no botão com o ícone 
de \textbf{lixeira}.

\begin{figure}[H]
    \centering
    \includegraphics[width=.39\textwidth]{/gmail/exclusao_reg_leitura_adiar/Imagem2.png}
    \caption{}
\end{figure}


\subsubsection{Gerenciando a Lixeira}
Se você acabou excluindo um e-mail por engano, saiba que ele pode ser 
recuperado. O Gmail mantém os e-mails que você enviou para a lixeira guardados 
por 30 dias em caso de engano. Após esse tempo, serão excluídos definitivamente.
Para encontrar os seus e-mails excluídos, clique na pasta “Lixeira”, localizada 
no menu à esquerda. Dentro da lixeira, marque a caixa de seleção do e-mail que 
deseja restaurar. O menu de ações aparecerá novamente no topo da página, clique 
no ícone representado por uma \textbf{pasta}, nomeado como “Mover para”, selecione a 
caixa de entrada ou a pasta correspondente para a qual deseja enviar o e-mail.

\begin{figure}[H]
    \centering
    \includegraphics[width=.39\textwidth]{/gmail/exclusao_reg_leitura_adiar/Imagem3.png}
    \caption{}
\end{figure}

Após essas ações, o e-mail retornará imediatamente ao local selecionado. 
Agora, nos casos em que você deseja excluir o e-mail permanentemente e liberar espaço de 
armazenamento, na pasta “Lixeira” selecione a mensagem desejada e 
clique no botão “Excluir definitivamente”. Lembrando, essa ação não pode ser 
desfeita, então certifique-se de que é o e-mail correto ou que realmente deseja 
realizar essa ação.

\begin{figure}[H]
    \centering
    \includegraphics[width=.39\textwidth]{/gmail/exclusao_reg_leitura_adiar/Imagem4.png}
    \caption{}
\end{figure}

Para você limpar a lixeira, basta clicar no botão “esvaziar a lixeira agora”, 
assim todos os e-mails que estão na lixeira serão excluídos definitivamente.

\begin{figure}[H]
    \centering
    \includegraphics[width=.39\textwidth]{/gmail/exclusao_reg_leitura_adiar/Imagem5.png}
    \caption{}
\end{figure}


\section{Registro de leitura}
O Gmail utiliza um sistema visual simples para ajudar a diferenciar as 
mensagens que já foram lidas daquelas que ainda precisam de atenção. Os e-mails 
não lidos aparecem em \textbf{negrito}. Para alternar entre os status lido e não 
lido, basta visualizar a mensagem ou alterar manualmente, clicando na caixa de 
seleção e no botão com ícone de \textbf{carta}, localizado na barra de ações que 
será aberta.

\begin{figure}[H]
    \centering
    \includegraphics[width=.39\textwidth]{/gmail/exclusao_reg_leitura_adiar/Imagem6.png}
    \caption{}
\end{figure}


\section{Adiar e-mails}
A funcionalidade “Adiar” é uma ferramenta muito útil para quem costuma receber 
e-mails importantes em momentos inoportunos. Essa funcionalidade remove a 
mensagem temporariamente da sua caixa de entrada e a faz retornar como uma 
nova mensagem na data e hora que você escolher.


\subsection{Adiando um e-mail}
Estando na caixa de entrada, selecione o e-mail que deseja adiar clicando na 
caixa de seleção. No menu superior que aparecerá, clique no ícone de 
\textbf{relógio}. Uma janela irá abrir solicitando até quando você deseja adiar 
este e-mail, selecione o tempo desejado para o retorno do e-mail.

\begin{figure}[H]
    \centering
    \includegraphics[width=.39\textwidth]{/gmail/exclusao_reg_leitura_adiar/Imagem7.png}
    \caption{}
\end{figure}


\subsection{Gerenciando e-mails adiados}
Assim que você adiar um e-mail, ele ficará guardado na pasta “Adiados”, 
localizada no menu lateral esquerdo. Caso você decida responder à 
mensagem antes do previsto, ou adiar ainda mais o e-mail, apenas 
selecione o e-mail a partir da caixa de seleção, no menu, e clique novamente no 
ícone de relógio. Uma nova janela irá abrir, basta clicar na opção “retomar” ou 
selecionar novamente o tempo desejado.

\begin{figure}[H]
    \centering
    \includegraphics[width=.39\textwidth]{/gmail/exclusao_reg_leitura_adiar/Imagem8.png}
    \caption{}
\end{figure}