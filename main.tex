% !TeX program = lualatex
% !TeX root = main.tex
% !TeX encoding = UTF-8

\documentclass[12pt,a4paper, oneside]{report}

% ---- Detecção do idioma
\usepackage[brazil]{babel}

% ---- Tipografia e fontes
\usepackage{fontspec}
\defaultfontfeatures{Ligatures=TeX, Scale=MatchLowercase}
\setmainfont{Poppins}[
  Path=fonts/Poppins/,
  Extension=.ttf,
  UprightFont=*-Regular,
  BoldFont=*-Bold,
  ItalicFont=*-Italic,
  BoldItalicFont=*-BoldItalic
]
% ---- Use fallback for missing glyphs (like arrows)
\usepackage{newunicodechar}
\newunicodechar{→}{\ensuremath{\rightarrow}} % seta direita
%\newunicodechar{←}{\ensuremath{\leftarrow}}
%\newunicodechar{↔}{\ensuremath{\leftrightarrow}}
%\newunicodechar{⇒}{\ensuremath{\Rightarrow}}
%\newunicodechar{⇔}{\ensuremath{\Leftrightarrow}}



% ---- Layout e utilidades
\usepackage{geometry}
\geometry{margin=2.2cm}
\usepackage{graphicx}
\graphicspath{{images/}} % Pasta padrão das figuras
\usepackage[labelfont=bf,labelsep=space]{caption} % Formato da legenda
\usepackage{xcolor}
\usepackage{enumitem}
\setlist{noitemsep, topsep=2pt, leftmargin=*}
\usepackage{booktabs}
\usepackage{array}
\usepackage{titlesec}
\usepackage{tcolorbox}
\tcbuselibrary{skins,breakable}
\usepackage{hyperref}
\usepackage{siunitx}
\sisetup{
  %locale = DE,               % vírgula decimal e ponto como separador de milhar
  space-before-unit = false, % "87%" em vez de "87 %"
  number-unit-product = {},  % remove o espaço entre número e unidade
}
\usepackage{titlesec}
\usepackage{float}
\usepackage[backend=biber,style=authoryear]{biblatex}

% ---- Remove o rótulo do capítulo na página (fica só o título "Introdução")
\titleformat{\chapter}[hang]
  {\bfseries\Huge} % estilo do título
  {\thechapter.}               % <-- sem rótulo ("Capítulo 2")
  {0.37em}
  {}

% Ajustar espaçamentos acima/abaixo do título do capítulo
\titlespacing*{\chapter}{0pt}{-1ex}{2ex}

% ---- Metadados da apostila
\newcommand{\titulo}{Google para Todos}
\newcommand{\autor}{Turma Extensão III — ADS/IFRS - Campus: Farroupilha (2025/2)}
\newcommand{\versao}{v1.0}
\newcommand{\data}{2025/2}

% ---- Cores (paleta inspirada no Google)
\definecolor{dicas}{HTML}{4285F4}% azul (nome mantido para compatibilidade com o código abaixo)
\definecolor{alerta}{HTML}{DB4437}
\definecolor{conhecimentoExtra}{HTML}{F4B400}
\definecolor{formulas}{HTML}{0F9D58}
\definecolor{primary}{HTML}{020202}
% Correções: cores referenciadas mas não definidas
\definecolor{ggray}{HTML}{5F6368}
\definecolor{ggreen}{HTML}{0F9D58}
\definecolor{gyellow}{HTML}{F4B400}
\definecolor{gred}{HTML}{DB4437}

% ---- Hyperlinks

\newif\ifprint

% Toggle: Mudar entre perfil de tela e impressão
\printfalse              % tela (padrão)
% \printtrue             % impressão

\ifprint
  % ---- Perfil IMPRESSÃO (B/W, sem cores/caixas)
  \hypersetup{
    hidelinks,                % links ativos, sem cor/caixa
    pdfauthor=\autor,
    pdftitle=\titulo,
    pdfstartview=FitH
  }
\else
  % ---- Perfil TELA (links coloridos, navegação)
  \hypersetup{
    colorlinks=true,
    allcolors=primary,        % usa a cor 'primary' para tudo
    linktoc=all,              % todo item do sumário vira link
    breaklinks=true,          % quebra de linha em URLs longas
    pdfauthor=\autor,
    pdftitle=\titulo,
    pdfpagemode=UseOutlines   % abre com bookmarks
  }
\fi

% ---- Títulos
\titleformat{\section}{\Large\bfseries\color{primary}}{\thesection}{0.6em}{}
\titleformat{\subsection}{\large\bfseries\color{ggray}}{\thesubsection}{0.6em}{}
\titleformat{\subsubsection}{\bfseries}{\thesubsubsection}{0.6em}{}

% ---- Caixas reutilizáveis
\newtcolorbox{objetivos}[1][]{
  breakable, colback=primary!5, colframe=primary, title=Objetivos, fonttitle=\bfseries, #1}
\newtcolorbox{passos}[1][]{
  breakable, colback=ggreen!5, colframe=formulas, title=Passo a passo, fonttitle=\bfseries, #1}
\newtcolorbox{dica}[1][]{
  breakable, colback=gyellow!10, colframe=conhecimentoExtra, title=Dica, fonttitle=\bfseries, #1}
\newtcolorbox{atencao}[1][]{
  breakable, colback=gred!5, colframe=alerta, title=Atenção, fonttitle=\bfseries, #1}
\newtcolorbox{checagem}[1][]{
  breakable, colback=white, colframe=dicas, title=Checklist de domínio, fonttitle=\bfseries, #1}

 % ---- Macros úteis
\newcommand{\tecla}[1]{\fbox{#1}}
\newcommand{\produto}[1]{\textbf{#1}} % destacar nome de ferramenta


\begin{document}

% !TeX root = ../main.tex
% ---- Capa simples
\begin{titlepage}
  \thispagestyle{empty} % esconde o número da página
  \centering
  {\LARGE\bfseries \titulo\par}
  {\large \versao\par}
  {\large \autor\par}
  {\small Licença: lorem ipsum\par}
  {\small Última atualização: \today}
\end{titlepage}


% (opcional) pré-texto em romanos
\pagenumbering{roman}
\tableofcontents
\clearpage
% corpo em arábicos a partir daqui
\pagenumbering{arabic}

% sections/exemplo/exemplo.tex
% !TeX root = ../main.tex

\chapter{Modelo de Capítulo — Lorem Ipsum}
\label{chap:lorem}

\begin{objetivos}
Ao final, você será capaz de:
\begin{itemize}
  \item Descrever o propósito do módulo;
  \item Executar um fluxo básico de tarefa;
  \item Identificar e resolver problemas comuns.
\end{itemize}
\end{objetivos}

\section{O que é e por que usar}
\produto{Lorem ipsum} dolor sit amet, consectetur adipiscing elit. Integer
aliquet, mauris non feugiat porta, ante massa gravida nibh, in venenatis lorem
nibh non dolor. Use \tecla{Ctrl+F} para buscar rapidamente no documento.

\section{Conceitos-chave}
\begin{itemize}
  \item Termo A — definição curta e objetiva;
  \item Termo B — quando usar e limitações;
  \item Termo C — relação com os demais conceitos.
\end{itemize}

\section{Passo a passo essencial}
\begin{passos}
\begin{enumerate}
  \item Acesse o sistema e faça login.
  \item Crie um recurso \emph{white} e nomeie segundo o padrão:
        \texttt{AAAA-MM-DD\_Projeto\_Descrição}.
  \item Realize a ação principal e valide o resultado esperado.
  \item Compartilhe com permissões mínimas necessárias.
\end{enumerate}
\end{passos}

\begin{dica}
Use a busca avançada para localizar rapidamente itens por tipo, proprietário e
data de modificação.
\end{dica}

\section{Exemplo de figura}
\begin{figure}[!ht]
  \centering
  % Placeholder de imagem (substitua por \includegraphics[width=.9\textwidth]{figs/...})
  \rule{0.9\textwidth}{6cm}
  \caption{Área reservada para imagem de exemplo.}
  \label{fig:exemplo}
\end{figure}

Como mostrado na \autoref{fig:exemplo}, mantenha as capturas com boa legibilidade.

\section{Exemplo de tabela}
\begin{table}[!ht]
\centering
\begin{tabular}{@{}ll@{}}
\toprule
Recurso & Descrição \\
\midrule
Item A  & Explicação resumida do item A. \\
Item B  & Explicação resumida do item B. \\
\bottomrule
\end{tabular}
\caption{Tabela de exemplo com \texttt{booktabs}.}
\label{tab:exemplo}
\end{table}

Consulte a \autoref{tab:exemplo} para o resumo dos itens.

\section{Atalhos e truques úteis}
\begin{itemize}
  \item \tecla{Ctrl+K} → inserir link;
  \item \tecla{Ctrl+Shift+C} → copiar formatação (exemplo);
  \item Buscas salvas para reutilizar filtros frequentes.
\end{itemize}

\section{Problemas comuns e soluções rápidas}
\begin{atencao}
\textbf{Não consigo compartilhar:} confirme o e-mail e o papel do usuário.\\
\textbf{Arquivo muito grande:} compacte ou use upload via aplicativo de desktop.\\
\textbf{Conflitos de edição:} use comentários e modo de \emph{sugestões}.
\end{atencao}

\section{Checklist de domínio}
\begin{checagem}
Marque mentalmente o que você faz sem consultar:
\begin{itemize}[leftmargin=*]
  \item \texttt{[ ]} Acessa e navega pelas áreas principais;
  \item \texttt{[ ]} Executa a tarefa essencial do módulo;
  \item \texttt{[ ]} Configura compartilhamento/segurança corretamente;
  \item \texttt{[ ]} Resolve um problema comum.
\end{itemize}
\end{checagem}


%\chapter{Introdução}
%% sections/introducao/introducao.tex
% !TeX root = ../main.tex

\section{Apresentação da apostila}

Olá, seja bem-vindo(a) à apostila “Google Para Todos”! Neste material, 
você irá encontrar um passo-a-passo completo para aprender a utilizar as 
principais ferramentas online e gratuitas do Google Workspace. Além de 
operar essas funcionalidades, você terá a habilidade de aplicá-las em 
cenários do cotidiano que, cada vez mais, estão imersos na tecnologia, 
como trabalho, estudos e até projetos pessoais. Este é um guia feito para 
facilitar seu aprendizado e ajudar você a utilizar, de forma confiante, 
a tecnologia no dia a dia.

Para contextualizar o tema desta apostila, apresentamos alguns dados de 
pesquisas sobre o cenário tecnológico no Brasil: 

\begin{itemize}
    \item Nos últimos anos, segundo o Movimento Brasil Competitivo (MBC; FGV, 2022), 
    as ocupações profissionais relacionadas às atividades digitais apresentaram 
    um crescimento de \SI{4.9}{\percent} em relação às demais ocupações, o que torna evidente 
    a importância do conhecimento tecnológico e digital para o mercado;
    
    \item A Pesquisa Nacional por Amostra de Domicílios Contínua – PNAD Contínua (2022),
    também apresentou a relevância das ferramentas digitais ao indicar que 7,4 
    milhões de brasileiros estavam trabalhando por meios remotos (home office) 
    no ano de 2022.
    
    \item Segundo a Pesquisa de Inovação Semestral 2022 do IBGE (2023), a computação em 
    nuvem é a tecnologia avançada mais adotada entre as empresas brasileiras, 
    sendo que 87\% delas possuem 500 ou mais empregados, 76,8\% de 250 a 499 
    empregados e 68\% de 100 a 249 empregados. Isso revela que, tecnologias em 
    nuvem, como o Google Workspace, são as mais utilizadas no mercado de trabalho, 
    visto que são soluções acessíveis, colaborativas e produtivas. 
\end{itemize}
    

%\chapter{Google Drive}
%\input{sections/google_drive/google_drive}

\chapter{Documentos Google}
% sections/documentos/documentos.tex
% !TeX root = ../main.tex

\section{Introdução ao Documentos Google}
O Google Documentos é uma ferramenta de escrita e edição de texto produzida e 
mantida pela Google, seu uso é online e gratuito à todos os usuários de contas 
Google. O acesso à plataforma é realizado por qualquer navegador (Google Chrome,
 Microsoft Edge, Safari, etc…) através do link 
 \hyperlink{https://docs.google.com/}{https://docs.google.com/}, você precisará 
realizar o login na sua conta Google e já terá acesso à página com alguns 
modelos de documentos pré-prontos. 

\begin{figure}[H]
    \centering
    \includegraphics[width=.9\textwidth]{/documentos/Imagem 9.png}
    \caption{Ao clicar em “Documento em branco”, você será direcionado à um novo 
        arquivo vazio.}
\end{figure}

\begin{figure}[H]
    \centering
    \includegraphics[width=.9\textwidth]{/documentos/Imagem 10.png}
    \caption{}
\end{figure}


A seguir, serão apresentados alguns elementos disponíveis na tela para 
contextualizar algumas funcionalidades básicas do Google Documentos:

\subsubsection{Título do arquivo}
O título do arquivo fica no canto superior esquerdo da página. Para renomeá-lo é 
só clicar no campo e escrever o novo título.

\subsubsection{Menus}
Os menus ficam localizados logo abaixo do título do arquivo. As opções são: 
“Arquivo”, “Editar”, “Ver”, “Inserir”, “Formatar”, “Ferramentas”, “Extensões” e 
“Ajuda”.  Ao clicar nos menus, outros submenus irão aparecer.

\subsubsection{Barra de ferramentas}
A barra de ferramentas fica logo abaixo dos menus. Ao longo da apostila, as 
funcionalidades dos itens da barra de ferramentas serão explicados.


\begin{figure}[H]
    \centering
    \includegraphics[width=.9\textwidth]{/documentos/Imagem 11.png}
    \caption{}
\end{figure}


\subsubsection{Página}
A página do arquivo é o grande campo branco, nele você conseguirá construir o 
seu arquivo. Ela fica no centro da tela, é só clicar com o mouse e começar a 
escrever.

\subsubsection{Ajustar o zoom}
Caso a página esteja muito pequena ou muito grande em sua tela, é possível 
ajustar o zoom seguindo os passos:

\begin{enumerate}
    \item Clicar na opção “Zoom” da barra de ferramentas (É um menu com o texto 
    “100\%”);
    \item Selecionar a opção desejada. Aumente a porcentagem para aumentar o 
    zoom e vice versa.
\end{enumerate}

\begin{dica}
Ajustar o zoom aumenta a aparência das letras no seu computador, mas mantém o tamanho correto para a impressão.
\end{dica}


\subsubsection{Réguas}
As réguas ficam na lateral esquerda e superior da tela.

\begin{checagem}[title=Para exibir as réguas]
    Caso você não esteja vendo as réguas da página, você pode seguir os passos abaixo:
    \begin{enumerate}[leftmargin=*]
      \item Clicar sobre o menu “Ver”;
      \item Clicar sobre a opção “Exibir régua”;
    \end{enumerate}
\end{checagem}

Para exibir as réguas
Caso você não esteja vendo as réguas da página, você pode seguir os passos 
abaixo:
Clicar sobre o menu “Ver”;
Clicar sobre a opção “Exibir régua”;


\subsubsection{Ajustar as margens da página}
Ao mexer a seta azul da direita, você ajusta o tamanho da margem direita, ou 
seja, até onde o texto chega na página.

Ao clicar à esquerda da seta da esquerda, você consegue mover a margem esquerda 
da página inteira.

Ao mexer a seta azul da esquerda, você consegue ajustar a margem esquerda da 
linha que está com o cursor.

Ao mexer somente a barra horizontal azul, você passa para um ajuste mais fino, 
você ajusta a margem esquerda da segunda linha de texto, permitindo que você 
diferencie o começo dos parágrafos, por exemplo.

\begin{figure}[H]
    \centering
    \includegraphics[width=.9\textwidth]{/documentos/Imagem 12.png}
    \caption{}
\end{figure}


% sections/documentos/menu_inserir.tex
% !TeX root = ../main.tex]

\section{Menu Inserir}


\subsection{Imagens}
Para inserir imagens no arquivo, tudo depende de onde que a imagem está, quando 
trabalhando com imagens que estão na internet ou no computador muitas vezes 
copiar e colar pode ser o suficiente. Porém, quando essa opção não funciona, 
podemos utilizar o menu “Inserir”.

\begin{figure}[H]
    \centering
    \includegraphics[width=.9\textwidth]{/documentos/Imagem 1.png}
    \caption{}
\end{figure}


\subsubsection{Fazer upload do computador}
\begin{enumerate}
    \item Selecionar a opção correspondente no menu “Inserir”;
    \item Localizar a imagem no seu computador;
    \item Clicar em “Abrir”;
\end{enumerate}


\subsubsection{Pesquisar na Web}
\begin{enumerate}
    \item Selecionar a opção correspondente no menu “Inserir”;
    \item Utilizar palavras-chave para buscar a imagem desejada;
    \item Clicar na imagem escolhida;
    \item Clicar no botão “Inserir” que vai aparecer no canto inferior direito;
\end{enumerate}


\subsubsection{Google Drive}
\begin{enumerate}
    \item Selecionar a opção correspondente no menu “Inserir”;
    \item Localizar a imagem no seu drive;
    \item Clicar no botão “Inserir” que vai aparecer no canto inferior direito;
\end{enumerate}


\subsubsection{Google Fotos}
\begin{enumerate}
    \item Selecionar a opção correspondente no menu “Inserir”;
    \item Localizar a imagem no seu Google Fotos;
    \item Clicar no botão “Inserir” que vai aparecer no canto inferior direito;
\end{enumerate}


\subsubsection{Câmera}
\begin{enumerate}
    \item Essa opção poderá ser utilizada quando o aparelho possuir uma câmera.
    \item Selecionar a opção correspondente no menu “Inserir”;
    \item Caso apareça a opção no seu navegador, permitir que o mesmo utilize a 
    câmera para tirar a foto dessa vez;
    \item Capture a foto e clique no botão “Inserir”;
\end{enumerate}


\subsubsection{Por URL}
\begin{enumerate}
    \item Selecionar a opção correspondente no menu “Inserir”;
    \item Colar o endereço da imagem;
    \item Clicar no botão “Inserir” que vai aparecer no canto inferior direito;
\end{enumerate}

\begin{dica}
Caso o endereço que você copiou esteja dando erro, verifique que você selecionou 
a opção “Copiar endereço da imagem” para copiar o endereço correto. 
\end{dica}


\subsubsection{Para ajustar o tamanho da imagem:}
\begin{enumerate}
    \item Selecionar a imagem;
    \item Pressionar os oito quadrados azuis nos vértices e arestas da imagem, 
    e arrastar;
\end{enumerate}

\begin{figure}[H]
    \centering
    \includegraphics[width=.9\textwidth]{/documentos/Imagem 2.png}
    \caption{}
\end{figure}


\subsubsection{Para rotacionar a imagem:}
\begin{enumerate}
    \item Selecionar a imagem;
    \item Pressionar e arrastar o círculo azul que aparece acima da imagem;
\end{enumerate}


\subsubsection{Para cortar a imagem:}
\begin{enumerate}
    \item Clicar com o botão direito na imagem;
    \item Selecionar a opção “Cortar imagem”;
    \item As bordas pretas que aparecem são as margens da imagem, é possível 
        ajustar tanto o tamanho da imagem, como o de suas bordas;
    \item Clique fora da imagem para completar o ajuste;
\end{enumerate}

\begin{figure}[H]
    \centering
    \includegraphics[width=.8\textwidth]{/documentos/Imagem 3.png}
    \caption{}
\end{figure}


\subsubsection{Para remover as alterações de corte na imagem:}
\begin{enumerate}
    \item Clicar com o botão direito na imagem;
    \item Selecionar a opção “Redefinir imagem”;
\end{enumerate}


\subsubsection{Para adicionar texto alternativo:}
\begin{enumerate}
    \item Clicar com o botão direito na imagem;
    \item Selecionar a opção “Texto alternativo”;
    \item Escrever o texto na aba lateral direita que foi aberta;
\end{enumerate}


\subsubsection{Para acessar o mais opções de imagem:}
\begin{enumerate}
    \item Clicar com o botão direito na imagem;
    \item Selecionar a opção “Opções de imagem”;
    \item Abre um menu lateral com mais opções, possui descrição na própria 
        ferramenta;
\end{enumerate}

\begin{figure}[H]
    \centering
    \includegraphics[width=.8\textwidth]{/documentos/Imagem 4.png}
    \caption{}
\end{figure}

\begin{dica}
O texto alternativo serve para acessibilidade, se o seu documento será consumido 
por uma pessoa com deficiência visual que utiliza descrição de áudio, é 
interessante adicionar uma descrição da imagem para ajudar com contextualização.
\end{dica}


\subsection{Tabelas}
As tabelas podem ser adicionadas no documento para ajudar a organizar a informação. 

\subsubsection{Para realizar a inserção de uma tabela no documento:}
Acessar o menu Inserir > Tabela;
Selecionar o tamanho da tabela desejado no esquema de grade;

DICA: Os números abaixo de esquema de grade se refere ao número de [colunas] x [linhas]

[IMAGEM 5]

DICA: Sob o menu “Elemento Básicos”, a ferramenta disponibiliza também alguns modelos prontos. O usuário é incentivado a investigar esses modelos e utilizá-los em parte ou integralmente.

\subsubsection{Para acessar os menus de coluna e linha:}
Ao passar o mouse pelas laterais esquerda e superior da tabela, os menus irão aparecer;

[IMAGEM 6]

\subsubsection{Para inserir uma coluna ou linha:}
É possível adicionar uma coluna ou linha clicando no símbolo de mais [+] nos menus de coluna e linha. A linha será adicionada abaixo da linha que tem o menu aberto. A coluna será adicionada à direita da que tem o menu aberto.

\subsubsection{Para trocar a ordem das colunas e linhas:}
É possível trocar a posição das colunas e linhas ao clicar e segurar o símbolo de seis pontos cinza claro que aparece nos menus de coluna e linha.

\subsubsection{Para fixar e desafixar o cabeçalho da tabela:}
Ao clicar no botão com ícone de tachinha no menu da linha, é possível fixar e desafixar o cabeçalho. É possível fixar mais de uma linha.

DICA: Fixar o cabeçalho, significa que, em tabelas que ocupam mais de uma página, as linhas fixadas vão aparecer no topo de todas as páginas.

DICA: O cabeçalho vai estar fixado quando o ícone de tachinha com a barra transversal estiver aparecendo. De forma a simbolizar que a operação de desafixação vai ser realizada ao clicar novamente.

\subsubsection{Para ordenar a tabela:}
No menu da coluna que se deseja ordenar, clicar no botão com ícone de três barras paralelas;
Selecionar se a ordem deve ser crescente ou decrescente.

DICA: Ordenar uma coluna mantém os valores da linha agrupados. Não edita a tabela, apenas organiza as informações.

\subsubsection{Para alterar as bordas da tabela:}
Ao passar o mouse por uma célula, aparece um botão com ícone de seta no canto superior direito. 
Ao clicar no botão é possível alterar a borda de uma célula ou grupo de células;

DICA: Para ajustes de cor e tamanho da borda, acessar mais configurações da tabela, e abrir o menu “Opções da tabela”. As opções estarão disponíveis sob o título “Cor”.

\subsubsection{Para acessar mais configurações da tabela:}
O menu oferece opções mais específicas de inserção e deleção de linhas e colunas. Assim como as funções também disponibilizadas nos menus de linha e coluna.
Clicar com o botão direito na tabela. Algumas operações são realizadas com relação à célula da tabela que foi clicada.

[IMAGEM 7]

\subsubsection{Descrição das outras opções do menu:}
Inserir linha de título: insere uma linha com uma única célula sobre a tabela;
Dividir célula: permite criar subdivisões de colunas e linhas em uma célula, o texto permanece na célula superior à esquerda.
Mesclar células: aparece na posição do botão anterior, quando mais de uma célula é selecionada.
Desfazer mesclagem de células: aparece no menu após uma mesclagem, reverte a operação;
Definir tipo de coluna: permite que o usuário defina um formato de informação, todos os valores naquela coluna devem ser preenchidos nesse formato.
Distribuir linhas e Distribuir colunas: Após ajustar o tamanho lateral e vertical total desejado, você pode utilizar essas opções para distribuir os espaços da tabela igualmente. É possível também selecionar um grupo de linhas ou colunas para distribuir o espaço.
Opções da tabela: Abre um menu lateral com mais opções, das quais a maioria está bem descrita dentro da própria ferramenta, descrições pertinentes a seguir;
Tabela > Estilo: a primeira opção mantém a tabela como único elemento em uma linha, o texto fica acima e abaixo da tabela. A segunda opção permite que o texto também apareça ao lado da tabela quando a tabela for menos larga que a página.

[IMAGEM 8]

Descrição imagens:
Ressaltar o menu Inserir > Imagem, eu pensei em um quadrado ao redor
 
Ressaltar as opções para imagem (segundo bloco)
Uma seta apontando para o menu lateral
Ressaltar o menu Inserir > Tabela
Duas setas apontando para os menus de linha e coluna
Ressaltar as opções para tabela (de inserir até opções de tabela)
Uma seta apontando para o menu lateral



\end{document}