% !TeX program = lualatex
% !TeX root = main.tex
% !TeX encoding = UTF-8

\documentclass[12pt,a4paper, oneside]{report}

% ---- Detecção do idioma
\usepackage[brazil]{babel}

% ---- Tipografia e fontes
\usepackage{fontspec}
\defaultfontfeatures{Ligatures=TeX, Scale=MatchLowercase}
\setmainfont{Poppins}[
  Path=fonts/Poppins/,
  Extension=.ttf,
  UprightFont=*-Regular,
  BoldFont=*-Bold,
  ItalicFont=*-Italic,
  BoldItalicFont=*-BoldItalic
]
% ---- Use fallback for missing glyphs (like arrows)
\usepackage{newunicodechar}
\usepackage{menukeys}
\newunicodechar{→}{\ensuremath{\rightarrow}} % seta direita
%\newunicodechar{←}{\ensuremath{\leftarrow}}
%\newunicodechar{↔}{\ensuremath{\leftrightarrow}}
%\newunicodechar{⇒}{\ensuremath{\Rightarrow}}
%\newunicodechar{⇔}{\ensuremath{\Leftrightarrow}}

% ---- Evitar avisos de underfull/overfull hbox
\hbadness=10000   % não reporta Underfull \hbox

% ---- Layout e utilidades
\usepackage{geometry}
\geometry{margin=2.2cm}
\usepackage{graphicx}
\graphicspath{{images/}} % Pasta padrão das figuras
\usepackage[labelfont=bf,labelsep=space]{caption} % Formato da legenda
\usepackage{xcolor}
\usepackage{enumitem}
\setlist{noitemsep, topsep=2pt, leftmargin=*}
\usepackage{booktabs}
\usepackage{array}
\usepackage{titlesec}
\usepackage{tcolorbox}
\tcbuselibrary{skins,breakable}
\usepackage{hyperref}
\usepackage{siunitx}
\sisetup{
  %locale = DE,               % vírgula decimal e ponto como separador de milhar
  space-before-unit = false, % "87%" em vez de "87 %"
  number-unit-product = {},  % remove o espaço entre número e unidade
}
\usepackage{titlesec}
\usepackage{float}
\usepackage{csquotes}
\usepackage[backend=biber,style=authoryear]{biblatex}
\usepackage{etoolbox} 

% Numerar até o nível \paragraph
\setcounter{secnumdepth}{3} % numera até \paragraph
\setcounter{tocdepth}{2} 	% aparece no sumário

% Formatação para \paragraph como “subsubsubsection”
\titleformat{\paragraph}[block]{\normalsize\bfseries}{\theparagraph}{1em}{}
\titlespacing*{\paragraph}{0pt}{1ex}{1ex}

\renewcommand{\topfraction}{0.9}
\renewcommand{\textfraction}{0.1}
\renewcommand{\floatpagefraction}{0.9}

% Ajustar espaçamentos acima/abaixo do título do capítulo
\titlespacing*{\chapter}{0pt}{-1ex}{2ex}

% ---- Metadados da apostila
\newcommand{\titulo}{Google para Todos}
\newcommand{\autor}{Turma Extensão III — ADS/IFRS — Campus: Farroupilha (2025/2)}
\newcommand{\versao}{v1.0}
\newcommand{\data}{2025/2}

% ---- Cores (paleta inspirada no Google)
\definecolor{dicas}{HTML}{4285F4}% azul (nome mantido para compatibilidade com o código abaixo)
\definecolor{alerta}{HTML}{DB4437}
\definecolor{conhecimentoExtra}{HTML}{F4B400}
\definecolor{formulas}{HTML}{0F9D58}
\definecolor{primary}{HTML}{020202}
% Correções: cores referenciadas mas não definidas
\definecolor{ggray}{HTML}{5F6368}
\definecolor{ggreen}{HTML}{0F9D58}
\definecolor{gyellow}{HTML}{F4B400}
\definecolor{gred}{HTML}{DB4437}

% ---- Cor dos capítulos
\definecolor{chaptercolor}{HTML}{2E86C1}
\definecolor{chaptercolor1}{HTML}{2E86C1} % Azul
\definecolor{chaptercolor2}{HTML}{28B463} % Verde
\definecolor{chaptercolor3}{HTML}{CA6F1E} % Laranja
\definecolor{chaptercolor4}{HTML}{884EA0} % Roxo

\newcommand{\getchaptercolor}{%
	\ifcase\value{chapter}%
	\color{chaptercolor1}% (0 - TOC, etc.)
	\or{}
	\colorlet{chaptercolor}{chaptercolor1}% 1
	\or{}
	\colorlet{chaptercolor}{chaptercolor2}% 2
	\or{}
	\colorlet{chaptercolor}{chaptercolor3}% 3
	\or{}
	\colorlet{chaptercolor}{chaptercolor4}% 4
	\else
	\colorlet{chaptercolor}{chaptercolor1}% fallback
	\fi
}
% ---- Hyperlinks

\newif\ifprint{}

% Toggle: Mudar entre perfil de tela e impressão
\printfalse{}            % tela (padrão)
% \printtrue             % impressão

\ifprint{}
  % ---- Perfil IMPRESSÃO (B/W, sem cores/caixas)
  \hypersetup{
    hidelinks,                % links ativos, sem cor/caixa
    pdfauthor=\autor,
    pdftitle=\titulo,
    pdfstartview=FitH
  }
\else
  % ---- Perfil TELA (links coloridos, navegação)
  \hypersetup{
    colorlinks=true,
    allcolors=primary,        % usa a cor 'primary' para tudo
    linktoc=all,              % todo item do sumário vira link
    breaklinks=true,          % quebra de linha em URLs longas
    pdfauthor=\autor,
    pdftitle=\titulo,
    pdfpagemode=UseOutlines   % abre com bookmarks
  }
\fi


% ---- Títulos
\titleformat{\section}{\Large\bfseries\color{primary}}{\thesection}{0.6em}{}
\titleformat{\subsection}{\large\bfseries\color{ggray}}{\thesubsection}{0.6em}{}
\titleformat{\subsubsection}{\bfseries}{\thesubsubsection}{0.6em}{}

% ---- Caixas reutilizáveis
\newtcolorbox{objetivos}[1][]{
  breakable, colback=primary!5, colframe=primary, title=Objetivos, fonttitle=\bfseries, #1}
\newtcolorbox{passos}[1][]{
  breakable, colback=ggreen!5, colframe=formulas, title=Passo a passo, fonttitle=\bfseries, #1}
\newtcolorbox{dica}[1][]{
  breakable, colback=gyellow!10, colframe=conhecimentoExtra, title=Dica, fonttitle=\bfseries, #1}
\newtcolorbox{atencao}[1][]{
  breakable, colback=gred!5, colframe=alerta, title=Atenção, fonttitle=\bfseries, #1}
\newtcolorbox{checagem}[1][]{
  breakable, colback=white, colframe=dicas, title=Checklist de domínio, fonttitle=\bfseries, #1}

 % ---- Macros úteis
\newcommand{\tecla}[1]{\fbox{#1}}
\newcommand{\produto}[1]{\textbf{#1}} % destacar nome de ferramenta

\newcommand{\ChapterTitleFormat}[1]{
	\getchaptercolor{}
	\ifcsdef{@toclevel@chapter}{
		#1
	}{
		% ---- Cria a faixa colorida ---
		\begin{tikzpicture}[remember picture,overlay]
			\fill[chaptercolor!90!black]
			(current page.north west)
			rectangle
			([yshift=-1.5cm]current page.north east); % --- Controla o tamanho dela
		\end{tikzpicture}
		% --- Controla o quão pra cima o título do capítulo deve estar
		\vspace*{-2.6cm}
		
		\tcbox[
		colback=white,
		colframe=chaptercolor!90!black,
		boxrule=0.8pt,
		arc=8pt, % -- Controla a curvatura das bordas
		left=8pt,right=8pt,top=6pt,bottom=3pt, % --- Controla o padding interno
		boxsep=0pt,
		enhanced,
		]{
			\ifnum\value{chapter}>0
				{\huge\textcolor{chaptercolor!90!black}{\bfseries \thechapter. #1}}% -- Com númeral
			\else
				{\huge\textcolor{chaptercolor!90!black}{\bfseries #1}}% --- Sem númeral
			\fi
		}
		% --- Controla o quão pra cima o texto do elemento seguinte deve estar
		\vspace{-1.2cm}
	}
}

\titleformat{\chapter}[block]
{\large\bfseries\color{black}} % Formatação base
{} % label
{0pt} % sep
{\ChapterTitleFormat}

\newdimen\trianglesize{}
\trianglesize=2cm % <- change this to make the triangle larger/smaller

\AddToHook{shipout/background}{%
	\ifnum\value{chapter}>0
	\getchaptercolor{}
	\begin{tikzpicture}[remember picture,overlay]
		
    	\ifodd\value{page}
		% === Triangulo direito ===
			\fill[chaptercolor] 
			([xshift=0pt,yshift=-0.6\trianglesize]current page.north east) -- % base on right edge (inside)
			([xshift=-0.6\trianglesize,yshift=0pt]current page.north east) -- % base on top edge (inside)
			([xshift= 0.8\trianglesize,yshift= 0.8\trianglesize]current page.north east) -- % apex OUTSIDE
			cycle;
			
			\draw[line width=0.12\trianglesize, chaptercolor]
			([xshift=0cm]current page.north east) --
			([xshift=0cm]current page.south east);
		\else
		% === Triangulo esquerdo ===
			\fill[chaptercolor]
			([xshift=0pt,yshift=-0.6\trianglesize]current page.north west) --
			([xshift=0.6\trianglesize,yshift=0pt]current page.north west) --
			([xshift=-0.8\trianglesize,yshift= 0.8\trianglesize]current page.north west) --
			cycle;
			
			\draw[line width=0.12\trianglesize, chaptercolor]
			([xshift=0cm]current page.north west) --
			([xshift=0cm]current page.south west);
		\fi
		
		

	\end{tikzpicture}%
	\fi
}

\addbibresource{sections/motor_de_busca/Motor de Busca.bib}
\addbibresource{sections/google_drive/google_drive.bib}

\newcommand{\codeblock}[1]{%
	\vspace{1em}%
	\begin{tcolorbox}[colback=gray!5,colframe=gray!40,boxrule=0.4pt,arc=8pt,enhanced jigsaw,]
		\begin{center}
			\texttt{#1}
		\end{center}
	\end{tcolorbox}
	\vspace{1em}%
}

\newcommand{\hseparator}{
	\vspace{1em}
	\noindent\textcolor{gray}{\rule{\linewidth}{1pt}}
	\vspace{1em}
}

\begin{document}

% !TeX root = ../main.tex
% ---- Capa simples
\begin{titlepage}
  \thispagestyle{empty} % esconde o número da página
  \centering
  {\LARGE\bfseries \titulo\par}
  {\large \versao\par}
  {\large \autor\par}
  {\small Licença: lorem ipsum\par}
  {\small Última atualização: \today}
\end{titlepage}


% (opcional) pré-texto em romanos
\pagenumbering{roman}
\tableofcontents
\clearpage
% corpo em arábicos a partir daqui
\pagenumbering{arabic}

% sections/exemplo/exemplo.tex
% !TeX root = ../main.tex

\chapter{Modelo de Capítulo — Lorem Ipsum}
\label{chap:lorem}

\begin{objetivos}
Ao final, você será capaz de:
\begin{itemize}
  \item Descrever o propósito do módulo;
  \item Executar um fluxo básico de tarefa;
  \item Identificar e resolver problemas comuns.
\end{itemize}
\end{objetivos}

\section{O que é e por que usar}
\produto{Lorem ipsum} dolor sit amet, consectetur adipiscing elit. Integer
aliquet, mauris non feugiat porta, ante massa gravida nibh, in venenatis lorem
nibh non dolor. Use \tecla{Ctrl+F} para buscar rapidamente no documento.

\section{Conceitos-chave}
\begin{itemize}
  \item Termo A — definição curta e objetiva;
  \item Termo B — quando usar e limitações;
  \item Termo C — relação com os demais conceitos.
\end{itemize}

\section{Passo a passo essencial}
\begin{passos}
\begin{enumerate}
  \item Acesse o sistema e faça login.
  \item Crie um recurso \emph{white} e nomeie segundo o padrão:
        \texttt{AAAA-MM-DD\_Projeto\_Descrição}.
  \item Realize a ação principal e valide o resultado esperado.
  \item Compartilhe com permissões mínimas necessárias.
\end{enumerate}
\end{passos}

\begin{dica}
Use a busca avançada para localizar rapidamente itens por tipo, proprietário e
data de modificação.
\end{dica}

\section{Exemplo de figura}
\begin{figure}[!ht]
  \centering
  % Placeholder de imagem (substitua por \includegraphics[width=.9\textwidth]{figs/...})
  \rule{0.9\textwidth}{6cm}
  \caption{Área reservada para imagem de exemplo.}
  \label{fig:exemplo}
\end{figure}

Como mostrado na \autoref{fig:exemplo}, mantenha as capturas com boa legibilidade.

\section{Exemplo de tabela}
\begin{table}[!ht]
\centering
\begin{tabular}{@{}ll@{}}
\toprule
Recurso & Descrição \\
\midrule
Item A  & Explicação resumida do item A. \\
Item B  & Explicação resumida do item B. \\
\bottomrule
\end{tabular}
\caption{Tabela de exemplo com \texttt{booktabs}.}
\label{tab:exemplo}
\end{table}

Consulte a \autoref{tab:exemplo} para o resumo dos itens.

\section{Atalhos e truques úteis}
\begin{itemize}
  \item \tecla{Ctrl+K} → inserir link;
  \item \tecla{Ctrl+Shift+C} → copiar formatação (exemplo);
  \item Buscas salvas para reutilizar filtros frequentes.
\end{itemize}

\section{Problemas comuns e soluções rápidas}
\begin{atencao}
\textbf{Não consigo compartilhar:} confirme o e-mail e o papel do usuário.\\
\textbf{Arquivo muito grande:} compacte ou use upload via aplicativo de desktop.\\
\textbf{Conflitos de edição:} use comentários e modo de \emph{sugestões}.
\end{atencao}

\section{Checklist de domínio}
\begin{checagem}
Marque mentalmente o que você faz sem consultar:
\begin{itemize}[leftmargin=*]
  \item \texttt{[ ]} Acessa e navega pelas áreas principais;
  \item \texttt{[ ]} Executa a tarefa essencial do módulo;
  \item \texttt{[ ]} Configura compartilhamento/segurança corretamente;
  \item \texttt{[ ]} Resolve um problema comum.
\end{itemize}
\end{checagem}


%\chapter{Introdução}
%% sections/introducao/introducao.tex
% !TeX root = ../main.tex

\section{Apresentação da apostila}

Olá, seja bem-vindo(a) à apostila “Google Para Todos”! Neste material, 
você irá encontrar um passo-a-passo completo para aprender a utilizar as 
principais ferramentas online e gratuitas do Google Workspace. Além de 
operar essas funcionalidades, você terá a habilidade de aplicá-las em 
cenários do cotidiano que, cada vez mais, estão imersos na tecnologia, 
como trabalho, estudos e até projetos pessoais. Este é um guia feito para 
facilitar seu aprendizado e ajudar você a utilizar, de forma confiante, 
a tecnologia no dia a dia.

Para contextualizar o tema desta apostila, apresentamos alguns dados de 
pesquisas sobre o cenário tecnológico no Brasil: 

\begin{itemize}
    \item Nos últimos anos, segundo o Movimento Brasil Competitivo (MBC; FGV, 2022), 
    as ocupações profissionais relacionadas às atividades digitais apresentaram 
    um crescimento de \SI{4.9}{\percent} em relação às demais ocupações, o que torna evidente 
    a importância do conhecimento tecnológico e digital para o mercado;
    
    \item A Pesquisa Nacional por Amostra de Domicílios Contínua – PNAD Contínua (2022),
    também apresentou a relevância das ferramentas digitais ao indicar que 7,4 
    milhões de brasileiros estavam trabalhando por meios remotos (home office) 
    no ano de 2022.
    
    \item Segundo a Pesquisa de Inovação Semestral 2022 do IBGE (2023), a computação em 
    nuvem é a tecnologia avançada mais adotada entre as empresas brasileiras, 
    sendo que 87\% delas possuem 500 ou mais empregados, 76,8\% de 250 a 499 
    empregados e 68\% de 100 a 249 empregados. Isso revela que, tecnologias em 
    nuvem, como o Google Workspace, são as mais utilizadas no mercado de trabalho, 
    visto que são soluções acessíveis, colaborativas e produtivas. 
\end{itemize}
    

%\chapter{Google Drive}
%% sections/google_drive/introducao.tex
% !TeX root = ../../../main.tex

\section{Introdução ao Google Drive}
O Google Drive, ou apenas Drive, é uma ferramenta de armazenamento e 
compartilhamento de arquivos na \gls{nuvem} desenvolvida pela Google. Apresentado pela 
primeira vez em 2012, ele permite ao usuário salvar arquivos de qualquer tipo 
(documentos, imagens, vídeos, etc) e também acessá-los de qualquer lugar.

Com 15GB (gigabytes) de armazenamento em sua versão gratuita (que são compartilhados com 
Gmail e Google Fotos), possui planos de assinatura com valores variados que permitem ao 
usuário ampliar a capacidade máxima de armazenamento em nuvem.

É disponibilizado não somente como um site na web, mas também como uma 
aplicação para \textit{\gls{desktop}} (Windows e MacOS) e um aplicativo \gls{mobile} 
(iOS e Android).

\section{\gls{homepage} e menu lateral}
Após fazer o \gls{login} em sua conta Google para entrar no Drive, você verá a 
seguinte página:

\begin{figure}[htbp]
    \centering
    \includegraphics[width=1\textwidth]{/google_drive/imagem_5.png}
    \caption{Captura de tela exibindo a \gls{homepage} do Google Drive.}
\end{figure}
	
Da direita para a esquerda, na parte superior da aplicação, pode-se ver:

\begin{itemize}
    \item Ícone de \textbf{Perfil do Usuário}: ao clicar sobre este botão, serão exibidas algumas informações 
    da sua conta Google, contas alternativas para mudar o perfil ativo e botões para adicionar uma nova conta, sair da atual ou de todas;
    \item Ícone do \textbf{Google Apps} ou "quadrado de pontinhos": ao ser clicado, abre um atalho para os outros \gls{aplicativos} da Google, facitando o acesso;
    \item Botão de \textbf{Configurações} ou "engrenagem": ao clicar nele, abre um menu com opções de configurações do Drive, como temas, gerenciamento de 
    aplicativos conectados, configurações gerais e de notificação;
    \item Botão de \textbf{Suporte}: abre uma janela com algumas opções de ajuda e diretrizes do Drive;
    \item \textbf{\gls{barrapesquisa}}, com \textbf{ícone de pesquisa avançada} ao lado direito (similar a três barras de volume), que permite a
    aplicação de filtros diversos para pesquisas de arquivos e diretórios mais específicos em seu Drive.
    \item Pode-se encontrar, ainda, um botão em formato de estrela do \textbf{Gemini}, a inteligência artificial da Google, caso o mesmo esteja ativo na conta.
\end{itemize}

Já, no lado esquerdo da tela, pode-se observar:

\begin{itemize}
    \item Botão "\textbf{Novo}": utilizado para a criação de pastas, \gls{upload} de arquivos/pastas e criação de documentos com os aplicativos Google (Documentos, Planilhas, Apresentações, etc);
    \item Botão "\textbf{Pessoal}": direciona o usuário para a \gls{homepage} (página principal) do Drive;
    \item Botão "\textbf{Atividades}": presente somente em contas com assinatura Google \gls{workspace}. Mostra um registro do que foi feito e por quem foi feito, a respeito de alterações, permissões, etc.;
    \item Botão "\textbf{Espaços de trabalho}": presente somente em contas Google \gls{workspace}, permite montar agrupamentos de arquivos/pastas sem alterar a localização real deles;
    \item Botão "\textbf{Computadores}": não presente na imagem, referente ao \gls{backup} e sincronização de arquivos do computador na versão \gls{desktop} do Google Drive; 
    \item Botão "\textbf{Meu Drive}": mostra o seu Drive, com todas as pastas e arquivos carregados;
    \item Botão "\textbf{Drives compartilhados}": presente somente em contas Google \gls{workspace}. Trata-se de um drive comum para membros de um grupo ou organização. Não pertence a um único indivíduo;
    \item Botão "\textbf{Compartilhados comigo}": lista todos os arquivos e pastas que foram compartilhadas com o usuário;
    \item Botão "\textbf{Recentes}": mostra pastas e arquivos recentemente acessados em seu Drive;
    \item Botão "\textbf{Com estrela}": mostra os arquivos marcados com uma estrela;
    
    \begin{dica}
        Você pode adicionar um arquivo ou pasta aos favoritos (com estrela) clicando com o botão direito do mouse sobre o item desejado e selecionando a opção 'Organizar' e 'Adicionar a "Com Estrela"'. 
        Pode-se utilizar, ainda, o atalho \tecla{Ctrl} + \tecla{Alt} + \tecla{S}. Essa prática facilita o acesso rápido a arquivos importantes. 
    \end{dica}

    \item Botão "\textbf{\gls{spam}}": mostra os arquivos classificados como "\gls{spam}". Tratam-se de arquivos com conteúdos indesejados ou potencialmente maliciosos. 
    O Drive realiza essa classificação automaticamente.;
    \item Botão "\textbf{Lixeira}": lista os arquivos excluídos pelo usuário. 
    Para mais informações sobre como excluir arquivos, leia a seção \textbf{3.7 Excluir arquivos}. Estes arquivos serão excluídos permanentemente após 30 dias na lixeira;
    \item Botão "\textbf{Armazenamento}": mostra uma relação com a distribuição de uso do seu armazenamento em nuvem entre os \gls{aplicativos} Google.
\end{itemize}

Também é possível ver, ao lado direito, uma \gls{aba} com complementos da Google, por 
\textit{\gls{default}} (padrão): Agenda, Contatos, \gls{keep} e Tarefas, com um botão de "cruz" (Instalar complementos) para 
adicionar novas funcionalidades. Esta \gls{aba} pode ser minimizada e estendida ao 
clicar no botão em forma de seta na parte inferior da guia.

\section{Upload de arquivos}
Para fazer o \gls{upload} de pastas ou arquivos, basta clicar no botão "Novo", localizado no menu lateral. Ao fazer isso, aparecerá um menu que permite ao usuário enviar arquivos, pastas, criar uma pasta no seu ambiente do drive, ou então, criar arquivos dos \gls{aplicativos} padrões da Google, bem como de outros apps que já tenham sido vinculados à sua conta.

\begin{figure}[H]
    \centering
    \includegraphics[width=.55\textwidth]{/google_drive/imagem_1.png}
    \caption{Captura e tela do menu de \gls{upload} do Google Drive.}
\end{figure}

Basta, então, selecionar o \gls{arquivo} que deseja "subir" ao seu armazenamento em nuvem e esperar o tempo necessário para o \gls{upload}. Caso o usuário utilize o Google Chrome ou Firefox como navegador, também é possível fazer o envio de arquivos e pastas arrastando seus ícones, com o cursor do mouse, do local de origem para o ambiente do Drive, e os soltando. Ainda, é possível subir arquivos por meio da versão para \textit{\gls{desktop}} do Google Drive.

\section{Criar pastas}
O uso de pastas é essencial para a melhor utilização do sistema, pois “[...] a 
organização das informações permite que o processo de tomada de decisões seja 
eficaz e rápido, impedindo barreiras e atrasos em diversos processos.”.

Por isso, é necessário saber como criá-las. Para criar uma nova pasta, clique no botão "Novo" e, no menu suspenso, selecione a opção "Nova pasta". Essa ação abre uma nova janela onde será preciso digitar um nome para pasta. Com isso feito, clique em "Criar".

\section{Abrir arquivos}
Para abrir um \gls{arquivo} do Google Drive, basta clicar sobre seu ícone duas vezes. O mesmo vale para acessar um diretório (pasta). 

“Os arquivos criados com o Google apps são abertos no 
navegador ou no app para dispositivos móveis. Outros tipos de \gls{arquivo} na sua pasta do Drive são abertos nos apps correspondentes. 
Por exemplo, o Adobe Reader para arquivos \gls{pdf}.''

\section{Mover arquivos}
Para mover um ou mais arquivo(s) ou pasta(s), clique no botão "Mais opções" (ícone de “três pontinhos”) ou com o botão direito do mouse sobre o \gls{arquivo} que você deseja mover. 
Isto abrirá um menu com várias opções. Clique ou passe o mouse sobre a opção "Organizar" e clique em "Mover".

\begin{figure}[H]
    \centering
    \includegraphics[width=.8\textwidth]{/google_drive/imagem_2.png}
    \caption{Captura de tela do menu de ações adicionais para um \gls{arquivo} do Google Drive}
\end{figure}

Com isso, será aberta uma janela sobreposta. Nela você pode escolher o local para o qual deseja mover o \gls{arquivo}. Com o local selecionado, basta clicar no botão "Mover".

\section{Excluir arquivos}
Para excluir arquivos ou pastas, clique com o botão direito do mouse, ou clique no botão "Mais opções" (ícone de “três pontinhos”), sobre o arquivo
desejado, para abrir as opções adicionais. Escolha a opção "Mover para a lixeira" e, novamente, "Mover para a lixeira" para excluir, ou "Cancelar" para cancelar a ação. Alternativamente, o usuário pode apertar a tecla \textbf{Delete} no teclado com o \gls{arquivo} selecionado, o que também abrirá uma janela pedindo confirmação para mover para a lixeira.

“O \gls{arquivo} permanecerá na lixeira por 30 dias antes de ser excluído  automaticamente.'' Ao colocar o \gls{arquivo} na lixeira, caso ele seja de sua autoria, será excluído do seu Drive e uma cópia será feita para pessoas com as quais você o compartilhou. Caso ele seja de outra pessoa, será apenas removido do seu Drive. 

Para excluir um arquivo definitivamente, abra a lixeira clicando no botão "Lixeira" do menu lateral esquerdo. Nela, clique em "Esvaziar lixeira" para excluir definitivamente tudo que estiver lá. Se quiser excluir um item específico, clique sobre ele com o botão direito do mouse, ou então no ícone de “três pontinhos”, e selecione a opção "\textbf{Excluir definitivamente}". 
Há também a opção de "\textbf{Restaurar}" para restaurar um \gls{arquivo} a seu local de origem, ou seja, a pasta onde estava antes de ser excluído.

\begin{figure}[H]
    \centering
    \includegraphics[width=1\textwidth]{/google_drive/imagem_3.png}
    \caption{Lixeira do Drive, captura de tela/Vinícius Maurer}
\end{figure}

\section{Baixar arquivos}
Para baixar um \gls{arquivo} ou pasta, clique com o botão direito do mouse sobre ele, ou nos “três pontinhos”, e selecione "Baixar", ou então, clique diretamente no ícone de download mostrado ao selecionar ou passar o mouse por cima do \gls{arquivo}.

\section{Renomear arquivos}
Para renomear um \gls{arquivo} ou pasta, clique com o botão direito do mouse, ou nos “três pontinhos”, e selecione "Renomear". Com isso, basta digitar o novo nome e confirmar clicando em "Ok".

Outra alternativa é clicar diretamente no ícone de renomear (simbolizado por uma caneta), mostrado ao selecionar ou passar o mouse por cima do \gls{arquivo}.

\section{Marcar arquivos com estrela}
Para marcar um \gls{arquivo} ou pasta com estrela, clique com o botão direito do mouse sobre ele, ou nos “três pontinhos”, e, no menu "Organizar", selecione a opção "\textbf{Adicionar a 'Com estrela'}". Todo \gls{arquivo} que o usuário marcar "com estrela" será mostrado na \gls{aba} "Com estrela", localizada na guia lateral esquerda da tela - vide seção \textbf{3.2 Homepage e menu lateral}.

\section{Pesquisar arquivos}
Para pesquisar um \gls{arquivo} ou pasta, clique na \gls{barrapesquisa} localizada na parte superior da tela e digite o termo que deseja pesquisar. Adicionalmente, você pode usar a \textbf{pesquisa avançada}, clicando no ícone ao lado direito da \gls{barrapesquisa}. 
Com ela, você pode buscar pelas opções: "Tipo", "Proprietário", "Com as palavras", "Nome do Item", "Local", "Data de modificação", "Aprovações e assinaturas eletrônicas", "Compartilhado com" e "Acompanhamentos", especificando ainda mais o que deseja encontrar.

\begin{figure}[H]
    \centering
    \includegraphics[width=.9\textwidth]{/google_drive/imagem_4.png}
    \caption{Captura de tela de uma pesquisa avançada no Drive}
\end{figure}
%% sections/google_drive/compartilhamento_e_permissoes.tex
% !TeX root = ../../main.tex

\section{Benefícios e Riscos do Compartilhamento}
O compartilhamento de arquivos e pastas no Google Drive para computadores simplifica significativamente a colaboração. Ele permite que várias pessoas trabalhem no mesmo documento em tempo real, eliminando a necessidade de envio de arquivos por e-mail, sem ter que lidar com diferentes versões e problemas de compatibilidade. Com as permissões adequadas, se garante que as pessoas certas tenham o nível de acesso adequado, seja para visualizar, comentar ou editar o conteúdo, melhorando a produtividade e a organização do trabalho em equipe.

Apesar dos benefícios, o compartilhamento inadequado pode levar a riscos de segurança. Se não gerenciar as permissões corretamente, informações sensíveis podem ser acessadas, copiadas ou até mesmo alteradas por pessoas não autorizadas. Por exemplo, dar permissão de "Editor" a alguém que só precisa visualizar um documento pode permitir que essa pessoa compartilhe o arquivo com terceiros sem o seu conhecimento. Por isso, é fundamental definir o nível de acesso cuidadosamente.

\section{Como Compartilhar Arquivos e Pastas}

Esta seção vai te guiar pelo processo de compartilhamento de arquivos e gerenciamento de permissões usando o Google Drive. Para compartilhar um arquivo ou pasta diretamente do seu computador, siga os passos abaixo.

Primeiro, localize o arquivo ou pasta desejado dentro da sua pasta do Google Drive ou no computador caso esteja usando o aplicativo desktop. Clique com o botão direito no arquivo ou pasta e selecione a opção "Compartilhar". Dentro do menu de compartilhamento, existem duas opções:

\begin{figure}[htbp]
	\centering
	\includegraphics[width=0.5\textwidth]{/google_drive/imagem_6.png}
	\caption{Menu de compartilhamento}
\end{figure}

\begin{itemize}
	\item Compartilhamento direto (com e-mail):
	\begin{itemize}
		\item Como funciona: O usuário adiciona o endereço de e-mail de uma ou mais pessoas diretamente no menu de compartilhamento;
		\item Vantagens: Oferece maior controle sobre quem pode acessar o arquivo. Apenas as pessoas convidadas explicitamente podem ver o conteúdo. Se pode gerenciar as permissões de cada pessoa individualmente e revogar o acesso a qualquer momento;
		\item Quando usar: Ideal para documentos confidenciais ou quando se sabe exatamente com quem precisa compartilhar.
	\end{itemize}
	\item Compartilhamento com link:
	\begin{itemize}
		\item Como funciona: Gera um link para o arquivo e define o nível de acesso geral, como "Qualquer pessoa com o link" ou "Restrito";
		\item Vantagens: Extremamente prático para compartilhar com um grande número de pessoas. Não é necessário digitar e-mails e o link pode ser facilmente enviado em mensagens ou documentos;
		\item Quando usar: Ótimo para compartilhar materiais públicos, como guias de estudo, portfólios ou outros documentos que não exigem controle rigoroso. No entanto, se o link cair nas mãos erradas o acesso ao seu arquivo pode ser comprometido, dependendo do nível de acesso concedido, conforme veremos a seguir.
	\end{itemize}
\end{itemize}

Após definido o método de compartilhamento, é necessário definir o nível de acesso. Ao lado do nome de cada pessoa ou link criado, há a possibilidade de definir as permissões de acesso, através de um botão que abre um menu. As opções são:

\begin{itemize}
	\item Leitor: A pessoa pode apenas visualizar o conteúdo;
	\item Comentador: A pessoa pode visualizar e adicionar comentários;
	\item Editor: A pessoa pode editar, compartilhar com outras pessoas e fazer alterações no arquivo;
	\item Proprietário: A pessoa que criou o arquivo/pasta tem permissão total sobre o mesmo. Este nível de acesso pode ser transferido para um novo usuário.
\end{itemize}

Após selecionar o nível de acesso, finalize o processo clicando em "Salvar” para compartilhar. Caso tenha escolhido via link, não se esqueça de copiar o mesmo. Com o compartilhamento criado, existem algumas configurações adicionais que podem ser selecionadas a partir da engrenagem de configuração, presente dentro do menu de compartilhamento. São elas:

\begin{itemize}
	\item Permitir que os editores mudem as permissões e compartilhem;
	\item Pessoas que podem baixar, copiar e imprimir o conteúdo;
	\item Pessoas que comentaram e visualizaram.
\end{itemize}

Essas configurações servem para melhorar o controle de acessos e permissões, colaborando para a criação de um ambiente mais seguro para os usuários.
%% sections/google_drive/\gls{desktop}.tex
% !TeX root = ../../main.tex

\section{Diferenciais da Versão Desktop do Google Drive}

Enquanto a versão \gls{web} do Google Drive permite acessar seus arquivos de qualquer lugar 
com conexão à internet, a versão para computador se integra ao sistema do seu 
dispositivo, facilitando o uso no dia a dia. Com ela, é possível abrir, mover e editar 
arquivos diretamente pelo explorador de arquivos do seu sistema (como o Windows 
Explorer ou o Finder, no Mac), sem precisar acessar o navegador.

Em vez de abrir um navegador, fazer \gls{login} e navegar pela interface \gls{web}, o aplicativo 
\gls{desktop} permite que o usuário traga seus arquivos da \gls{nuvem} como se eles estivessem 
salvos localmente. O usuário pode arrastar, soltar, copiar e colar arquivos e pastas 
diretamente para a pasta do Google Drive em seu computador e eles serão automaticamente 
sincronizados com a \gls{nuvem}. Isso elimina a necessidade de \gls{upload} manual e torna o 
processo de salvamento de arquivos mais rápido e intuitivo.

Outro grande diferencial é o controle sobre como e onde seus arquivos são armazenados 
no seu computador. Com a versão para \gls{desktop}, o usuário pode escolher entre "stream" 
e "mirror" de arquivos:

\begin{itemize}
	\item \textbf{Stream de arquivos}: Esta opção é ideal para economizar espaço em 
	disco. Os arquivos ficam na \gls{nuvem} e são baixados para o seu computador somente 
	quando precisar abri-los.
	\item \textbf{Mirror de arquivos}: Se precisar de acesso \gls{offline} a todos os seus 
	arquivos, esta opção mantém uma cópia local e outra na \gls{nuvem}, garantindo que você 
	tenha seus documentos sempre disponíveis, mesmo sem internet.
\end{itemize}

\section{Configurando o Google Drive para Computador}
Esta seção trata-se de um guia passo a passo, do \gls{download} à configuração inicial, para 
que o usuário possa começar a usar a ferramenta de forma rápida e eficiente.

\subsection{Como Baixar a Versão para Desktop}
\begin{enumerate}
	\item Acesse o site oficial do Google Drive em https://drive.google.com/drive/home.
	\item No canto superior direito da tela, clique na engrenagem e clique na opção: 
	“Use o Drive no computador” ou acesse o \gls{link} https://workspace.google.com/products/drive/\#\gls{download}.
	\item Clique no botão para iniciar o \gls{download} do \gls{arquivo} de instalação. O sistema 
	operacional do seu computador (Windows ou Mac) será detectado automaticamente.
\end{enumerate}

\subsection{Guia de Instalação e Configuração}
Após o \gls{download}, siga as instruções simples de instalação. A configuração inicial é o 
passo mais importante para definir como o Drive funcionará no seu computador. Após a 
instalação, o aplicativo solicitará que faça \gls{login} com sua conta do Google.

%% sections/google_drive/sincronizacao_e_backup.tex
% !TeX root = ../../main.tex

\section{Backup e Sincronização}
\subsection{Backup}
O ato de realizar backups é muito importante para evitar perdas de arquivos, pois, com ele, caso algo aconteça com o original, você sempre terá uma “cópia de segurança”.

No Google Drive, há mais de uma maneira de realizá-lo. A mais simples é, simplesmente, copiar o arquivo desejado. Para isso, basta clicar nele com o botão direito,  ou nos “três pontinhos”, e depois clicar na opção “Fazer uma cópia”, gerando um arquivo idêntico ao original. Esta ação não pode ser feita em pastas.

A outra é utilizando o Google Drive para desktop. Desta forma, você pode fazer cópias dos arquivos de seu computador para o Drive,podendo acessar a cópia diretamente de seu computador, tanto no modo Stream quanto no Mirror, porém, com a diferença de que no Stream é necessária uma conexão com a internet, enquanto no mirror, não.

\subsection{Sincronização}
A sincronização de arquivos no Drive é feita por meio de sua versão desktop, permitindo que você acesse seus arquivos salvos no Drive diretamente de seu computador. Esses arquivos podem ser salvo em modo Stream (streaming) ou Mirror (espelhamento), esta diferença já foi explicada de forma resumida no capítulo sobre os diferenciais da versão desktop do Google Drive e, para melhorar o entendimento de suas diferenças, confira a tabela abaixo:

\begin{figure}[htbp]
	\centering
	\includegraphics[width=1\textwidth]{/google_drive/imagem_7.png}
	\caption{Tabela comparativa entre as opções de sincronização do Google Drive, Fonte: \cite{google2025streammirror}}
\end{figure}

Clicando no ícone de Configurações (“engrenagem”) e depois na aba ‘Preferências’. Você pode ver e alterar seu modo de sincronização ativado, clicando novamente em Configurações (dentro da janela de Preferências). É possível alterar configurações sobre o limite de uso do cache, escolher entre usar uma unidade de armazenamento ou uma pasta para a sincronização, alterar as configurações em relação ao Google Fotos, dentre outras opções.

Também é possível ver: seus arquivos disponíveis de forma off-line, lista de erros, página de ajuda, sobre, enviar feedback ou sair da conta, ao clicar no ícone de Configurações na homepage.

Já na parte esquerda da tela, é possível ver o botão ‘Abrir a pasta do Drive’ que, como o próprio nome já diz, abre a pasta correspondente a sincronização do Google Drive - isto é, o local onde você pode acessar seus arquivos do Drive no computador.

Além disso, é possível ver sua atividade de sincronização, ou seja, quais arquivos foram sincronizados, estão sendo sincronizados ou ainda não foram sincronizados. É possível também ver suas notificações. Tais como: atualizações e alterações em arquivos compartilhados.

%% sections/google_drive/integracao_ferramentas.tex
% !TeX root = ../../main.tex

\section{Integração com outras ferramentas}

O Google Drive vai muito além de ser só um “lugar para guardar arquivos”. O que realmente faz a diferença é a forma como ele se conecta com outros \gls{aplicativos}, que o transforma em uma espécie de hub central de organização, criação e colaboração. Essa flexibilidade permite simplificar tarefas, automatizar processos e agilizar a resolução de suas necessidades.

\subsection{Como conectar novos \gls{aplicativos}}
A maneira mais prática de expandir as funções do seu Drive é utilizando o Google \Gls{workspace} \Gls{marketplace}. Nele se encontram \gls{aplicativos} de diferentes áreas que podem ser integrados diretamente à sua conta. Para instalar, siga o passo a passo:

\begin{enumerate}
	\item Abra o Google Drive no navegador;
	\item Clique em “Novo” (canto superior esquerdo);
	\item Vá em “Mais” $\rightarrow$ “Conectar mais \gls{aplicativos}”;
	\item O \Gls{marketplace} será aberto.
\end{enumerate}

\begin{figure}[htbp]
	\centering
	\includegraphics[width=1\textwidth]{/google_drive/imagem_9.png}
	\caption{Menu do \Gls{marketplace}}
\end{figure}

Com o \Gls{marketplace} aberto, é só pesquisar os \gls{aplicativos} que você necessita. Quando encontrar, clique em Instalar e o app será vinculado à sua conta automaticamente.

\begin{atencao}
	Nem todos os apps são gratuitos. Alguns exigem assinatura ou pagamento único para liberar todas as funções. Então, antes de instalar, vale dar uma olhada nos detalhes para não ser “pego de surpresa”.
\end{atencao}

\section{Exemplos de integrações úteis}
\subsection{Diagramas e fluxos de trabalho}

Ferramentas como Lucidchart, draw.io e \Gls{miro} permitem criar fluxogramas, organogramas, mapas mentais e até \Gls{modeloer}. Ao integrá-los com o Drive, todos os arquivos ficam salvos automaticamente, o que facilita a colaboração em equipe e evita a dispersão de diferentes versões por diretórios desordenados.

\begin{figure}[htbp]
	\centering
	\includegraphics[width=1\textwidth]{/google_drive/imagem_13.png}
	\caption{Captura de tela da ferramenta \Gls{miro}}
\end{figure}

Exemplo Prático (\Gls{miro}): Para trazer um documento do Drive para o seu quadro do \Gls{miro}, basta criar um novo quadro em branco. Na barra de ferramentas do \Gls{miro}, há a possibilidade de selecionar a opção de inserir documentos do Google Drive. Ao utilizar o \gls{link} público do documento (configurado com a permissão correta), o \Gls{miro} exibe o conteúdo desse documento diretamente no seu quadro, facilitando as sessões de brainstorming e \gls{feedback} visual sem que o usuário precise sair da ferramenta.

\subsection{Design e edição}

Se busca uma ferramenta de design, o Canva pode ser um ótimo aliado. Ele se conecta ao Drive e permite importar imagens, editar projetos e salvar tudo na pasta certa. Para edição de PDFs e imagens, ferramentas como Lumin \gls{pdf} e Pixlr também são muito práticas: são abertas diretamente pelo Drive, fazem a edição e salvam sem precisar baixar nada no computador.

\subsection{Programação e ciência de dados}
Para quem é da área da tecnologia e programação, o Google Colab é um dos melhores exemplos de integração. Ele permite criar e compilar notebooks (anotações + blocos de código) em Python (ou outras linguagens) diretamente da \gls{nuvem}, com o armazenamento dos arquivos em seu Drive. Assim, dá para trabalhar com análise de dados, treinar modelos de \gls{machinelearning} ou colaborar em projetos de programação sem instalar nada localmente.

\begin{figure}[htbp]
	\centering
	\includegraphics[width=1\textwidth]{/google_drive/imagem_12.png}
	\caption{Captura de tela da ferramenta Google Colab}
\end{figure}

\subsection{Automação de tarefas}
Plataformas como \Gls{zapier} e \Gls{ifttt} conectam o Drive a centenas de outros \gls{aplicativos}. Podem ser criados “gatilhos” automáticos, como no exemplo prático demonstrado a seguir:
\begin{itemize}
	\item \Gls{gatilho}: um anexo chega no seu Gmail.
	\item Ação: o \gls{arquivo} é salvo automaticamente na pasta “Faturas” do Drive.
\end{itemize}
Isso economiza tempo, reduz erros e garante que ninguém fique de fora da informação.

\subsection{Assinaturas eletrônicas}
Ferramentas como \gls{docusign} e \Gls{pandadoc} também podem ser integradas ao Drive. Você manda um contrato direto da sua pasta, a pessoa assina digitalmente e a versão final já volta para o Drive. Simples e rápido, sem precisar imprimir nada.



%% sections/google_drive/colaboracao.tex
% !TeX root = ../../main.tex

\section{Segurança}

Além das permissões de compartilhamento, o Google Drive oferece outras medidas de segurança, tais como o bloqueio de arquivos, que impede a edição ou comentário em documentos por indivíduos além do proprietário, mesmo que tenham tais permissões. Contudo, colaboradores “editores” ainda podem retirar o bloqueio. Para evitar que isto ocorra, clique com o botão direito do mouse, ou nos “três pontinhos”, e vá para a opção "Informações sobre o arquivo". Depois clique em "Bloquear". Por fim, basta confirmar o bloqueio.

Pode-se verificar se o arquivo está bloqueado se, ao lado do nome do usuário, estiver presente um ícone de cadeado, como na imagem abaixo:

\begin{figure}[htbp]
	\centering
	\includegraphics[width=1\textwidth]{/google_drive/imagem_10.png}
	\caption{Arquivo bloqueado no Drive, Captura de tela/Vinícius Maurer}
\end{figure}

Para desbloquear, siga os mesmos passos citados anteriormente, porém, agora clique em "Desbloquear" e, depois, confirme a ação.

O Drive também bloqueia automaticamente arquivos detectados como \textit{malware}, \textit{spam} ou \textit{phishing}, mas não busca por estes em arquivos maiores que 100MB, algo que geralmente é exibido por meio de um aviso. Estes arquivos devem aparecer na sua aba de "Spam", presente no menu lateral esquerdo.

\begin{figure}[htbp]
	\centering
	\includegraphics[width=1\textwidth]{/google_drive/imagem_11.png}
	\caption{Aviso de arquivo acima dos limites da verificação de segurança, Captura de tela/ Vinicius Maurer}
\end{figure}

Também é possível consultar os avisos de segurança de um arquivo ao clicar nele com o botão direito - ou nos “três pontinhos” -, ir para "Informações sobre o arquivo" e clicar em "Limitações de segurança".


%\chapter{Motor de Busca}
%% !TeX root = ../main.tex
% sections/motor_de_busca/motor_de_busca.tex

\section{Introdução ao Motor de Busca}
O motor de busca do Google, conhecido mundialmente como \textbf{Google Search}, é a ferramenta central acessada ao visitar o site \href{https://google.com}{google.com}. Mais do que um simples buscador, ele se tornou sinônimo de pesquisa na internet, sendo um dos produtos digitais mais influentes e utilizados da história.

A história dessa ferramenta começa em 1995, quando Larry Page e Sergey Brin, ainda estudantes da Universidade de Stanford, criaram um projeto chamado \textbf{Backrub}. A ideia inicial era analisar a relevância de cada página com base nos links que recebia, criando assim uma forma de classificação que se tornaria o algoritmo conhecido como PageRank (\cite{pagePageRankCitationRanking1999}). Esse mecanismo revolucionou a forma de navegar na internet, tornando os resultados muito mais úteis e organizados. Pouco tempo depois, o projeto recebeu o nome de \textbf{Google}, inspirado no termo matemático “\textit{googol}” (o número 1 seguido de 100 zeros), representando a missão ambiciosa de seus criadores: “\textit{organizar a informação do mundo e torná-la universalmente acessível e útil}” (\cite{googleHowWeStarted}).

Desde então, o Google Search não apenas manteve sua relevância, mas evoluiu continuamente. Novos recursos foram incorporados ao longo dos anos, como a pesquisa por imagens, notícias, vídeos, mapas, compras e até comandos por voz. Hoje, a ferramenta consegue interpretar intenções de busca, corrigir erros ortográficos automaticamente, sugerir alternativas e até responder diretamente a perguntas simples por meio de caixas de destaque conhecidas como \textit{featured snippets} (\cite{googleHowWeStarted}).

Mesmo após quase três décadas de sua criação, o Google Search permanece indispensável para usuários de todas as idades e perfis. Segundo dados fornecidos pela empresa e apresentados por \citeauthor{srinivasanAIPersonalizationFuture2025} (\citeyear{srinivasanAIPersonalizationFuture2025}) em uma postagem no blog para anunciantes, são realizadas mais de 5 trilhões de pesquisas por ano em todo o mundo, abrangendo desde dúvidas cotidianas até pesquisas acadêmicas e profissionais. Esse alcance gigantesco mostra como a ferramenta se consolidou como um elemento essencial da vida digital.

Por isso, compreender como utilizá-la de forma estratégica é cada vez mais importante. Saber aplicar filtros, operadores de busca e recursos avançados pode facilitar o acesso a informações específicas, economizar tempo e melhorar a precisão dos resultados. Em um cenário onde a quantidade de dados disponíveis cresce, dominar o uso do Google Search é, sem dúvida, uma habilidade fundamental para navegar com eficiência no mundo conectado em que vivemos.

%% sections/motor_de_busca/motor_de_busca.tex
% !TeX root = ../../main.tex

\section{O Motor de Busca}
Ao acessar a página \href{https://google.com}{google.com} com um navegador de internet, chega-se a tela apresentada na figura \ref{fig:o_motor_de_busca1}. Ela é composta por 4 elementos principais, que seguindo a ordem apresentada na imagem são: a barra de busca, o botão para iniciar a pesquisa, o botão "Estou com sorte" e o menu de atalhos.

\begin{figure}[!ht]
	\centering
	\includegraphics[width=.9\textwidth]{images/motor_de_busca/o_motor_de_busca1.png}
	\caption{Interface do motor de busca.}
	\label{fig:o_motor_de_busca1}
\end{figure}

O elemento mais importante desta tela é a barra de busca, nela será possível digitar os termos que se quer procurar na base de dados do Google, em seguida o botão “Pesquisa Google” tem o mesmo efeito prático que clicar Enter no teclado, que será enviar a busca e retornar os dados, em seguida temos o botão “Estou com sorte”, ele ao invés de procurar e mostrar os resultados irá acessar o primeiro resultado da pesquisa diretamente e por fim há o menu de atalhos, por ali é possível acessar diversas outras ferramentas da empresa, muitas delas já apresentadas nesta apostila, como o Google Docs, Agenda, Drive, entre outros.

Além dos elementos realçados na figura \ref{fig:o_motor_de_busca1} há outros links menos utilizados, como a localização atual, dados de privacidade, referência de como funciona a pesquisa, termos de uso, configurações gerais e atalhos para o \Gls{googleads} e \Gls{googlebusiness}.

Voltando para o elemento principal, a \gls{barrapesquisa}, nela há 3 botões, sendo eles, da esquerda para a direita: o teclado virtual, o microfone para digitação por voz e um atalho para o Google Lens, que permite uma pesquisa inversa de imagem, algo que será explorado nas seções seguintes.

O uso da ferramenta é simples. Na caixa de pesquisa digita-se aquilo que quer procurar, como: “maçã”. E o Google em seguida irá mostrar os resultados para o termo pesquisado, como demonstrado na figura \ref{fig:o_motor_de_busca2}.

\begin{figure}[!ht]
	\centering
	\includegraphics[width=.9\textwidth]{images/motor_de_busca/o_motor_de_busca2.png}
	\caption{Pesquisa no motor de busca.}
	\label{fig:o_motor_de_busca2}
\end{figure}

Ao realizar uma pesquisa simples, como no exemplo dado com o termo “maçã”, o Google irá apresentar uma página de resultados organizada em diferentes blocos de informação. Estes blocos não se limitam apenas a links para outros sites, mas também podem incluir imagens, vídeos, notícias e mapas.

Além dos resultados orgânicos, isto é, aqueles que aparecem de forma natural conforme os critérios do \gls{algoritmo}, o Google também apresenta resultados patrocinados. Estes são anúncios pagos por empresas que desejam aparecer com maior destaque para determinadas palavras-chave. Normalmente, esses links são identificados pela palavra “Patrocinado” ou “Anúncio” logo abaixo do título, permitindo que o usuário saiba que se trata de publicidade.

Outro recurso importante é a seção de pesquisa relacionada, como demonstrado na figura \ref{fig:o_motor_de_busca3}, que aparece ao final da página. Ali o usuário encontra sugestões de outros termos próximos ao que ele digitou, o que pode ser útil para refinar a busca e encontrar resultados mais específicos.

\begin{figure}[!ht]
	\centering
	\includegraphics[width=.9\textwidth]{images/motor_de_busca/o_motor_de_busca3.png}
	\caption{Pesquisa no motor de busca.}
	\label{fig:o_motor_de_busca3}
\end{figure}

Na parte superior da tela de resultados, logo abaixo da barra de busca, o usuário também pode escolher filtros de pesquisa. Estes filtros permitem restringir a busca a categorias específicas, como “Imagens”, “Vídeos”, “Notícias”, “Shopping” e “Mapas”. Assim, em vez de ter acesso apenas a páginas da \gls{web}, o usuário pode rapidamente encontrar aquilo que procura em diferentes formatos.

Ainda pode se perguntar como o Google decide qual \gls{link} é mais relevante na pesquisa, para isso ele considera centenas de fatores. Entre os principais estão a relevância do conteúdo em relação ao termo pesquisado, a qualidade do site, a autoridade conquistada por meio de links de outros sites confiáveis (conhecido como \textit{PageRank}), além da experiência do usuário, como tempo de carregamento e adaptação a dispositivos móveis. Assim, quanto mais útil, confiável e bem estruturada for uma página, maiores são as chances de ela aparecer nas primeiras posições dos resultados (\cite{pagePageRankCitationRanking1999}).

Por fim, é importante notar que a experiência de busca no Google também é influenciada pela personalização dos resultados, que leva em conta fatores como a localização geográfica do usuário, o idioma configurado no navegador, o histórico de navegação e até interesses pessoais identificados por meio de pesquisas anteriores. Isso significa que duas pessoas que pesquisam o mesmo termo, em locais diferentes ou com perfis distintos, podem receber resultados diferentes. Essa personalização busca tornar a experiência mais relevante e prática, aproximando o usuário das informações que ele provavelmente considera mais úteis. 



%% sections/motor_de_busca/motor_de_busca.tex
% !TeX root = ../../main.tex

\section{Operadores}

A ferramenta permite a pesquisa digitando qualquer termo na barra de pesquisa, porém, para melhor utilizar o poder dela, é necessário aprender sobre os operadores de pesquisa. Eles são termos específicos que, ao serem digitados, permitem alterar o comportamento da ferramenta como um todo, podendo aumentar a precisão dos resultados.

O operador mais simples que se pode usar é colocar palavras entre aspas. Ao fazer isso, é possível filtrar todos os sites cujo termo aparece exatamente como escrito, evitando o uso de sinônimos ou palavras similares.

Perceba na figura \ref{fig:operadores1}, como os resultados apresentam variações da palavra “desenhada”, porém, não apresentam ela nos resultados. Já na figura \ref{fig:operadores2} é possível perceber que a palavra está presente nas páginas dos resultados.

\begin{figure}[h]
	\centering
	\includegraphics[width=.9\textwidth]{images/motor_de_busca/operadores1.png}
	\caption{Resultado da pesquisa Maçã Desenhada.}
	\label{fig:operadores1}
\end{figure}

%\vspace*{\fill}
%\newpage

\begin{figure}[h]
	\centering
	\includegraphics[width=.9\textwidth]{images/motor_de_busca/operadores2.png}
	\caption{Resultado da pesquisa Maçã “Desenhada”}
	\label{fig:operadores2}
\end{figure}


Combinado com as aspas, existem outros dois operadores importantes para filtrar ainda mais os resultados, são eles: \textbf{AND} e \textbf{OR}. Estas são palavras no inglês que significam, respectivamente, E e OU. Assim podemos criar uma pesquisa onde os resultados serão de maçãs ou bananas e estas especificamente serão desenhadas.

\begin{figure}[h]
	\centering
	\includegraphics[width=.9\textwidth]{images/motor_de_busca/operadores3.png}
	\caption{Resultado da pesquisa (Maçã OR Banana) AND “Desenhada”}
	\label{fig:operadores3}
\end{figure}


Note que na figura \ref{fig:operadores3} há a presença de outro elemento na pesquisa, os parênteses. Com eles é possível agrupar termos e pesquisas para facilitar a leitura e filtrar os resultados ainda mais. Sem a presença deles a pesquisa seria a seguinte: maçã OR banana AND “desenhada”, nela o motor iria trazer dados sobre maçãs ou bananas desenhadas, porém, com os parênteses traz resultados sobre maçãs ou bananas desenhadas.

Agora suponha que não se quer trazer os dados sobre maçãs ou bananas verdes, pois se quer buscar apenas por imagens delas maduras. Assim surge o operador \textbf{-}, com ele é possível remover um termo da pesquisa, como demonstrado na figura \ref{fig:operadores4}.

\begin{figure}[h]
	\centering
	\includegraphics[width=.9\textwidth]{images/motor_de_busca/operadores4.png}
	\caption{Resultado da pesquisa (Maçã OR Banana) AND “Desenhada” -verde}
	\label{fig:operadores4}
\end{figure}

Estes são os operadores mais simples que o motor de pesquisa do Google oferece, porém não são todos, existem ainda os operadores complexos que irão filtrar ainda mais os resultados. São eles palavras específicas que serão utilizadas com : no final, ex.: \textbf{site}, \textbf{inurl}, \textbf{filetype}, \textbf{define}, \textbf{before}, \textbf{after}, dentre outros.

Para explicar estes operadores propõe-se o seguinte caso de uso: foi visto um edital no site do IFRS Farroupilha, o qual foi acessado em alguma data no mês de Agosto de 2025. Sabe-se que ele se referia a Certificação de Conhecimento. Com isso em mente pode-se formular uma pesquisa direcionada:

\codeblock{"certificação de conhecimentos" AND site:ifrs.edu.br AND inurl:farroupilha AND after:2025-08-1}

Esta pesquisa já é mais complexa que as outras, por isso é necessário explicá-la. Começa com o termo-chave: o edital é sobre certificação de conhecimento, e mais, sabe-se que este trecho estará presente no site, por isso o uso de aspas. Em seguida há o uso do filtro \textbf{site}, com ele se especifica em qual site pesquisar, que no caso é o do IFRS. Na sequência se encontra o \textbf{inurl}, que irá filtrar pelo termo presente no link do site, isto é, ele irá detectar o termo "farroupilha" presente no link. Por fim, se tem o \textbf{after} que irá filtrar por todos os resultados que o Google achar após a data de 1 de Agosto de 2025.

Ao observar o resultado da pesquisa direcionada, demonstrada na figura \ref{fig:operadores4}, é possível observar que o Google trouxe apenas um resultado que é justamente o que se queria encontrar.

\begin{figure}[h]
	\centering
	\includegraphics[width=.9\textwidth]{images/motor_de_busca/operadores5.png}
	\caption{Resultado da pesquisa direcionada}
	\label{fig:operadores5}
\end{figure}

Um detalhe importante a se observar é o formato que a data foi colocada. Na pesquisa foi utilizado a seguinte formatação: 2025-08-01. Este formato de data é o formato da ISO 8601 (\cite{ISOISO86011988}), ele especifica que a formatação deve seguir a seguinte ordem: ano, mês e dia separados por hífen.

Para referência, a tabela \ref{tab:operador1} apresenta os principais operadores de pesquisa e seus respectivos usos. Se ainda restar alguma dúvida, a próxima seção apresenta diferentes cenários de uso para exemplificar a aplicação dos operadores em pesquisas reais.

\begin{table}[h]
	\centering
\begin{tabular}{>{\ttfamily}p{3cm} p{6cm} p{5cm}}
		\toprule
		\textnormal{\textbf{Operador}} & \textbf{Explicação} & \textbf{Uso} \\
		\midrule
		``...'' & Pesquisa pelo termo exato & ``maçã'' \\
		\addlinespace
		OR & Entre um ou outro & maçã OR banana \\
		\addlinespace
		AND & Ambos & maçã AND banana \\
		\addlinespace
		-... & Remove os resultados cujo termo aparece & -verde \\
		\addlinespace
		(...) & Agrupa termos de pesquisa & (maçã OR morango) AND vermelho \\
		\addlinespace
		filetype & Pesquisa pelo tipo de documento & filetype:pdf \\
		\addlinespace
		site & Pesquisa dentro de um site & site:ifrs.edu.br \\
		\addlinespace
		related & Pesquisa por sites relacionados ao site especificado & related:ifrs.edu.br \\
		\addlinespace
		intitle & Pesquisa pelo termo presente no título do site & intitle:educação \\
		\addlinespace
		inurl & Pesquisa pelo termo presente na URL & inurl:farroupilha \\
		\addlinespace
		intext & Pesquisa pelo termo presente no texto do site & intext:certificação \\
		\addlinespace
		before & Pesquisa por resultados catalogados antes de X data & before:2020-01-01 \\
		\addlinespace
		after & Pesquisa por resultados catalogados depois de X data & after:2020-01-01 \\
		\bottomrule
	\end{tabular}
	\caption{Operadores de Pesquisa}
	\label{tab:operador1}
\end{table}

Ainda existem outros operadores de pesquisa, porém eles são considerados inconsistentes, pois funcionam de maneiras imprevisíveis e por isso não serão abordados nesta apostila.


%% sections/motor_de_busca/motor_de_busca.tex
% !TeX root = ../../main.tex

\section{Ideias de uso}

Por fim, chegamos ao ponto onde conhecemos os operadores e como utilizar a ferramenta. Porém, é possível que ainda tenha um pouco de dificuldade em como montar as pesquisas utilizando tudo que vimos. Por isso, neste capítulo, que pode ser utilizado como uma referência para suas futuras pesquisas, iremos construir juntos pesquisas elaboradas para cenários hipotéticos.

\textbf{Cenário 1: }Você está estudando e precisa encontrar artigos acadêmicos ou documentos oficiais sobre o impacto ambiental da mineração no Brasil, evitando notícias de jornais ou blogs de opinião.

\codeblock{"impacto ambiental" mineração (site:.gov.br OR site:.edu.br) filetype:pdf -notícias}

Essa pesquisa utilizou as aspas para focar no tópico a ser pesquisado, adicionou a palavra chave mineração, mas sem o uso das aspas, permitindo aparecer sinônimos e variações da palavra. Além disso foi adicionado filtros de sites, podendo ser do governo (.gov.br) ou sites de educação (.edu.br). Para trazer mais detalhes, foi filtrado apenas por PDFs e por fim foi removido todas as notícias utilizando a negação da palavra chave “notícias”.

\hseparator

\textbf{Cenário 2:} Você deseja ver como a inteligência artificial (IA) está evoluindo, porém com o recente aumento da popularidade do tema fica difícil de encontrar notícias antigas.

\codeblock{"inteligência artificial" (site:.com OR site:.org) before:2015-01-01}

Nessa pesquisa usamos aspas em “inteligência artificial” para obter o termo exato. O operador site: limitou os resultados a domínios confiáveis e abrangentes (.com e .org). Já o before restringiu a pesquisa a conteúdos publicados antes de 2015, permitindo localizar artigos e notícias mais antigas. Assim, conseguindo visualizar como o tema era abordado antes da popularização atual. 

\hseparator

\textbf{Cenário 3: }Você está planejando uma viagem para o Japão, mas os guias turísticos apresentam informações que você já cansou de ler, por isso resolve procurar por experiências de viajantes em fóruns.

\codeblock{"viagem ao Japão" (site:reddit.com OR site:quora.com OR site:tripadvisor.com/forum)}

As aspas em “viagem ao Japão” fixam a expressão exata. O uso do operador site: restringe a busca a sites que funcionam como fóruns ou espaços de discussão (Reddit, Quora e o fórum do TripAdvisor). Isso elimina guias comerciais e blogs promocionais, priorizando relatos e discussões de pessoas reais que já viajaram.

\hseparator

\textbf{Cenário 4:} Você quer acessar livros de filosofia que já estejam em domínio público, preferencialmente em bibliotecas digitais.

\codeblock{"filosofia"AND "domínio público" (site:archive.org OR site:dominiopublico.gov.br) filetype:pdf}

As aspas fixam as expressões “filosofia” e “domínio público”. O operador site: restringe a bibliotecas digitais conhecidas, como Archive.org e Domínio Público. O filetype:pdf traz versões digitais completas das obras. Assim, evita-se material pago ou restrito.

\hseparator

\textbf{Cenário 5:} Você escreveu um trabalho excelente, porém o professor comentou que por mais que o conteúdo esteja bom ele não segue as normas ABNT. Então você decide que quer localizar versões digitais das normas da ABNT sobre formatação de trabalhos.

\codeblock{"ABNT" Formatação de Trabalhos (site:abnt.org.br OR site:.edu.br) filetype:pdf}

As aspas garantem que a palavra ABNT apareça e a pesquisa por formatação de trabalhos ajuda a ferramenta a direcionar o conteúdo. O site: limita a sites oficiais e acadêmicos, como a ABNT e universidades. O filetype:pdf busca documentos em formato pronto para consulta.

\hseparator

\textbf{Cenário 6:} Você quer encontrar receitas veganas de sobremesas que possam ser feitas em menos de 30 minutos. Porém, está cansado de bolo e está sem chocolate em casa.

\codeblock{"receita vegana" sobremesa "até 30 minutos" -chocolate -bolo}

As aspas fixam “receita vegana” e “até 30 minutos”. As palavras sem aspas (sobremesa) permitem variações de termos. O operador - exclui resultados com chocolate ou bolo, filtrando para outras opções de sobremesa rápida.
%% sections/motor_de_busca/motor_de_busca.tex
% !TeX root = ../../main.tex


\newpage
\section{Boas práticas}

Realizar pesquisas no Google aparentemente é uma tarefa simples, porém, há formas para otimizá-las. As boas práticas de busca são um conjunto de recomendações que visam melhorar a eficiência e a precisão dos resultados obtidos ao utilizar um motor de busca como o Google Search. Seguem algumas recomendações gerais para melhorar a qualidade das pesquisas realizadas.

Inicialmente, uma das formas mais eficazes de melhorar os resultados de uma pesquisa no Google é ser específico na escolha das palavras-chave. Pesquisas muito amplas costumam trazer milhares de páginas irrelevantes, enquanto termos mais direcionados ajudam a filtrar a informação. Por exemplo, ao invés de procurar apenas por: \textbf{história Brasil}.

\begin{figure}[h]
	\centering
	\includegraphics[width=.9\textwidth]{images/motor_de_busca/boas_praticas1.png}
	\caption{Resultado da pesquisa história Brasil.}
	\label{fig:boas_praticas1}
\end{figure}

É possível refinar a pesquisa e escrever da seguinte maneira: \textbf{história do Brasil período colonial economia açúcar}. Assim, o motor de busca entende com mais clareza o que se deseja encontrar, entregando resultados mais próximos da necessidade real.

\newpage
\begin{figure}[h]
	\centering
	\includegraphics[width=.9\textwidth]{images/motor_de_busca/boas_praticas2.png}
	\caption{Resultado da pesquisa história do Brasil período colonial economia açúcar.}
	\label{fig:boas_praticas2}
\end{figure}

Após encontrar uma estrutura clara e específica, se pode começar a utilizar os operadores. Por exemplo as aspas, explicadas nas seções anteriores, são extremamente poderosas. Essa prática é especialmente útil em trabalhos acadêmicos ou quando se sabe que um dos termos chaves são tão importantes que precisam aparecer da maneira que foram escritos. Por exemplo, a busca por: \textbf{bolo de “chocolate”} garante que os resultados tragam exatamente esse termo, evitando páginas que falem apenas de bolos em geral.

\begin{figure}[h]
	\centering
	\includegraphics[width=.9\textwidth]{images/motor_de_busca/boas_praticas3.png}
	\caption{Resultado da pesquisa bolo de “chocolate”.}
	\label{fig:boas_praticas3}
\end{figure}

Depois de refinar os termos há a possibilidade de utilizar operadores mais avançados, aumentando o controle sobre os resultados exibidos. Por isso, ao escrever a sua pesquisa use como referência a tabela dos operadores apresentada na seção anterior. E, caso se perder, pegue como base um dos casos apresentados e mude a pesquisa para que se adapte ao que quer procurar.

Se ainda assim os resultados não estiverem muito bons, comece a refinar os termos, removendo tudo aquilo que não é um termo chave para a pesquisa. Assim, ao invés de digitar: \textbf{quais são os sintomas da dengue}, é mais eficiente escrever: \textbf{sintomas dengue}.

\begin{figure}[h]
	\centering
	\includegraphics[width=.9\textwidth]{images/motor_de_busca/boas_praticas4.png}
	\caption{Resultado da pesquisa sintomas dengue.}
	\label{fig:boas_praticas4}
\end{figure}

Por fim, nenhuma pesquisa será realmente eficaz se não houver atenção à credibilidade da fonte. O Google pode trazer resultados diferentes de acordo com a localização, idioma ou histórico de navegação, e nem sempre os primeiros links são os mais confiáveis. Muito menos o resultado da inteligência artificial produzida ao pesquisar, como demonstrado na figura \ref{fig:boas_praticas5}, onde destacado pelo texto “Visão geral criada por IA”, pode trazer reflexões rápidas e diretas, porém pode ser acompanhada de informações falsas e "alucinações da ferramenta de IA". Por isso, é fundamental avaliar se o site tem autoridade no tema, se apresenta dados atualizados e se não há indícios de informação duvidosa. Em uma pesquisa acadêmica sobre saúde, por exemplo, é muito mais confiável utilizar resultados de portais como a OMS, em vez de blogs pessoais sem referências científicas.

\begin{figure}[h]
	\centering
	\includegraphics[width=.9\textwidth]{images/motor_de_busca/boas_praticas5.png}
	\caption{Resultado da pesquisa sintomas dengue.}
	\label{fig:boas_praticas5}
\end{figure}


%\chapter{Gmail}
%% sections/gmail/introducao_gmail.tex
% !TeX root = ../../../main.tex

\section{Introdução ao Gmail}
O Gmail é uma plataforma de e-mails desenvolvida em 2004 pela Google. No 
decorrer dos anos o Gmail tornou-se extremamente popular, pela sua simplicidade, 
eficiência e segurança. As principais funções do Gmail vão além do simples 
“enviar e receber” mensagens eletrônicas. Conforme o Gmail crescia em 
popularidade, ele acabou se tornando uma central de comunicação integrada aos 
serviços Google, assim como as aplicações Google Agenda, Google Drive e Google 
Meet. Ele possui, também, uma organização inteligente com marcadores, filtros 
automáticos e mecanismos de busca avançada, que facilitam a busca por qualquer 
conversa. Todas essas funções se traduzem em benefícios diretos ao usuário, a 
partir dos quais desenvolvemos essa apostila para descrever processos e auxiliar 
o uso da plataforma no dia a dia.
%% sections/gmail/categorias_marcadores_filtros.tex
% !TeX root = ../../../main.tex

\section{Categorias}
\subsection{O que são as Categorias}
As categorias são elementos utilizados para organização das mensagens da caixa 
de entrada. Elas separam os conteúdos em diferentes grupos. O grupo “Principal” 
é definido como a caixa de entrada padrão para mensagens no Gmail. Há, também, 
as categorias “Social”, “Atualizações”, “Fóruns” e “Promoções”, que podem ser 
selecionadas para filtragem das mensagens.

\subsection{Como acessar o campo de categorias}
O campo “Categorias” pode ser acessado ao clicar no “Menu Principal” (ícone de 
três barrinhas horizontais), no canto superior esquerdo da tela, caso a barra 
lateral não esteja sendo exibida. Após, basta selecionar o item “Mais” para 
exibir os demais filtros de pesquisa. Por fim, clique sobre o item “Categoria”.

\begin{figure}[H]
    \centering
    \includegraphics[width=.39\textwidth]{/gmail/categorias_marcadores_filtros/Imagem1.png}
    \caption{}
\end{figure}

Após abrir a opções de categorias, serão exibidas as seguintes opções:
\begin{itemize}
    \item Social: tratam-se dos e-mails que contêm mensagens de redes sociais e 
    sites de compartilhamento de mídia;
    \item Atualizações: agrupa os e-mails que contêm confirmações, notificações, 
    extratos e lembretes que não precisam de atenção imediata;
    \item Fóruns: corresponde aos e-mails com mensagens de grupos online, fóruns 
    de discussão e listas de e-mails;
    \item  Promoções: tratam-se dos e-mails que contêm informações de descontos, 
    ofertas e promoções;
\end{itemize}

\begin{figure}[H]
    \centering
    \includegraphics[width=.6\textwidth]{/gmail/categorias_marcadores_filtros/Imagem2.png}
    \caption{}
\end{figure}

\subsection{Como adicionar os campos de categoria na caixa de entrada}
Para adicionar qualquer uma dessas 4 abas em sua caixa de entrada, comece 
acessando a aba “Configurações”, no canto superior direito da tela.\newline

Em seguida, selecione a opção “Mostrar todas as configurações”.

\begin{figure}[H]
    \centering
    \includegraphics[width=.55\textwidth]{/gmail/categorias_marcadores_filtros/Imagem3.png}
    \caption{}
\end{figure}

Após, na parte superior da tela, selecione a opção “Caixa de entrada”.

\begin{figure}[H]
    \centering
    \includegraphics[width=.6\textwidth]{/gmail/categorias_marcadores_filtros/Imagem4.png}
    \caption{}
\end{figure}

Por padrão, a categoria “Principal” já estará selecionada, e não pode ser 
removida. Você pode adicionar os demais filtros (“Promoções”, “Social”, 
“Atualizações” e “Fóruns”) clicando sobre as caixinhas ao lado de seus nomes.

\begin{figure}[H]
    \centering
    \includegraphics[width=.6\textwidth]{/gmail/categorias_marcadores_filtros/Imagem5.png}
    \caption{}
\end{figure}

Após as alterações, role a página para baixo (com o scroll do mouse ou a barra 
lateral da janela) e clique no botão “Salvar alterações”, para definir as 
modificações.

\begin{figure}[H]
    \centering
    \includegraphics[width=.6\textwidth]{/gmail/categorias_marcadores_filtros/Imagem6.png}
    \caption{}
\end{figure}

Após realizar as alterações, você irá retornar automaticamente para a caixa de 
entrada, onde estarão sendo exibidos os campos de categoria selecionados na 
parte superior da tela.

\begin{figure}[H]
    \centering
    \includegraphics[width=.6\textwidth]{/gmail/categorias_marcadores_filtros/Imagem7.png}
    \caption{}
\end{figure}

Para desfazer essas alterações, basta reproduzir os passos anteriores e 
desmarcar as caixinhas ao lado do nome das opções de categorias que não gostaria 
mais de exibir.

\section{Marcadores}
\subsection{Para que servem os marcadores}
Os marcadores são uma forma eficiente de organizar suas mensagens. Com eles, 
você pode adicionar uma etiqueta ao e-mail, tornando mais fácil a tarefa de 
encontrar seus emails de maior relevância.

\subsection{Como criar um novo marcador}
Para acessar o campo de marcadores você deve ir na aba esquerda do “Menu 
Principal” e selecionar a opção “Mais”. Em seguida, clique sobre a “Criar novo 
marcador”.

\begin{figure}[H]
    \centering
    \includegraphics[width=.6\textwidth]{/gmail/categorias_marcadores_filtros/Imagem8.png}
    \caption{}
\end{figure}

Após selecionar a opção, será aberta uma janela, onde você poderá adicionar um 
nome ao seu novo marcador. Se já tiver algum marcador com a opção abaixo, você 
pode definir sub-marcadores para organização.

\begin{figure}[H]
    \centering
    \includegraphics[width=.6\textwidth]{/gmail/categorias_marcadores_filtros/Imagem9.png}
    \caption{}
\end{figure}

Os marcadores criados serão exibidos abaixo da opção “Marcadores”.

\begin{figure}[H]
    \centering
    \includegraphics[width=.6\textwidth]{/gmail/categorias_marcadores_filtros/Imagem10.png}
    \caption{}
\end{figure}

\subsection{Como aplicar marcadores aos e-mails}
Para aplicar um marcador a um email, você deve clicar sobre a mensagem a qual 
você quer aplicar o marcador e selecionar o símbolo de marcadores na parte 
superior da tela.

\begin{figure}[H]
    \centering
    \includegraphics[width=.6\textwidth]{/gmail/categorias_marcadores_filtros/Imagem11.png}
    \caption{}
\end{figure}

Ao selecionar o símbolo de marcadores, será aberta uma janela onde você pode 
adicionar o e-mail a um dos seus marcadores criados ou atribuir um ou mais 
campos de categoria.

\begin{figure}[H]
    \centering
    \includegraphics[width=.6\textwidth]{/gmail/categorias_marcadores_filtros/Imagem12.png}
    \caption{}
\end{figure}

Após adicionar o e-mail ao marcador definido, você terá acesso àquele e-mail em 
sua caixa de entrada ou, também, através da aba de “Marcadores”.

\begin{figure}[H]
    \centering
    \includegraphics[width=.6\textwidth]{/gmail/categorias_marcadores_filtros/Imagem13.png}
    \caption{}
\end{figure}

\subsection{Como gerenciar marcadores}
Para gerenciar os marcadores, você pode ir até a aba esquerda do “Menu 
Principal” e selecionar a opção “Mais”.\newline

Selecione a opção “Gerenciar Marcadores”

\begin{figure}[H]
    \centering
    \includegraphics[width=.6\textwidth]{/gmail/categorias_marcadores_filtros/Imagem14.png}
    \caption{}
\end{figure}

Será exibida uma tela que apresenta alguns marcadores-padrão definidos pelo 
sistema. Os marcadores criados pelo usuário serão exibidos na parte inferior 
desta tela. Com isso, você tem a possibilidade de “Criar novo marcador”, 
“mostrar”, “ocultar”, “remover” ou “editar” cada um dos marcadores existentes.

\begin{figure}[H]
    \centering
    \includegraphics[width=.6\textwidth]{/gmail/categorias_marcadores_filtros/Imagem15.png}
    \caption{}
\end{figure}

\section{Filtrar e-mails}
Com a ferramenta para filtro de e-mails, você pode encontrar e-mails enviados 
para alguém, e-mails enviados a você, ou até definir critérios para procura de 
e-mails de forma fácil e rápida. Para acessá-la, você deve clicar sobre o ícone 
“Mostrar opções de pesquisa”, que se encontra ao lado da barra de busca 
“Pesquisar e-mail”, na parte superior da tela em sua caixa de entrada.

\begin{figure}[H]
    \centering
    \includegraphics[width=.6\textwidth]{/gmail/categorias_marcadores_filtros/Imagem16.png}
    \caption{}
\end{figure}

\subsection{Tipos de filtros e suas respectivas funcionalidades}
\begin{itemize}
    \item “De” --- Com este filtro você pode procurar e-mails que uma pessoa 
    específica enviou para você.
    \item “Para” --- Com ele você procura e-mails que enviou para um destinatário 
    específico.
    \item “Assunto“ ---  Este filtro permite que você encontre e-mails através da 
    busca pelo campo “Assunto” da mensagem (o popular “título do e-mail”).
    \item “Contém as palavras” --- Com ele você pode adicionar uma ou mais 
    palavras que estejam no corpo do texto do(s) e-mail(s) procurado(s).                            % chktex 36
    \item “Não tem” --- Com ele você pode adicionar uma ou mais palavras as quais 
    não gostaria que estivessem incluídas no corpo do texto dos e-mails 
    procurados.
    \item “Tamanho” --- Com este filtro você encontra emails por espaço de 
    armazenamento ocupado, podendo, inclusive, definir a busca por e-mails 
    menores ou maiores que a quantidade especificada. Também é possível 
    selecionar se a medição será realizada em MB (megabytes), KB (kilobytes) ou 
    bytes.
    \item “Data entre” --- Pode utilizá-lo para definir datas iniciais e finais de 
    envio para procura de um e-mail, estipulando um intervalo de tempo ou um dia 
    específico.
    \item “Pesquisar” --- Com ele você pode definir um parâmetro específico para 
    procurar um e-mail, como, por exemplo, filtrar se ele encontra-se na caixa 
    de entrada, se possui algum filtro ou marcador específico, ou se está na 
    lixeira.
    \item Campo “Com anexo” --- Com esta opção selecionada, serão exibidos apenas 
    os e-mails que possuem ao menos um arquivo anexo a eles.
\end{itemize}

\begin{figure}[H]
    \centering
    \includegraphics[width=.6\textwidth]{/gmail/categorias_marcadores_filtros/Imagem17.png}
    \caption{}
\end{figure}

Com uma ou mais opções de busca definidas, você poderá utilizar a função “Criar 
um filtro”, que exibirá uma série de ações que podem ser atribuídas às mensagens 
encontradas com os parâmetros definidos. 

\begin{figure}[H]
    \centering
    \includegraphics[width=.6\textwidth]{/gmail/categorias_marcadores_filtros/Imagem18.png}
    \caption{}
\end{figure}

\begin{figure}[H]
    \centering
    \includegraphics[width=.6\textwidth]{/gmail/categorias_marcadores_filtros/Imagem19.png}
    \caption{}
\end{figure}

\subsubsection{As ações disponíveis são}
\begin{itemize}
    \item “Ignorar a caixa de entrada (arquivar)” --- Os emails irão diretamente 
        a aba “Arquivar”, ou seja, não irão mais aparecer na caixa de entrada.
    \item “Marcar como lida” --- As mensagens serão marcadas automaticamente 
        como “lidas”.
    \item “Marcar como estrela” --- Os e-mails serão marcados automaticamente 
        “com estrela”.
    \item “Aplicar o marcador:” --- Com ele você pode definir um marcador para 
        atribuir aos emails.
    \item “Encaminhar” --- Você poderá adicionar um email para que os 
        respectivos e-mails filtrados sejam encaminhados automaticamente.
    \item “Excluir” --- Os emails buscados serão excluídos automaticamente.
    \item “Nunca enviar para Spam” --- Os emails não serão atribuídos a aba 
        “Spam”.
    \item “Sempre marcar como importante” --- Irá marcá-los automaticamente como 
        “importante”.
    \item “Nunca marcar como importante” --- Os emails em questão não serão 
        atribuídos à aba “importante”.
    \item “Categorizar como:” --- Permite atribuir categorias automaticamente 
        aos e-mails buscados, sendo elas: “Principal”, “Social”, “Atualizações”, 
        “Fóruns” e “Promoções”.
    \item “Também aplicar filtros a conversas correspondentes” --- Irá aplicar 
        aos emails já existentes a opção desejada.
\end{itemize}

Por fim, basta clicar sobre o botão “Criar filtro”

\begin{figure}[H]
    \centering
    \includegraphics[width=.6\textwidth]{/gmail/categorias_marcadores_filtros/Imagem20.png}
    \caption{}
\end{figure}

Para apagar um dos filtros criados, é necessário entrar nas “Configurações” 
novamente.\newline

Em seguida, selecione a opção “Mostrar todas as configurações”.

\begin{figure}[H]
    \centering
    \includegraphics[width=.6\textwidth]{/gmail/categorias_marcadores_filtros/Imagem3.png}
    \caption{}
\end{figure}

Clique sobre a aba “Filtros e endereços bloqueados”

\begin{figure}[H]
    \centering
    \includegraphics[width=.6\textwidth]{/gmail/categorias_marcadores_filtros/Imagem21.png}
    \caption{}
\end{figure}

Nesta aba serão exibidas opções para criar novos filtros, bem como serão 
mostrados todos os filtros já existentes, que podem ser removidos ou editados.

\begin{figure}[H]
    \centering
    \includegraphics[width=.6\textwidth]{/gmail/categorias_marcadores_filtros/Imagem22.png}
    \caption{}
\end{figure}
%% sections/gmail/configuracoes_avancadas.tex
% !TeX root = ../../main.tex

\section{Configurações de Organização Avançadas}
Apresenta como o usuário pode personalizar sua experiência no Gmail, inclui 
ajustes de idioma, aparência, ferramentas, organização da caixa de entrada e 
integração com outros serviços do Google.


\subsection{Como acessar as configurações}
Para acessar as configurações do Gmail, clique no ícone de engrenagem localizado 
no canto superior direito da tela.

\begin{figure}[H]
    \centering
    \includegraphics[width=.39\textwidth]{/gmail/configuracoes_avancadas/Imagem1.png}
    \caption{}
\end{figure}

\textbf{Ao abrir o menu, algumas opções de acesso rápido já são exibidas:}
\begin{itemize}
    \item \textbf{Apps do Gmail:} Permite escolher se os aplicativos integrados, 
    como Chat e Meet, serão exibidos ou não.
    \item \textbf{Densidade:} Define como os e-mails aparecem na lista, podendo 
    escolher entre compacto, confortável ou padrão, dependendo da preferência do 
    usuário e do tamanho da tela.
    \item \textbf{Tema:} Permite personalizar a aparência do Gmail, escolhendo 
    imagens de fundo oferecidas pela plataforma ou enviando imagens próprias.
    \item \textbf{Tipo de caixa de entrada:} Define a forma de priorização das 
    mensagens, permitindo destacar e-mails importantes ou organizar conforme 
    categorias específicas.
    \item \textbf{Painel de leitura:} Dá ao usuário a opção de abrir e-mails 
    diretamente na lista, à direita ou abaixo da caixa de entrada, facilitando a 
    leitura.
    \item \textbf{Conversa por e-mail:} Agrupa mensagens com o mesmo assunto em 
    uma sequência de conversa, tornando mais fácil acompanhar o histórico de 
    respostas.
\end{itemize}

Para acessar todas as opções de configuração, clique em “Mostrar todas as 
configurações”.

\begin{figure}[H]
    \centering
    \includegraphics[width=.39\textwidth]{/gmail/configuracoes_avancadas/Imagem2.png}
    \caption{}
\end{figure}


\subsection{Configuração geral}
Na Configuração geral, o usuário pode ajustar diversas preferências como:

% TODO: Colocar em negrito os nomes dos itens da lista
\begin{enumerate}
    \item Idioma: Define o idioma da interface e permite ajustar configurações de outros produtos Google. Também oferece suporte a ferramentas de inserção de texto para múltiplos idiomas.
    \item Tamanho máximo da página: Permite escolher quantas conversas ou contatos serão exibidos por página.
    \item Cancelar envio: Possibilita desfazer o envio de um e-mail por um período definido, evitando envios acidentais.
    \item Comportamento de resposta padrão: Permite definir se ao responder um e-mail, a resposta será enviada apenas ao remetente ou a todos os destinatários.
    \item Ações de passar o cursor: Habilita ou desabilita ícones de ação rápida que aparecem ao passar o mouse sobre um e-mail.
    \item Enviar e arquivar: Adiciona um botão que envia a resposta e arquiva a mensagem ao mesmo tempo.
    \item Estilo de texto padrão: Permite definir a fonte, tamanho e cor do texto usado nos e-mails.
    \item Imagens: Escolhe se as imagens nos e-mails são carregadas automaticamente ou se a exibição deve ser solicitada.
    \item E-mail dinâmico: Habilita recursos interativos dentro dos e-mails, como formulários e botões.
    \item Gramática: Oferece sugestões automáticas de correção gramatical enquanto o usuário digita um e-mail.
    \item Ortografia: Verifica e destaca erros ortográficos, sugerindo alterações para manter a escrita correta.
    \item Correção automática: Aplica automaticamente as correções ortográficas sugeridas pelo Gmail.
    
    \begin{figure}[H]
        \centering
        \includegraphics[width=.9\textwidth]{/gmail/configuracoes_avancadas/Imagem3.png}
        \caption{}
    \end{figure}

    \item Escrita inteligente: Sugere palavras ou frases enquanto o usuário digita.
    \item Personalização da Escrita inteligente: Ajusta as sugestões com base no histórico de mensagens do usuário.
    \item Visualização de conversas: Agrupa mensagens com o mesmo assunto, facilitando a leitura do histórico completo de uma conversa.
    \item Alertas: Sugere e-mails que podem ter sido esquecidos ou que precisam de acompanhamento.
    \item Resposta inteligente: Oferece respostas curtas e prontas com base no conteúdo do e-mail recebido.
    \item Painel de visualização: Permite dividir a tela em painéis para visualizar a lista de e-mails e o conteúdo de uma mensagem simultaneamente.
    \item Recursos inteligentes: Permite que informações do Gmail sejam usadas em outros aplicativos Google, como Chat e Meet.
    \item Recursos inteligentes do Google Workspace: Permite que informações de e-mails sejam utilizadas em outros serviços do Google Workspace.
    \item Notificações na área de trabalho: Ativa alertas de e-mail diretamente no computador, avisando sobre novas mensagens recebidas.
    \item Estrelas: Permite personalizar ícones de marcação para destacar mensagens importantes.
    \item Atalhos do teclado: Habilita comandos rápidos via teclado para realizar ações rápidas.
    \item Marcadores de botão: Define se os botões da interface exibem apenas ícones ou ícones acompanhados de texto.
    
    \begin{figure}[H]
        \centering
        \includegraphics[width=.9\textwidth]{/gmail/configuracoes_avancadas/Imagem4.png}
        \caption{}
    \end{figure}

    \item Minha foto: Permite adicionar ou alterar a foto de perfil do usuário.
    \item Criar contatos para preenchimento automático: Define se novos endereços de e-mail devem ser salvos automaticamente ao enviar mensagens.
    \item Assinatura: Permite criar assinaturas personalizadas que aparecem automaticamente no final do e-mail enviado.
    \item Indicadores de nível pessoal: Exibe indicadores para mostrar se o e-mail foi enviado diretamente ao usuário ou para um grupo.
    \item Snippets: Exibe um trecho do conteúdo do e-mail logo abaixo do assunto na caixa de entrada.
    \item Resposta automática de férias: Permite programar respostas automáticas para um período específico.
\end{enumerate}

\begin{atencao}
Após realizar alterações, é necessário clicar no botão “Salvar alterações” na 
parte inferior da tela para que todas as configurações sejam aplicadas.
\end{atencao}

\begin{figure}[H]
    \centering
    \includegraphics[width=.39\textwidth]{/gmail/configuracoes_avancadas/Imagem5.png}
    \caption{}
\end{figure}

\subsection{Configuração de Marcadores}
A aba Configuração de Marcadores permite ao usuário gerenciar e personalizar os 
marcadores existentes e criar novos, facilitando a organização da caixa de 
entrada.\newline

Marcadores funcionam como etiquetas que classificam e-mails, ajudando a 
encontrar mensagens importantes de forma rápida. Alguns marcadores vem como 
padrão, como Com estrela, Adiados, Spam, Lixeira, entre outros. Mas o usuário 
pode criar marcadores personalizados conforme suas necessidades.

\begin{passos}
    \begin{enumerate}
        \item Acesse a aba Marcadores nas configurações.
        \item Selecione quais marcadores deseja exibir na barra lateral 
            esquerda.
        \item Para criar um novo marcador, clique em “Criar novo marcador”, 
            insira o nome desejado e, se necessário, defina um marcador pai para 
            hierarquia e organização.
    \end{enumerate}
\end{passos}

\begin{figure}[H]
    \centering
    \includegraphics[width=.39\textwidth]{/gmail/configuracoes_avancadas/Imagem6.png}
    \caption{}
\end{figure}

\subsection{Configuração da Caixa de entrada}
A aba Configuração da Caixa de entrada permite personalizar a forma como as 
mensagens são exibidas.

\begin{enumerate}
    \item Tipo de Caixa de entrada: Permite definir quais e-mails serão exibidos primeiro.
    \item Categorias: Ativa abas como Principal, Social, Promoções, Atualizações e Fóruns, que organizam automaticamente as mensagens conforme o conteúdo.
    \item Painel de leitura: Permite dividir a tela, visualizando simultaneamente a lista de e-mails e o conteúdo de uma mensagem.
    \item Marcadores de importância: Exibe ícones para destacar e-mails considerados importantes, com base no histórico de uso do usuário e nos critérios do Gmail.
    \item E-mails filtrados: Mostra ou oculta mensagens que já foram direcionadas automaticamente por filtros criados pelo usuário.
\end{enumerate}

\begin{atencao}
Após realizar alterações, clique no botão “Salvar alterações” na parte inferior 
da tela para aplicá-las.
\end{atencao}

\begin{figure}[H]
    \centering
    \includegraphics[width=.39\textwidth]{/gmail/configuracoes_avancadas/Imagem7.png}
    \caption{}
\end{figure}


\subsection{Configuração de Contas e importação}
Essa aba permite gerenciar contas vinculadas e importar mensagens de outros 
e-mails.

\begin{enumerate}
    \item Alterar configurações da conta: Permite modificar informações básicas da conta, como senha, métodos de recuperação e outras preferências de segurança.
    \item Enviar e-mail como: Permite configurar endereços alternativos de envio.
    \item Verificar o e-mail de outras contas: Importa mensagens de outras contas de e-mail para centralizar tudo na mesma caixa de entrada.
    \item Permitir acesso à sua conta: Dá permissão a outros usuários para ler, responder e organizar e-mails em seu nome, sem compartilhar a senha.
\end{enumerate}

\begin{figure}[H]
    \centering
    \includegraphics[width=.39\textwidth]{/gmail/configuracoes_avancadas/Imagem8.png}
    \caption{}
\end{figure}


\subsubsection{Configuração de Filtros e endereços bloqueados}
Essa aba ajuda o usuário a organizar e proteger a caixa de entrada.

\begin{itemize}
    \item Permite criar filtros personalizados que executam ações 
        automaticamente, como arquivar, marcar como lido, aplicar marcadores, 
        encaminhar ou excluir e-mails.
    \item Também possibilita bloquear endereços específicos, garantindo que 
        mensagens de remetentes indesejados não cheguem à caixa de entrada.
\end{itemize}

\begin{figure}[H]
    \centering
    \includegraphics[width=.39\textwidth]{/gmail/configuracoes_avancadas/Imagem9.png}
    \caption{}
\end{figure}

\subsubsection{Configuração de Encaminhamento}
A aba Encaminhamento permite redirecionar automaticamente e-mails recebidos para 
outra conta e definir regras de encaminhamento com base em filtros.

\begin{atencao}
Após qualquer alteração, clique em “Salvar alterações” para que as configurações 
sejam aplicadas.
\end{atencao}

\begin{figure}[H]
    \centering
    \includegraphics[width=.39\textwidth]{/gmail/configuracoes_avancadas/Imagem10.png}
    \caption{}
\end{figure}

\subsection{Configuração de Complemento}
A aba Complemento permite ao usuário instalar e gerenciar extensões e 
integrações no Gmail.

\begin{itemize}
    \item É possível adicionar ferramentas de terceiros, como gerenciadores de 
        tarefas, CRMs ou apps de produtividade.
    \item Os complementos ajudam a automatizar processos, centralizar 
        informações e melhorar a organização.
\end{itemize}

\begin{figure}[H]
    \centering
    \includegraphics[width=.39\textwidth]{/gmail/configuracoes_avancadas/Imagem11.png}
    \caption{}
\end{figure}

\subsection{Configuração de Chat e Meet}
Esta seção integra os serviços de comunicação do Google no Gmail, permitindo 
configurações detalhadas para Chat e Meet.

\begin{enumerate}
    \item Chat: Ativa ou desativa o Google Chat no Gmail.
    \item Configurações do Chat: Ajusta a posição do Chat na tela e permite 
        ativar ou desativar o histórico de conversas.
    \item Configurações do Meet: Permite gerenciar preferências de reuniões, 
        como disponibilidade e integração de convites, diretamente pelo Gmail.
    \item Meet: Mostra ou oculta a seção do Google Meet na barra lateral.
\end{enumerate}

\begin{atencao}
Após realizar alterações, clique no botão “Salvar alterações” para aplicá-las.
\end{atencao}

\begin{figure}[H]
    \centering
    \includegraphics[width=.39\textwidth]{/gmail/configuracoes_avancadas/Imagem12.png}
    \caption{}
\end{figure}

\subsection{Configurações Avançadas}
A aba Configurações Avançadas disponibiliza funcionalidades adicionais que podem 
ser habilitadas ou desabilitadas conforme a necessidade do usuário.

\begin{enumerate}
    \item Avanço automático: Após realizar uma ação em um e-mail, o Gmail abre 
        automaticamente o próximo da lista.
    \item Modelos: Permite criar e utilizar mensagens pré-formatadas, 
        facilitando o envio de e-mails frequentes ou repetitivos.
    \item Atalhos de teclado personalizados: Habilita a criação de combinações 
        de teclas personalizadas, aumentando a produtividade para usuários que 
        preferem atalhos.
    \item Ícone de mensagem não lida: Exibe no ícone da aba do navegador, a 
        quantidade de e-mails não lidos.
\end{enumerate}

\begin{atencao}
Após realizar alterações, clique no botão “Salvar alterações” para aplicá-las.
\end{atencao}

\begin{figure}[H]
    \centering
    \includegraphics[width=.39\textwidth]{/gmail/configuracoes_avancadas/Imagem13.png}
    \caption{}
\end{figure}

\subsubsection{Configurações Off-line}
Esta aba permite ao usuário utilizar o Gmail mesmo sem conexão com a internet.

\begin{itemize}
    \item Ideal para situações em que o usuário não tem internet constante.
    \item As mensagens enviadas offline ficam na caixa de saída até que o Gmail 
    sincronize com o servidor ao reconectar.
\end{itemize}

\begin{figure}[H]
    \centering
    \includegraphics[width=.39\textwidth]{/gmail/configuracoes_avancadas/Imagem14.png}
    \caption{}
\end{figure}

\subsubsection{Configurações de Temas}
A aba Temas permite ao usuário personalizar a aparência visual do Gmail, 
ajustando cores, fundos e estilos da interface.

\begin{itemize}
    \item É possível escolher entre imagens de fundo fornecidas pelo Gmail ou 
    enviar imagens próprias.
\end{itemize}

\begin{figure}[H]
    \centering
    \includegraphics[width=.39\textwidth]{/gmail/configuracoes_avancadas/Imagem15.png}
    \caption{}
\end{figure}

%% sections/gmail/dicas_e_boas_praticas.tex
% !TeX root = ../../main.tex

\section{Dicas e Boas Práticas do Gmail}
Evite realizar operações de arquivar, excluir, aplicar etiquetas ou mover 
muitos e-mails ao mesmo tempo, pois isso pode causar instabilidade no sistema e 
tornar o processo mais lento, o recomendável é realizar essas ações com menos de 
mil mensagens por vez.\newline

Utilize a pesquisa e filtros do Gmail. Com apenas algumas palavras-chave é 
possível encontrar e-mails específicos em toda sua conta, e com o auxílio dos filtros pode-se classificar e-mails por remetente, assunto, aplicar etiquetas, marcar 
como lido e encaminhar para outra conta. Isso torna seu uso mais 
eficiente.

Para reduzir a quantidade de mensagens indesejadas, como spam ou promoções que 
você não deseja mais receber, basta abrir o e-mail e clicar na opção “Cancelar 
inscrição”. Essa ação interrompe o envio de novas mensagens daquele remetente, 
ajudando a manter a caixa de entrada mais limpa.\newline

O Gmail permite que o usuário gerencie várias contas em um único lugar, podendo 
alternar entre elas sem a necessidade de sair e entrar novamente.\newline

Ativar as configurações de resposta automática ou modelos de mensagens é útil 
para enviar mensagens frequentes com respostas semelhantes, isso ajuda a 
economizar tempo e padronizar a comunicação.\newline

Para manter sua conta segura é recomendável aplicar a autenticação em duas 
etapas, adicionando mais uma camada de proteção no login.\newline

É importante saber identificar sinais de phishing, como remetentes estranhos, 
links suspeitos ou erros ortográficos. Nunca clique em links ou baixe anexos de 
remetentes desconhecidos sem verificar a autenticidade.

O Gmail permite configurar assinaturas personalizadas, que são inseridas ao 
final do e-mail enviado, tornando suas mensagens mais profissionais.\newline

O Gmail é integrado ao Google Agenda, permitindo que o usuário crie eventos 
diretamente a partir de um e-mail recebido. Basta clicar nos três pontos no 
canto superior da mensagem e selecionar “Criar evento”. Isso ajuda a organizar 
reuniões, prazos e lembretes sem sair da caixa de entrada.\newline


\subsection{Comandos}
Ao ativar a opção de atalhos de teclado nas configurações do Gmail, é possível 
executar tarefas de forma muito mais rápida. Esses comandos ajudam a agilizar 
ações como abrir a caixa de entrada, escrever novos e-mails, arquivar ou excluir 
mensagens e navegar entre diferentes seções.

\newpage

\textbf{Principais atalhos:}\newline
\begin{itemize}
    \item \tecla{c} → Compor novo e-mail;\newline
    \item \tecla{d} → Abrir composição em nova janela;\newline
    \item \tecla{/} → Colocar o cursor na barra de pesquisa;\newline
    \item \tecla{e} → Arquivar e-mail selecionado;\newline
    \item \tecla{\#} → Excluir e-mail selecionado;\newline
    \item \tecla{r} → Responder e-mail;\newline
    \item \tecla{a} → Responder a todos;\newline
    \item \tecla{f} → Encaminhar e-mail;\newline
    \item \tecla{Shift} + \tecla{i} → Marcar como lido;\newline
    \item \tecla{Shift} + \tecla{u} → Marcar como não lido;\newline
    \item \tecla{g} + \tecla{i} → Ir para a caixa de entrada;\newline
    \item \tecla{g} + \tecla{s} → Ir para e-mails com estrela;\newline
    \item \tecla{g} + \tecla{t} → Ir para e-mails enviados;\newline
    \item \tecla{*} + \tecla{a} → Selecionar todos os e-mails;\newline
    \item \tecla{*} + \tecla{n} → Desmarcar todos os e-mails;\newline
    \item \tecla{x} → Selecionar conversa atual;\newline
    \item \tecla{z} → Desfazer última ação;\newline
    \item \tecla{k} → Ir para a conversa mais nova;\newline
    \item \tecla{j} → Ir para a conversa mais antiga;\newline
    \item \tecla{o} ou \tecla{Enter} → Abrir conversa selecionada;\newline
    \item \tecla{n} → Ir para a próxima mensagem em uma conversa;\newline
    \item \tecla{p} → Voltar para a mensagem anterior em uma conversa;\newline
    \item \tecla{s} → Marcar ou desmarcar estrela na conversa;\newline
    \item \tecla{y} → Remover a conversa da visualização atual;\newline
    \item \tecla{Shift} + \tecla{t} → Adicionar conversa as tarefas;\newline
    \item \tecla{Shift} + \tecla{\#} → Mover conversa para a lixeira;\newline
    \item \tecla{Ctrl} + \tecla{i} → Colocar texto em itálico;\newline
    \item \tecla{Ctrl} + \tecla{b} → Colocar texto em negrito;\newline
    \item \tecla{Ctrl} + \tecla{u} → Colocar texto Sublinhado;\newline
    \item \tecla{Alt} + \tecla{Shift} + \tecla{5} → Colocar texto riscado;\newline
\end{itemize}





%% sections/gmail/escrever_enviar_download_anexos_emails_com_estrela.tex
% !TeX root = ../../main.tex


\section{Como Escrever um e-mail}
Este guia irá demonstrar o processo completo para compor e enviar uma mensagem 
de forma eficaz. Para tornar o aprendizado mais prático, vamos simular uma 
situação real: “O envio de um currículo para uma oportunidade de emprego”. 
Acompanhe como construir o e-mail, anexar arquivos, inserir links importantes, 
acessar os rascunhos e agendar o envio do e-mail.


\subsection{Iniciando um novo e-mail}
Clique no botão “Escrever” no canto superior esquerdo da tela. Uma janela de 
“Nova mensagem” será aberta no canto inferior direito, assim você poderá começar 
a compor seu e-mail.

\begin{figure}[H]
    \centering
    \includegraphics[width=.39\textwidth]{/gmail/escrever_enviar_download_anexos_emails_com_estrela/Imagem1.png}
    \caption{}
\end{figure}

Para uma melhor visualização, você pode maximizar essa janela clicando no ícone de tela cheia.

\begin{figure}[H]
    \centering
    \includegraphics[width=.39\textwidth]{/gmail/escrever_enviar_download_anexos_emails_com_estrela/Imagem2.png}
    \caption{}
\end{figure}

\subsection{Preenchendo os campos da mensagem}
Agora, preencha os campos essenciais do seu e-mail:

\begin{itemize}
    \item Destinatários / Para: O endereço de e-mail do destinatário.
    \item Assunto: Escreva um título claro e objetivo para a sua mensagem.
    \item Corpo do e-mail: Escreva o texto da sua mensagem no espaço principal.
\end{itemize}

\begin{figure}[H]
    \centering
    \includegraphics[width=.39\textwidth]{/gmail/escrever_enviar_download_anexos_emails_com_estrela/Imagem3.png}
    \caption{}
\end{figure}


\subsubsection{Anexando um Arquivo}
Para incluir um documento, como um currículo (PDF) ou uma foto, clique no ícone 
de \textbf{clipe de papel} na barra de ferramentas inferior. Uma janela do seu computador 
será aberta para que você possa navegar e selecionar o arquivo desejado, depois 
de localizar o arquivo clique em “Abrir”.

\begin{figure}[H]
    \centering
    \includegraphics[width=.39\textwidth]{/gmail/escrever_enviar_download_anexos_emails_com_estrela/Imagem4.png}
    \caption{}
\end{figure}

Após selecionar o arquivo e clicar em “Abrir”, ele aparecerá na parte inferior 
da sua mensagem, mostrando uma barra de carregamento, após essa barra ser 
completa, o arquivo está pronto para ser enviado.

\begin{figure}[H]
    \centering
    \includegraphics[width=.39\textwidth]{/gmail/escrever_enviar_download_anexos_emails_com_estrela/Imagem5.png}
    \caption{}
\end{figure}

Caso precise adicionar um link para um site ou perfil de rede social, como o 
LinkedIn, selecione o texto que deseja transformar em link e clique no ícone de 
\textbf{corrente} barra de ferramentas. Uma caixa de diálogo aparecerá para você, apenas 
cole a URL e clique em “Aplicar”.

\begin{figure}[H]
    \centering
    \includegraphics[width=.39\textwidth]{/gmail/escrever_enviar_download_anexos_emails_com_estrela/Imagem6.png}
    \caption{}
\end{figure}


\subsection{Acessando os Rascunhos}
Em caso de você acabar fechando a mensagem, não se preocupe em perder seu 
trabalho. O Gmail salva automaticamente seu progresso, se isso acontecer poderá 
encontrar seu e-mail salvo na pasta “Rascunhos”, localizada no menu à esquerda, 
assim podendo continuar de onde você parou.

\begin{figure}[H]
    \centering
    \includegraphics[width=.39\textwidth]{/gmail/escrever_enviar_download_anexos_emails_com_estrela/Imagem7.png}
    \caption{}
\end{figure}


\subsubsection{Enviando ou programando o e-mail}
Para enviar o e-mail imediatamente, basta clicar no botão “Enviar”.

\begin{figure}[H]
    \centering
    \includegraphics[width=.39\textwidth]{/gmail/escrever_enviar_download_anexos_emails_com_estrela/Imagem8.png}
    \caption{}
\end{figure}

Ao invés de enviar o e-mail imediatamente, você pode agendar o envio para um 
horário mais apropriado, para isso, ao lado do botão “Enviar”, clique na 
\textbf{seta para baixo} e selecione a opção “Programar envio”.

\begin{figure}[H]
    \centering
    \includegraphics[width=.39\textwidth]{/gmail/escrever_enviar_download_anexos_emails_com_estrela/Imagem8.1.png}
    \caption{}
\end{figure}

Uma janela aparecerá com sugestões de data e hora ou a opção “Escolher data e 
hora”, para que você personalize o envio.

\begin{figure}[H]
    \centering
    \includegraphics[width=.39\textwidth]{/gmail/escrever_enviar_download_anexos_emails_com_estrela/Imagem9.png}
    \caption{}
\end{figure}


\subsubsection{Gerenciando e-mails Programados}
Após programar o envio do e-mail, ele não irá aparecer na caixa de “Enviados”, 
mas sim em uma caixa chamada “Programados”. Você pode acessá-la no menu à 
esquerda para conferir os e-mails que estão aguardando o momento do envio.

\begin{figure}[H]
    \centering
    \includegraphics[width=.39\textwidth]{/gmail/escrever_enviar_download_anexos_emails_com_estrela/Imagem10.png}
    \caption{}
\end{figure}

Ao clicar no e-mail dentro da pasta “Programados”, você pode revisar seu 
conteúdo e o horário agendado.

\begin{figure}[H]
    \centering
    \includegraphics[width=.39\textwidth]{/gmail/escrever_enviar_download_anexos_emails_com_estrela/Imagem11.png}
    \caption{}
\end{figure}

Para cancelar o envio de um e-mail programado, você deve abrir novamente a pasta 
“Programados”, clicar sobre o e-mail que deseja cancelar, e clicar no botão 
“Cancelar envio”.

\begin{figure}[H]
    \centering
    \includegraphics[width=.39\textwidth]{/gmail/escrever_enviar_download_anexos_emails_com_estrela/Imagem11.1.png}
    \caption{}
\end{figure}


\subsection{Verificando e-mails Enviados}
Para acessar os e-mails enviados por você, basta clicar na pasta “Enviados”, 
localizada no menu à esquerda.

\begin{figure}[H]
    \centering
    \includegraphics[width=.39\textwidth]{/gmail/escrever_enviar_download_anexos_emails_com_estrela/Imagem12.png}
    \caption{}
\end{figure}


\section{Como realizar o download dos anexos}
Aprenda a baixar facilmente os arquivos que você recebe por e-mail, como 
documentos, fotos ou planilhas, diretamente para seu computador.


\subsection{Localize e Abra o e-mail}
Identifique na pasta “Caixa de Entrada”, ou em qualquer outra pasta, a mensagem 
que contém o anexo que deseja baixar. Para abrir o anexo pode ser seguido de 
duas maneiras, uma sendo normalmente exibido abaixo do “Assunto” e do resumo do 
corpo do email.

\begin{figure}[H]
    \centering
    \includegraphics[width=.39\textwidth]{/gmail/escrever_enviar_download_anexos_emails_com_estrela/Imagem13.png}
    \caption{}
\end{figure}

Ou também clicando na mensagem para abri-la e rolando a página até o final. A 
seção de anexos estará visível logo abaixo do corpo do e-mail.

\begin{figure}[H]
    \centering
    \includegraphics[width=.39\textwidth]{/gmail/escrever_enviar_download_anexos_emails_com_estrela/Imagem14.png}
    \caption{}
\end{figure}


\section{Como Usar a Função ``Com Estrela''}
A função “Com estrela” é uma forma rápida de destacar e-mails importantes para 
que você possa encontrá-los facilmente mais tarde, funcionando como uma lista de 
favoritos.


\subsubsection{Como Marcar e Desmarcar um e-mail com Estrela}
Identifique na pasta “Caixa de Entrada”, ou em qualquer outra pasta, a mensagem 
que deseja destacar. Ao lado esquerdo do nome do remetente, você verá um ícone 
de estrela vazia.

\begin{figure}[H]
    \centering
    \includegraphics[width=.39\textwidth]{/gmail/escrever_enviar_download_anexos_emails_com_estrela/Imagem13.1.png}
    \caption{}
\end{figure}

Para marcar o e-mail, basta clicar no ícone da \textbf{estrela}. Para desmarcar, basta 
clicar na estrela novamente.


\subsubsection{Como acessar seus e-mails marcados com Estrela}
Para visualizar de maneira separada todos os e-mails que você marcou com 
estrela, clique na pasta “Com estrela”, localizada no menu à esquerda.

\begin{figure}[H]
    \centering
    \includegraphics[width=.39\textwidth]{/gmail/escrever_enviar_download_anexos_emails_com_estrela/Imagem15.png}
    \caption{}
\end{figure}
%% sections/gmail/exclusao_reg_leitura_adiar.tex
% !TeX root = ../../../main.tex


\section{Como Excluir e Recuperar e-mails excluídos}
Este guia irá demonstrar o processo completo para manter uma caixa de entrada 
limpa. Para isso, vamos explorar as funções de excluir, recuperar e excluir 
definitivamente e-mails indesejados.


\subsubsection{Excluindo um e-mail}
Existem duas maneiras possíveis de excluir um e-mail indesejado: diretamente 
pela caixa de entrada ou estando com a mensagem aberta. Para excluir a partir da 
caixa de entrada, basta marcar a caixa de seleção ao lado esquerdo do e-mail --- 
podem ser marcadas mais de uma caixa de seleção simultaneamente --- em seguida, 
um menu de ações aparecerá no topo da página. Apenas clique no botão com o ícone 
de \textbf{lixeira}.

\begin{figure}[H]
    \centering
    \includegraphics[width=.39\textwidth]{/gmail/exclusao_reg_leitura_adiar/Imagem1.png}
    \caption{}
\end{figure}

Para excluir o e-mail com a mensagem aberta, basta clicar no botão com o ícone 
de \textbf{lixeira}.

\begin{figure}[H]
    \centering
    \includegraphics[width=.39\textwidth]{/gmail/exclusao_reg_leitura_adiar/Imagem2.png}
    \caption{}
\end{figure}


\subsubsection{Gerenciando a Lixeira}
Se você acabou excluindo um e-mail por engano, saiba que ele pode ser 
recuperado. O Gmail mantém os e-mails que você enviou para a lixeira guardados 
por 30 dias em caso de engano. Após esse tempo, serão excluídos definitivamente.
Para encontrar os seus e-mails excluídos, clique na pasta “Lixeira”, localizada 
no menu à esquerda. Dentro da lixeira, marque a caixa de seleção do e-mail que 
deseja restaurar. O menu de ações aparecerá novamente no topo da página, clique 
no ícone representado por uma \textbf{pasta}, nomeado como “Mover para”, selecione a 
caixa de entrada ou a pasta correspondente para a qual deseja enviar o e-mail.

\begin{figure}[H]
    \centering
    \includegraphics[width=.39\textwidth]{/gmail/exclusao_reg_leitura_adiar/Imagem3.png}
    \caption{}
\end{figure}

Após essas ações, o e-mail retornará imediatamente ao local selecionado. 
Agora, nos casos em que você deseja excluir o e-mail permanentemente e liberar espaço de 
armazenamento, na pasta “Lixeira” selecione a mensagem desejada e 
clique no botão “Excluir definitivamente”. Lembrando, essa ação não pode ser 
desfeita, então certifique-se de que é o e-mail correto ou que realmente deseja 
realizar essa ação.

\begin{figure}[H]
    \centering
    \includegraphics[width=.39\textwidth]{/gmail/exclusao_reg_leitura_adiar/Imagem4.png}
    \caption{}
\end{figure}

Para você limpar a lixeira, basta clicar no botão “esvaziar a lixeira agora”, 
assim todos os e-mails que estão na lixeira serão excluídos definitivamente.

\begin{figure}[H]
    \centering
    \includegraphics[width=.39\textwidth]{/gmail/exclusao_reg_leitura_adiar/Imagem5.png}
    \caption{}
\end{figure}


\section{Registro de leitura}
O Gmail utiliza um sistema visual simples para ajudar a diferenciar as 
mensagens que já foram lidas daquelas que ainda precisam de atenção. Os e-mails 
não lidos aparecem em \textbf{negrito}. Para alternar entre os status lido e não 
lido, basta visualizar a mensagem ou alterar manualmente, clicando na caixa de 
seleção e no botão com ícone de \textbf{carta}, localizado na barra de ações que 
será aberta.

\begin{figure}[H]
    \centering
    \includegraphics[width=.39\textwidth]{/gmail/exclusao_reg_leitura_adiar/Imagem6.png}
    \caption{}
\end{figure}


\section{Adiar e-mails}
A funcionalidade “Adiar” é uma ferramenta muito útil para quem costuma receber 
e-mails importantes em momentos inoportunos. Essa funcionalidade remove a 
mensagem temporariamente da sua caixa de entrada e a faz retornar como uma 
nova mensagem na data e hora que você escolher.


\subsection{Adiando um e-mail}
Estando na caixa de entrada, selecione o e-mail que deseja adiar clicando na 
caixa de seleção. No menu superior que aparecerá, clique no ícone de 
\textbf{relógio}. Uma janela irá abrir solicitando até quando você deseja adiar 
este e-mail, selecione o tempo desejado para o retorno do e-mail.

\begin{figure}[H]
    \centering
    \includegraphics[width=.39\textwidth]{/gmail/exclusao_reg_leitura_adiar/Imagem7.png}
    \caption{}
\end{figure}


\subsection{Gerenciando e-mails adiados}
Assim que você adiar um e-mail, ele ficará guardado na pasta “Adiados”, 
localizada no menu lateral esquerdo. Caso você decida responder à 
mensagem antes do previsto, ou adiar ainda mais o e-mail, apenas 
selecione o e-mail a partir da caixa de seleção, no menu, e clique novamente no 
ícone de relógio. Uma nova janela irá abrir, basta clicar na opção “retomar” ou 
selecionar novamente o tempo desejado.

\begin{figure}[H]
    \centering
    \includegraphics[width=.39\textwidth]{/gmail/exclusao_reg_leitura_adiar/Imagem8.png}
    \caption{}
\end{figure}
%% sections/gmail/responder_encaminhar_marcar_emails.tex
% !TeX root = ../../../main.tex

\section{Como responder e encaminhar e-mails}
Após receber e ler os seus e-mails, as ações mais comuns são continuar a 
conversa ou compartilhar a informação com outras pessoas. Para isso, o Gmail 
oferece as funções “Responder” e “Encaminhar”. Aprender a utilizar essas 
ferramentas é fundamental para uma comunicação digital eficaz.


\subsubsection{Como responder a um e-mail}
Use a função de responder para enviar uma mensagem de volta ao remetente 
original, assim, poderá continuar a conversa.\newline

Para isso, abra o e-mail que você deseja responder. Na parte inferior da 
mensagem você encontrará o botão "Responder". Em seguida, uma nova caixa de 
diálogo aparecerá abaixo do e-mail. Digite sua resposta nesse novo espaço. 
Após a conclusão do texto da resposta, clique no botão Enviar.

\begin{figure}[H]
    \centering
    \includegraphics[width=.39\textwidth]{/gmail/responder_encaminhar_marcar_emails/Imagem1.png}
    \caption{}
\end{figure}

\subsubsection{Como encaminhar um e-mail}
Use a função de encaminhar quando for necessário enviar uma cópia exata de um 
e-mail que você recebeu para uma pessoa que não estava na conversa 
originalmente.\newline

Para isso, abra o e-mail que você deseja encaminhar e clique no botão 
"Encaminhar", localizado ao lado do botão de resposta. Na sequência, uma nova caixa 
de diálogo será aberta. No campo “Para”, digite o endereço de e-mail do 
destinatário. Se desejar, você pode escrever uma mensagem adicional acima do 
conteúdo original que está sendo encaminhado.

\begin{figure}[H]
    \centering
    \includegraphics[width=.39\textwidth]{/gmail/responder_encaminhar_marcar_emails/Imagem2.png}
    \caption{}
\end{figure}


\section{E-mails como importante}
O marcador "Importante" do Gmail é uma ferramenta inteligente que ajuda você a 
priorizar suas mensagens. O próprio Gmail tenta prever o que é importante para o 
usuário, mas você também pode marcar e desmarcar e-mails manualmente. Isso ajuda 
a organizar sua caixa de entrada e a encontrar rapidamente as mensagens que 
exigem mais atenção.


\subsubsection{Como marcar um e-mail como importante}
Para sinalizar manualmente que uma mensagem é uma prioridade, estando na sua 
Caixa de Entrada, selecione o e-mail desejado clicando na caixa de seleção. Em 
seguida, clique no ícone de \textbf{três pontos} na barra de ferramentas 
superior. Após abrir o menu de opções, selecione "Marcar como importante".

\begin{figure}[H]
    \centering
    \includegraphics[width=.39\textwidth]{/gmail/responder_encaminhar_marcar_emails/Imagem3.png}
    \caption{}
\end{figure}


\subsubsection{Como desmarcar um e-mail como importante}
Todos os e-mails marcados como importantes, seja por você ou pelo Gmail, são 
reunidos em uma pasta específica para fácil acesso. Para visualizá-los, basta 
clicar na pasta "Importante", localizada no menu lateral esquerdo da tela. Com 
isso, você consegue marcar o e-mail como não importante repetindo as mesmas ações: 
clicar na caixa de seleção, no ícone de três pontos na barra de ferramentas 
superior e, por fim, na opção "Marcar como não importante".

\begin{figure}[H]
    \centering
    \includegraphics[width=.39\textwidth]{/gmail/responder_encaminhar_marcar_emails/Imagem4.png}
    \caption{}
\end{figure}


\section{Como Lidar com E-mails de Spam}
\subsubsection{O que é Spam?}
Spam são e-mails indesejados, geralmente enviados em massa para um grande número 
de destinatários sem o consentimento deles. Na maioria das vezes, são mensagens 
de publicidade, mas também podem ser perigosas, contendo tentativas de golpe 
(phishing), vírus ou links maliciosos. Nunca clique em links ou baixe anexos de 
e-mails de spam.


\subsubsection{Como Marcar um E-mail como Spam (Denunciar Spam)}
Ao receber um e-mail de spam em sua caixa de entrada, você pode denunciá-lo para 
removê-lo e ajudar o filtro do Gmail. Para isso, basta selecionar a mensagem 
suspeita na sua caixa de entrada (sem a necessidade de abri-la) e clicar no ícone 
de atenção localizado no menu superior.

\begin{figure}[H]
    \centering
    \includegraphics[width=.39\textwidth]{/gmail/responder_encaminhar_marcar_emails/Imagem5.png}
    \caption{}
\end{figure}


\subsubsection{Como Acessar e Gerenciar a Caixa de Spam}
Como um e-mail legítimo pode ser classificado como spam por engano, é uma boa 
prática verificar esta pasta periodicamente. Para acessá-la, clique na pasta "Spam" no menu 
lateral esquerdo. Caso encontre um e-mail que não é spam, basta selecioná-lo e 
clicar no botão "Não é spam" no topo da tela para que ele retorne à sua Caixa de 
Entrada.

\begin{figure}[H]
    \centering
    \includegraphics[width=.39\textwidth]{/gmail/responder_encaminhar_marcar_emails/Imagem6.png}
    \caption{}
\end{figure}

\chapter{Agenda}
% sections/agenda/introducao_google_agenda.tex
% !TeX root = ../../../main.tex

\section{Introdução ao Google Agenda}
O Google Agenda é uma ferramenta de gerenciamento de tempo e agendamento de compromissos, lançada pelo Google em 2006. Desde então, consolidou-se como uma das plataformas mais utilizadas em seu segmento, graças à sua versatilidade, acessibilidade e eficiência na organização de atividades pessoais e profissionais.

Mais do que um simples calendário digital, o Google Agenda funciona como um assistente pessoal inteligente. Um de seus principais diferenciais é a integração com outros serviços do ecossistema Google (\cite{GoogleAgendaBaseConhecimento}). O Gmail, por exemplo, permite a criação automática de eventos a partir de e-mails recebidos, enquanto o Google \Gls{meet} gera links de videoconferência diretamente nos compromissos agendados. Essas funcionalidades ampliam significativamente a utilidade e a praticidade da plataforma.

Além disso, o Google Agenda oferece recursos avançados, como o gerenciamento de múltiplas agendas, lembretes personalizados e opções de compartilhamento (\cite{GoogleAgendaBaseConhecimento}), tornando a administração do tempo mais dinâmica, eficiente e colaborativa.

Com base nesses recursos, este capítulo foi desenvolvido com o objetivo de apresentar as principais funcionalidades do Google Agenda e orientar o usuário em sua utilização no dia a dia, promovendo uma rotina mais organizada e produtiva.
% sections/agenda/configuracoes.tex
% !TeX root = ../../../main.tex

\section{Configurações}
A aba de configurações do Google Agenda é o espaço onde o usuário pode 
personalizar e ajustar o funcionamento de acordo com suas preferências. Nela, é 
possível alterar o idioma, o fuso horário, escolher a forma de exibição dos 
eventos, definir horários de trabalho e disponibilidade, configurar as 
notificações e lembretes de seus eventos e compromissos e também é possível 
gerenciar múltiplas agendas, definir cores para seus diferentes tipos de 
compromisso. Facilitando a vida do usuário, tornando sua experiência mais 
agradável e fácil de usar. Outro ponto importante é a possibilidade de gerenciar 
o compartilhamento de agendas, permitindo que outras pessoas visualizem e 
editem, facilitando bastante na organização de um grupo ou de uma equipe. 


\subsubsection{Como acessar as configurações}
Para acessar as configurações do Google Agenda, clique no ícone de engrenagem 
localizado no canto superior direito da tela.

\begin{figure}[H]
    \centering
    \includegraphics[width=.39\textwidth]{/agenda/configuracoes/Imagem1.png}
    \caption{}
\end{figure}

\begin{itemize}
    \item Lixeira: Todos os eventos que foram previamente excluídos, ficaram lá por até 30 dias e depois serão excluídos permanentemente, ele te dá uma chance de recuperar itens que foram excluídos.
    \item Aparência: Permite personalizar o calendário de acordo com as preferências do usuário, possibilitando a escolha entre diferentes temas e ajustando a densidade das informações exibidas na tela.
    \item Imprimir: Permite gerar e salvar um arquivo PDF ou realizar a impressão de um intervalo de tempo da agenda.
    \item Instalar complemento: Permite integrar ao Google Agenda aplicativos de terceiros para melhorar a experiência do usuário.
\end{itemize}

Para acessar todas as opções de configuração, clique em “Configurações”.

\begin{figure}[H]
    \centering
    \includegraphics[width=.39\textwidth]{/agenda/configuracoes/Imagem2.png}
    \caption{}
\end{figure}


\subsubsection{Configuração geral}
Na Configuração geral, o usuário pode ajustar diversas preferências que personalizam o funcionamento do Google Agenda:

\begin{enumerate}
    \item Idioma e região: Define o idioma, a região e o formato de data e hora da sua agenda
    \item Fuso horário: Define o fuso horário da sua agenda, além de poder adicionar um segundo fuso horário.
    \item Relógio Mundial: Permite exibir no menu principal do Google Agenda vários fusos horários de diversos locais do mundo. 
    
    \begin{figure}[H]
        \centering
        \includegraphics[width=.39\textwidth]{/agenda/configuracoes/Imagem3.png}
        \caption{}
    \end{figure}

    \item Configurações de eventos: Define critérios para criar um novo evento, como duração padrão e as permissões de convidados.
    \item Configurações de notificação: Permite configurar como o usuário será avisado sobre os eventos, podendo alterar como será notificado e o tipo de som da notificação do evento.

    \begin{figure}[H]
        \centering
        \includegraphics[width=.39\textwidth]{/agenda/configuracoes/Imagem4.png}
        \caption{}
    \end{figure}

    \item Opções de visualização: Ajusta a aparência da agenda e a forma como os eventos são mostrados, podendo escolher entre diferentes modos de exibição, como diária, semanal ou mensal.
    \item Recursos inteligentes do Google Workspace: Personaliza a experiência do usuário com recursos de outros produtos do Google, apps como o Gmail, Agenda, Chat, Meet e Drive. Ativando esse recurso você pode permitir que o Gmail crie eventos automaticamente com base em confirmações.
    
    \begin{figure}[H]
        \centering
        \includegraphics[width=.39\textwidth]{/agenda/configuracoes/Imagem5.png}
        \caption{}
    \end{figure}

    \item Horário e local de trabalho: Define sua disponibilidade, informando horários que você trabalha ou estuda para caso alguém tente agendar uma reunião com você.
    \item Atalhos do teclado: Permite utilizar comandos no teclado para facilitar e agilizar a locomoção pela agenda. 
    \item Offline: Permite utilizar da agenda principal mesmo quando estiver sem internet, dando acesso ao usuário a compromissos já sincronizados.

    \begin{figure}[H]
        \centering
        \includegraphics[width=.39\textwidth]{/agenda/configuracoes/Imagem6.png}
        \caption{}
    \end{figure}
\end{enumerate}

\subsection{Adicionar agenda}
Adiciona novos calendários e permite acompanhar outras agendas e eventos públicos, facilitando a organização de seus compromissos e eventos.

\begin{enumerate}
    \item Inscrever-se na agenda: Permite adicionar agendas de outras pessoas, aparecendo a agenda que foram compartilhadas junto com a sua, podendo ver compromissos e eventos de outras pessoas.
        
    \begin{figure}[H]
        \centering
        \includegraphics[width=.39\textwidth]{/agenda/configuracoes/Imagem7.png}
        \caption{}
    \end{figure}

    \item Criar nova agenda: Cria um novo calendário separado da sua agenda principal, útil para manter compromissos separados e organizados por categoria.
    
    \begin{figure}[H]
        \centering
        \includegraphics[width=.39\textwidth]{/agenda/configuracoes/Imagem8.png}
        \caption{}
    \end{figure}

    \item Procurar agendas de interesse: Disponibiliza agendas públicas, assim o usuário pode acompanhar automaticamente eventos como feriados nacionais, eventos esportivos, entre outros.

    \begin{figure}[H]
        \centering
        \includegraphics[width=.39\textwidth]{/agenda/configuracoes/Imagem9.png}
        \caption{}
    \end{figure}

    \item Do URL: Permite adicionar uma agenda externa utilizando um endereço de URL, integrando calendários de outras plataformas.

    \begin{figure}[H]
        \centering
        \includegraphics[width=.39\textwidth]{/agenda/configuracoes/Imagem10.png}
        \caption{}
    \end{figure}
\end{enumerate}


\subsection{Importar e Exportar}
Essas opções permitem transferir eventos entre diferentes contas ou aplicativos de agenda.

\begin{enumerate}
    \item Importar:Permite trazer eventos através de um arquivo no formato .ics para dentro do Google Agenda.
    \item Exportar: Permite baixar os seus eventos do Google agenda em um arquivo .zip que contém um ou mais arquivos no formato .ics.
    
    \begin{figure}[H]
        \centering
        \includegraphics[width=.39\textwidth]{/agenda/configuracoes/Imagem11.png}
        \caption{}
    \end{figure}
\end{enumerate}


\subsubsection{Configurações das minhas agendas}
Logo abaixo do título dessa seção, na barra lateral esquerda, aparecem os nomes das agendas do usuário. Ao clicar em qualquer uma delas, são exibidas as opções de configuração específicas dessa agenda.


\subsubsection{Configurações de agendas criadas}

\begin{enumerate}
    \item Configurações da agenda: Define nome, descrição, fuso horário e outras preferências básicas da agenda selecionada.
    \item Compartilhar com: Permite compartilhar a sua agenda com outras pessoas, além de poder permitir que outros usuários apenas olhem e também possam editar a sua agenda.
    
    \begin{figure}[H]
        \centering
        \includegraphics[width=.39\textwidth]{/agenda/configuracoes/Imagem12.png}
        \caption{}
    \end{figure}

    \item Autorizações de acesso a eventos: Ajusta a quantidade de detalhes que outros usuários poderão ver quando você compartilha a sua agenda.
    \item Notificações de eventos: Define quando e como você será notificado de um evento normal.
    \item Notificações de eventos de dia inteiro: Define quando e como você será notificado de um evento de dia inteiro.
    
    \begin{figure}[H]
        \centering
        \includegraphics[width=.39\textwidth]{/agenda/configuracoes/Imagem13.png}
        \caption{}
    \end{figure}

    \item Outras notificações: Define como você será notificado quando a alguma ação relacionada a sua agenda, como quando alguém adiciona um evento ou quando alguém responde a algum convite.
    \item Integrar agenda: Mostra links e códigos para integrar a sua agenda em outros serviços.
    \item Remover agenda: Permite excluir uma agenda criada ou se desinscrever de uma agenda que compartilharam com você.
    
    \begin{figure}[H]
        \centering
        \includegraphics[width=.39\textwidth]{/agenda/configuracoes/Imagem14.png}
        \caption{}
    \end{figure}
\end{enumerate}


\subsubsection{Configuração da agenda de aniversário}

\begin{enumerate}
    \item Configuração da agenda: Permite ajustar o fuso horário de exibição da agenda.
    \item Configuração de permissão: Define as configurações de permissão da agenda de aniversário.
    \item Notificações de eventos de dia inteiro: Define quando e como você será notificado de um evento de aniversário.
    \item Sincronizar com os contatos:Conecta os aniversários dos seus contatos do Google diretamente na agenda.
    
    \begin{figure}[H]
        \centering
        \includegraphics[width=.39\textwidth]{/agenda/configuracoes/Imagem15.png}
        \caption{}
    \end{figure}
\end{enumerate}
% sections/agenda/dicas_e_boas_praticas.tex
% !TeX root = ../../../main.tex

\section{Dicas e boas práticas do Google Agenda}

\begin{dica}
\begin{itemize}
    \item Utilize cores diferentes para categorizar eventos, facilitando a visualização. Por exemplo: estudo, pessoal, trabalho e lazer;
    \item Considere usar mais de uma agenda para sua rotina: uma específica para tarefas de trabalho e outra para compromissos pessoais;
    \item Adicione descrições relevantes aos eventos, como endereços, links, documentos importantes ou orientações;
    \item Defina lembretes automáticos para seus eventos, ativando notificações por e-mail ou \textit{push} para lembrar-se com antecedência;
    \item Para compromissos recorrentes, utilize a funcionalidade de eventos recorrentes, evitando duplicar manualmente eventos semelhantes;
    \item Integre o Google Agenda ao Gmail para adicionar automaticamente eventos de confirmações recebidas por e-mail;
    \item Ajuste e defina seus horários disponíveis para facilitar o agendamento de compromissos por outras pessoas;
    \item Compartilhe sua agenda com quem necessita verificar sua disponibilidade, fazendo uso desse recurso colaborativo;
    \item Utilize a função “Encontrar um horário” para marcar eventos com várias pessoas, evitando conflitos de agenda;
    \item Ajuste as permissões de visualização da sua agenda, permitindo mostrar apenas disponibilidade ou detalhes completos, conforme sua preferência;
    \item Utilize a busca avançada para localizar eventos específicos em agendas grandes, filtrando por local, data, criador ou outros critérios.
\end{itemize}
\end{dica}

\subsection{comandos}
Ao ativar a opção de atalhos de teclado nas configurações do Google Agenda, é possível executar ações de maneira rápida 
e prática. Comandos de atalho simplificam a realização de tarefas como: criar novos eventos, alternar entre as 
visualizações de dia, semana ou mês, navegar pelos próximos períodos e até o acesso à busca. \newline

\textbf{Principais atalhos:}\newline
\begin{itemize}
    \item \tecla{j} ou \tecla{n} → Alterar a visualização da agenda para o período seguinte;\newline
    \item \tecla{r} → Atualizar a agenda;\newline
    \item \tecla{t} → Ir para o dia atual;\newline
    \item \tecla{+} → Ir para a seção “Adicionar uma agenda”;\newline
    \item \tecla{/} → Posicionar o cursor na caixa de pesquisa;\newline
    \item \tecla{s} → Ir para a página “Configurações”;\newline
    \item \tecla{g} → Ir para uma data específica;\newline
    \item \tecla{1} ou \tecla{d} → Visualização de dia;\newline
    \item \tecla{2} ou \tecla{w} → Visualização de semana;\newline
    \item \tecla{3} ou \tecla{m} → Visualização de mês;\newline
    \item \tecla{4} + \tecla{x} → Visualização personalizada;\newline
    \item \tecla{5} + \tecla{a} → Visualização de compromisso;\newline
    \item \tecla{c} → Criar um novo evento;\newline
    \item \tecla{e} → Ver os detalhes de um evento;\newline
    \item \tecla{Backaspace} ou \tecla{Delete} → Excluir eventos;\newline
    \item \tecla{z} → Desfazer;\newline
    \item \tecla{Ctrl} + \tecla{s} → Salvar o evento a partir da página detalhes;\newline
    \item \tecla{Esc} → Voltar da página de detalhes do evento para a grade da agenda.\newline
\end{itemize}

% sections/agenda/gerenciamento_de_multiplas_agendas.tex
% !TeX root = ../../../main.tex

\section{Gerenciamento de múltiplas agendas}
O Google Agenda possui a funcionalidade de gerenciamento de múltiplas agendas. Confira a seguir como criar agendas, se inscrever em agendas de outros usuários, dentre outras funções.

\subsection{Como criar novas agendas}
A partir da tela inicial do Google Agenda, observe o painel localizado na lateral esquerda da página e localize a seção ``Outras agendas''. Clique no botão ``Adicionar outras agendas'' (ícone de sinal de adição à esquerda do título da seção). 

\begin{figure}[H]
    \centering
    \includegraphics[width=.39\textwidth]{/agenda/gerenciamento_de_multiplas_agendas/Imagem1.png}
    \caption{Botão que abre menu para adição de agenda em destaque}
\end{figure}

Ao clicar neste botão, se abre um menu suspenso com diferentes opções para adicionar agendas ao seu Google Agenda. Clique na opção ``Criar nova agenda''. Ao realizar essa ação, a tela mudará para o formulário de criação de agenda, com os seguintes campos:

\begin{itemize}
    \item \textbf{Nome}: nome da nova agenda;
    \item \textbf{Descrição}: descrição em texto simples da agenda;
    \item \textbf{Fuso horário}: defina o fuso horário da nova agenda de acordo com a necessidade;
    \item \textbf{Proprietário}: campo não editável com o nome do proprietário da agenda;
    \item \textbf{Organização}: este campo não editável é somente exibido se a conta Google proprietária da agenda pertencer a alguma organização.
\end{itemize}

Com as informações da agenda preenchidas, clique em ``Criar agenda''.

\begin{figure}[H]
    \centering
    \includegraphics[width=.39\textwidth]{/agenda/gerenciamento_de_multiplas_agendas/Imagem2.png}
    \caption{Exemplo de criação de agenda com detalhes preenchidos e prestes a ser criada}
\end{figure}

Com isso feito, aguarde o sistema do Google Agenda concluir a criação da nova agenda, a qual será exibida no painel do lado esquerdo da página e os campos do formulário de criação de agenda serão limpos, permitindo a criação de outras agendas se assim desejar.


\subsection{Como se inscrever em outras agendas}
A partir da tela inicial do Google Agenda, clique no botão ``Adicionar outras agendas'' e no menu suspenso clique na opção ``Inscrever-se na agenda''. Realizar essa ação levará o usuário para o formulário de inscrição em agendas de contatos e de pessoas que estejam dentro da mesma organização do usuário.


\subsubsection{Como adicionar agendas de interesse}
Na tela inicial do Google Agenda, clique no botão ``Adicionar nova agenda'' e no menu suspenso clique na opção ``Procurar agendas de interesse''. Ao clicar nesta opção, o usuário será direcionado para uma página com várias opções de agendas para se inscrever de acordo com seus interesses.

\begin{figure}[H]
    \centering
    \includegraphics[width=.39\textwidth]{/agenda/gerenciamento_de_multiplas_agendas/Imagem3.png}
    \caption{Tela de inscrição em agendas de interesse}
\end{figure}

Para se inscrever em qualquer uma das agendas disponíveis, basta clicar na caixa de seleção localizada ao lado esquerdo do nome da agenda. 

Há a possibilidade de visualizar agendas das categorias ``Feriados religiosos internacionais'', ``Esportes'' e ``Outros''. Basta posicionar o mouse em cima da opção desejada, isso fará com que seja exibido um ícone de olho na extremidade direita da opção selecionada, clique nesse ícone para abrir uma nova \gls{aba} do navegador com a agenda selecionada.


\subsection{Como adicionar uma agenda por \gls{url}}\label{subsec:agenda_externa}
Na tela inicial do Google Agenda, clique no botão ``Adicionar nova agenda'' e no menu suspenso clique na opção ``Do \gls{url}''. O usuário será direcionado para a página de adição de agenda por \gls{url}.\@

Esta forma de adição serve para se inscrever em agendas \gls{caldav}.\@ Isto é, se o usuário possui um serviço \gls{online} que provê agenda digital, como um provedor de e-mail, pode ser interessante integrar a agenda hospedada nesse provedor externo, assim possibilita centralizar as agendas em uma única ferramenta.

Entre em contato com o seu provedor de serviços \gls{online} para saber se tens serviço de agenda disponível. Caso positivo, localize na ferramenta de seu provedor o local para acessar a \gls{url} da agenda, copie-a e cole-a no campo ``\gls{url} da agenda'', conforme a imagem abaixo.

\begin{figure}[H]
    \centering
    \includegraphics[width=.39\textwidth]{/agenda/gerenciamento_de_multiplas_agendas/Imagem4.png}
    \caption{Adicionando agenda por \gls{url}}
\end{figure}

Ao colar a \gls{url} no campo citado, será habilitado o botão ``Adicionar agenda''. Antes de clicá-lo, note que há uma caixa de seleção localizada logo acima deste botão, marque-a se deseja tornar a agenda prestes a ser incluída disponível para o público. Com as configurações feitas, clique em ``Adicionar agenda'' para concluir o processo de adição. Realizar esta ação adiciona a agenda na seção ``Outras agendas'' no painel esquerdo da tela inicial do Google Agenda.


\subsection{Importação e exportação de agendas}
O Google agenda possui ferramentas para importação e exportação de agendas. Confira a seguir como acessá-las e usá-las.


\subsection{Como importar uma agenda}
A partir da tela inicial do Google Agenda, clique no botão ``Adicionar nova agenda'' e no menu suspenso clique na opção ``Importar''. O usuário será redirecionado para a tela de importação e exportação de agendas.

Antes de iniciar o processo de importação, primeiro exporte uma agenda hospedada em algum serviço externo à plataforma Google. Para mais informações, consulte a Seção~\ref{subsec:agenda_externa} para mais informações sobre agendas fora do Google Agenda. Com a agenda externa exportada, pode-se prosseguir.

Na seção ``Importar'', clique no botão ``Selecionar \gls{arquivo} no seu computador''. Isso abrirá uma janela de seleção de arquivos. Selecione o \gls{arquivo} a ser importado, selecione a agenda de destino e clique no botão ``Importar''. Logo após será exibido uma caixa de diálogo informando quantos eventos foram importados.

\begin{figure}[H]
    \centering
    \includegraphics[width=.39\textwidth]{/agenda/gerenciamento_de_multiplas_agendas/Imagem5.png}
    \caption{Tela de importação/exportação de agendas com um \gls{arquivo} com eventos prestes a serem importados}
\end{figure}

\begin{figure}[H]
    \centering
    \includegraphics[width=.39\textwidth]{/agenda/gerenciamento_de_multiplas_agendas/Imagem6.png}
    \caption{Janela de diálogo informando quantidade de eventos importados}
\end{figure}

Os eventos importados serão exibidos na tela inicial do Google Agenda, veja exemplo na imagem a seguir.

\begin{figure}[H]
    \centering
    \includegraphics[width=.39\textwidth]{/agenda/gerenciamento_de_multiplas_agendas/Imagem7.png}
    \caption{Eventos importados na agenda}
\end{figure}


\subsection{Como exportar uma agenda}
Para acessar a exportação de agendas, pode-se realizar os mesmos passos para acessar a tela de importação. Também se pode chegar na mesma tela acessando as Configurações do Google Agenda.

\begin{figure}[H]
    \centering
    \includegraphics[width=.39\textwidth]{/agenda/gerenciamento_de_multiplas_agendas/Imagem8.png}
    \caption{Acessando as configurações do Google Agenda}
\end{figure}

Na tela de configurações, clique na opção ``Importar e exportar'' localizada no painel localizado na lateral esquerda da página. Na tela de importação/exportação de agendas, clique no botão ``Exportar''. Executar essa ação fará o \gls{download} de um \gls{arquivo} .\gls{zip} contendo um \gls{arquivo} .\gls{ics} com os eventos das agendas selecionadas. Use este \gls{arquivo} para importar os eventos em uma outra ferramenta de gerenciamento de agendamentos. %chktex 26

% sections/agenda/gerenciamento_entradas_no_calendario.tex
% !TeX root = ../../../main.tex

\section{Gerenciando Entradas no Calendário}
A presente seção irá demonstrar o processo completo para realizar o agendamento de um evento. Para exemplificar, usaremos a seguinte ocasião: ``Marcar uma reunião com colegas de trabalho da faculdade''. Acompanhe, abaixo, como criar esse evento.

\subsection{Como criar um evento}
Na tela inicial, clique no botão ``Criar'' que se encontra na parte esquerda da tela. No menu expansível, selecione a opção ``Evento''.

\begin{figure}[H]
    \centering
    \includegraphics[width=.39\textwidth]{/agenda/gerenciamento_entradas_no_calendario/Imagem1.png}
    \caption{}
\end{figure}

Isso abrirá uma janela sobreposta, com os campos de título, data/hora do evento, etc. Preencha o campo ``Adicionar título'' para definir o título do evento. Em seguida, é necessário definir a data e hora do evento, o que pode ser feito clicando em cima de cada informação (data, horário de início e horário de término) e selecionando a opção desejada dentre as listadas. É possível selecionar ou digitar os valores desejados.

\begin{figure}[H]
    \centering
    \includegraphics[width=.39\textwidth]{/agenda/gerenciamento_entradas_no_calendario/Imagem2.png}
    \caption{Evento com os valores definidos}
\end{figure}

Alternativamente, pode-se definir o evento com duração para o dia inteiro, basta clicar na caixa de seleção ``Dia inteiro''. Ao marcar essa caixa, serão ocultados os seletores de data e horário de início e fim.

Com os dados básicos definidos, já se pode concluir a criação do evento clicando no botão ``Salvar''. A realização dessa ação fechará a janela de cadastro de evento e retornará à tela principal. O evento recém criado poderá ser visto na agenda, na coluna da data na qual ele foi agendado.


\subsubsection{Definindo repetição de evento}
Se o evento que está sendo criado ou editado se repete periodicamente, existe a possibilidade de defini-lo para tal. Para marcar a repetição de um evento, clique no menu suspenso com o texto ``Não se repete''. Observe que abrem-se diversas opções para repetição de evento. Como no exemplo apresentado a situação que mais se ajusta é a marcação de reuniões semanais, foi selecionada a opção ``Semanal''.

\begin{figure}[H]
    \centering
    \includegraphics[width=.39\textwidth]{/agenda/gerenciamento_entradas_no_calendario/Imagem3.png}
    \caption{}
\end{figure}

Com essa opção selecionada, se salvarmos o evento da forma que está agora, serão criadas reuniões todas às quintas-feiras no horário definido, mas sem um fim determinado para a repetição.


\paragraph{Definindo repetição de evento com fim determinado}
Para determinar o fim da repetição de um evento, clique no menu suspenso usado para definir a repetição e selecione a opção ``Personalizar''. Isso abrirá a janela ``Recorrência personalizada'', onde poderá se definir melhor a repetição do evento. No exemplo escolhido, as reuniões serão todas as quintas-feiras do mês de Setembro de 2025, portanto selecionamos o dia 25 de Setembro como data final das reuniões.

\begin{figure}[H]
    \centering
    \includegraphics[width=.39\textwidth]{/agenda/gerenciamento_entradas_no_calendario/Imagem4.png}
    \caption{}
\end{figure}

Com a recorrência definida, clique em ``Concluir'' na janela de ``Recorrência personalizada''.


\subsubsection{Adicionar convidados}
Para adicionar convidados, basta clicar na opção ``Adicionar convidados'' e digitar o nome ou e-mail de quem você deseja convidar. Enquanto estiver digitando, o sistema irá te fornecer sugestões, se encontrar quem quiser convidar na lista exibida, é só clicar no nome do convidado que ele será adicionado ao evento.

\begin{figure}[H]
    \centering
    \includegraphics[width=.39\textwidth]{/agenda/gerenciamento_entradas_no_calendario/Imagem5.png}
    \caption{}
\end{figure}

Note que quando adicionar o primeiro convidado ao evento, o Google Agenda já irá pré-definir uma reunião pelo Google \Gls{meet}. Mais a frente comentaremos sobre configuração de videoconferência.


\paragraph{Permissões de convidados}
Logo abaixo da lista de convidados, temos a opção ``Permissões de convidados'', clicando nela aparecerão caixas de seleção com as quais podemos definir o que os convidados podem fazer com o evento. No exemplo, foi removida a permissão ``Convidar outras pessoas'' e mantida a opção ``Ver lista de convidados''.


\subsubsection{Configurando videoconferência do Google \Gls{meet}}
Logo abaixo das permissões de convidados temos a seção de configurações de reunião pelo Google \Gls{meet}. No lado direito desta seção temos 4 (quatro) botões, em ordem da esquerda para a direita: 

\begin{itemize}
    \item Copiar informações da videoconferência: copia as informações da videoconferência para que você as compartilhe por Whatsapp, por exemplo;
    \item Opções da videochamada: abre uma janela para configurar as opções da videoconferência. As opções são autoexplicativas, portanto não entraremos em detalhes de cada uma delas;
    \item Mais detalhes da videoconferência: exibe informações extras sobre a videoconferência;
    \item Remover conferência: remove as configurações da videoconferência, como o exemplo é uma reunião presencial, essa opção foi usada.
\end{itemize}


\subsubsection{Adicionar local de evento}
Para adicionar o local do evento, clique na opção ``Adicionar local'' e digite o nome ou endereço do local do evento. Veja que, tendo um local selecionado, será exibido um botão ``Visualizar no Maps'', permitindo que você veja o local selecionado através da aplicação Google Maps.

\begin{figure}[H]
    \centering
    \includegraphics[width=.39\textwidth]{/agenda/gerenciamento_entradas_no_calendario/Imagem6.png}
    \caption{}
\end{figure}


\subsubsection{Adicionar descrição}
Para adicionar uma descrição, clique na opção ``Adicionar descrição ou um anexo do Google Drive'', isso abre uma caixa de texto com opções de formatação. Digite nessa caixa a descrição que deseja dar ao evento.


\paragraph{Adicionar ata da reunião}
Se o evento necessitar de uma ata, temos a opção de criar uma a partir do próprio evento, por meio da opção ``Criar ata da reunião''. Caso você tenha clicado por engano na opção, você pode remover a ata clicando no ícone ``X'' que se encontra ao lado do nome do \gls{arquivo} da ata.


\paragraph{Adicionar anexo do Google Drive}
Para adicionar um anexo do Google Drive ao evento, basta clicar na opção ``Adicionar um anexo do Google Drive''. Isso abrirá uma janela para que você selecione o \gls{arquivo} desejado. Para mais informações sobre como operar o Google Drive, leia o capítulo específico que trata dessa ferramenta.

Com os itens selecionados, clique no botão inserir.

Caso tenha anexado algum \gls{arquivo} por engano e queira removê-lo dos anexos, clique no ícone ``X'' ao lado do nome do anexo que pretende remover.

\begin{figure}[H]
    \centering
    \includegraphics[width=.39\textwidth]{/agenda/gerenciamento_entradas_no_calendario/Imagem7.png}
    \caption{}
\end{figure}


\subsubsection{Configurações extras}
Para acessar configurações de disponibilidade, visibilidade e notificações, clique na área da janela logo abaixo da seção onde se configura a descrição do evento. Geralmente essa seção possui o seu nome, conforme a imagem a seguir.

\begin{figure}[H]
    \centering
    \includegraphics[width=.39\textwidth]{/agenda/gerenciamento_entradas_no_calendario/Imagem8.png}
    \caption{}
\end{figure}

Essa seção possui as seguintes opções, apresentadas na próxima sequência de subitens.


\paragraph{Ícone de pasta --- Disponibilidade}
Tem a finalidade de definir sua disponibilidade durante o evento. No caso, pode-se dizer se estará disponível ou ocupado.


\paragraph{Ícone de cadeado --- Visibilidade do evento}
Serve para definir o nível de visibilidade do evento. Ao selecionar diferentes opções deste menu, o ícone de ajuda (ponto de interrogação dentro de um círculo) abrirá um balão de dica (\gls{tooltip}) sobre a opção selecionada.


\paragraph{Ícone de sino --- Notificações}
Podem-se configurar notificações para se lembrar do evento. Por padrão, o Google Agenda emite uma notificação com 10 minutos de antecedência ao horário marcado. Para mudar o horário da primeira notificação, clique no menu suspenso que contém o texto ``10 minutos antes''.

Há a possibilidade de personalizar a notificação selecionando a opção ``Personalizar\ldots'', a qual abre uma janela com mais configurações para a notificação, como a forma de entrega dela e o tempo de antecedência ao evento. 

Podemos configurar mais notificações clicando no botão ``Adicionar notificação''. 

Para remover uma notificação, arraste o mouse para cima da linha da notificação que deseja excluir até que apareça um ícone ``X''. Clique neste ícone para remover a notificação.



\subsubsection{Botão ``Mais opções''}
Para acessar mais opções de criação de agenda, clique no botão ``Mais opções''.

\begin{figure}[H]
    \centering
    \includegraphics[width=.39\textwidth]{/agenda/gerenciamento_entradas_no_calendario/Imagem9.png}
    \caption{}
\end{figure}

Ao clicar nesse botão, a janela sobreposta irá ocupar a página inteira, dando uma visão geral do evento. Essa visualização também permite que se alterem quaisquer campos e opções mencionados anteriormente.

\begin{figure}[H]
    \centering
    \includegraphics[width=.39\textwidth]{/agenda/gerenciamento_entradas_no_calendario/Imagem10.png}
    \caption{}
\end{figure}


\subsubsection{\Gls{aba} ``Encontrar um horário''}
Essa \gls{aba} identifica os períodos livres da sua agenda para que o evento seja agendado de modo a evitar conflito com outros. Também pode-se ver eventos dos cronogramas dos convidados caso estes não estejam privados.


\subsubsection{Concluir criação de evento}
A qualquer momento após ter definido o título do evento, clique no botão ``Salvar'' para criar o evento.

\begin{figure}[H]
    \centering
    \includegraphics[width=.39\textwidth]{/agenda/gerenciamento_entradas_no_calendario/Imagem11.png}
    \caption{}
\end{figure}

\begin{figure}[H]
    \centering
    \includegraphics[width=.39\textwidth]{/agenda/gerenciamento_entradas_no_calendario/Imagem12.png}
    \caption{}
\end{figure}

Se adicionar convidados ao evento, uma janela de diálogo vai pedir se você deseja enviar e-mails de convite. Selecione a opção desejada.

\begin{figure}[H]
    \centering
    \includegraphics[width=.39\textwidth]{/agenda/gerenciamento_entradas_no_calendario/Imagem13.png}
    \caption{}
\end{figure}

Após a criação do evento, o mesmo será exibido na agenda.


\subsection{Como criar uma tarefa}
Este guia irá demonstrar o processo completo para compor o agendamento de uma tarefa. Para isso, usaremos a seguinte situação: ``Marcar um lembrete para ir ao mercado''. Acompanhe como criar a tarefa:

Na tela inicial, clique no botão ``Criar'' que se encontra na parte esquerda da tela. No menu expansível, selecione a opção ``Tarefa''.

\begin{figure}[H]
    \centering
    \includegraphics[width=.39\textwidth]{/agenda/gerenciamento_entradas_no_calendario/Imagem14.png}
    \caption{}
\end{figure}

Isso abrirá uma janela sobreposta muito semelhante à de criação de agenda, embora com menos opções.

Preencha o campo ``Adicionar título'' para definir o título da tarefa. Em seguida, é necessário definir a data e hora da tarefa, que no exemplo é um lembrete. Clique na área que mostra a data atual, isso irá abrir campos para seleção dessas informações. Para selecionar ou digitar os valores desejados, clique nas caixas de seleção de data e de hora de início e fim.

\begin{figure}[H]
    \centering
    \includegraphics[width=.39\textwidth]{/agenda/gerenciamento_entradas_no_calendario/Imagem15.png}
    \caption{}
\end{figure}

Alternativamente, pode-se definir a tarefa para o dia inteiro, basta clicar na caixa de seleção ``Dia inteiro''. Marcando essa caixa, serão ocultados os seletores de data e horário. Com esses dados básicos definidos, já se pode concluir a criação dessa tarefa clicando no botão ``Salvar''.


\subsubsection{Definindo repetição de tarefa}
Como nos eventos, também há a possibilidade de configurar recorrência para tarefas. As opções disponíveis são as mesmas encontradas na criação de eventos.


\subsubsection{Adicionar descrição}
Para adicionar uma descrição à tarefa, basta clicar na caixa de texto onde consta ``Adicionar uma descrição'' e inserir o texto desejado.


\subsubsection{Selecionar a lista de tarefas}
Logo abaixo da descrição da tarefa existe um menu suspenso com o texto ``Minhas tarefas''. Essa é a lista de tarefas padrão do Google Agenda. Se você já tiver outra lista de tarefas cadastrada, pode selecioná-la usando este menu.


\subsubsection{Concluir criação de tarefa}
Com os campos preenchidos, clique no botão ``Salvar'' para criar a tarefa. Ao clicar nesse botão, a tarefa aparecerá em sua agenda.

\begin{figure}[H]
    \centering
    \includegraphics[width=.39\textwidth]{/agenda/gerenciamento_entradas_no_calendario/Imagem16.png}
    \caption{}
\end{figure}


\subsection{Como criar um agendamento de horário.}
Este guia tem por objetivo demonstrar o processo de configuração de agendamentos de horário com reserva. O exemplo utilizado para ilustrar essa função será o da configuração de agendamentos de um médico. Abaixo, seguem os passos necessários para se fazer tal configuração. 

Na tela inicial, clique no botão criar e no menu suspenso clique em agendamento de horários. 

\begin{figure}[H]
    \centering
    \includegraphics[width=.39\textwidth]{/agenda/gerenciamento_entradas_no_calendario/Imagem17.png}
    \caption{}
\end{figure}

Fazendo isso, será aberto um painel à esquerda da tela para a configuração do cronograma de agendamentos. Note que não há todas as opções visíveis nesta etapa, como indicado pelo botão ``Avançar'' no final desse painel. Será discutido brevemente sobre cada uma das seções de ambas as partes deste painel.


\subsubsection{Título do Agendamento e Duração dos Horários}
Primeiro, preencha o campo ``Adicionar título'' para poder identificar a grade de agendamento. Em seguida, tem-se a opção de duração dos horários. Clicando sobre o menu suspenso, abrem-se mais opções de duração, com a possibilidade de personalizar a duração, caso necessário.


\subsubsection{Disponibilidade Geral e Janela de Programação}
Em seguida, temos a seção ``Disponibilidade geral''. Nela é definido em quais dias da semana haverá disponibilidade de agendamento, bem como a recorrência da grade que está sendo configurada. No exemplo, um médico não poderá atender a partir das 11:30 nas terças e quintas-feiras, sendo necessário configurar o ajuste como na imagem a seguir.

\begin{figure}[H]
    \centering
    \includegraphics[width=.39\textwidth]{/agenda/gerenciamento_entradas_no_calendario/Imagem18.png}
    \caption{}
\end{figure}

Logo abaixo, temos a seção ``Janela de programação''. É nela onde se definem início e fim de disponibilidade de reserva, bem como tempos máximo e mínimo de antecedência para marcação de reserva. 


\subsubsection{Disponibilidade Ajustada}
Após configurar a janela de agendamento, temos a configuração de ``Disponibilidade ajustada''. Nela podem ser adicionados ajustes na grade de horários em dias específicos.


\subsubsection{Configurações de Eventos Agendados}
Essa seção conta com opções para configurar intervalo entre os horários, número máximo de agendamentos por dia e permissões para que os convidados possam convidar outras pessoas.


\subsubsection{Verificação de Disponibilidade}
Ao configurar uma grade de horários de agendamento, na sessão ``Agendas'', pode-se deixar horários de agendamento indisponíveis caso estes coincidam com marcações de eventos nas agendas pertencentes a conta Google que se está realizando a configuração. Essa opção vem marcada por padrão, mas pode ser desmarcada, se assim o desejar.


\subsubsection{Definindo Coorganizadores}
Ao organizar uma grade de agendamentos, ao final do painel de configuração, há uma seção denominada ``Coorganizadores''. É por meio dela que se pode gerir quais outros usuários também podem editar a grade.


\subsubsection{Foto e nome da página de agendamento}
Esta seção se torna visível após clicar sobre o botão ``Avançar'' na etapa inicial de criação de grade de agendamentos. Nela pode-se visualizar como será exibida a sua identidade na página de agendamento.


\subsubsection{Local e videoconferência}
Nesta seção é definido o meio pelo qual o atendimento será feito: se será remoto ou presencial. Para definir essa opção, clique no menu suspenso ``Selecionar como e onde fazer uma reunião''.


\subsubsection{Descrição}
Há a possibilidade de adicionar uma descrição para a grade de agendamentos. Com ela, pode-se descrever melhor o serviço/atendimento prestado.


\subsubsection{Personalizar formulário de reserva}
Logo após a seção de descrição da grade, você pode personalizar o formulário de reserva. Por padrão, o Google Agenda pré-configura os campos ``Nome'', ``Sobrenome'' e ``Endereço on-line'', os quais são obrigatórios e não podem ser alterados. Para adicionar itens, basta clicar no botão ``Adicionar um item''. Há também a possibilidade de habilitar obrigatoriedade de verificação de e-mail.


\subsubsection{Lembretes e confirmações}
Nesta seção é onde se configuram os e-mails de lembrete. Por padrão, o Google Agenda pré-configura um e-mail para um dia antes do agendamento. Podem ser adicionados mais lembretes conforme a necessidade.


\subsubsection{Salvando a grade de agendamento}
Para concluir a criação da grade, clique no botão ``Salvar'' que se encontra ao final do painel depois de clicar no botão ``Avançar''. Ao clicar nesse botão, o painel de configuração será fechado e, em seguida, a sua visualização da agenda será atualizada, exibindo os dias de agendamento com uma faixa de destaque. À esquerda da tela será exibido uma caixa com um resumo da grade de agendamentos recém-criada, conforme imagem a seguir.

\begin{figure}[H]
    \centering
    \includegraphics[width=.39\textwidth]{/agenda/gerenciamento_entradas_no_calendario/Imagem19.png}
    \caption{}
\end{figure}


\subsection{Como configurar um evento de ausência}
A seguir será demonstrado como criar uma evento de ausência no Google Agenda. O exemplo que será usado é um evento de ausência por causa de férias. Confira a seguir os passos para criar esse tipo de registro:

Na tela inicial do Google Agenda, clique no botão ``Criar'' e no menu suspenso, clique na opção ``Ausente''. 

\begin{figure}[H]
    \centering
    \includegraphics[width=.39\textwidth]{/agenda/gerenciamento_entradas_no_calendario/Imagem20.png}
    \caption{Acessado a criação de evento de ausência}
\end{figure}

Essa ação abrirá uma janela para configuração do evento de ausência.


\subsubsection{Título, período da ausência, recorrência e recusa de reuniões}
Como na configuração de eventos, aqui também há configurações de título, duração e recorrência nesse tipo de entrada da agenda. 

Note que o período de ausência pode abranger tanto um único dia, em horário específico, quanto vários dias inteiros. Por padrão, o Google Agenda inicia o evento com a data do dia seguinte, abrangendo o dia inteiro. Como exemplo, será marcada uma semana para representar o período de férias do dono da agenda.

\begin{figure}[H]
    \centering
    \includegraphics[width=.39\textwidth]{/agenda/gerenciamento_entradas_no_calendario/Imagem21.png}
    \caption{}
\end{figure}

Veja que logo abaixo da configuração de recorrência há opção de recusar reuniões de forma automática, selecione as opções de acordo com o desejado.


\subsubsection{Mensagem do evento de ausência}
Logo abaixo das configurações de recusa de reuniões, há a configuração da mensagem do evento. Use esse campo para deixar um breve recado do motivo da ausência.


\subsubsection{Visibilidade da ausência}
Seguindo a definição da mensagem, há a configuração do nível de visibilidade do evento de ausência, marcado com um ícone de cadeado. Ao lado direito do menu suspenso desta opção, encontra-se um ícone de ajuda (ponto de exclamação dentro de um círculo). 

Ao passar o mouse em cima dele, é exibido um ``\gls{tooltip}'', que descreve o efeito de cada opção do menu.


\subsubsection{Salvando evento de ausência}
Clique no botão ``Salvar'' para concluir a criação do evento de ausência de acordo com as configurações selecionadas. Realizar essa ação exibirá uma caixa de diálogo pedindo se confirma recusa automática conforme configurado, como exemplificado na imagem abaixo:

\begin{figure}[H]
    \centering
    \includegraphics[width=.39\textwidth]{/agenda/gerenciamento_entradas_no_calendario/Imagem22.png}
    \caption{}
\end{figure}

Clicar no botão ``Salvar e recusar'' fecha a janela de configuração e exibe na agenda as marcações de ausência, caso o período configurado coincida com o da visualização da tela antes de iniciar o processo de criação do evento.

\begin{figure}[H]
    \centering
    \includegraphics[width=.39\textwidth]{/agenda/gerenciamento_entradas_no_calendario/Imagem23.png}
    \caption{Aviso de férias criado}
\end{figure}


\subsection{Como configurar um evento ``Hora de se concentrar''}
Esse tipo de evento serve para avisar às outras pessoas que você estará indisponível e não aceitará interrupções durante o período marcado. Para criar um evento de ``Hora de se concentrar'', na tela inicial do Google Agenda, clique no botão ``Criar'' e, em seguida, em ``Hora de se concentrar''. Essa ação abrirá uma janela como a da imagem a seguir.

\begin{figure}[H]
    \centering
    \includegraphics[width=.39\textwidth]{/agenda/gerenciamento_entradas_no_calendario/Imagem24.png}
    \caption{}
\end{figure}


\subsubsection{Título, data, duração e recorrência}
As mesmas configurações de título, data, duração e recorrência encontradas na marcação de eventos genéricos também aparecem na criação deste tipo de evento. Para mais informações, leia as seções que tratam destas configurações no guia ``Como criar um evento''.


\subsubsection{Silenciar notificações de \gls{chat}}
Logo abaixo das opções de data, duração e recorrência, há a opção ``Não perturbe'' a qual permite bloquear notificações de \gls{chat} do Google \Gls{workspace}. Essa opção é útil caso não queira ser interrompido por alguma mensagem do \gls{chat} do Google.


\subsubsection{Recusa automática de reuniões}
Esta opção já foi explorada anteriormente no guia ``Como configurar um evento de ausência''. Leia o item correspondente para mais explicações sobre esta configuração.


\subsubsection{Adicionar local}
Há, também, a possibilidade de adicionar um local para este tipo de evento. Para mais informações, leia o item que trata desta configuração no guia ``Como criar um evento''.


\subsubsection{Adicionar descrição ou um anexo do Google Drive}
Esta opção é idêntica à encontrada no guia ``Como criar um evento'', confira o item correspondente para mais informações.


\subsubsection{Configurações extras}
Para acessar configurações de visibilidade e notificações, clique na seção da janela logo abaixo da seção onde se configura a descrição do evento. A disponibilidade estará fixa como ``Ocupado''. Para mais detalhes sobre estas configurações, recorra ao item encontrado no guia ``Como criar um evento''.


\subsubsection{Salvando o evento de ``Hora de se concentrar''}
Com as configurações feitas, clique no botão ``Salvar''. 


\textbf{Observação}: Caso tenha marcado para recusar reuniões automaticamente, aparecerá uma caixa de diálogo pedindo se realmente deseja fazer a recusa automática. Clique em ``Salvar e recusar'' para confirmar e concluir a criação do evento.

Com a criação do evento concluída, o mesmo será exibido na agenda caso sua data e hora coincidam com o período selecionado em tela, como observado na imagem a seguir.

\begin{figure}[H]
    \centering
    \includegraphics[width=.39\textwidth]{/agenda/gerenciamento_entradas_no_calendario/Imagem25.png}
    \caption{}
\end{figure}


\subsection{Como configurar um local de trabalho}
Para configurar um local de trabalho, a partir da tela inicial do Google Agenda, clique no botão ``Criar'' e, em seguida, na opção ``Local de trabalho''. Essa ação exibirá uma janela como a mostrada na imagem a seguir.

\begin{figure}[H]
    \centering
    \includegraphics[width=.39\textwidth]{/agenda/gerenciamento_entradas_no_calendario/Imagem26.png}
    \caption{}
\end{figure}


\subsubsection{Configurar Período, Horário e Recorrência}
Pode-se configurar estas opções da mesma forma que outros tipos de eventos. Leia a explicação no guia ``Como criar um evento'' para mais detalhes.


\subsubsection{Escolher local}
Por padrão, o Google Agenda já disponibiliza dois locais: ``Casa'' e ``Escritório''. Para mais opções, clique no botão ``Outros locais'' e selecione a opção desejada no menu suspenso.


\subsubsection{Salvar local de trabalho}
Com as opções selecionadas, clique no botão ``Salvar'' para guardar as alterações. A realização dessa ação fechará a janela de configuração e exibirá o local de trabalho na agenda conforme exemplificado na imagem a seguir.


\begin{figure}[H]
    \centering
    \includegraphics[width=.39\textwidth]{/agenda/gerenciamento_entradas_no_calendario/Imagem27.png}
    \caption{}
\end{figure}


\subsection{Excluir eventos da agenda}
Nas seções a seguir, haverão instruções envolvendo exclusão temporária, recuperação e exclusão definitiva de entradas da agenda.


\subsubsection{Exclusão de registros}
Para excluir qualquer registro da agenda, clique no registro desejado. Essa ação fará com que seja exibida uma janela com um resumo sobre o evento selecionado, conforme a imagem abaixo.

\begin{figure}[H]
    \centering
    \includegraphics[width=.39\textwidth]{/agenda/gerenciamento_entradas_no_calendario/Imagem28.png}
    \caption{}
\end{figure}

Dentro dessa janela, clique no ícone de lixeira. Se o evento for recorrente, será exibida uma janela de diálogo. Selecione a opção desejada e clique em ``OK''.


\subsubsection{Como acessar a lixeira}
Para acessar a lixeira, a partir da tela inicial do Google Agenda, abra o menu de Configurações (ícone de engrenagem na parte superior da página) e clique na opção ``Lixeira''.

\begin{figure}[H]
    \centering
    \includegraphics[width=.39\textwidth]{/agenda/gerenciamento_entradas_no_calendario/Imagem29.png}
    \caption{Caminho para acessar a lixeira}
\end{figure}


\subsubsection{Como recuperar eventos da lixeira}
Para recuperar eventos da lixeira, primeiro acesse-a e a partir deste ponto há algumas formas de recuperação:



\begin{itemize}
    \item \textbf{Recuperar um único evento:}  
    Na linha do evento que se deseja recuperar, clique no ícone de seta torcida para a esquerda, conforme a imagem a seguir.

    \begin{figure}[H]
        \centering
        \includegraphics[width=.39\textwidth]{/agenda/gerenciamento_entradas_no_calendario/Imagem30.png}
        \caption{Observe o ícone em destaque: clicar neste ícone na linha do registro desejado faz com que ele seja restaurado.}
    \end{figure}

    \item \textbf{Recuperar múltiplos eventos:}  
    Para recuperar mais de um evento, selecione os eventos desejados usando as caixas de seleção que se encontram na parte esquerda da lista. Caso deseje restaurar todos os itens, existe uma caixa de seleção na linha de cabeçalho que permite selecionar toda a lista.  
    Ao selecionar qualquer item, aparece um contador de itens selecionados logo acima da lista, acompanhado de botões para restaurar e excluir. Clique no botão de restaurar para recuperar os itens selecionados.

    \begin{figure}[H]
        \centering
        \includegraphics[width=.39\textwidth]{/agenda/gerenciamento_entradas_no_calendario/Imagem31.png}
        \caption{Exemplo de restauração de múltiplos itens. Clicar no ícone em destaque restaura os itens selecionados.}
    \end{figure}

\end{itemize}


\subsubsection{Como excluir eventos definitivamente}
Para excluir definitivamente eventos da lixeira, primeiro acesse-a. A partir deste ponto, há algumas formas de exclusão:

\begin{itemize}
    \item \textbf{Excluir um único evento:}  
    O processo é similar ao de recuperação de um único evento: basta clicar no ícone de lixeira na linha do evento que se deseja excluir;

    \item \textbf{Excluir múltiplos eventos:}  
    O processo é o mesmo da restauração de múltiplos eventos, porém o ícone a ser clicado é o de lixeira;

    \item \textbf{Esvaziar lixeira:}  
    Clique no botão ``Esvaziar lixeira''.  
    Será exibida uma janela de diálogo pedindo confirmação.  
    Clique em ``Esvaziar'' para concluir ou em ``Cancelar'' para manter os itens.

    \begin{figure}[H]
        \centering
        \includegraphics[width=.39\textwidth]{/agenda/gerenciamento_entradas_no_calendario/Imagem32.png}
        \caption{Caixa de diálogo exibida ao clicar no botão ``Esvaziar lixeira''.}
    \end{figure}
\end{itemize}

\begin{dica}
    Itens na lixeira são excluídos automaticamente após 30 dias.
\end{dica}
% ----------- parte II ----------- %


\subsection{Modos de Visualização - Modo Agenda}
O Google Agenda possui alguns modos de visualização para facilitar a programação do usuário. 

Para ver os modos disponíveis, clique no menu suspenso em destaque na imagem a seguir:


\begin{figure}[H]
    \centering
    \includegraphics[width=.39\textwidth]{/agenda/gerenciamento_entradas_no_calendario_pt_II/Imagem0.png}
    \caption{}
\end{figure}


Selecionar um dos modos altera a visualização para o modo correspondente. Veja a seguir uma explicação sobre cada um deles.

\begin{dica}
Vale ressaltar que todos os modos vão levar em consideração o dia selecionado no calendário mensal localizado na lateral esquerda da página. Selecionar qualquer data deste calendário a tornará "data ativa". Tenha isso em mente quando estiver alternando entre os modos.
\end{dica}


\subsubsection{Visualização de Semana}

Este é o modo padrão de visualização do Google Agenda: exibindo a semana da data selecionada. O usuário pode alternar entre as semanas usando as setas na parte superior da tela.

\begin{figure}[H]
    \centering
    \includegraphics[width=.39\textwidth]{/agenda/gerenciamento_entradas_no_calendario/Imagem1.png}
    \caption{Tela inicial do Google Agenda no modo de visualização "semana", note as setas de navegação em destaque}
\end{figure}

Se o usuário estiver em uma semana que não seja a atual e queira voltar a ver a atual, basta clicar no botão "Hoje" localizado ao lado das setas de navegação.


\subsubsection{Visualização de Dia}
Este modo exibe somente um dia por tela. É um modo útil caso haja muitas entradas em um mesmo dia. Por padrão, este modo exibe a data selecionada e a navegação é feita da mesma forma que o modo de visão semanal.


\subsubsection{Visualização de Mês}
Este modo de visualização é semelhante a um calendário mensal tradicional, exibindo o mês inteiro na tela. Quanto à navegação, é realizada da mesma forma que os outros modos de visualização.

\subsubsection{Visualização de Ano}
Ao selecionar este modo de visualização, será apresentado um calendário do ano selecionado. Clicar em qualquer dia deste calendário exibirá um cartão flutuante exibindo os eventos marcados para o dia selecionado. Os controles de navegação seguem inalterados em relação aos outros modos de visualização.

\begin{figure}[H]
    \centering
    \includegraphics[width=.39\textwidth]{/agenda/gerenciamento_entradas_no_calendario/Imagem2.png}
    \caption{Google Agenda no modo de calendário anual}
\end{figure}

\subsubsection{Visualização de Programação}
Selecionar o modo "Programação" exibirá uma lista corrida dos próximos eventos da agenda a partir da data selecionada. Observe na imagem abaixo um exemplo desse modo de visualização:

\begin{figure}[H]
    \centering
    \includegraphics[width=.39\textwidth]{/agenda/gerenciamento_entradas_no_calendario/Imagem3.png}
    \caption{}
\end{figure}

Os controles de navegação alternam dia por dia, exibindo o dia selecionado no topo da lista.


\subsubsection{Visualização de 4 dias}
Este modo exibe quatro dias por vez, a partir da data selecionada. Os controles de navegação neste modo não alteram a data selecionada, apenas alteram a visualização a cada quatro dias para frente ou para trás, conforme o usuário clicar nas setas.


\subsection{Modos de Visualização - Modo Tarefas}
O Google Agenda possui um modo de visualização exclusivo para as tarefas. Confira a seguir como acessar e operar este modo de visualização.

A partir da tela inicial, observe que, logo ao lado do seletor do modo de visualização da agenda, há um botão bipartido em que sua parte esquerda contém o ícone de calendário e, na outra, um círculo com um "tique". Clique nesta segunda parte. Realizando esta ação, o usuário será direcionado para o modo de tarefas.

\begin{figure}[H]
    \centering
    \includegraphics[width=.39\textwidth]{/agenda/gerenciamento_entradas_no_calendario/Imagem4.png}
    \caption{O botão em destaque alterna para o modo de tarefas}
\end{figure}

\begin{figure}[H]
    \centering
    \includegraphics[width=.39\textwidth]{/agenda/gerenciamento_entradas_no_calendario/Imagem5.png}
    \caption{Modo de tarefas ativo, com duas tarefas de exemplo}
\end{figure}

Neste modo, podem ser realizadas as seguintes ações:


\subsubsection{Criar tarefa}
Para criar uma nova tarefa, clique no botão "Adicionar uma tarefa", o qual está localizado no topo da lista de tarefas. Ao clicar nesse botão, será exibido uma caixa de texto para adicionar o título da tarefa, conforme imagem abaixo.

\begin{figure}[H]
    \centering
    \includegraphics[width=.39\textwidth]{/agenda/gerenciamento_entradas_no_calendario/Imagem6.png}
    \caption{}
\end{figure}

Logo abaixo do título, há uma seção para adicionar detalhes para a tarefa em texto simples, esta seção é opcional. 

Depois das seções de título e descrição, há uma barra com opções para configurar a data da tarefa, com os seguintes botões:
\begin{itemize}
    \item \textbf{Hoje:} Marca a tarefa para o dia atual;
    \item \textbf{Amanhã:} Marca para o dia seguinte ao dia de criação da tarefa;
    \item \textbf{Ícone de calendário:} Abre uma janela de diálogo com um calendário para que o usuário selecione a data desejada.
\end{itemize}
    
Ao lado direito da seção de configuração de tarefa há dois botões:
\begin{itemize}
    \item \textbf{Canto superior direito - Botão "Opções":} Este botão estará disponível depois que você definir o título da tarefa. Ele abre um menu suspenso com as seguintes opções principais:
    \begin{itemize}
        \item \textbf{Adicionar a "Com estrela":} Adiciona a tarefa a uma lista de "destaque";
        \item \textbf{Adicionar uma subtarefa:} Abre uma subseção para configurar uma tarefa filha à tarefa selecionada;
        \item \textbf{Excluir:} remove a tarefa selecionada.
    \end{itemize}
    \item \textbf{Canto inferior direito - Botão "Repetir":} Este botão abre uma janela de diálogo para configurar recorrência, sua forma de configuração é semelhante a encontrada ao criar eventos recorrentes no modo "Agenda".
\end{itemize}

Para concluir a criação da tarefa, basta clicar em qualquer lugar fora da seção de cadastro, isso recolherá as opções e exibirá a tarefa na lista.

\begin{figure}[H]
    \centering
    \includegraphics[width=.39\textwidth]{/agenda/gerenciamento_entradas_no_calendario/Imagem7.png}
    \caption{}
\end{figure}

\subsubsection{Editar tarefa} % 9.2
Para editar uma tarefa, basta clicar nela, isso expandirá as opções de configuração da tarefa para que você faça as alterações necessárias.

\subsubsection{Marcar uma tarefa como concluída} % 9.3
Observe que, ao lado esquerdo do título de cada tarefa, há um círculo. Ao passar o mouse por cima, será exibida uma marcação "\gls{tooltip}" com o texto "Marcar como concluída". Clicar nesse elemento marca a respectiva tarefa como concluída e a move para a seção "Concluída", abaixo das tarefas pendentes.

\begin{figure}[H]
    \centering
    \includegraphics[width=.39\textwidth]{/agenda/gerenciamento_entradas_no_calendario/Imagem8.png}
    \caption{Tarefa pendente prestes a ser marcada como concluída.}
\end{figure}

\begin{figure}[H]
    \centering
    \includegraphics[width=.39\textwidth]{/agenda/gerenciamento_entradas_no_calendario/Imagem9.png}
    \caption{Tarefa marcada é movida para seção apartada das pendentes.}
\end{figure}

\subsubsection{Marcar tarefa ``com estrela''} % 9.4
Se deseja dar destaque para uma tarefa específica, posicione o \gls{cursor} do mouse em cima da tarefa desejada. Observe que aparecerão dois botões no lado superior esquerdo da caixa da tarefa: o botão de opções (três pontos dispostos na vertical) e o botão com ícone de estrela - clique neste segundo botão. Esta ação faz com que a tarefa selecionada fique visível não somente na visão "Todas as tarefas", mas também na visão "Com estrela", conforme imagem a seguir.

\begin{figure}[H]
    \centering
    \includegraphics[width=.39\textwidth]{/agenda/gerenciamento_entradas_no_calendario/Imagem10.png}
    \caption{}
\end{figure}

\subsubsection{Excluir tarefa} % 9.5
Para excluir uma tarefa, posicione o \gls{cursor} do mouse sobre a tarefa a ser excluída e clique no botão de opções (ícone de três pontos dispostos verticalmente). Isso abrirá um menu suspenso e, neste menu, clique na opção "Excluir". Realizar esta ação excluirá imediatamente a tarefa da lista, porém, há a possibilidade de desfazer esta ação. Basta clicar no botão "Desfazer", localizado na caixa de diálogo temporária que será exibida no canto inferior esquerdo da página, logo após excluir a tarefa.

\begin{figure}[H]
    \centering
    \includegraphics[width=.39\textwidth]{/agenda/gerenciamento_entradas_no_calendario/Imagem11.png}
    \caption{Tarefa prestes a ser excluída.}
\end{figure}

\begin{figure}[H]
    \centering
    \includegraphics[width=.39\textwidth]{/agenda/gerenciamento_entradas_no_calendario/Imagem12.png}
    \caption{Tarefa excluída, note a opção para desfazer a exclusão em destaque.}
\end{figure}

\subsubsection{Listas de tarefas: Criar, renomear, imprimir e excluir} % 9.6
A visualização de tarefas do Google Agenda possibilita, também, que o usuário consiga gerenciar múltiplas listas de tarefas. Essa funcionalidade permite manter o controle de projetos pessoais que envolvem múltiplas etapas, por exemplo. Leia, a seguir, como gerenciar uma nova lista de tarefas, usando como exemplo a organização no local de trabalho com o método 5S.

\paragraph{Criar uma nova lista} % 9.6.1
Na visualização inicial do modo de Tarefas, observe a lateral esquerda da página e localize a opção "Criar nova lista". Clicar nesta opção abrirá uma janela de diálogo simples para definir o nome da lista. Com o nome definido, clique em "Concluir". Essa ação adiciona a nova lista na visualização inicial da visão de tarefas.

\begin{figure}[H]
    \centering
    \includegraphics[width=.39\textwidth]{/agenda/gerenciamento_entradas_no_calendario/Imagem13.png}
    \caption{Criação de uma nova lista. A opção em destaque abre a janela "Criar nova lista".}
\end{figure}

\paragraph{Renomear uma lista de tarefas} % 9.6.2
Para renomear uma lista de tarefas, clique no botão de opções localizado no canto superior direito da lista desejada (ícone de três pontos na vertical). Ao fazer isso, será exibido um menu suspenso com opções de ordenação das tarefas da lista e, logo abaixo delas, estarão as opções de gerenciamento da lista. Clique em "Renomear lista". 

\begin{figure}[H]
    \centering
    \includegraphics[width=.39\textwidth]{/agenda/gerenciamento_entradas_no_calendario/Imagem14.png}
    \caption{Opção para renomear lista em destaque.}
\end{figure}

Clicar nesta opção abre uma janela de diálogo com um campo para redefinir o nome da lista. Digite o novo nome da lista neste campo e clique em "Concluir".

\paragraph{Imprimir lista de tarefas} % 9.6.3
É provável que o usuário nem sempre tenha acesso à sua conta Google, seja por estar utilizando um dispositivo alternativo desvinculado de sua conta ou, por exemplo, estar realizando trabalhos em um setor onde não tem acesso à dispositivos. Nesses casos, é provável, também, que ele esteja realizando alguma de suas tarefas planejadas. Mas como ter controle sobre isso?  . É nesse momento que a opção “Imprimir lista” pode ser útil. Acompanhe, a seguir, como imprimir uma lista de tarefas.

A partir da visualização de tarefas do Google Agenda, clique no botão de opções da lista desejada. No menu suspenso que se abre, clique na opção "Imprimir lista". Fazendo isso, será aberta uma janela de impressão do navegador do usuário, selecione as opções desejadas e clique em "Imprimir".

\begin{figure}[H]
    \centering
    \includegraphics[width=.39\textwidth]{/agenda/gerenciamento_entradas_no_calendario/Imagem15.png}
    \caption{Acessando a opção para imprimir uma lista de tarefas.}
\end{figure}

\begin{figure}[H]
    \centering
    \includegraphics[width=.39\textwidth]{/agenda/gerenciamento_entradas_no_calendario/Imagem16.png}
    \caption{Pré-visualização de impressão no navegador Mozilla Firefox.}
\end{figure}

\paragraph{Excluir lista de tarefas} % 9.6.4
Para excluir uma lista de tarefas, clique no botão de opções da lista desejada e, no menu suspenso, clique em "Excluir lista". Selecionar esta opção exibirá uma janela de diálogo para confirmar a exclusão definitiva da lista selecionada. Vale destacar que, caso confirme a ação, não haverá opções para desfazê-la, portanto, somente realize este procedimento com as listas que realmente deseja remover.

% --- Seção 10 ---

\subsection{Modo de Pesquisa no Modo Agenda} % 10.
Para entrar no modo de pesquisa no modo Agenda, clique no ícone de lupa localizado na parte superior da página. Observe a imagem a seguir:

\begin{figure}[H]
    \centering
    \includegraphics[width=.39\textwidth]{/agenda/gerenciamento_entradas_no_calendario/Imagem17.png}
    \caption{}
\end{figure}

\subsubsection{Pesquisa por Texto} % 10.1
Ao clicar no ícone, será exibida uma \gls{barrapesquisa} simples na qual o usuário pode pesquisar eventos pelo conteúdo dos seguintes campos:
\begin{itemize}
    \item Título
    \item Nome de convidados
    \item Localização
    \item Descrição
\end{itemize}

Observação: O modo de pesquisa não busca por tarefas, somente eventos.

\paragraph{Opções de Pesquisa} % 10.1.1
Ao clicar na seta localizada na lateral direita da caixa de pesquisa, abrem-se mais opções de pesquisa, tais como:
\begin{itemize}
    \item Pesquisar no: este menu suspenso permite que o usuário selecione uma agenda ou categoria de agendas para ser pesquisada.
    \item O quê: serve para a pesquisa por palavras-chave presentes no evento.
    \item Quem: este campo permite pesquisar eventos pelo nome de convidados, organizador ou criador.
    \item Onde: este campo realiza a busca de eventos por local
    \item Não tem: use este campo para procurar eventos que não tenham as expressões que forem preenchidas nesta caixa.
    \item Data: ao clicar nas caixas "Data de início" e "Data de término", será exibido um calendário mensal do mês selecionado, permitindo pesquisa por eventos compreendidos no intervalo deste filtro.
\end{itemize}

\begin{figure}[H]
    \centering
    \includegraphics[width=.39\textwidth]{/agenda/gerenciamento_entradas_no_calendario/Imagem18.png}
    \caption{Caixa de pesquisa com as opções em exibição. Observe o botão em destaque usado para mostrar as opções.}
\end{figure}

Para executar a pesquisa, clique no botão "Pesquise", localizado abaixo dos campos mencionados.

Na imagem a seguir, temos um exemplo de pesquisa utilizando as opções discutidas anteriormente.

\begin{figure}[H]
    \centering
    \includegraphics[width=.39\textwidth]{/agenda/gerenciamento_entradas_no_calendario/Imagem19.png}
    \caption{}
\end{figure}

\paragraph{Resultado de pesquisa} % 10.1.2
O resultado da pesquisa será exibido de forma semelhante ao modo de visualização "programação", com os eventos em lista crescente por data.

\begin{figure}[H]
    \centering
    \includegraphics[width=.39\textwidth]{/agenda/gerenciamento_entradas_no_calendario/Imagem20.png}
    \caption{Exemplo de resultado de pesquisa a partir dos filtros aplicados na imagem anterior.}
\end{figure}

\paragraph{Sair do modo de pesquisa} % 10.1.3
Para sair do modo de pesquisa, basta usar a tecla "Esc" (Escape) do teclado para retornar ao modo selecionado antes da realização da pesquisa.

%\chapter{Documentos Google}
%% sections/documentos/documentos.tex
% !TeX root = ../main.tex

\section{Introdução ao Documentos Google}
O Google Documentos é uma ferramenta de escrita e edição de texto produzida e 
mantida pela Google, seu uso é online e gratuito à todos os usuários de contas 
Google. O acesso à plataforma é realizado por qualquer navegador (Google Chrome,
 Microsoft Edge, Safari, etc…) através do link 
 \hyperlink{https://docs.google.com/}{https://docs.google.com/}, você precisará 
realizar o login na sua conta Google e já terá acesso à página com alguns 
modelos de documentos pré-prontos. 

\begin{figure}[H]
    \centering
    \includegraphics[width=.9\textwidth]{/documentos/Imagem 9.png}
    \caption{Ao clicar em “Documento em branco”, você será direcionado à um novo 
        arquivo vazio.}
\end{figure}

\begin{figure}[H]
    \centering
    \includegraphics[width=.9\textwidth]{/documentos/Imagem 10.png}
    \caption{}
\end{figure}


A seguir, serão apresentados alguns elementos disponíveis na tela para 
contextualizar algumas funcionalidades básicas do Google Documentos:

\subsubsection{Título do arquivo}
O título do arquivo fica no canto superior esquerdo da página. Para renomeá-lo é 
só clicar no campo e escrever o novo título.

\subsubsection{Menus}
Os menus ficam localizados logo abaixo do título do arquivo. As opções são: 
“Arquivo”, “Editar”, “Ver”, “Inserir”, “Formatar”, “Ferramentas”, “Extensões” e 
“Ajuda”.  Ao clicar nos menus, outros submenus irão aparecer.

\subsubsection{Barra de ferramentas}
A barra de ferramentas fica logo abaixo dos menus. Ao longo da apostila, as 
funcionalidades dos itens da barra de ferramentas serão explicados.


\begin{figure}[H]
    \centering
    \includegraphics[width=.9\textwidth]{/documentos/Imagem 11.png}
    \caption{}
\end{figure}


\subsubsection{Página}
A página do arquivo é o grande campo branco, nele você conseguirá construir o 
seu arquivo. Ela fica no centro da tela, é só clicar com o mouse e começar a 
escrever.

\subsubsection{Ajustar o zoom}
Caso a página esteja muito pequena ou muito grande em sua tela, é possível 
ajustar o zoom seguindo os passos:

\begin{enumerate}
    \item Clicar na opção “Zoom” da barra de ferramentas (É um menu com o texto 
    “100\%”);
    \item Selecionar a opção desejada. Aumente a porcentagem para aumentar o 
    zoom e vice versa.
\end{enumerate}

\begin{dica}
Ajustar o zoom aumenta a aparência das letras no seu computador, mas mantém o tamanho correto para a impressão.
\end{dica}


\subsubsection{Réguas}
As réguas ficam na lateral esquerda e superior da tela.

\begin{checagem}[title=Para exibir as réguas]
    Caso você não esteja vendo as réguas da página, você pode seguir os passos abaixo:
    \begin{enumerate}[leftmargin=*]
      \item Clicar sobre o menu “Ver”;
      \item Clicar sobre a opção “Exibir régua”;
    \end{enumerate}
\end{checagem}

Para exibir as réguas
Caso você não esteja vendo as réguas da página, você pode seguir os passos 
abaixo:
Clicar sobre o menu “Ver”;
Clicar sobre a opção “Exibir régua”;


\subsubsection{Ajustar as margens da página}
Ao mexer a seta azul da direita, você ajusta o tamanho da margem direita, ou 
seja, até onde o texto chega na página.

Ao clicar à esquerda da seta da esquerda, você consegue mover a margem esquerda 
da página inteira.

Ao mexer a seta azul da esquerda, você consegue ajustar a margem esquerda da 
linha que está com o cursor.

Ao mexer somente a barra horizontal azul, você passa para um ajuste mais fino, 
você ajusta a margem esquerda da segunda linha de texto, permitindo que você 
diferencie o começo dos parágrafos, por exemplo.

\begin{figure}[H]
    \centering
    \includegraphics[width=.9\textwidth]{/documentos/Imagem 12.png}
    \caption{}
\end{figure}


%% sections/documentos/menu_inserir.tex
% !TeX root = ../main.tex]

\section{Menu Inserir}


\subsection{Imagens}
Para inserir imagens no arquivo, tudo depende de onde que a imagem está, quando 
trabalhando com imagens que estão na internet ou no computador muitas vezes 
copiar e colar pode ser o suficiente. Porém, quando essa opção não funciona, 
podemos utilizar o menu “Inserir”.

\begin{figure}[H]
    \centering
    \includegraphics[width=.9\textwidth]{/documentos/Imagem 1.png}
    \caption{}
\end{figure}


\subsubsection{Fazer upload do computador}
\begin{enumerate}
    \item Selecionar a opção correspondente no menu “Inserir”;
    \item Localizar a imagem no seu computador;
    \item Clicar em “Abrir”;
\end{enumerate}


\subsubsection{Pesquisar na Web}
\begin{enumerate}
    \item Selecionar a opção correspondente no menu “Inserir”;
    \item Utilizar palavras-chave para buscar a imagem desejada;
    \item Clicar na imagem escolhida;
    \item Clicar no botão “Inserir” que vai aparecer no canto inferior direito;
\end{enumerate}


\subsubsection{Google Drive}
\begin{enumerate}
    \item Selecionar a opção correspondente no menu “Inserir”;
    \item Localizar a imagem no seu drive;
    \item Clicar no botão “Inserir” que vai aparecer no canto inferior direito;
\end{enumerate}


\subsubsection{Google Fotos}
\begin{enumerate}
    \item Selecionar a opção correspondente no menu “Inserir”;
    \item Localizar a imagem no seu Google Fotos;
    \item Clicar no botão “Inserir” que vai aparecer no canto inferior direito;
\end{enumerate}


\subsubsection{Câmera}
\begin{enumerate}
    \item Essa opção poderá ser utilizada quando o aparelho possuir uma câmera.
    \item Selecionar a opção correspondente no menu “Inserir”;
    \item Caso apareça a opção no seu navegador, permitir que o mesmo utilize a 
    câmera para tirar a foto dessa vez;
    \item Capture a foto e clique no botão “Inserir”;
\end{enumerate}


\subsubsection{Por URL}
\begin{enumerate}
    \item Selecionar a opção correspondente no menu “Inserir”;
    \item Colar o endereço da imagem;
    \item Clicar no botão “Inserir” que vai aparecer no canto inferior direito;
\end{enumerate}

\begin{dica}
Caso o endereço que você copiou esteja dando erro, verifique que você selecionou 
a opção “Copiar endereço da imagem” para copiar o endereço correto. 
\end{dica}


\subsubsection{Para ajustar o tamanho da imagem:}
\begin{enumerate}
    \item Selecionar a imagem;
    \item Pressionar os oito quadrados azuis nos vértices e arestas da imagem, 
    e arrastar;
\end{enumerate}

\begin{figure}[H]
    \centering
    \includegraphics[width=.9\textwidth]{/documentos/Imagem 2.png}
    \caption{}
\end{figure}


\subsubsection{Para rotacionar a imagem:}
\begin{enumerate}
    \item Selecionar a imagem;
    \item Pressionar e arrastar o círculo azul que aparece acima da imagem;
\end{enumerate}


\subsubsection{Para cortar a imagem:}
\begin{enumerate}
    \item Clicar com o botão direito na imagem;
    \item Selecionar a opção “Cortar imagem”;
    \item As bordas pretas que aparecem são as margens da imagem, é possível 
        ajustar tanto o tamanho da imagem, como o de suas bordas;
    \item Clique fora da imagem para completar o ajuste;
\end{enumerate}

\begin{figure}[H]
    \centering
    \includegraphics[width=.8\textwidth]{/documentos/Imagem 3.png}
    \caption{}
\end{figure}


\subsubsection{Para remover as alterações de corte na imagem:}
\begin{enumerate}
    \item Clicar com o botão direito na imagem;
    \item Selecionar a opção “Redefinir imagem”;
\end{enumerate}


\subsubsection{Para adicionar texto alternativo:}
\begin{enumerate}
    \item Clicar com o botão direito na imagem;
    \item Selecionar a opção “Texto alternativo”;
    \item Escrever o texto na aba lateral direita que foi aberta;
\end{enumerate}


\subsubsection{Para acessar o mais opções de imagem:}
\begin{enumerate}
    \item Clicar com o botão direito na imagem;
    \item Selecionar a opção “Opções de imagem”;
    \item Abre um menu lateral com mais opções, possui descrição na própria 
        ferramenta;
\end{enumerate}

\begin{figure}[H]
    \centering
    \includegraphics[width=.8\textwidth]{/documentos/Imagem 4.png}
    \caption{}
\end{figure}

\begin{dica}
O texto alternativo serve para acessibilidade, se o seu documento será consumido 
por uma pessoa com deficiência visual que utiliza descrição de áudio, é 
interessante adicionar uma descrição da imagem para ajudar com contextualização.
\end{dica}


\subsection{Tabelas}
As tabelas podem ser adicionadas no documento para ajudar a organizar a informação. 

\subsubsection{Para realizar a inserção de uma tabela no documento:}
Acessar o menu Inserir > Tabela;
Selecionar o tamanho da tabela desejado no esquema de grade;

DICA: Os números abaixo de esquema de grade se refere ao número de [colunas] x [linhas]

[IMAGEM 5]

DICA: Sob o menu “Elemento Básicos”, a ferramenta disponibiliza também alguns modelos prontos. O usuário é incentivado a investigar esses modelos e utilizá-los em parte ou integralmente.

\subsubsection{Para acessar os menus de coluna e linha:}
Ao passar o mouse pelas laterais esquerda e superior da tabela, os menus irão aparecer;

[IMAGEM 6]

\subsubsection{Para inserir uma coluna ou linha:}
É possível adicionar uma coluna ou linha clicando no símbolo de mais [+] nos menus de coluna e linha. A linha será adicionada abaixo da linha que tem o menu aberto. A coluna será adicionada à direita da que tem o menu aberto.

\subsubsection{Para trocar a ordem das colunas e linhas:}
É possível trocar a posição das colunas e linhas ao clicar e segurar o símbolo de seis pontos cinza claro que aparece nos menus de coluna e linha.

\subsubsection{Para fixar e desafixar o cabeçalho da tabela:}
Ao clicar no botão com ícone de tachinha no menu da linha, é possível fixar e desafixar o cabeçalho. É possível fixar mais de uma linha.

DICA: Fixar o cabeçalho, significa que, em tabelas que ocupam mais de uma página, as linhas fixadas vão aparecer no topo de todas as páginas.

DICA: O cabeçalho vai estar fixado quando o ícone de tachinha com a barra transversal estiver aparecendo. De forma a simbolizar que a operação de desafixação vai ser realizada ao clicar novamente.

\subsubsection{Para ordenar a tabela:}
No menu da coluna que se deseja ordenar, clicar no botão com ícone de três barras paralelas;
Selecionar se a ordem deve ser crescente ou decrescente.

DICA: Ordenar uma coluna mantém os valores da linha agrupados. Não edita a tabela, apenas organiza as informações.

\subsubsection{Para alterar as bordas da tabela:}
Ao passar o mouse por uma célula, aparece um botão com ícone de seta no canto superior direito. 
Ao clicar no botão é possível alterar a borda de uma célula ou grupo de células;

DICA: Para ajustes de cor e tamanho da borda, acessar mais configurações da tabela, e abrir o menu “Opções da tabela”. As opções estarão disponíveis sob o título “Cor”.

\subsubsection{Para acessar mais configurações da tabela:}
O menu oferece opções mais específicas de inserção e deleção de linhas e colunas. Assim como as funções também disponibilizadas nos menus de linha e coluna.
Clicar com o botão direito na tabela. Algumas operações são realizadas com relação à célula da tabela que foi clicada.

[IMAGEM 7]

\subsubsection{Descrição das outras opções do menu:}
Inserir linha de título: insere uma linha com uma única célula sobre a tabela;
Dividir célula: permite criar subdivisões de colunas e linhas em uma célula, o texto permanece na célula superior à esquerda.
Mesclar células: aparece na posição do botão anterior, quando mais de uma célula é selecionada.
Desfazer mesclagem de células: aparece no menu após uma mesclagem, reverte a operação;
Definir tipo de coluna: permite que o usuário defina um formato de informação, todos os valores naquela coluna devem ser preenchidos nesse formato.
Distribuir linhas e Distribuir colunas: Após ajustar o tamanho lateral e vertical total desejado, você pode utilizar essas opções para distribuir os espaços da tabela igualmente. É possível também selecionar um grupo de linhas ou colunas para distribuir o espaço.
Opções da tabela: Abre um menu lateral com mais opções, das quais a maioria está bem descrita dentro da própria ferramenta, descrições pertinentes a seguir;
Tabela > Estilo: a primeira opção mantém a tabela como único elemento em uma linha, o texto fica acima e abaixo da tabela. A segunda opção permite que o texto também apareça ao lado da tabela quando a tabela for menos larga que a página.

[IMAGEM 8]

Descrição imagens:
Ressaltar o menu Inserir > Imagem, eu pensei em um quadrado ao redor
 
Ressaltar as opções para imagem (segundo bloco)
Uma seta apontando para o menu lateral
Ressaltar o menu Inserir > Tabela
Duas setas apontando para os menus de linha e coluna
Ressaltar as opções para tabela (de inserir até opções de tabela)
Uma seta apontando para o menu lateral


%\chapter{Planilhas}
%% !TeX root = ../../main.tex
% sections/planilhas/introducao.tex

\section{Introdução}
Uma das ferramentas mais poderosas e conhecidas dentro do \textbf{Workspace}, é o \textbf{Google Planilhas} (conhecido também como \textbf{Google Sheets}). Seja para organizar as finanças pessoais, gerenciar um projeto da faculdade ou analisar dados de um negócio, ele se torna um grande aliado. Nesta seção, será introduzida a ferramenta, juntamente com suas funcionalidades.

\subsection{O que é?}
De maneira simples, uma planilha é um documento composto por linhas e colunas de células. Nessas células é possível inserir textos, números e fórmulas que permitam fazer cálculos, análises e etc.

O Google Planilhas é uma alternativa online e gratuita ao Microsoft Excel, ferramenta que possui ampla base de usuários de longa data. Ele funciona diretamente no navegador, não sendo necessário instalar (o que pode ser uma vantagem em relação ao Excel) e além disso, permite integração com outras ferramentas do Workspace, como o Google Drive e o Google Forms.

\subsection{Para que serve?}

O Google Planilhas é útil para:
\begin{itemize}
	\item Organizar informações pessoais em tabelas de forma clara e estruturada.
	\item Realizar cálculos automáticos com fórmulas e funções matemáticas.
	\item Criar gráficos e dashboards para análise visual de dados.
	\item Compartilhar documentos com edição simultânea por várias pessoas.
	\item Planejamentos pessoais e administrativos, cronogramas de projetos, planos de estudo.
\end{itemize}

\subsection{Exemplos práticos de uso}
Podemos pensar no Google Sheets como um caderno digital inteligente, capaz de realizar tarefas e organizar informações de forma muito eficiente. Abaixo estão alguns exemplos e cenários em que essa poderosa ferramenta pode ser sua aliada:
\begin{itemize}
	\item Orçamento pessoal, anotando as suas despesas do mês, como aluguel e alimentação, além de automatizar cálculos e previsões para você saber para onde seu dinheiro está indo.
	
	\item Planejamento de viagem, criando um roteiro dia a dia com os passeios, endereços e horários, além de controlar os gastos previstos com transporte e hospedagem.
	
	\item Lista de convidados para um evento, elaborando uma lista com o nome de todas as pessoas que você quer convidar e anotando ao lado quem já confirmou presença.
	
	\item Controle de estoque simples, listando todos os seus produtos para saber exatamente quantos itens ainda tem disponíveis para venda.
	
	\item Registro de treinos na academia, anotando os exercícios, pesos e repetições de cada dia, ajudando a visualizar seu progresso ao longo do tempo.
\end{itemize}

\subsection{Primeiros passos}
Assumindo que você já possui uma conta \textbf{Google} (caso não possua, é elucidado anteriormente nesta apostila), há duas principais maneiras de criar uma planilha.

Se já tens conhecimento da ferramenta \textbf{Google Drive} (também elucidada aqui), é interessante que crie seus documentos a partir de lá, pois assim podem ser organizados ao seu diretório pessoal em nuvem, junto aos seus outros documentos. Nada o impede de importá-lo ao seu diretório posteriormente, caso opte por criar o arquivo diretamente no site da ferramenta.

\begin{itemize}
	\item Google Drive (\href{https://drive.google.com/}{drive.google.com}): Acesse (ou crie) o diretório que deseja hospedar o arquivo e no canto superior esquerdo vá no botão “Novo” > “Planilhas Google”.
	
	\item Site da ferramenta (\href{https://sheets.google.com/}{sheets.google.com}): No início da página, clique no botão “+”, rotulado como “Nova planilha em branco”.
	
	\item Bônus: Na barra de endereços do seu navegador em uma nova guia, apenas escreva “\href{https://sheet.new/}{sheet.new}”. Esse link te leva para uma planilha nova rapidamente!
\end{itemize}

Antes de trabalhar com a planilha e os dados em si, vamos primeiramente renomear o arquivo (essencial para que possamos ter um espaço organizado): vá ao canto superior esquerdo e onde está o nome do arquivo (que por padrão é “Planilha sem título”) e simplesmente dê um clique único com o botão esquerdo do mouse e assim, pode reescrevê-lo com o que desejar.

\begin{figure}[htbp]
	\centering
	\includegraphics[width=.9\textwidth]{images/planilhas/imagem_1.png}
	\caption{Interface do Google Planilhas}
	\label{fig:planilhas:introducao1}
\end{figure}


\begin{dica}
	Evite usar espaços no nome do arquivo, para garantir uma melhor compatibilidade com a eventual exportação do arquivo.
\end{dica}


%% !TeX root = ../../main.tex
% sections/planilhas/barra.tex

\section{Entendendo a barra superior}

Para dominar o Google Planilhas o primeiro passo é compreender a sua barra superior. Localizada na parte superior da interface da tela, é o principal centro de controle para todas as ações que você realizará em suas planilhas. É através dela que são organizadas as ferramentas da plataforma de maneira lógica e eficiente.

A barra superior do Google Planilhas é composta por três componentes distintos, cada um com uma finalidade específica para otimizar seu fluxo de trabalho. Compreender a função de cada um é fundamental para navegar de forma clara na ferramenta.   

\begin{figure}[h]
	\centering
	\includegraphics[width=.9\textwidth]{images/planilhas/imagem_2.png}
	\caption{Barra superior do Google Planilhas}
	\label{fig:planilhas:barra1}
\end{figure}

\begin{itemize}
	\item Barra de Menus: A fileira de texto no topo (Arquivo, Editar, Ver, etc.). Nela contém um índice completo de todas as funcionalidades disponíveis na plataforma.   
	
	\item Barra de Ferramentas de Acesso Rápido: A fileira de ícones logo abaixo da Barra de Menus. Ela oferece atalhos visuais para as ações mais comuns e frequentes, como formatação de texto e de números.   
	
	\item Barra de Fórmulas: A área que começa com o símbolo “fx”, sendo o principal local para a edição de fórmulas.   
\end{itemize}

\subsection{Barra de Menus}

A Barra de Menus (Arquivo, Editar, Ver, Inserir, Formatar, Dados, Ferramentas, Extensões, Ajuda) funciona como um mapa completo de tudo que o Google Planilhas pode fazer. Cada menu agrupa comandos relacionados por categoria, tornando a localização de qualquer ferramenta uma tarefa lógica e intuitiva. Mesmo que uma função tenha um atalho visual na barra de ferramentas, sua versão completa e todas as suas opções sempre poderão ser encontradas aqui.

Apesar disso, ações que demandam recorrer constantemente às opções dessa parte da interface, como aplicar negrito ou alterar a cor de uma célula, podem se tornar lentas e trabalhosas. É nesse ponto que a Barra de Ferramentas de Acesso Rápido entra em cena, pois ela é otimizada para eficiência, transformando ações repetitivas em um simples clique.

\subsection{Barra de Ferramentas de Acesso Rápido}

Servindo como uma forma de atalho para ações da Barra de Menus, a Barra de Ferramentas de Acesso Rápido tem a função de agilizar o trabalho. Ao invés de navegar pelos menus, é possível executar formatações e executar comandos com apenas um clique. Aqui se encontram ferramentas para desfazer e refazer ações, imprimir, aplicar formatações de uma célula para outra, controlar o zoom, formatar números como moeda ou porcentagem, alterar o estilo da fonte (negrito, itálico), definir cores de preenchimento e de texto, adicionar bordas e alinhar o conteúdo das células.

\subsection{Barra de Fórmulas}

Identificada pelo ícone “fx”, é uma das partes mais importantes da interface de uma planilha. Ela possui duas funções:
\begin{enumerate}
	\item Exibir o conteúdo real: Uma célula pode exibir um valor, como por exemplo “90”, mas seu conteúdo real pode ser uma função que calcula esse valor, por exemplo “=SOMA(D12:D18)”. A barra de fórmulas sempre mostra o conteúdo real por trás da célula.
	\item Permitir edições precisas: Através dela é possível editar e criar fórmulas longas e complexas com maior facilidade, oferecendo maior clareza e espaço do que a edição direta na célula. 
\end{enumerate}

\subsection{Navegando pela Barra de Menus}

	A seguir um detalhamento de cada menu que responderá a duas questões principais: \textbf{o que são} e \textbf{para que servem} suas principais funcionalidades, explicando com exemplos práticos para facilitar a compreensão e aplicação no dia a dia.

\subsection{Menu Arquivo}
	Este menu concentra todas as ações que afetam o arquivo da planilha como um todo, desde sua criação até o compartilhamento e exportação.

\begin{figure}[h]
	\centering
	\includegraphics[width=.3\textwidth]{images/planilhas/imagem_3.png}
	\caption{Menu arquivo}
	\label{fig:planilhas:barra2}
\end{figure}

\begin{itemize}
	\item \textbf{Novo / Abrir:}
	\begin{itemize}
		\item \textbf{O que são?} O botão “Novo” é para criar uma nova planilha do zero ou a partir de um modelo. O “Abrir” para abrir um arquivo já existente no seu Google Drive.
		\item \textbf{Para que servem?} Iniciar um novo projeto ou continuar um anterior.
		\item \textbf{Como usar?} Clique em \textbf{Arquivo > Novo > Planilha} para um arquivo em branco. Para abrir clique em \textbf{Arquivo > Abrir} e navegue pelos seus arquivos no Google Drive.
	\end{itemize}
	
	\item \textbf{Fazer uma cópia:}
	\begin{itemize}
		\item \textbf{O que é?} Cria um clone exato e independente da planilha atual, mas com um novo nome. 
		\item \textbf{Para que serve?} Recriar planilhas a partir de uma estrutura pronta ao invés de criar tudo do zero novamente. 
		\item \textbf{Exemplo prático?} Você tem uma planilha de "Orçamento Mensal" com toda a estrutura de despesas e receitas. No início de cada mês você vai em \textbf{Arquivo > Fazer uma cópia}, renomeia para "Orçamento - {Novo mês}" e preenche com os novos dados, mantendo o modelo original intacto para os meses seguintes.
	\end{itemize}
	
	\item \textbf{Compartilhar:}
	\begin{itemize}
		\item \textbf{O que é?} Cria um link do arquivo para que outras pessoas tenham acesso a sua planilha de forma on-line. (Para mais informações consultar o capítulo Compartilhamento).
		\item \textbf{Para que serve?} Trabalhar em equipe no mesmo arquivo simultaneamente e em tempo real, controlando quem pode apenas ver, quem pode comentar e quem pode editar o conteúdo.
		\item \textbf{Exemplo prático?} Você cria uma planilha para organizar um churrasco com amigos. Usando o botão "Compartilhar", você envia um link de "Editor" para eles, permitindo que todos adicionem os itens que vão levar e vejam as atualizações dos outros instantaneamente.
	\end{itemize}
	
	\item \textbf{Baixar:}
	\begin{itemize}
		\item \textbf{O que é?} Exporta e salva uma cópia da sua planilha no seu computador em diferentes formatos.
		\item \textbf{Para que serve?} Para compartilhar seu trabalho com pessoas que usam outros softwares - como Microsoft Excel - ou para criar uma versão estática do seu documento, como um PDF.
		\item \textbf{Exemplo prático?} Você precisa enviar um relatório de vendas para um cliente que só utiliza Excel. Você pode ir em \textbf{Arquivo > Baixar > Microsoft Excel (.xlsx)}. Um arquivo.xlsx será baixado e o cliente poderá abri-lo sem problemas.
	\end{itemize}
	
	\item \textbf{Histórico de versões:}
	\begin{itemize}
		\item \textbf{O que é?} Registra todas as alterações feitas na planilha, mostrando quem as fez e quando.
		\item \textbf{Para que serve?} Permite visualizar e restaurar versões anteriores do seu trabalho, sendo uma ferramenta poderosa para reverter erros ou recuperar informações apagadas.
		\item \textbf{Exemplo prático?} Imagine que você deletou acidentalmente uma aba inteira com dados importantes. Em vez de se desesperar, você pode acessar \textbf{Arquivo > Histórico de versões > Ver histórico de versões}. Uma barra lateral mostrará todas as versões salvas. Você pode encontrar a versão de 5 minutos atrás, antes do erro, e clicar em "Restaurar esta versão" para recuperar todo o seu trabalho perdido.
	\end{itemize}
	
\end{itemize}

\subsection{Menu Editar:}
Este menu contém as ferramentas fundamentais para manipular o conteúdo das células.
	
	\begin{itemize}
		\begin{figure}[h]
			\centering
			\includegraphics[width=.3\textwidth]{images/planilhas/imagem_4.png}
			\caption{Menu editar}
			\label{fig:planilhas:barra3}
		\end{figure}
		
		\item \textbf{Desfazer / Refazer:}
		\begin{itemize}
			\item \textbf{O que são?} Comandos básicos para reverter a última ação (Desfazer) ou reaplicar uma ação que foi desfeita (Refazer).
			\item \textbf{Para que serve?} Para corrigir erros de forma instantânea.
		\end{itemize}
		\item \textbf{Recortar / Copiar / Colar:}
		\begin{itemize}
			\item \textbf{O que é?} Ações para mover (Recortar) ou duplicar (Copiar) dados de um local para outro (Colar).
			\item \textbf{Para que serve?} Para reorganizar a estrutura da sua planilha ou replicar informações e fórmulas rapidamente.
		\end{itemize}
		\item \textbf{Colar especial:}
		\begin{itemize}
			\item \textbf{O que é?} Uma versão avançada do comando “Colar” que permite escolher exatamente o que você quer colar de uma célula copiada.
			\item \textbf{Para que serve?} Evitar problemas comuns, como copiar uma fórmula e ela quebrar em um novo local ou copiar um valor e trazer junto uma formatação indesejada.
			\item \textbf{Exemplos práticos:}
			\begin{itemize}
				\item \textbf{Colar apenas os valores:} Você tem uma célula com a fórmula “=SOMA(B2:B10)” que resulta em “R\$1.500,00”. Se você copiar e colar normalmente, a fórmula será ajustada para o novo local. Mas se usar \textbf{Editar > Colar especial > Apenas os valores}, você colará o texto estático “1.500” sem a fórmula ou a formatação de moeda.
				\item \textbf{Colar apenas a formatação:} Você criou um cabeçalho com fundo azul, texto branco e em negrito. Para aplicar este mesmo estilo a outro cabeçalho, copie o original, selecione o novo cabeçalho e vá em \textbf{Editar > Colar especial > Apenas a formatação}. O estilo será aplicado sem alterar o texto.
			\end{itemize}
		\end{itemize}
	\end{itemize}
	
	\subsection{Menu Ver:}
	Controla a aparência da área de trabalho, permitindo personalizar o que é exibido na tela sem alterar os dados.
	\begin{itemize}
		\begin{figure}[h]
			\centering
			\includegraphics[width=.3\textwidth]{images/planilhas/imagem_5.png}
			\caption{Menu editar}
			\label{fig:planilhas:barra4}
		\end{figure}
		\item \textbf{Mostrar:}
		\begin{itemize}
			\item \textbf{O que é?} Submenu que permite ativar ou desativar elementos da interface.
			\item \textbf{Para que serve?} Para limpar a tela e focar no que lhe é importante.
			\item \textbf{Exemplo prático:} Antes de apresentar um dashboard, vá em \textbf{Ver > Mostrar} e desmarque as Linhas de grade. Isso remove as linhas cinzas que separam as células, dando à planilha uma aparência mais limpa e profissional.
		\end{itemize}
		\item \textbf{Congelar:}
		\begin{itemize}
			\item \textbf{O que é?} Ferramenta para travar linhas e colunas no campo de visão enquanto você navega.
			\item \textbf{Para que serve?} Garante que cabeçalhos ou identificadores importantes estejam sempre visíveis.
			\item \textbf{Exemplo prático:} Em uma lista com 200 produtos, clique na linha de cabeçalho da tabela e vá em \textbf{Ver > Congelar > 1 linha}. Assim, os cabeçalhos permanecem visíveis ao rolar a planilha.
		\end{itemize}
		\item \textbf{Zoom:}
		\begin{itemize}
			\item \textbf{O que é?} Ajusta o nível de ampliação da planilha.
			\item \textbf{Para que serve?} Melhora a legibilidade (aumentando o zoom) ou oferece uma visão geral (diminuindo o zoom).
			\item \textbf{Como usar:} Vá em \textbf{Ver > Zoom} e selecione a porcentagem desejada.
		\end{itemize}
	\end{itemize}
	
	\subsection{Menu Inserir:}
	Permite adicionar elementos visuais, interativos e informativos à planilha.
	\begin{itemize}
		\begin{figure}[htbp]
			\centering
			\includegraphics[width=.3\textwidth]{images/planilhas/imagem_6.png}
			\caption{Menu inserir}
			\label{fig:planilhas:barra5}
		\end{figure}
		\item \textbf{Gráfico:}
		\begin{itemize}
			\item \textbf{O que é?} Forma visual de representar dados em colunas, linhas, pizza, etc.
			\item \textbf{Para que serve?} Facilita a análise e a apresentação de informações complexas.
			\item \textbf{Exemplo prático:} Selecione os dados e clique em \textbf{Inserir > Gráfico}. Escolha o tipo desejado para comparar visualmente o desempenho dos produtos.
		\end{itemize}
		\item \textbf{Imagem:}
		\begin{itemize}
			\item \textbf{O que é?} Insere figuras ou ilustrações dentro das células ou na planilha.
			\item \textbf{Para que serve?} Melhora a visualização e contextualização dos dados.
			\item \textbf{Exemplo prático:} Em uma planilha de vendedores, insira a foto de cada pessoa em \textbf{Inserir > Imagem > Na célula}.
		\end{itemize}
		\item \textbf{Link:}
		\begin{itemize}
			\item \textbf{O que é?} Cria um endereço clicável que leva a outra página, documento ou parte da planilha.
			\item \textbf{Para que serve?} Facilita o acesso rápido a informações relacionadas.
			\item \textbf{Exemplo prático:} Em uma planilha de e-commerce, adicione links diretos para cada transação.
		\end{itemize}
		\item \textbf{Caixa de seleção:}
		\begin{itemize}
			\item \textbf{O que é?} Recurso interativo para marcar ou desmarcar opções em uma célula.
			\item \textbf{Para que serve?} Útil para criar listas de verificação e indicar status de tarefas.
			\item \textbf{Exemplo prático:} Crie uma coluna “Status” e insira caixas de seleção em \textbf{Inserir > Caixa de seleção} para marcar tarefas concluídas.
		\end{itemize}
	\end{itemize}
	
	\subsection{Menu Formatar:}
	Permite alinhar, ajustar espaçamentos e aplicar estilos visuais.
	\begin{itemize}
		\begin{figure}[htbp]
			\centering
			\includegraphics[width=.3\textwidth]{images/planilhas/imagem_7.png}
			\caption{Menu formatar}
			\label{fig:planilhas:barra6}
		\end{figure}
		\item \textbf{Tema:}
		\begin{itemize}
			\item \textbf{O que é?} Conjunto de estilos que define aparência da planilha.
			\item \textbf{Para que serve?} Uniformiza cores e fontes.
			\item \textbf{Exemplo prático:} Aplique um tema corporativo com as cores do logotipo da empresa.
		\end{itemize}
		\item \textbf{Número:}
		\begin{itemize}
			\item \textbf{O que é?} Define como valores numéricos são exibidos (moeda, porcentagem, data etc.).
			\item \textbf{Para que serve?} Facilita a compreensão dos dados.
			\item \textbf{Exemplo prático:} Selecione uma coluna e vá em \textbf{Formatar > Número > Moeda} para exibir “R\$” nos valores.
		\end{itemize}
		\item \textbf{Negrito / Itálico / Alinhamento:}
		\begin{itemize}
			\item \textbf{O que é?} Formatações básicas que alteram a aparência e posição do texto.
			\item \textbf{Para que serve?} Destaca informações importantes e melhora a leitura.
			\item \textbf{Exemplo prático:} Aplique negrito aos resultados finais e alinhe colunas numéricas.
		\end{itemize}
		\item \textbf{Formatação condicional:}
		\begin{itemize}
			\item \textbf{O que é?} Aplica estilos automáticos com base em regras definidas.
			\item \textbf{Para que serve?} Destaca dados relevantes e padrões.
			\item \textbf{Exemplo prático:} Use \textbf{Formatar > Formatação condicional} para destacar valores negativos em vermelho.
		\end{itemize}
		\item \textbf{Cores alternadas:}
		\begin{itemize}
			\item \textbf{O que é?} Alterna cores entre linhas.
			\item \textbf{Para que serve?} Facilita a leitura de grandes tabelas.
			\item \textbf{Exemplo prático:} Aplique \textbf{Formatar > Cores alternadas} para diferenciar visualmente cada linha.
		\end{itemize}
	\end{itemize}
	
	\subsection{Menu Dados:}
	Oferece ferramentas para organizar e analisar informações.
	\begin{itemize}
		\begin{figure}[htbp]
			\centering
			\includegraphics[width=.3\textwidth]{images/planilhas/imagem_8.png}
			\caption{Menu dados}
			\label{fig:planilhas:barra7}
		\end{figure}
		\item \textbf{Classificar:}
		\begin{itemize}
			\item \textbf{O que é?} Organiza dados em ordem crescente ou decrescente.
			\item \textbf{Para que serve?} Facilita a localização de informações.
			\item \textbf{Exemplo prático:} Use \textbf{Dados > Classificar de A a Z} para ordenar nomes de funcionários.
		\end{itemize}
		\item \textbf{Criar um filtro:}
		\begin{itemize}
			\item \textbf{O que é?} Exibe apenas dados que atendem a critérios específicos.
			\item \textbf{Para que serve?} Facilita análises sem alterar a planilha.
			\item \textbf{Exemplo prático:} Aplique \textbf{Dados > Criar filtro} para visualizar apenas vendas de um produto.
		\end{itemize}
		\item \textbf{Validação de dados:}
		\begin{itemize}
			\item \textbf{O que é?} Define regras para os valores aceitos em uma célula.
			\item \textbf{Para que serve?} Garante precisão e padronização nos dados.
			\item \textbf{Exemplo prático:} Configure para aceitar apenas números inteiros entre 1 e 10.
		\end{itemize}
	\end{itemize}
	
	\subsection{Menu Ferramentas:}
	Oferece funcionalidades adicionais que expandem as capacidades da planilha.
	\begin{itemize}
		\begin{figure}[htbp]
			\centering
			\includegraphics[width=.3\textwidth]{images/planilhas/imagem_9.png}
			\caption{Menu ferramentas}
			\label{fig:planilhas:barra8}
		\end{figure}
		
		\item \textbf{Criar um formulário:}
		\begin{itemize}
			\item \textbf{O que é?} Cria formulários vinculados à planilha.
			\item \textbf{Para que serve?} Coleta informações automaticamente.
			\item \textbf{Exemplo prático:} Crie um formulário escolar com nome, idade e turma, com respostas registradas na planilha.
		\end{itemize}
		\item \textbf{Verificação ortográfica:}
		\begin{itemize}
			\item \textbf{O que é?} Verifica a ortografia do texto nas células.
			\item \textbf{Para que serve?} Corrige erros de escrita automaticamente.
			\item \textbf{Exemplo prático:} Use para corrigir nomes de cidades digitados incorretamente.
		\end{itemize}
		\item \textbf{Regras de notificação:}
		\begin{itemize}
			\item \textbf{O que é?} Configura alertas para mudanças em planilhas compartilhadas.
			\item \textbf{Para que serve?} Mantém usuários informados sobre alterações.
			\item \textbf{Exemplo prático:} Em \textbf{Ferramentas > Regras de notificação}, configure alertas imediatos ou diários para modificações.
		\end{itemize}
	\end{itemize}
	
	\subsection{Menu Extensões:}
	Permite expandir as funcionalidades do Google Planilhas.
	\begin{itemize}
		\begin{figure}[htbp]
			\centering
			\includegraphics[width=.3\textwidth]{images/planilhas/imagem_11.png}
			\caption{Menu extensões}
			\label{fig:planilhas:barra10}
		\end{figure}
		\item \textbf{Complementos:}
		\begin{itemize}
			\item \textbf{O que é?} Ferramentas extras que adicionam recursos ao Google Planilhas.
			\item \textbf{Para que serve?} Aumenta as possibilidades da ferramenta.
			\item \textbf{Exemplo prático:} Instale complementos para importar dados do Google Analytics ou gerar gráficos personalizados.
		\end{itemize}
		\item \textbf{Apps Script / Macros:}
		\begin{itemize}
			\item \textbf{O que é?} Ferramentas para automatizar tarefas repetitivas ou criar scripts personalizados.
			\item \textbf{Para que serve?} Economizam tempo e reduzem erros manuais.
			\item \textbf{Exemplo prático:} Crie uma macro para formatar tabelas automaticamente ou scripts que enviem relatórios por e-mail.
		\end{itemize}
	\end{itemize}
	
	\subsection{Menu Ajuda:}
	Recurso essencial para encontrar informações, resolver dúvidas e aprender a usar a ferramenta.
	\begin{itemize}
		\begin{figure}[htbp]
			\centering
			\includegraphics[width=.3\textwidth]{images/planilhas/imagem_12.png}
			\caption{Menu ajuda}
			\label{fig:planilhas:barra12}
		\end{figure}
		\item \textbf{Ajuda do Planilhas:}
		\begin{itemize}
			\item \textbf{O que é?} Área de suporte e tutoriais sobre o Google Planilhas.
			\item \textbf{Para que serve?} Ensina funções e soluções para diversos problemas.
			\item \textbf{Exemplo prático:} Use para relembrar funções ou aprender a criar gráficos passo a passo.
		\end{itemize}
		\item \textbf{Pesquisar os menus:}
		\begin{itemize}
			\item \textbf{O que é?} Barra de pesquisa que encontra rapidamente comandos dentro dos menus.
			\item \textbf{Para que serve?} Economiza tempo ao localizar opções sem navegar manualmente.
			\item \textbf{Exemplo prático:} Digite “formatação condicional” na barra e acesse o comando diretamente.
		\end{itemize}
	\end{itemize}

%% !TeX root = ../../main.tex
% sections/planilhas/formatacao.tex

\section{Manipulando a Planilha}
Dando seguimento às partes importantes do Google Planilhas, vamos aprender a trabalhar e manipular a planilha em si. 

\subsection{Inserção de dados}
Podemos iniciar clicando em uma das células dispostas na planilha, o que faz com que ela seja selecionada. Cada célula possui uma coordenada única, formada pela combinação da letra da coluna e do número da linha. Essas coordenadas podem ser identificadas facilmente no campo ao lado da Barra de Fórmulas, apresentada anteriormente.


\begin{figure}[h]
	\centering
	\includegraphics[width=.9\textwidth]{images/planilhas/imagem_12.png}
	\caption{Coluna e linha de uma célula}
	\label{fig:planilhas:formatacao1}
\end{figure}

Após selecionar a célula desejada, basta digitar o valor (sendo texto, número, data ou fórmula) e pressionar Enter. Para editar um conteúdo já existente, clique duas vezes sobre a célula ou edite diretamente na Barra de Fórmulas após selecioná-la.

\subsection{Formatação de células}
Depois de inserir os dados, uma prática indispensável é formatar a planilha. A formatação de células permite alterar aspectos como estilo de texto, cores e bordas, tornando as informações mais legíveis e organizadas. A formatação também ajuda a destacar pontos-chave da tabela, o que facilita a interpretação.

\begin{figure}[h]
	\centering
	\includegraphics[width=.9\textwidth]{images/planilhas/imagem_13.png}
	\caption{Formatação das células}
	\label{fig:planilhas:formatacao2}
\end{figure}

Exemplos práticos de formatação:
\begin{itemize}
	\item Aplicar negrito aos cabeçalhos da tabela.
	\item Alterar a cor de fundo de uma coluna para destacar valores importantes.
	\item Inserir bordas para separar diferentes blocos de informações.
	\item Usar cores no texto para categorizar dados (ex.: vermelho para gastos, verde para receitas).
\end{itemize}

A quantidade de customização fica sempre a seu critério e bom senso. Uma planilha bem organizada visualmente facilita a leitura, mas o excesso pode acabar poluindo a visualização.

Uma boa dica é utilizar o botão “Pintar Formatação” (representado por um ícone de rolo de pintura, na Barra de Ferramentas). Ele permite copiar o estilo de uma célula ou intervalo e aplicar rapidamente em outros locais da planilha, economizando tempo e garantindo a consistência visual.

\subsection{Ajustar colunas e linhas}
Com frequência, os conteúdos não cabem na largura padrão da coluna ou na altura da linha. Mas é possível ajustar manualmente o tamanho posicionando o \gls{cursor} na borda da coluna ou linha até que ele se transforme em uma seta dupla, e então arrastar para expandir ou reduzir.

\begin{figure}[h]
	\centering
	\includegraphics[width=.3\textwidth]{images/planilhas/imagem_14.png}
	\caption{Ajuste das colunas}
	\label{fig:planilhas:formatacao3}
\end{figure}

Outro recurso útil é o ajuste automático, que é feito com um duplo clique na borda da coluna/linha, assim adaptando automaticamente o tamanho em relação ao conteúdo presente.

\subsection{Validação de dados}

Para evitar erros de digitação e manter um padrão de preenchimento, é possível utilizar a validação de dados. Ao aplicar regras de validação a uma célula ou a um intervalo de células, restringimos os tipos de valores aceitos, garantindo maior qualidade e consistência nas informações, especialmente em bases de dados extensas e complexas.

Exemplo prático: em uma planilha de controle de pagamentos, configurar uma lista suspensa que permita apenas as opções “Pago” ou “Pendente”.


\begin{figure}[h]
	\centering
	\includegraphics[width=.3\textwidth]{images/planilhas/imagem_15.png}
	\caption{Tabela com colunas com valores limitados}
	\label{fig:planilhas:formatacao4}
\end{figure}


Para utilizar a ferramenta: Selecione o intervalo desejado; No menu superior, clique em “Dados” > “Validação de dados”; Uma barra lateral de regras abrirá com as seguintes opções:
\begin{itemize}
	\item \textbf{Aplicar ao intervalo:} define as células onde a regra será usada.
	\item \textbf{Critérios:} permite escolher o tipo de regra (no caso, um menu suspenso).
	
	\item \textbf{Itens do menu:} aqui inserimos as opções permitidas, como “Pago” e “Pendente”. É possível até personalizar as cores dos itens.
	
	\item \textbf{Opções avançadas:} você pode mostrar uma mensagem de ajuda ou determinar o que acontece se alguém tentar inserir valores fora da lista.
	
	\item \textbf{Estilo de exibição:} define como a lista será mostrada (ícone, seta ou texto simples).
\end{itemize}


\begin{figure}[h]
	\centering
	\includegraphics[width=.3\textwidth]{images/planilhas/imagem_16.png}
	\caption{Menu de validação de dados}
	\label{fig:planilhas:formatacao5}
\end{figure}

No exemplo, foi marcada a caixa “Rejeitar a entrada” para impedir que valores diferentes de “Pago” ou “Pendente” sejam digitados. Assim, a planilha se torna mais confiável e organizada.

\subsection{Filtros}
Ao trabalhar com grandes volumes de informações, os filtros no Google Planilhas permitem destacar apenas os dados relevantes sem a necessidade de excluir nada da planilha. Eles podem ser aplicados a colunas inteiras, ocultando os registros que não se enquadram nos critérios definidos. Além disso, é possível combinar vários filtros simultaneamente, como por nome e por cidade.

O Google Planilhas oferece diferentes opções de filtragem:
\begin{itemize}
	\item \textbf{Filtrar por cor:} possibilita selecionar dados com base na cor de preenchimento da célula ou na cor do texto. Isso é útil para destacar visualmente informações importantes e depois isolá-las para análise.
	
	\item \textbf{Filtrar por condição:} aplica regras lógicas, como "maior que", "menor que", "contém texto" ou "data anterior a". Essa opção traz flexibilidade para criar filtros dinâmicos, ajustando-se a critérios específicos.
	
	\item \textbf{Filtrar por valores:} permite escolher ou desmarcar valores específicos presentes na coluna. É a forma mais direta de filtrar, pois você define exatamente quais entradas deseja visualizar.	
\end{itemize}

Além dos filtros básicos, existe também a funcionalidade de \textbf{“Visualizações de filtro”}. Ela possibilita que diferentes usuários criem e salvem suas próprias visualizações em uma mesma planilha, sem interferir na exibição dos demais. Esse recurso é especialmente útil em ambientes colaborativos, onde cada pessoa pode analisar os dados de acordo com seus objetivos. Assim, cada usuário pode nomear e alternar entre suas visualizações personalizadas, mantendo a planilha organizada e eficiente, com foco nos dados mais relevantes para cada análise.

\begin{figure}[h]
	\centering
	\includegraphics[width=.3\textwidth]{images/planilhas/imagem_17.png}
	\caption{Menu de configuração da coluna}
	\label{fig:planilhas:formatacao6}
\end{figure}

Por exemplo: em uma planilha de vendas, é possível aplicar um filtro para mostrar apenas os clientes de uma cidade específica ou as vendas realizadas em determinado período.

%% !TeX root = ../../main.tex
% sections/planilhas/formula.tex

\section{Transformando dados em informação}
	Após explorar a interface e as ferramentas de formatação, este capítulo avança para o núcleo funcional do Google Planilhas. As funcionalidades aqui apresentadas permitem tornar os dados inteligentes e interativos.
Serão abordados três tópicos fundamentais:
\begin{itemize}
	\item \textbf{Fórmulas:} Responsáveis por automatizar cálculos e implementar lógicas complexas.
	\item \textbf{Atalhos de teclado:} Utilizados para otimizar o fluxo de trabalho, economizando tempo e esforço em tarefas repetitivas.
	\item \textbf{Ferramentas para organização:} Técnicas para estruturar, gerenciar e proteger dados, garantindo integridade e usabilidade especialmente em projetos colaborativos e com grande volume de dados.
\end{itemize}

\section{Fórmulas essenciais}

Como já dito, as fórmulas permitem a execução de cálculos e a automatização de tarefas que, de outra forma, seriam manuais e propensas a erros. A partir delas que as planilhas começam a se tornar ambientes dinâmicos e não mais estáticos.

\subsection{Entendendo a Anatomia de uma Fórmula}
Toda fórmula no Google Planilhas, sem exceção, começa com o sinal de igual (=). A partir disso, o restante da estrutura será composta pelos seguintes elementos-chave:

\begin{itemize}
	\item \textbf{Funções:} São comandos predefinidos que realizam uma operação específica. Por exemplo “SOMA” ou “MÉDIA”.
	
	\item \textbf{Argumentos:} São os dados que a função utiliza para realizar seu cálculo. Eles são inseridos entre parênteses () e separados por ponto e vírgula (;). Por exemplo, na fórmula “=SOMA(A1;  B1)”, A1 e B1 são os argumentos.
	
	\item \textbf{Referências de célula:} São usadas para referenciar as células que contém os dados que servirão de argumentos, por exemplo “A1”.  Isso torna a fórmula dinâmica; se o valor em A1 mudar, o resultado da fórmula será atualizado automaticamente.
	
	\item \textbf{Intervalos:} Para operar conjuntos contínuos de células, utiliza-se um intervalo indicado por dois pontos (:). Por exemplo, A1:A10 faz referência a todas as células compreendidas no intervalo de A1 até A10.
\end{itemize}

\subsection{Fórmulas Matemáticas e de Contagem}

São as funções mais básicas e frequentemente utilizadas, formando a base para a maioria das análises quantitativas.

\begin{itemize}
	\item \textbf{SOMA:}
	\begin{itemize}
		\item \textbf{O que é?} Uma função que adiciona todos os números em um intervalo de células.
		
		\item\textbf{Para que serve?} Calcular a soma total de intervalos de forma precisa.
		Sintaxe: =SOMA(valor1; valor2; [...])
		
		\item \textbf{Exemplo prático:} Em uma planilha de controle de despesas com valores nas células de C2 a C20, a fórmula “=SOMA(C2:C20)” na célula C21 calcularia o gasto total do período.
	\end{itemize}
	
	\item \textbf{MÉDIA:}
	\begin{itemize}
		\item \textbf{O que é?} Uma função que calcula a média aritmética de um conjunto de números.
		
		\item \textbf{Para que serve?} Encontrar o valor central ou a média aritmética de uma série de dados.\\
		\textbf{Sintaxe:} =MÉDIA(valor1; valor2; [...])
		
		\item \textbf{Exemplo prático:} Para obter a nota média de uma turma de alunos cujas notas estão no intervalo B2:B30, a fórmula “=MÉDIA(B2:B30)” seria a solução.
	\end{itemize}
	
	\item \textbf{MÁXIMO / MÍNIMO:}
	\begin{itemize}
		\item \textbf{O que é?} Duas funções distintas que encontram, respectivamente, o maior (MÁXIMO) e o menor (MÍNIMO) valor numérico dentro de um intervalo.
		
		\item \textbf{Para que serve?} Identificar rapidamente valores extremos em um conjunto de dados.\\
		\textbf{Sintaxe:} =MÁXIMO(intervalo) e =MÍNIMO(intervalo)
		
		\item \textbf{Exemplo prático:} Em um inventário de produtos com preços listados de D2 a D100, “=MÁXIMO(D2:D100)” mostraria o preço do item mais caro enquanto “=MÍNIMO(D2:D100)” identificaria o mais barato.
	\end{itemize}
	
	\item \textbf{CONT.NÚM / CONT.VALORES:}
	\begin{itemize}
		\item \textbf{O que são?} Duas funções de contagem com uma importante diferença. CONT.NÚM conta apenas células que contêm números, enquanto CONT.VALORES conta todas as células que não estão vazias.
		
		\item \textbf{Para que servem?} CONT.NÚM serve para informar o total de registros numéricos existentes em um intervalo. CONT.VALORES retorna o total de entradas preenchidas.\\
		\textbf{Sintaxes:} =CONT.NÚM(valor1; valor2; [...]) e =CONT.VALORES(valor1; valor2; [...])
		
		\item \textbf{Exemplo prático:} Em uma lista de tarefas, “=CONT.VALORES(A2:A100)” contaria quantas tarefas foram descritas. Se a coluna B tivesse as datas de conclusão, “=CONT.NÚM(B2:B100)” contaria quantas tarefas já foram finalizadas (assumindo que apenas as concluídas têm data).
	\end{itemize}
\end{itemize}
	
\subsection{Fórmulas de Pesquisa:}
Funções que automatizam a busca por dados correspondentes entre listas e tabelas.
	\begin{itemize}
	\item \textbf{PROCV:}
	\begin{itemize}
		\item \textbf{O que é?} Uma ferramenta de busca que procura por um valor específico em uma coluna da tabela e retorna o valor correspondente de uma coluna diferente na mesma linha.
		
		\item \textbf{Para que serve?} Automatizar a busca e o cruzamento de informações entre listas.\\
		\textbf{Sintaxe:} =PROCV(chave\_de\_pesquisa; intervalo; índice; [classificado])
		
		\item \textbf{Exemplo prático:} Imagine duas tabelas: uma com “ID do Produto” e “Quantidade Vendida” e outra com “ID do Produto”, “Nome do Produto” e “Preço”. Para descobrir o preço de um item vendido na primeira tabela, você usaria o PROCV. A fórmula “=PROCV(A2; 'Tabela de Preços'!A:C; 3; 0)” procuraria o ID do produto da célula A2 na tabela de preços e retornaria o valor da terceira coluna (o preço).
	\end{itemize}
\end{itemize}

	
\subsection{Fórmulas Lógicas:}
Fórmulas que permitem tomadas de decisão automáticas pela planilha, retornando diferentes resultados com base em condições predefinidas.
	\begin{itemize}
	
	\item \textbf{SE:}
	\begin{itemize}
		\item \textbf{O que é?} Uma função que avalia um teste lógico, retornando um valor se a condição for verdadeira e outro valor se for falsa.
		
		\item \textbf{Para que serve?} Automatizar respostas e classificações com base em determinados critérios.\\
		\textbf{Sintaxe:} =SE(condição; valor\_se\_verdadeiro; valor\_se\_falso)
		
		\item \textbf{Exemplo prático:} Em uma planilha de notas de alunos com a nota final na célula C2, a fórmula “=SE(C2>=7; 'Aprovado'; 'Reprovado')” exibiria automaticamente o status do aluno.
	\end{itemize}
	
	\item \textbf{E / OU:}
	\begin{itemize}
		\item \textbf{O que são?} Funções auxiliares frequentemente utilizadas dentro da fórmula “SE” para testar múltiplas condições simultaneamente. A fórmula “E” retorna VERDADEIRO somente se todas as condições forem atendidas, enquanto a fórmula “OU” retorna VERDADEIRO se pelo menos uma das condições for atendida.
		
		\item \textbf{Para que servem?} Criar testes lógicos mais complexos e com múltiplos critérios.\\
		\textbf{Sintaxes:} =E(condição1; condição2; [...]) e =OU(condição1; condição2; [...])
		
		\item \textbf{Exemplo prático:} Para conceder um bônus a um vendedor que atingiu a meta de vendas de R\$10.000 (célula B2) e tem mais de 2 anos na empresa (célula C2), a fórmula seria: “=SE(E(B2>10000; C2>2); 'Bônus Concedido'; 'Sem Bônus')”.
	\end{itemize}
\end{itemize}
	
\section{Atalhos de teclado:}

Embora a Barra de Ferramentas de Acesso Rápido ofereça ícones para as ações mais comuns, o próximo passo em relação a produtividade e otimização de tempo são os atalhos de teclado. O uso de atalhos minimiza a necessidade de alternar entre o teclado e o mouse, permitindo que o usuário mantenha um fluxo de trabalho contínuo e mais rápido.

Para visualizar a lista completa de atalhos a qualquer momento, basta pressionar  \tecla{Ctrl + /} (em Windows) ou \tecla{\cmd  + /} (em macOS).

\subsection{Tabelas de atalhos essenciais}
As tabelas a seguir apresentam os principais atalhos para o dia a dia, organizados por função e sistema operacional, servindo como uma referência prática para o uso diário.

\section{Atalhos Universais Mais Importantes}
\begin{table}[!ht]
	\centering
	\begin{tabular}{@{}lll@{}}
		\toprule
		\textbf{Ação} & \textbf{Atalho (Windows)} & \textbf{Atalho (macOS)} \\ 
		\midrule
		Copiar & Ctrl + C & \cmd + C \\
		Colar & Ctrl + V & \cmd + V \\
		Colar Apenas Valores & Ctrl + Shift + V & \cmd + Shift + V \\
		Recortar & Ctrl + X & \cmd + X \\
		Desfazer & Ctrl + Z & \cmd + Z \\
		Refazer & Ctrl + Y & \cmd + Y \\
		Negrito & Ctrl + B & \cmd + B \\
		Itálico & Ctrl + I & \cmd + I \\
		Inserir \Gls{link} & Ctrl + K & \cmd + K \\
		Selecionar Tudo & Ctrl + A & \cmd + A \\
		Salvar & Ctrl + S & \cmd + S \\
		\bottomrule
	\end{tabular}
	\caption{Principais atalhos universais utilizados nas planilhas.}
	\label{tab:atalhos_universais}
\end{table}

\section{Atalhos de Navegação e Seleção}
\begin{table}[!ht]
	\centering
	\begin{tabular}{@{}lll@{}}
		\toprule
		\textbf{Ação} & \textbf{Atalho (Windows)} & \textbf{Atalho (macOS)} \\ 
		\midrule
		Selecionar coluna inteira & Ctrl + Espaço & \cmd + Espaço \\
		Selecionar linha inteira & Shift + Espaço & Shift + Espaço \\
		Ir para o início da linha & Home & Fn + Seta para a esquerda \\
		Ir para o fim da linha & End & Fn + Seta para a direita \\
		Ir para o início da planilha & Ctrl + Home & \cmd + Fn + Seta para a esquerda \\
		Ir para o fim da planilha & Ctrl + End & \cmd + Fn + Seta para a direita \\
		\bottomrule
	\end{tabular}
	\caption{Atalhos de navegação e seleção de células em planilhas.}
	\label{tab:atalhos_navegacao}
\end{table}

\section{Organização de dados}

A organização dos dados é um dos pontos mais importantes em qualquer planilha. Recursos como ordenação, mesclagem, congelamento e proteção de intervalos tornam o trabalho mais estruturado e seguro, especialmente em projetos colaborativos.

\subsection{Ordenação de Dados}
Ordenar dados é essencial para tornar informações mais legíveis e fáceis de analisar. No Google Sheets, é possível organizar os valores em ordem crescente (A a Z, 0 a 9) ou decrescente (Z a A, 9 a 0).

\begin{itemize}
	\item \textbf{Como ordenar:} Selecione o intervalo de dados ou a coluna desejada, vá até o menu Dados > Classificar página e escolha entre A-Z (crescente) ou Z-A (decrescente).
	\item \textbf{Exemplo prático:} Em uma lista de alunos com as notas na coluna B, ao ordenar de maior para menor, os primeiros registros exibem automaticamente os melhores resultados.
\end{itemize}

\begin{figure}[h]
	\centering
	\includegraphics[width=.5\textwidth]{images/planilhas/imagem_18.png}
	\caption{Exemplo de tabela}
	\label{fig:planilhas:formula1}
\end{figure}

\subsection{Mesclar Células}
A mesclagem combina duas ou mais células em uma única, geralmente usada para títulos ou destaques visuais.

\begin{itemize}
	\item \textbf{Como mesclar:} Selecione as células, clique em Formatar > Mesclar células, então escolha a opção desejada: mesclar tudo, mesclar horizontalmente ou mesclar verticalmente.
	\item \textbf{Atenção:} ao mesclar, somente o conteúdo da célula superior esquerda é mantido; os outros dados são descartados.
	\item \textbf{Exemplo prático:} Em uma tabela de controle mensal, pode-se mesclar as células A1 até D1 para criar um título centralizado chamado “Relatório de Vendas - Setembro”.
\end{itemize}

\begin{figure}[h]
	\centering
	\includegraphics[width=.6\textwidth]{images/planilhas/imagem_19.png}
	\caption{Exemplo de tabela com células mescladas}
	\label{fig:planilhas:formula2}
\end{figure}

\subsection{Congelar Linhas e Colunas}

O recurso de congelar mantém linhas ou colunas fixas na tela durante a rolagem da planilha. Isso facilita a leitura em planilhas extensas.
\begin{itemize}
	\item \textbf{Como congelar:} Selecione a linha ou coluna que deseja manter visível, vá até o menu Ver > Congelar, escolha entre 1 linha, 2 linhas, até a linha atual, ou equivalente para colunas.
	\item \textbf{Exemplo prático:} Em uma planilha de inventário com centenas de produtos, congelar a linha 1 (cabeçalhos) garante que os títulos das colunas, como “Produto” e “Preço”, fiquem sempre visíveis ao rolar a tela.
	
	\begin{figure}[h]
		\centering
		\includegraphics[width=.6\textwidth]{images/planilhas/imagem_20.png}
		\caption{Exemplo de tabela com linha congelada}
		\label{fig:planilhas:formula3}
	\end{figure}
\end{itemize}

\subsection{Proteger Intervalos}
A proteção de intervalos evita alterações indesejadas em partes específicas da planilha. Isso é útil em ambientes colaborativos, garantindo que apenas usuários autorizados possam editar determinadas áreas.

\begin{itemize}
	\item  \textbf{Como proteger:} Selecione o intervalo ou célula, clique em Dados > Proteger páginas e intervalos e defina quem pode editar (apenas você ou pessoas específicas).
	
	\item \textbf{Exemplo prático:} Em um relatório financeiro compartilhado, o intervalo contendo fórmulas de cálculo pode ser protegido para que apenas o administrador consiga alterar, enquanto os demais colaboradores podem inserir dados em outras células normalmente.	
\end{itemize}

	\begin{figure}[h]
		\centering
		\includegraphics[width=.6\textwidth]{images/planilhas/imagem_21.png}
		\caption{Janela indicando tentativa de edição em célula protegida}
		\label{fig:planilhas:formula4}
	\end{figure}

%% !TeX root = ../../main.tex
% sections/planilhas/graficos.tex

\section{Transformando dados em visualizações}
Após a organização e a formatação dos dados em uma planilha, o próximo passo para extrair valor real dessas informações é a visualização. Os gráficos traduzem a complexidade dos dados em um formato visual que o cérebro pode processar de forma muito mais rápida e intuitiva.

No entanto, a eficácia de um gráfico depende da escolha do tipo correto para os dados e para a visualização que se deseja obter. Cada tipo de gráfico tem uma finalidade específica. A tabela abaixo serve como um guia rápido para ajudar na seleção do gráfico mais apropriado para sua necessidade.

\begin{table}[!ht]
	\centering
	\resizebox{\textwidth}{!}{%
		\begin{tabular}{@{}p{2.5cm}p{4cm}p{5cm}p{4.5cm}@{}}
			\toprule
			\textbf{Tipo de Gráfico} & \textbf{Finalidade Principal} & \textbf{Exemplo de Pergunta que Responde} & \textbf{Estrutura de Dados Ideal} \\
			\midrule
			Coluna & Comparar valores entre categorias distintas. & “Qual produto vendeu mais em janeiro?” & Uma coluna para categorias (texto) e colunas subsequentes para valores (números). \\
			Linha & Mostrar a evolução de dados ao longo do tempo. & “Como nossas vendas mensais mudaram ao longo do último ano?” & Uma coluna para o eixo do tempo (datas, meses) e colunas subsequentes para os valores a serem acompanhados. \\
			Pizza & Exibir a proporção de cada categoria em um todo. & “Qual a porcentagem de nosso orçamento é gasta com marketing?” & Uma coluna para categorias (texto) e uma única coluna para os valores correspondentes (números). \\
			\bottomrule
		\end{tabular}
	}
	\caption{Relação entre tipos de gráficos, suas finalidades e estrutura de dados recomendada.}
	\label{tab:tipos_graficos}
\end{table}

\subsection{Criando seu primeiro gráfico}

A criação de um gráfico no Google Sheets é um processo direto, onde a qualidade e a clareza do seu gráfico final são um reflexo direto da organização da sua tabela de origem.

\subsubsection{Passo 1: Seleção de dados}
Antes de inserir um gráfico, é fundamental garantir que seus dados estejam estruturados de forma lógica e clara. A prática recomendada é organizar os dados em colunas com um cabeçalho claro e descritivo na primeira linha de cada coluna. Por exemplo, em uma tabela de vendas, a primeira coluna pode ser "Mês", a segunda "Produto A" e a terceira "Produto B".

Após certificar que os dados estão devidamente estruturados, é preciso selecionar o intervalo de células que você deseja visualizar. Clique na primeira célula do seu conjunto de dados e arraste o mouse para incluir todas as linhas e colunas relevantes, incluindo os cabeçalhos.

\subsubsection{Passo 2: Inserindo o gráfico}
Com o intervalo de dados selecionado, navegue até a Barra de Menus e clique em \textbf{Inserir > Gráfico}.

\begin{figure}[h]
	\centering
	\includegraphics[width=.6\textwidth]{images/planilhas/imagem_26.png}
	\caption{Inserção do gráfico}
	\label{fig:planilhas:graficos1}
\end{figure}

O Google Planilhas analisará os dados selecionados e inserirá automaticamente um tipo de gráfico que ele considera mais apropriado. Um painel lateral chamado Editor de Gráficos também aparecerá à direita da tela. É através deste editor que todo o controle sobre o gráfico é exercido.

\subsection{Conhecendo o Editor de Gráficos}

O Editor de Gráficos é dividido em duas abas principais, a primeira serve para definir a estrutura e os dados, a segunda para definir a aparência.

\subsubsection{Aba Configuração}

Nesta aba se encontram os fundamentos do gráfico, definindo o tipo de gráfico, o intervalo de dados e como as colunas e linhas são usadas para os Eixos e as Séries. É o local para garantir que os dados estão sendo representados corretamente.

\subsubsection{Aba Personalizar}

Nesta aba se encontram todos os aspectos estéticos do gráfico, como cores, fontes, títulos, legendas, linhas de grade e etc. É aqui que um gráfico funcional se transforma em uma visualização de fácil leitura.

\section{Principais tipos de gráficos}

Apesar de o Google Planilhas apresentar diversos tipos de gráficos, três deles formam a base para a grande maioria dos gráficos: Coluna, Linha e Pizza. É importante notar que, embora visualmente distintos, a estrutura de dados deles segue um padrão comum: colunas que contêm categorias ou rótulos (texto, datas) e colunas que contêm valores (números).


\subsection{Gráfico de Coluna}
\begin{itemize}
	\item \textbf{O que é?} Uma categoria de gráficos com colunas dispostas na vertical, ideal para comparar valores entre diferentes categorias discretas.
	\item \textbf{Para que serve?} Perfeito para responder a perguntas como “Qual vendedor teve o melhor desempenho?” ou “Qual foi a receita em cada trimestre?”. 
	\item \textbf{Estrutura de dados ideal:} Os dados devem ser organizados com as categorias a serem comparadas em uma coluna (que se tornará o eixo horizontal, ou eixo X) e os valores numéricos correspondentes em colunas adjacentes (que formarão as barras no eixo vertical, ou eixo Y).
\end{itemize}

\begin{figure}[h]
	\centering
	\includegraphics[width=.6\textwidth]{images/planilhas/imagem_22.png}
	\caption{Exemplo de gráfico de coluna}
	\label{fig:planilhas:graficos2}
\end{figure}


\subsection{Gráfico de Linha}
\begin{itemize}
	\item \textbf{O que é?} Uma categoria de gráficos que conecta pontos de dados com uma linha, ideal para identificar tendências, padrões, flutuações e aceleração ou desaceleração em uma métrica. 
	\item \textbf{Para que serve?} Perfeito para responder perguntas como “Nossa base de usuários está crescendo?” ou “Como a temperatura variou ao longo do dia?”.
	\item \textbf{Estrutura de dados ideal:} Uma primeira coluna com os dados do eixo do tempo (dias, meses, anos) e as colunas seguintes com os valores numéricos que você deseja acompanhar ao longo desse tempo.
\end{itemize}

\begin{figure}[h]
	\centering
	\includegraphics[width=.6\textwidth]{images/planilhas/imagem_23.png}
	\caption{Exemplo de gráfico de linha}
	\label{fig:planilhas:graficos3}
\end{figure}


\subsection{Gráfico de Pizza}
\begin{itemize}
	\item \textbf{O que é?} Uma categoria de gráficos onde um círculo é subdividido em partes individuais que se relacionam com um todo. Cada parte representa uma categoria, e o tamanho da fatia é proporcional à sua porcentagem do total. 
	\item \textbf{Para que serve?} Ideal para responder a perguntas como “Qual porcentagem do nosso tráfego vem de cada rede social?” ou “Como nosso orçamento está dividido entre os departamentos?”.
	\item \textbf{Estrutura de dados ideal:} Uma coluna para os nomes das categorias e uma única coluna adjacente com seus valores numéricos correspondentes. 
\end{itemize}

\begin{figure}[h]
	\centering
	\includegraphics[width=.6\textwidth]{images/planilhas/imagem_24.png}
	\caption{Exemplo de gráfico de pizza}
	\label{fig:planilhas:graficos4}
\end{figure}

\subsection{Personalização Avançada}
Após a criação de um gráfico básico, o Google Sheets oferece uma vasta gama de opções de personalização para refinar a aparência e a clareza da visualização.  
O \textbf{Editor de Gráficos}, acessível no painel lateral, é a ferramenta central para todas essas modificações, dividindo as opções em duas abas principais: \textbf{Configurações} e \textbf{Personalizar}.  
Enquanto a aba de \textbf{Configuração} lida com a estrutura e os dados, a aba de \textbf{Personalizar} é onde a mágica estética realmente acontece, transformando um gráfico funcional em uma ferramenta de comunicação visual impactante.

\subsubsection{Estilo do Gráfico}
O estilo geral define a primeira impressão do gráfico. Nesta seção do Editor de Gráficos, é possível ajustar elementos como o tipo de gráfico, cores de fundo, bordas e fonte padrão.  

\begin{itemize}
	\item \textbf{Tipo de Gráfico:} Embora o Google Sheets sugira um tipo inicial, é possível alterá-lo para explorar outras representações visuais.
	\item \textbf{Cor de Fundo:} Define a cor de plano de fundo da área do gráfico.
	\item \textbf{Borda do Gráfico:} Adiciona uma borda ao redor, destacando-o do restante do conteúdo.
	\item \textbf{Fonte:} Afeta a legibilidade de todos os textos do gráfico, incluindo títulos, rótulos e legendas.
\end{itemize}

\begin{figure}[h]
	\centering
	\includegraphics[width=.6\textwidth]{images/planilhas/imagem_25.png}
	\caption{Personalização de gráficos}
	\label{fig:planilhas:graficos5}
\end{figure}

\subsubsection{Títulos, Eixos e Legendas}
Esses elementos são cruciais para contextualizar e interpretar o gráfico.  
\begin{itemize}
	\item \textbf{Título do Gráfico:} Deve ser curto, claro e indicar o principal insight.
	\item \textbf{Títulos dos Eixos:} Identificam o que está sendo representado, como “Meses”, “Vendas (R\$)” ou “Número de Clientes”.
	\item \textbf{Legendas:} Identificam séries de dados por meio de cores ou padrões e podem ter posição, fonte e tamanho personalizados.
\end{itemize}

\subsubsection{Formatando as Séries de Dados}
As séries de dados são o elemento central do gráfico.  
\begin{itemize}
	\item \textbf{Cor da Série:} Altere as cores das barras, linhas ou fatias de pizza.
	\item \textbf{Estilo da Linha ou Marcadores:} Ajuste espessura, tipo de traço e marcadores (círculos, quadrados etc.).
	\item \textbf{Rótulos de Dados:} Exibem valores diretamente no gráfico, devendo ser usados com moderação.
	\item \textbf{Linha de Tendência:} Evidencia padrões ou projeções (linear, exponencial, polinomial etc.).
\end{itemize}

\begin{figure}[h]
	\centering
	\includegraphics[width=.6\textwidth]{images/planilhas/imagem_30.png}
	\caption{Personalização da série de dados}
	\label{fig:planilhas:graficos6}
\end{figure}

\subsubsection{Linhas de Grade e Marcas}
As linhas de grade e marcas de escala facilitam a leitura e interpretação dos valores.  
\begin{itemize}
	\item \textbf{Linhas de Grade:} Ajudam na estimativa de valores. Podem ser principais ou secundárias, com cor e estilo personalizáveis.
	\item \textbf{Marcas de Escala (ticks):} Traços posicionados nos eixos para indicar valores correspondentes. Ajustáveis em posição, comprimento e frequência.
\end{itemize}

%% !TeX root = ../../main.tex
% sections/planilhas/tabelas.tex

\section{Tabelas Dinâmicas}
Após a inserção e organização dos dados em uma planilha, uma das formas mais eficientes de analisar informações e identificar padrões é utilizar as tabelas dinâmicas. Esse recurso permite organizar, resumir e filtrar grandes volumes de dados de maneira flexível, facilitando a visualização de resultados sem a necessidade de fórmulas complexas.

As tabelas dinâmicas são especialmente úteis em contextos de relatórios financeiros, controle de despesas, acompanhamento de desempenho e quaisquer cenários em que seja necessário cruzar informações de forma dinâmica.

\subsection{Criando uma Tabela Dinâmica}
O processo de criação de uma tabela dinâmica no Google Planilhas é simples e intuitivo. A ferramenta automaticamente reconhece o intervalo de dados e oferece opções para definir como as informações serão resumidas e exibidas.

\subsubsection{Passo 1: Seleção dos dados}
Antes de criar a tabela dinâmica, é importante garantir que a planilha esteja bem estruturada, com cabeçalhos claros (devidamente preenchidos) e dados organizados em colunas. No exemplo abaixo, há uma tabela com as colunas “Descrição”, “Valor” e “Situação”, representando diferentes despesas.

\subsubsection{Passo 2: Inserindo a tabela dinâmica}
Com o intervalo selecionado, acesse o menu superior e clique em Inserir > Tabela dinâmica.

\begin{figure}[h]
	\centering
	\includegraphics[width=.5\textwidth]{images/planilhas/imagem_27.png}
	\caption{Menu de criação de tabela dinâmica}
	\label{fig:planilhas:tabelas1}
\end{figure}

Uma janela será exibida pedindo para confirmar o intervalo de dados e escolher onde a tabela será criada: em uma nova página ou em uma página existente. A opção mais comum é criar em uma nova página, mantendo o relatório separado da planilha original.

\begin{figure}[h]
	\centering
	\includegraphics[width=.5\textwidth]{images/planilhas/imagem_28.png}
	\caption{Exemplo de criação de tabela dinâmica}
	\label{fig:planilhas:tabelas2}
\end{figure}

Ao clicar em Criar, uma nova aba será aberta com a estrutura da tabela dinâmica e o Editor de tabela dinâmica visível à direita da tela.

\subsection{Conhecendo o Editor de Tabela Dinâmica}
O editor é o painel de controle dessa ferramenta. É nele que o usuário define quais dados serão exibidos, como serão organizados e quais cálculos serão aplicados. Ele possui quatro principais áreas de configuração:
\begin{itemize}
	\item \textbf{Linhas:} determina os itens que aparecerão listados verticalmente na tabela (por exemplo, “Descrição”).
	\item \textbf{Colunas:} organizam as categorias na horizontal (por exemplo, “Mês” ou “Situação”).
	\item \textbf{Valores:} representam os dados numéricos que serão calculados. Podem ser somas, médias, contagens, entre outros.
	\item \textbf{Filtros:} permitem exibir apenas informações específicas conforme critérios definidos.
\end{itemize}
No exemplo ilustrado abaixo, o campo “Descrição” foi adicionado às Linhas, e o campo “Valor” foi adicionado em Valores, resultando em um resumo simples e direto das despesas listadas.

\begin{figure}[h]
	\centering
	\includegraphics[width=.5\textwidth]{images/planilhas/imagem_29.png}
	\caption{Exemplo de gráfico feito com tabela dinâmica}
	\label{fig:planilhas:tabelas3}
\end{figure}

\subsection{Personalizando a Exibição dos Dados}

As tabelas dinâmicas do Google Planilhas oferecem uma série de opções de personalização para tornar os dados mais claros e visualmente organizados:

\begin{itemize}
	\item \textbf{Ordenação:} é possível ordenar os valores em ordem crescente ou decrescente, tanto para linhas quanto para colunas.
	\item \textbf{Totais:} pode-se exibir o total geral ou subtotais de grupos de dados.
	Agrupamento: categorias semelhantes podem ser agrupadas para uma análise mais segmentada.
	\item \textbf{Filtros adicionais:} permitem restringir a visualização a determinados períodos, categorias ou faixas de valores.
\end{itemize}

Essas funcionalidades tornam as tabelas dinâmicas extremamente versáteis, permitindo explorar diferentes perspectivas sobre os mesmos dados sem alterar a planilha original.

\begin{dica}
	\begin{itemize}
		\item Mantenha os dados de origem sempre atualizados, pois as tabelas dinâmicas se ajustam automaticamente conforme as alterações.
		\item Evite células mescladas no intervalo de origem, pois isso pode dificultar a leitura dos dados pela ferramenta.
		\item Use nomes claros nos cabeçalhos, garantindo que cada coluna tenha uma identificação única.
		\item Combine com gráficos dinâmicos para transformar os resultados em visualizações mais intuitivas.
	\end{itemize}
\end{dica}

%% !TeX root = ../../main.tex
% sections/planilhas/compartilhamento.tex

\section{Trabalhando em equipe no Google Planilhas}
Além da funcionalidade de uma ferramenta de planilha individual, no Google Planilhas é possível trabalhar em colaboração com outras pessoas em tempo real. As funcionalidades de compartilhamento, comunicação e controle de versões são os pilares que permitem que equipes trabalhem de maneira eficiente, eliminando barreiras geográficas e otimizando fluxos de trabalho.

\subsection{Entendendo a colaboração em tempo real}
O principal diferencial do Google Planilhas nos ambientes de trabalho é a sua capacidade de permitir que múltiplos usuários visualizem e editem o mesmo documento simultaneamente. Este conceito elimina a prática obsoleta e propensa a erros de gerenciar múltiplas versões de um mesmo arquivo, como \textit{Relatorio\_Vendas\_v1.xlsx} ou \textit{Relatorio\_Vendas\_v2.xlsx}. Em vez disso, a informação é centralizada num único arquivo, acessível a todos os membros da equipe ao mesmo tempo. 

Por exemplo, ao organizar um evento como um churrasco entre amigos, uma planilha compartilhada pode ser usada para listar os itens necessários. Ao invés de um organizador centralizar as informações através de um grupo de mensagens, cada participante pode acessar a planilha e marcar os itens que irá levar. As atualizações são vistas por todos instantaneamente, evitando itens duplicados e garantindo que nada seja esquecido.

\subsection{Compartilhando sua planilha}

O processo é iniciado através do menu superior, navegando até \textbf{Arquivo > Compartilhar > Compartilhar} com outras pessoas. A partir daí, existem dois métodos principais para conceder acesso. 

\begin{figure}[h]
	\centering
	\includegraphics[width=.5\textwidth]{images/planilhas/imagem_32.png}
	\caption{Menu de compartilhamento}
	\label{fig:planilhas:compartilhamento1}
\end{figure}

O primeiro método é o Convite Direto. Ao inserir os endereços de e-mail dos colaboradores, um convite formal é enviado. Esta abordagem oferece um maior nível de controle e segurança pois o acesso fica estritamente vinculado às contas Google dos convidados. É o método preferencial para ambientes corporativos e projetos com informação sensível.

O segundo método é o Compartilhamento por Link. Esta opção gera um URL único que pode ser distribuído. Ao utilizar esta funcionalidade, é crucial configurar o nível de acesso associado ao link. A opção "Restrito" garante que apenas as pessoas adicionadas por e-mail possam abrir o link. A opção "Qualquer pessoa com o link" torna a planilha acessível a qualquer um que possua o URL, o que representa um risco de segurança significativo se o documento contiver informações confidenciais. Este tipo de acesso é mais apropriado para materiais de consulta pública ou documentos que não contenham dados sensíveis.

\subsection{Gerenciando permissões}

Nem todos os colaboradores necessitam do mesmo nível de acesso a uma planilha. O Google Planilhas oferece três níveis de permissão principais. 

\begin{itemize}
	\item \textbf{Leitor:} essa permissão permite apenas visualizar o conteúdo da planilha, incluindo dados, fórmulas e gráficos. Os utilizadores não podem fazer qualquer tipo de alteração.
	\item \textbf{Comentador:} esse papel permite que o utilizador visualize todo o conteúdo e adicione comentários em células específicas, mas sem a capacidade de editar os dados diretamente. 
	\item \textbf{Editor:} essa permissão concede poder total sobre a planilha. Um editor pode alterar o conteúdo das células, modificar a formatação, adicionar ou excluir abas e até mesmo gerir as configurações de compartilhamento, adicionando ou removendo outros colaboradores.
\end{itemize}


\begin{figure}[h]
	\centering
	\includegraphics[width=.5\textwidth]{images/planilhas/imagem_33.png}
	\caption{Exemplo de comentário de outro usuário}
	\label{fig:planilhas:compartilhamento2}
\end{figure}

\subsection{Comentários e notas}

As ferramentas de comentários do Google Planilhas foram projetadas para que as discussões ocorram diretamente dentro do documento, atreladas a células ou intervalos específicos, eliminando a ambiguidade de comunicações externas.

Para adicionar um comentário basta selecionar a célula desejada, clicar com o botão direito e escolher "Comentário". Uma caixa de diálogo aparecerá, permitindo a inserção da mensagem. Uma funcionalidade poderosa dentro dos comentários é o uso do símbolo "@" seguido pelo nome ou e-mail de um colaborador. Isto não só direciona a mensagem para a pessoa específica, mas também lhe envia uma notificação por e-mail, garantindo que a questão seja vista.

\begin{figure}[h]
	\centering
	\includegraphics[width=.5\textwidth]{images/planilhas/imagem_34.png}
	\caption{Exemplo de nota}
	\label{fig:planilhas:compartilhamento3}
\end{figure}

Um comentário possui um ciclo de vida: ele pode ser respondido, criando uma discussão; pode ser editado; e uma vez que a questão tenha sido resolvida, pode ser marcado como "Resolvido". Esta ação oculta o comentário da visualização principal, limpando a interface e servindo como um registo de que a tarefa foi concluída.

É importante distinguir entre Comentários e Notas. Enquanto os comentários são projetados para diálogos dinâmicos e discussões, as Notas (acessíveis também pelo menu do botão direito) servem para anotações estáticas e informativas. Uma nota é ideal para explicar o propósito de uma fórmula complexa, a origem de um dado específico ou para deixar instruções sobre como uma determinada célula deve ser preenchida.

\begin{figure}[h]
	\centering
	\includegraphics[width=.5\textwidth]{images/planilhas/imagem_35.png}
	\caption{Botão compartilhar}
	\label{fig:planilhas:compartilhamento4}
\end{figure}


\subsection{Histórico de versões}

É uma espécie de registo de segurança automático que captura todas as alterações feitas no documento ao longo do tempo, identificando quem fez cada alteração e quando. O histórico de versões atende a duas necessidades principais.

\begin{itemize}
	\item \textbf{Auditoria e rastreabilidade:} permite verificar exatamente quem modificou um valor específico e em que momento. Em ambientes de trabalho colaborativos, isto é fundamental para a responsabilização e para entender a evolução dos dados.
	
	\item \textbf{Recuperação de erros:} por exemplo, se uma aba inteira for acidentalmente apagada ou se uma fórmula crítica for corrompida, basta clicar no ícone do histórico de versões (\autoref{fig:planilhas:compartilhament5}) ou navegar para \textbf{Arquivo > Histórico de versões > Ver histórico de versões}. Uma barra lateral então é exibida com uma lista detalhada de todas as versões anteriores do documento, agrupadas por data e por autor da modificação. É possível selecionar qualquer versão anterior, visualizá-la, e com um único clique no botão "Restaurar esta versão", reverter o documento inteiro para aquele estado anterior, recuperando todo o trabalho perdido.
\end{itemize}

\begin{figure}[h]
	\centering
	\includegraphics[width=.9\textwidth]{images/planilhas/imagem_36.png}
	\caption{Interface do Google Planilhas}
	\label{fig:planilhas:compartilhament5}
\end{figure}



%% !TeX root = ../../main.tex
% sections/planilhas/dicas.tex

\section{Dicas e boas práticas}
A criação de uma planilha eficaz vai além da simples inserção de dados e fórmulas. Ela envolve um conjunto de boas práticas e princípios de design que garantem que o documento seja claro, confiável, fácil de usar e escalável.

\subsection{Organização e estrutura de dados}
A base de uma planilha eficiente é uma boa organização e uma estruturação de dados objetiva. Uma base sólida garante que as ferramentas mais avançadas da plataforma funcionem de maneira previsível e eficiente.

\subsection{Nomenclatura de arquivos clara e consistente}
A clareza começa com os nomes. Nomes genéricos como "Planilha sem título" ou "Página1" geram confusão e dificultam a navegação.

\begin{itemize}
	\item \textbf{Arquivos:} Adote um padrão de nomenclatura consistente para os arquivos que inclua informações relevantes como o conteúdo, a área responsável e a data (ex: Relatorio-Vendas\_Marketing\_2025-10). É recomendado também evitar o uso de espaços nos nomes dos arquivos para garantir uma melhor compatibilidade durante a exportação para outros formatos.   
	\item \textbf{Páginas:} Renomeie sempre as páginas com nomes curtos e descritivos que indiquem o seu conteúdo, como "Dados Brutos", "Dashboard", "Resumo Mensal" ou "Instruções".
\end{itemize}

\subsection{Cabeçalhos únicos e descritivos}
Cada coluna deve ter um cabeçalho único e descritivo, localizado na primeira linha do conjunto de dados. Estes cabeçalhos são identificadores funcionais dos conteúdos da respectiva coluna. Ferramentas essenciais, como a Ordenação, os Filtros e as Tabelas Dinâmicas dependem diretamente deles. A ausência de cabeçalhos ou a existência de cabeçalhos duplicados é a causa mais comum de erros ao tentar analisar dados.

\subsection{Evitar células mescladas}
A funcionalidade de mesclar células é útil para criar títulos que se estendem por várias colunas acima de uma tabela de dados. No entanto, utilizar células mescladas dentro de um intervalo de dados estruturado é uma prática que deve ser evitada pois ela quebra a estrutura de grade da planilha, onde cada dado ocupa uma única célula numa intersecção de linha e coluna.  

A principal consequência é o impedimento da seleção correta de colunas inteiras, causando falhas em operações de ordenação e filtro, tornando impossível a criação de uma Tabela Dinâmica a partir desses dados. Como alternativa para centralizar um título sobre uma única coluna, utilize a opção "Centralizar" no alinhamento horizontal.

\subsection{Legibilidade e impacto visual}
Uma planilha bem formatada não é apenas esteticamente agradável; ela é mais fácil e rápida de ler e interpretar. 

\subsection{Formatação com propósito}
O objetivo da formatação é clareza, não decoração. O excesso de cores, fontes e bordas pode poluir a visualização e dificultar a compreensão em vez de a facilitar. Algumas dicas de formatação:
Use negrito para destacar cabeçalhos e totais.   
Use cores de fundo sutis para agrupar informações relacionadas ou destacar linhas e colunas importantes.
Use bordas para separar claramente diferentes blocos de informação.  
Use cores de texto com significado, como verde para valores positivos/receitas e vermelho para valores negativos/despesas.   

\begin{figure}[h]
	\centering
	\includegraphics[width=.9\textwidth]{images/planilhas/imagem_37.png}
	\caption{Tabela formatada}
	\label{fig:planilhas:dica1}
\end{figure}

\subsection{Padronização eficiente com Pintar Formatação}
Para garantir a consistência visual em toda a planilha utilize a ferramenta "Pintar Formatação", representada pelo ícone de um rolo de pintura na barra de ferramentas. Esta ferramenta permite copiar todo o estilo de uma célula (fonte, cor de fundo, bordas, formato de número, etc.) e aplicá-lo rapidamente a outra célula ou intervalo. É a forma mais eficiente de garantir que todos os cabeçalhos, totais ou blocos de dados tenham uma aparência padronizada.

\begin{figure}[h]
	\centering
	\includegraphics[width=.9\textwidth]{images/planilhas/imagem_38.png}
	\caption{Barra de formatação, com a ferramenta "Pintar Formatação" destacada}
	\label{fig:planilhas:dica2}
\end{figure}

\subsection{Cores alternadas}
Aplicar uma formatação que alterna a cor de fundo entre linhas claras e escuras melhora drasticamente a legibilidade de tabelas largas e densas. Isto ajuda o olho a seguir uma linha específica da esquerda para a direita sem se desviar para as linhas adjacentes, reduzindo o cansaço visual e a probabilidade de erros de leitura. É possível automatizar esse processo através de \textbf{Formatar > Cores alternadas}.

\begin{figure}[h]
	\centering
	\includegraphics[width=.9\textwidth]{images/planilhas/imagem_39.png}
	\caption{Painel "Cores alternadas"}
	\label{fig:planilhas:dica3}
\end{figure}


%\printbibliography

\end{document}
